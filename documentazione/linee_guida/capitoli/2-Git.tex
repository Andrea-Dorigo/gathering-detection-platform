% \lstset{style=mystyle}


\chapter{Git}

\section{Aggiunta o modifica di files}
Poniamo che un componente del gruppo voglia aggiungere o modificare uno o pi\`{u} files nella repository del progetto. \\
Il primo passo \`{e} creare una nuovo feature branch tramite
\begin{lstlisting}
git flow feature start analisi-dati-uber
\end{lstlisting}
Questo comando crea un branch locale a partire dal develop chiamato \texttt{feature/analisi-dati-uber} e vi esegue il checkout automaticamente; tutti i prossimi commit e push avverranno su questo branch, evitando l'immissione di errori o file ancora incompleti o non verificati sul develop. \\
\textbf{Per rendere possibile la collaborazione sullo stesso branch} \`{e} possibile pubblicarlo in remoto con il comando
\begin{lstlisting}
git flow feature publish analisi-dati-uber
\end{lstlisting}
Chiunque desideri lavorare su questo branch necessita di immettere
\begin{lstlisting}
git flow feature track analisi-dati-uber
\end{lstlisting}
che crea un branch locale dallo stesso nome e lo sincronizza con quello remoto.

Al completamento \textbf{e verifica} della nuova funzionalit\`{a}, \`{e} possibile richiudere il branch nel \texttt{develop} ed eliminarlo tramite il comando \\
\texttt{git flow feature finish analisi-dati-uber}.\\
Verr\`{a} automaticamente generato un \texttt{merge commit}.
Il responsabile di progetto si riserva la pubblicazione delle \texttt{release}.\\
\\
\section{Link utili e best practices sull'utilizzo di git}
Potete trovare un ottimo elenco dei comandi di git flow pi\`{u} importanti e delle loro funzioni qui:
\url{https://danielkummer.github.io/git-flow-cheatsheet/}\\
\\
Un buon riassunto sull'utilizzo di git flow: \\
\url{https://www.atlassian.com/git/tutorials/comparing-workflows/gitflow-workflow}\\
\\
Un altro ottimo riassunto sulle best practice da seguire nei commit qui: \\
\url{https://www.theserverside.com/video/Follow-these-git-commit-message-guidelines}\\
\\
Fra le molte pratiche importanti da seguire nella scrittura del titolo del commit si evidenzia:
\begin{itemize}
\item descrivi cosa hai fatto e perch\'{e}, ma non come.
\item l'utilizzo dell'imperativo;
\end{itemize}
Per quest'ultimo punto vi riporto un paragrafo del link qui sopra che aiuta la comprensione (\`{e} presente un errore nel testo dell'articolo, che ho corretto qui sotto):
\begin{lstlisting}
Fixed the fencepost error //bad

Fixing the fencepost error //bad

Fix the fencepost error //good
\end{lstlisting}
The imperative mood is the one git commit message guideline that developers tend to violate most often. A good rule of thumb is that a git commit message can be appended to the statement "If applied, this commit will …." The resulting sentence should make grammatical sense. As you can see from the following three examples, gerunds and past tense commits fail the test, while the imperative tense does not:}
\begin{lstlisting}
If applied, this commit will Fixed the fencepost error //bad

If applied, this commit will Fixing the fencepost error //bad

If applied, this commit will Fix the fencepost error //good
\end{lstlisting}
