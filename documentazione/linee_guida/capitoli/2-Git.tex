% \lstset{style=mystyle}


\chapter{Git}

\section{Aggiunta o modifica di files}
Poniamo che un componente del gruppo voglia aggiungere o modificare uno o più files nella repository del progetto.
Il primo passo è creare una nuovo feature branch tramite il comando
\begin{lstlisting}
git flow feature start analisi-uber
\end{lstlisting}
Questo crea un branch locale a partire dal develop chiamato \texttt{feature/analisi-uber} e vi esegue il checkout automaticamente; tutti i prossimi commit e push avverranno su questo branch, evitando l'immissione di errori o file ancora incompleti o non verificati sul \texttt{develop}.
\textbf{Per rendere possibile la collaborazione sullo stesso branch} è possibile pubblicarlo in remoto con il comando
\begin{lstlisting}
git flow feature publish analisi-uber
\end{lstlisting}
Chiunque desideri lavorare su questo branch necessita di immettere
\begin{lstlisting}
 git checkout -b feature/analisi-uber origin/feature/analisi-uber
\end{lstlisting}
che crea un branch locale dallo stesso nome e lo sincronizza con quello remoto.
Al completamento \textbf{e verifica} della nuova funzionalità, è possibile richiudere il branch nel \texttt{develop} ed eliminarlo tramite il comando
\begin{lstlisting}
git flow feature finish analisi-uber
\end{lstlisting}
Verrà automaticamente generato un \texttt{merge commit}.
Il responsabile di progetto si riserva la pubblicazione delle \texttt{release}.

\section{Link utili e best practices sull'utilizzo di git}
Potete trovare un elenco dei comandi di git flow più importanti e relative funzioni
\href{https://danielkummer.github.io/git-flow-cheatsheet/}{qui} \cite{GitFlowCheatseet}. \\
Un buon riassunto sull'utilizzo di git flow si può trovare \href{https://www.atlassian.com/git/tutorials/comparing-workflows/gitflow-workflow}{a questo link} \cite{GitFlowWorkflowTutorial}.\\
Dell'altro materiale riguardante le best practice da seguire nei commit \href{https://www.theserverside.com/video/Follow-these-git-commit-message-guidelines}{a questo indirizzo} \cite{GitCommitGuidelines}. \\
Fra le molte pratiche di cui tener conto nella scrittura del titolo del commit si evidenzia:
\begin{itemize}
\item esso deve contenere la descrizionee di cosa è stato fatto e perchè, ma non come.
\item l'utilizzo dell'imperativo;
\end{itemize}
Per quest'ultimo punto vi riporto un paragrafo del link qui sopra che ne facilita la comprensione (è presente un errore nel testo dell'articolo, che ho corretto qui sotto):
\begin{lstlisting}
Fixed the fencepost error //bad

Fixing the fencepost error //bad

Fix the fencepost error //good
\end{lstlisting}
\textit{The imperative mood is the one git commit message guideline that developers tend to violate most often. A good rule of thumb is that a git commit message can be appended to the statement "If applied, this commit will …." The resulting sentence should make grammatical sense. As you can see from the following three examples, gerunds and past tense commits fail the test, while the imperative tense does not:}
\begin{lstlisting}
If applied, this commit will Fixed the fencepost error //bad

If applied, this commit will Fixing the fencepost error //bad

If applied, this commit will Fix the fencepost error //good
\end{lstlisting}
