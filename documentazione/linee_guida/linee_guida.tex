\documentclass[a4paper,12pt]{report}
\usepackage[utf8]{inputenc}
\usepackage{graphicx}
\usepackage{float}
\usepackage{tabularx}
\usepackage{makecell}
\usepackage{titlesec}
\usepackage{fancyhdr}
\usepackage{lastpage}
\usepackage{xurl}
\usepackage{hyperref}
\usepackage{geometry}
\usepackage{color}
\usepackage{microtype}
\usepackage{enumerate}
\usepackage{listings}
\usepackage[table,dvipsnames]{xcolor}
\setcounter{tocdepth}{5}
\setcounter{secnumdepth}{5} %SONO I DUE COMANDI PER AVERE LE SUBSUBSECTION NUMERATE E PRESENTI NELL'INDICE
\renewcommand{\contentsname}{Indice} %QUESTO SERVE PER AVERE L'INDICE CON IL NOME CHE VOGLIO IO
\titleformat{\chapter}[display]
{\normalfont\bfseries}{}{0pt}{\LARGE} %QUESTO SERVE PER AVERE SOLO IL NOME DEL CAPITOLO CHE VOGLIO IO
\titlespacing*{\chapter}{0cm}{0cm}{0.2cm}
\geometry{
	left=20mm,
	right=20mm,
}
\fancypagestyle{plain}{
	\fancyhf{}
	\lhead{\includegraphics[width=3cm]{../immagini/minilogo.jpg}}
	\chead{}
	\rhead{\fontsize{12}{10}Linee Guida}
	\lfoot{}
	\cfoot{\thepage\ di \pageref*{LastPage}}
	\rfoot{}
}

\definecolor{atomlightorange}{rgb}{0.88,0.76,0.55}
\definecolor{atomdarkgrey}{RGB}{59,62,75}

\usepackage{tikz}

% set listings
\lstset{%
    basicstyle=\footnotesize\ttfamily\color{atomlightorange},
    framesep=20pt,
		belowskip=10pt,
		aboveskip=10pt
}

% add frame environment
\usepackage[%
    framemethod=tikz,
    skipbelow=8pt,
    skipabove=13pt
]{mdframed}
\mdfsetup{%
    leftmargin=0pt,
    rightmargin=0pt,
    backgroundcolor=atomdarkgrey,
    middlelinecolor=atomdarkgrey,
    roundcorner=6
}

\usepackage{etoolbox}% >= v2.1 2011-01-03
\BeforeBeginEnvironment{lstlisting}{\begin{mdframed}\vspace{-0.7em}}
\AfterEndEnvironment{lstlisting}{\vspace{-0.5em}\end{mdframed}}

% needed for \lstcapt
\def\ifempty#1{\def\temparg{#1}\ifx\temparg\empty}

% make new caption command for listings
\usepackage{caption}
\newcommand{\lstcapt}[2][]{%
    \ifempty{#1}%
        \captionof{lstlisting}{#2}%
    \else%
        \captionof{lstlisting}[#1]{#2}%
    \fi%
    \vspace{0.75\baselineskip}%
}

\hypersetup{
    colorlinks=true,
    linkcolor=black,
    filecolor=black,
    urlcolor=blue,
		citecolor=black,
}

\pagestyle{plain}

\makeatletter
	 \def\thebibliography#1{\chapter*{Bibliografia\@mkboth
		 {Bibliografia}{Bibliografia}}\list
		 {[\arabic{enumi}]}{\settowidth\labelwidth{[#1]}\leftmargin\labelwidth
 \advance\leftmargin\labelsep
 \usecounter{enumi}}
 \def\newblock{\hskip .11em plus .33em minus .07em}
 \sloppy\clubpenalty4000\widowpenalty4000
 \sfcode`\.=1000\relax}
	 \makeatother


\begin{document}

\makeatletter
\begin{titlepage}
    \begin{center}
    \vspace*{-4,0cm}
    \author{Jawa Druids}
    \title{Linee Guida}
    \date{} %LASCIARE QUESTO CAMPO VUOTO, SE LO TOLGO STAMPA LA DATA CORRENTE
    \includegraphics[width=0.7\linewidth]{../immagini/DRUIDSLOGO.jpg}\\[4ex]
    {\huge \bfseries  \@title }\\[2ex]
    {\LARGE  \@author}\\[50ex]
    \vspace*{-8,0cm}
    \begin{table}[H]
        \centering
        \begin{tabular}{c|c}
            \textbf{Versione} & x.x.x \\
            \textbf{Data approvazione} & xx-xx-xxxx\\
            \textbf{Responsabile} & Nome Cognome\\
            \textbf{Redattori} & Andrea Dorigo \\
            \textbf{Verificatori} & Nome Cognome \\
        %MAKECELL SERVE PER POI ANDARE A CAPO ALL'INTERNO DELLA CELLA
            \textbf{Stato} & Redazione in corso\\
            \textbf{Lista distribuzione} & \makecell{Jawa Druids \\ Tullio \\ NomeAziendaProponente}\\
            \textbf{Uso} & Interno
        \end{tabular}
    \end{table}
					\vspace{.8cm}
	\hfill \break
    \fontsize{17}{10}\textbf{Sommario}\\
		\vspace{.3cm}
    Linee guide da seguire per una sessione lavorativa standard con git flow, Trello e Google Sheet.
    \end{center}
\end{titlepage}
\makeatother

\def\myformat#1{
	\centering\huge#1
}
\def\tabularxcolumn#1{m{#1}}
\quad
\section*{\myformat{Registro delle modifiche}}

{\rowcolors{2}{Apricot!90!Bittersweet!20}{Bittersweet!70!Apricot!40}
\begin{table}[H]
  \caption[Registro delle Modifiche]{}
	\vspace{-1.4cm}
  \begin{center}
    \begin{tabularx}{\textwidth}{|Y|c|c|c|c|}
    	\hline
    	\rowcolor{Melon}
    	\textbf{Modifica} & \textbf{Autore} & \textbf{Ruolo} & \textbf{Data} & \textbf{Versione}\\
      \hline
      \textit{Aggiunte sezioni §1.2, §1.3} & Andrea Dorigo & \textit{Responsabile} & 2-12-2020 & v0.0.2 \\
    	\hline
    	\textit{Aggiunta sezione §1.1} & Andrea Dorigo & \textit{Responsabile} & 30-11-2020 & v0.0.1 \\
    	\hline
    \end{tabularx}
  \end{center}
\end{table}

\tableofcontents{}
\chapter{Introduzione}

Lo scopo di questo documento è la stesura di un elenco di linee guida ed esempi che i componenti sono incoraggiati a seguire per migliorare l'efficacia della collaborazione.
A differenza delle Norme di Progetto, questo documento è redatto in un linguaggio più informale per facilitarne la comprensione, con spezzati di codice a scopo esemplificativo e riferimenti esterni sulle best practices da seguire.\\
Il documento può essere soggetto a modifiche ed aggiunte per tutta la durata del progetto.

\chapter{Git}

\section{Esempio di un'aggiunta di funzionalit\`{a} o documento}
Poniamo che un componente del gruppo voglia aggiungere una funzionalit\`{a} o documento al progetto in corso. \\
Il primo passo \`{e} creare una nuovo feature branch tramite il comando \\
\texttt{git flow feature start analisi-dati-uber}\\
Questo comando crea un branch a partire dal develop chiamato \texttt{feature/analisi-dati-uber} e vi sposta automaticamente su ques'ultimo. Ci\`{o} significa che tutti i prossimi commit e push avverranno su questo branch, evitando l'immissione di errori o file ancora incompleti e non verificati sul develop. \\
\textbf{Per rendere possibile la collaborazione sullo stesso branch} \`{e} possibile pubblicarlo con il comando \\
\texttt{git flow feature publish analisi-dati-uber}\\
A questo punto chiunque voglia lavorare sullo stesso branch puo' farlo con il comando \\
\texttt{git flow feature track analisi-dati-uber}.\\
Qui ora e' possibile fare i vari \texttt{commit} e \texttt{push} su questo nuovo branch, senza il rischio di aggiungere bug o errori nel develop.
Una volta terminata e testata la nuova funzionalita', si puo' richiudere il branch nel develop ed eliminarlo tramite il comando \\
\texttt{git flow feature finish analisi-dati-uber}.\\
Si consiglia, nei casi di feature/hotfix/bugfix importanti di richiedere prima l'approvazione del responsabile di progetto.\\
Il responsabile di progetto si riserva la pubblicazione delle \texttt{release}.\\
\\
Potete trovare un ottimo cheatsheet dei comandi di git flow e delle loro funzioni qui:
\url{https://danielkummer.github.io/git-flow-cheatsheet/}\\
\\
Un buon riassunto sull'utilizzo di git flow: \\
\url{https://www.atlassian.com/git/tutorials/comparing-workflows/gitflow-workflow}\\
\\
Un altro ottimo riassunto sulle best practice dei commit qui: \\
\url{https://www.theserverside.com/video/Follow-these-git-commit-message-guidelines}\\
\\
Fra le tante cose importanti nel titolo del commit si evidenzia:
\begin{itemize}
\item descrivi cosa hai fatto e perche', ma non come.
\item l'utilizzo dell'imperativo;
\end{itemize}
Per quest'ultimo punto vi riporto un paragrafo del link qui sopra che aiuta la comprensione (c'e' un errore nel testo dell'articolo, che ho corretto qui sotto):
\\
\texttt{Fixed the fencepost error //bad \\
\\
Fixing the fencepost error //bad \\
\\
Fix the fencepost error //good} \\
\\
\textit{The imperative mood is the one git commit message guideline that developers tend to violate most often. A good rule of thumb is that a git commit message can be appended to the statement "If applied, this commit will …." The resulting sentence should make grammatical sense. As you can see from the following three examples, gerunds and past tense commits fail the test, while the imperative tense does not:}\\
\\
\texttt{If applied, this commit will Fixed the fencepost //error\\
\\
If applied,  this commit will Fixing the fencepost //error\\
\\
If applied,  this commit will Fix the fencepost //good}\\

\chapter{Trello}

Trello è la piattaforma scelta dal gruppo per la gestione del progetto, data la famigliarit\`{a} di certi componenti del gruppo con esso e la comprovata efficacia.
Qui sono raccolti tutti gli stadi futuri/presenti/passati del progetto, e l'attivit\`{a} che ogni componente sta svolgendo.

La \textbf{creazione} e lo \textbf{spostamento} delle schede è riservata all'amministratore o al responsabile (o quantomeno è richiesta la loro approvazione).
L'\textbf{assegnazione} delle schede è riservata al responsabile (o quantomeno è richiesta la sua approvazione).

\chapter{Sessione lavorativa standard}

Al termine di una sessione lavorativa, è bene:
\begin{enumerate}[1.]
\item eseguire l'aggiornamento del registro delle modifiche nella documentazione (se necessario)
\item committare e pushare, per evitare la perdita del lavoro effettuato
\item segnare le ore di lavoro svolte sulla
\href{https://docs.google.com/spreadsheets/d/12esX1ISWQOKM-fjuHTLmAzRN0cWltksn7eGiPsBHBI0/edit?usp=sharing}{Tabella delle ore lavorative} \cite{TabellaDelleOreLavorative}.
\item aggiornare lo stato di avanzamento su
\href{https://trello.com/b/hIEOGbE9/jawadruids}{Trello} \cite{TrelloJawaDruids}.
\end{enumerate}

\begin{thebibliography}{}
    \bibitem{GitFlowCheatseet}
    Daniel Kummer.
    \textit{Git flow cheatsheet - efficient branching using git-flow by Vincent Driessen}.\\
		\url{https://danielkummer.github.io/git-flow-cheatsheet/}.
		\bibitem{GitFlowWorkflowTutorial}
		Atlassian Bitbucket.
		\textit{Git flow workflow tutorial}.\\
		\url{https://www.atlassian.com/git/tutorials/comparing-workflows/gitflow-workflow}.
		\bibitem{GitCommitGuidelines}
		Cameron McKenzie, TechTarget.
		\textit{How to write a Git commit message properly with examples}.\\
		\url{https://www.theserverside.com/video/Follow-these-git-commit-message-guidelines}.
		\bibitem{TabellaDelleOreLavorative}
		Jawa Druids, 2020.
		\textit{Tabella delle Ore Lavorative}.\\
		\url{https://docs.google.com/spreadsheets/d/12esX1ISWQOKM-fjuHTLmAzRN0cWltksn7eGiPsBHBI0/edit?usp=sharing}.
		\bibitem{TrelloJawaDruids}
		Jawa Druids, 2020.
		\textit{Trello - dashboard Jawa Druids}.\\
		\url{https://trello.com/b/hIEOGbE9/jawadruids}.
\end{thebibliography}


\end{document}
