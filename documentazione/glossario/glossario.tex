\input{packages}
\input{config}

\begin{document}

\makeatletter
\begin{titlepage}
	\begin{center}
		\vspace*{-5cm}
		\author{Jawa Druids} 
		\title{Nome documento}
		\date{} %LASCIARE QUESTO CAMPO VUOTO, SE LO TOLGO STAMPA LA DATA CORRENTE
		\includegraphics[width=0.7\linewidth]{../immagini/DRUIDSLOGO.jpg}\\[4ex]
		{\huge \bfseries  \@title }\\[2ex] 
		{\LARGE  \@author}\\[50ex]
		\vspace*{-9cm}
		\begin{table}[H]
			\renewcommand{\arraystretch}{1.4}
			\centering
			\begin{tabular}{r | l}
				\textbf{Versione} & x.x.x \\%RIGA PER INSERIRE LA VERSIONE ULTIMA DEL DOCUMENTO
				\textbf{Data approvazione} & xx-xx-xxxx\\
				\textbf{Responsabile} & Nome Cognome\\
				\textbf{Redattori} & Nome Cognome \\
				\textbf{Verificatori} & \makecell[tl]{Nome Cognome \\ Nome Cognome} \\
				%MAKECELL SERVE PER POI ANDARE A CAPO ALL'INTERNO DELLA CELLA
				\textbf{Stato} & Stato\\
				\textbf{Lista distribuzione} & \makecell[tl]{Nome Gruppo \\ Nome Professori \\ Nome Azienda Proponente}\\
				\textbf{Uso} & Esterno            
			\end{tabular}
		\end{table}
		\vspace{0.1cm}
		\hfill \break
		\fontsize{17}{10}\textbf{Sommario} \\
		\vspace{0.1cm}
		\textit{Glossario} 
	\end{center}
\end{titlepage}
\makeatother

\quad
\begin{center}
	\LARGE\textbf{Registro delle modifiche}
\end{center}

\def\tabularxcolumn#1{m{#1}}
{\rowcolors{2}{RawSienna!90!RawSienna!20}{RawSienna!70!RawSienna!40}


\begin{center}
	\renewcommand{\arraystretch}{1.4}
	\begin{longtable}[c]{|p{2cm-1\tabcolsep}|p{2cm}|p{3cm-2\tabcolsep}|p{2,5cm-2\tabcolsep}|p{4cm-2\tabcolsep}|p{2,5cm}|}
		\hline
		\rowcolor{airforceblue}
		\makecell[c]{\textbf{Versione}} & \makecell[c]{\textbf{Data}} & \makecell[c]{\textbf{Autore}} & \makecell[c]{\textbf{Ruolo}} & \makecell[c]{\textbf{Modifica}} & \makecell[c]{\textbf{Verificatore}} \\
		\hline
		\centering v1.0.0 & 2021-04-18 & Andrea Dorigo & \centering \textit{Responsabile} & \textit{Approvazione del documento} & \makecell[c]{-} \\
		\hline
		\centering v0.1.0 & 2021-04-17 &  \centering - & \centering - &  \textit{Revisione complessiva del documento} & Mattia Cocco \\
		\hline
		\centering v0.0.1 & 2021-04-15 & Andrea Cecchin & \centering \textit{Redattore} &\textit{Stesura del documento}  & Mattia Cocco \\
		\hline
	\end{longtable}
\end{center}


%COMANDO PER LA CREAZIONE DELL'INDICE

\tableofcontents{}
\chapter{A}
\chapter{B}
\chapter{C}
\chapter{D}
\chapter{E}
\chapter{F}
\chapter{G}
\chapter{H}
\chapter{I}
\chapter{J}
\chapter{K}
\chapter{L}
\chapter{M}\\
\chapter{N}
\chapter{O}\\
\chapter{P}
\chapter{Q}
\chapter{R}
\chapter{S}
\chapter{T}
\chapter{U}
\chapter{V}
\chapter{W}
\chapter{X}
\chapter{Y}
\chapter{Z}

%PER RENDERE PIÙ CHIARA LA STESURA DEI DOCUMENTI È MEGLIO LASCIARE SEPARATI IN FILE DIVERSI OGNI CAPITOLO

% \input{esempio} -- esempio di codice per inserire un nuovo capitolo

\end{document}