\input{packages}
\input{config}

\begin{document}
	
	\makeatletter
	\begin{titlepage}
		\begin{center}
			\vspace*{-4cm}
			\author{Jawa Druids} 
			\title{Glossario}
			\date{} %LASCIARE QUESTO CAMPO VUOTO, SE LO TOLGO STAMPA LA DATA CORRENTE
			\includegraphics[width=0.5\linewidth]{../immagini/DRUIDSLOGO.jpg}\\[4ex]
			{\huge \bfseries  \@title }\\[2ex] 
			{\LARGE  \@author}\\[50ex]
			\vspace*{-9cm}
			\begin{table}[H]
				\renewcommand{\arraystretch}{1.4}
				\centering
				\begin{tabular}{r | l}
					\textbf{Versione} & v3.0.0 \\%RIGA PER INSERIRE LA VERSIONE ULTIMA DEL DOCUMENTO
					\textbf{Data approvazione} & 2021-04-16\\
					\textbf{Responsabile} & Andrea Cecchin\\
					\textbf{Redattori} & \makecell[tl]{Andrea Cecchin} \\
					\textbf{Verificatori} & \makecell[tl]{Andrea Dorigo } \\
					%MAKECELL SERVE PER POI ANDARE A CAPO ALL'INTERNO DELLA CELLA
					\textbf{Stato} & Approvato\\
					\textbf{Lista distribuzione} & \makecell[tl]{Jawa Druids \\ Prof. Tullio Vardanega \\ Prof. Riccardo Cardin \\ Sync Lab}\\
					\textbf{Uso} & Esterno          
				\end{tabular}
			\end{table}
			\vspace{0.1cm}
			\hfill \break
			\fontsize{17}{10}\textbf{Sommario} \\
			\vspace{0.1cm}
			Questo documento raccoglie le definizioni di alcuni termini utilizzati negli altri documenti al fine di risolvere possibili ambiguità ed incomprensioni.
		\end{center}
	\end{titlepage}
	\makeatother

\quad
\begin{center}
	\LARGE\textbf{Registro delle modifiche}
\end{center}

\def\tabularxcolumn#1{m{#1}}
{\rowcolors{2}{RawSienna!90!RawSienna!20}{RawSienna!70!RawSienna!40}


\begin{center}
	\renewcommand{\arraystretch}{1.4}
	\begin{longtable}[c]{|p{2cm-1\tabcolsep}|p{2cm}|p{3cm-2\tabcolsep}|p{2,5cm-2\tabcolsep}|p{4cm-2\tabcolsep}|p{2,5cm}|}
		\hline
		\rowcolor{airforceblue}
		\makecell[c]{\textbf{Versione}} & \makecell[c]{\textbf{Data}} & \makecell[c]{\textbf{Autore}} & \makecell[c]{\textbf{Ruolo}} & \makecell[c]{\textbf{Modifica}} & \makecell[c]{\textbf{Verificatore}} \\
		\hline
		\centering v1.0.0 & 2021-04-18 & Andrea Dorigo & \centering \textit{Responsabile} & \textit{Approvazione del documento} & \makecell[c]{-} \\
		\hline
		\centering v0.1.0 & 2021-04-17 &  \centering - & \centering - &  \textit{Revisione complessiva del documento} & Mattia Cocco \\
		\hline
		\centering v0.0.1 & 2021-04-15 & Andrea Cecchin & \centering \textit{Redattore} &\textit{Stesura del documento}  & Mattia Cocco \\
		\hline
	\end{longtable}
\end{center}


\tableofcontents{}

\chapter{A} \label{A}
\begin{itemize}
		\item \textbf{Attività}\\
		Composizione di più compiti$_G$ semplici svolti da singoli, o gruppi, al fine di perseguire un obbiettivo comune.
		
		\item \textbf{AnaConda}\\
		Strumento che permette la creazione di ambienti virtuali per testare librerie esterne o che non si vogliono mantenere nella memoria fisica del computer.
	\end{itemize}
\chapter{B} \label{B}
\begin{itemize}
	\item \textbf{Back-End} \\
	Interfaccia con la quale il gestore di un sito web dinamico ne gestisce i contenuti e le funzionalità. A differenza del frontend$_G$, l'accesso al backend è riservato agli amministratori del sito che possono accedere dopo essersi autenticati.
	%%OPPURE Back End è un termine largamente utilizzato per caratterizzare le interfacce che hanno come destinatario un programma. Una applicazione back end è un programma con il quale l'utente interagisce indirettamente, in generale attraverso l'utilizzo di una applicazione front-end. In una struttura client/server il back-end è il server. Viene spesso utilizzato in contrapposizione con Front End.
\end{itemize}
\chapter{C} \label{C}
\begin{itemize}
	\item \textbf{Capitolato (d'appalto)}\\
	Documento tecnico, allegato generalmente ad un contratto d'appalto, a cui si fa riferimento per la definizione delle specifiche tecniche dei lavori che verranno successivamente svolti per effetto dello stesso contratto.

	\item \textbf{Committente}\\
	Soggetto che ordina ad altri l'esecuzione di un lavoro.
	
	\item \textbf{Compito}\\
	Rappresenta la componente elementare di un'attività, viene svolta da un soggetto per senso del dovere.
	Viene anche utilizzato al plurale: compiti.
	
	\item \textbf{Caso d'uso}\\
	Un caso d’uso è un insieme di scenari (sequenze di azioni) che hanno in comune uno scopo finale (obiettivo) per un utente (attore). 
	Viene anche utilizzato al plurale: casi d'uso.
\end{itemize}
\chapter{D} \label{D}
	\begin{itemize}		
		\item \textbf{Design Pattern}\\
		Soluzione progettuale generale ad un problema ricorrente. Si tratta di una descrizione, o modello logico, da applicare per la risoluzione di un problema che può presentarsi in diverse situazioni durante le fasi di progettazione e sviluppo del prodotto software.
		
		\item \textbf{Diagrammi di Gantt}\\
		Strumento di supporto alla gestione dei progetti, è costruito partendo da un asse orizzontale, a rappresentazione dell'arco temporale totale del progetto, suddiviso in fasi incrementali  e da un asse verticale, a rappresentazione delle mansioni o attività che costituiscono il progetto. Permette dunque la rappresentazione grafica di un calendario di attività$_G$, utile al fine di pianificare, coordinare e tracciare specifiche attività$_{\scaleto{G}{3pt}}$ in un progetto dando una chiara illustrazione dello stato d'avanzamento del progetto rappresentato.  Ad ogni attività$_{\scaleto{G}{3pt}}$ possono essere in generale associati una serie di attributi come la durata, le risorse ed il costo.
		
		\item \textbf{Discord}\\
		Software gratuito utilizzato come canale di comunicazione da parte del gruppo di lavoro.
		Accessibile mediante smartphone, tablet e computer.
		Permette la comunicazione testuale, condivisione di file, chiamate vocali e video chiamate.
		Inoltre permette la creazione interna di canali personalizzati in modo da organizzare al meglio la comunicazione tra le persone.
		
		\item \textbf{Docker}\\
		Docker è una piattaforma software che permette di creare, testare e distribuire applicazioni con la massima rapidità. Docker raccoglie il software in unità standardizzate chiamate container che offrono tutto il necessario per la loro corretta esecuzione, incluse librerie, strumenti di sistema, codice e runtime. Con Docker, è possibile distribuire e ricalibrare le risorse per un'applicazione in qualsiasi ambiente, tenendo sempre sotto controllo il codice eseguito.
		
		\item \textbf{Drag}\\
		Parola inglese col significato di trascinare, nel contesto utilizzato indicato lo spostamento della heat map$_{\scaleto{G}{3pt}}$ negli assi orrizzontali e verticali.
	\end{itemize}
\chapter{E} \label{E}
\begin{itemize}
	\item \textbf{ESLint}\\
	Uno strumento di analisi statica del codice Javascript, permette l'inserimento di regole personali da seguire e segue come principi di codifica l'ECMAScript.
\end{itemize}
\chapter{F} \label{F}
\begin{itemize}	
	\item \textbf{Fogli Google}\\
	Programma per fogli di calcolo facente parte della suite gratuita di Google Docs Editor offerta da Google.
		
	\item \textbf{Fornitore}\\
	Individuo, team o azienda incaricato di realizzare  il prodotto software richiesto dal proponente$_G$.		
	
	\item \textbf{Framework} \\
	Utilizzato per descrivere la struttura operativa nella quale viene elaborato un dato software.
	Un framework, in generale, include software di supporto, librerie, un linguaggio per gli script e altri software che possono aiutare a mettere insieme le varie componenti di un progetto.
	
	\item \textbf{Front-End/Front end/Frontend} \\
	Parte di un sistema software che gestisce l'interazione con l'utente o con sistemi esterni che producono dati di ingresso.
	Tali dati sono poi utilizzabili dal Back-end$_G$.


\end{itemize}
\chapter{G} \label{G}
\begin{itemize}
	\item \textbf{Git}\\
	Sistema di controllo gratuito a versione distribuita progettato per tenere traccia del lavoro svolto durante l'intero periodo di sviluppo del software.
	Utilizzato anche per tenere traccia di tutte le modifiche fatte nei file.
	I suoi punti di forza sono l'integrità dei dati e il supporto per flussi di lavoro distribuiti e non lineari.

	\item \textbf{Git-Flow}\\
	Si tratta di un ramo di Git$_G$, molto adatto alla collaborazione e scalarità del team di sviluppo.
	
	\item \textbf{Git-Hub}\\
	GitHub è un servizio di hosting per progetti software. Il nome deriva dal fatto che esso è una implementazione dello strumento di controllo versione distribuito Git$_G$.
	
	\item \textbf{Git-Kraken}\\
	Piattaforma grafica per Git$_G$.
	Utilizzata per tenere traccia graficamente della repository$_G$, vedere i branch$_G$, potendo anche scaricare il lavoro presente in locale, crearne di nuovi.
	Inoltre si può vedere lo storico di tutte le modifiche e la possibilità di fare nuovi commit.
	
	\item \textbf{GET}\\
	Il GET è una tipologia di richiesta al server che permette di ottenere informazioni riguardanti una risorsa.
	\end{itemize}
\chapter{H} \label{H}
\begin{itemize}
	\item \textbf{Heat-Map} \\
	 Rappresentazione grafica dei dati dove i singoli valori contenuti in una matrice sono rappresentati da colori.
\end{itemize}
\chapter{I} \label{I}
\begin{itemize}
	\item \textbf{Indice di Gulpease}\\
	Indice di leggibilità di un testo tarato sulla lingua italiana.
	Rispetto ad altri, ha il vantaggio di utilizzare la lunghezza delle parole in lettere anziché in sillabe, semplificandone il calcolo automatico.
	
	\item \textbf{Inspection}\\
	Tecnica di verifica dove il	\textit{Verificatore$_G$} eseguirà una lettura e un controllo del documento più mirato nei punti dove si sa già essere fonte di errori.
		
	\item \textbf{ISO/IEC 9126}\\
	Insieme di normative e linee guida, sviluppate dall’Organizzazione internazionale per la
	normazione (ISO) in collaborazione con la Commissione Elettrotecnica Internazionale (IEC),
	preposte a descrivere un modello di qualità del software.
	
	\item \textbf{ISO/IEC 15504}\\
	Insieme di documenti tecnici che permettono di valutare oggettivamente la qualità di un processo software a fini migliorativi. 
	Fornisce delle valutazioni sui processi in maniera ripetibile, oggettiva e comparabile. È uno degli standard internazionali dell’Organizzazione internazionale per la normalizzazione (ISO) e della Commissione elettrotecnica internazionale
	(IEC), sviluppato dal sottocomitato congiunto ISO e IEC.

	
	
	
\end{itemize}

\chapter{J} \label{J}
\begin{itemize}
	\item \textbf{Java} \\
	 Linguaggio di programmazione ad alto livello, orientato agli oggetti e a tipizzazione statica, che si appoggia sull'omonima piattaforma software di esecuzione, specificamente progettato per essere il più possibile indipendente dalla piattaforma hardware di esecuzione. Le principali caratteristiche di Java sono la portabilità, cioè il codice sorgente e' compilato in bytecode e può essere eseguito su ogni PC che ha JVM (Java Virtual Machine), e la robustezza.

	 \item \textbf{Javascript} \\
	 Linguaggio di programmazione orientato ad oggetti ed eventi.
	 Utilizzato specificatamente per la programmazione Web lato client.
	 Aggiunge al sito effetti dinamici tramite funzioni invocate da un'azione eseguita sulla pagine Web(es. click del mouse, movimento del mouse, caricamento pagina).

	 \item \textbf{JSON} \\
	 Acronimo di "JavaScript Object Notation".
	 Si tratta di un formato file adattato all'interscambio di dati fra applicazioni client-server.
	 Basato sul linguaggio Javascript$_G$.

	 \item \textbf{JUnit} \\
	 Framework di unit testing per il linguaggio di programmazione Java$_G$.
	 Necessita di una versione Java$_{\scaleto{G}{3pt}}$ 8 o maggiore.
	 
	 \item \textbf{Jupiter}\\
	 Jupiter Notebook è un'applicazione open-source che permette di creare e condividere documenti che contengono codice, equazioni, testo e grafici.
\end{itemize}

\chapter{K} \label{K}
\begin{itemize}
	\item \textbf{Kafka}\\
	Apache Kafka è una piattaforma open-source$_G$ di stream(si veda streaming$_G$) processing. Scritta in Java$_G$ e
	Scala e sviluppata dall’Apache Software Foundation, mira a creare una piattaforma a bassa
	latenza ed alta velocità per la gestione di feed dati in tempo reale.
	
\end{itemize}

\chapter{L} \label{L}
\begin{itemize}
	\item \textbf{Leaflet} \\
	Libreria opne-source$_G$ di JavaScript che permette la realizzazione di mappe interattive che funzionano efficientemente sia su deskopt che su mobile. 
\end{itemize}
\chapter{M} \label{M}
\begin{itemize}
	\item \textbf{Machine Learning} \\
	Metodo di analisi dati che automatizza la costruzione di modelli analitici. È una branca dell'Intelligenza Artificiale e si basa sull'idea che i sistemi possono imparare dai dati, identificare modelli autonomamente e prendere decisioni con un intervento umano ridotto al minimo.

	\item \textbf{Maven} \\
	(Apache) Maven è uno strumento di gestione progetti software basati su Java e Build Automation.
	Supporta anche altri linguaggi quali C\#, Ruby e Scala.
	Mediante il file POM.xml vengono descritte le dipendenze fra il progetto e le varie versioni di librerie necessarie nonché le dipendenze fra di esse.
	Maven effettua automaticamente il download delle librerie necessarie tra i vari repository$_G$ definiti scaricandoli in locale o in un repository centralizzato lato sviluppo.
	Ciò permette un maggior controllo in caso si debba andare a cercare una determinata libreria.

	\item \textbf{Meeting}\\
	Tradotto dall’inglese: riunione.
\end{itemize}

\chapter{N} \label{N}
	\begin{itemize}
		\item \textbf{Norma}\\
		Regola di condotta, stabilità da un singolo soggetto o da un collettivo, al fine di regolare un'attività pratica o di indicare il procedimento da seguire in determinate circostanze.
	\end{itemize}
\chapter{O} \label{O}
\begin{itemize}
	\item \textbf{Open-Source} \\
	Un software open-source è reso tale per mezzo di una licenza attraverso cui i detentori dei diritti favoriscono la modifica, lo studio, l'utilizzo e la redistribuzione del codice sorgente.
	
	\item \textbf{Outlier}\\
	Termine utilizzato in statistica per definire, in un insieme di osservazioni, un valore anomalo e aberrante. Un valore quindi chiaramente distante dalle altre osservazioni disponibili.
\end{itemize}
\chapter{P} \label{P}
	\begin{itemize}
		\item \textbf{Pandas}\\
		Strumento per la modellazione di dati per il linguaggio Python.
		\item \textbf{Prettier} \\
		È un formattatore di codice, supporta per linguaggi come: Javascript$_G$, Vue.js$_G$, CSS$_G$, HTML$_G$, JSON$_G$ etc.

		\item \textbf{Product Baseline}\\
		Tradotto: linea di base del prodotto.
		Contiene tutti i contenuti rilasciabili del progetto.

		\item \textbf{Proof of Concept}\\
		Prototipo di software che ha lo scopo di dimostrare che il progetto può essere sviluppato in modo conforme alle richieste.

		\item \textbf{Procedura}\\
		Nell'informatica s'intende una sequenza ordinata di operazioni da eseguire al fine di raggiungere un determinato scopo.

		\item \textbf{Proponente}\\
		Soggetto che propone il capitolato$_G$ d'appalto per un progetto.

		\item \textbf{Pylint} \\
		Strumento di verifica del codice sorgente per controllarne la qualità e i bug presenti all'interno del codice Python$_G$.

		\item \textbf{Python}\\
		Python è un linguaggio di programmazione ad alto livello, rilasciato pubblicamente per la prima volta nel 1991 dal suo creatore Guido van Rossum, supporta diversi paradigmi di programmazione, come quello orientato agli oggetti (con supporto all'ereditarietà multipla), quello imperativo e quello funzionale, ed offre una tipizzazione dinamica forte.
		Python è un linguaggio pseudocompilato: un interprete si occupa di analizzare il codice sorgente e, se sintatticamente corretto, di eseguirlo. Questa caratteristica rende Python un linguaggio portabile. Una volta scritto un sorgente, esso può essere interpretato ed eseguito sulla gran parte delle piattaforme attualmente utilizzate, semplicemente basta la presenza della versione corretta dell’interprete.

		\item \textbf{Pop-up}\\
		Termine inglese che indica gli elementi dell'interfaccia grafica che compaiono automaticamente durante l'uso di una applicazione.
	\end{itemize}

\chapter{Q} \label{Q} 
\begin{itemize}
	\item \textbf{Query} \\
	In informatica il termine query viene utilizzato per indicare l'interrogazione da parte di un utente di un database.
\end{itemize}

\chapter{R} \label{R}
	\begin{itemize}
		\item \textbf{Repository} \\
		Ambiente di un sistema informativo in cui vengono conservati e gestiti file, documenti e metadati relativi ad un’attività$_G$ di progetto.
		
		\item \textbf{Requisito}\\
		Una condizione, o capacità, che deve essere verificata o posseduta da un sistema o un componente di esso per soddisfare un contratto, uno standard, una specifica o qualsiasi altro documento formalmente specificato. L'insieme di tutti i requisiti formano la base del successivo sviluppo del sistema o del componente.
		Plurale: Requisiti.
		
		\item \textbf{REST}
		Representational state transfer (REST) è uno stile architetturale per i sistemi distribuiti. Il termine REST rappresenta un sistema di trasmissione di dati su HTTP senza ulteriori livelli.
	\end{itemize}
\chapter{S} \label{S}
\begin{itemize}
	\item \textbf{Slack} \\
	Lo Slack time è definito come la differenza tra la data di completamento pianificata, per un'attività$_G$, e la data richiesta dal \textit{Responsabile di Progetto}.
	
	\item \textbf{Spring}\\
	In informatica Spring è un framework$_G$ open-source$_G$ per lo sviluppo di applicazioni su piattaforma Java$_G$.
	
	\item \textbf{Streaming}\\
	Identifica un flusso di dati audio/video trasmessi da una sorgente a una o più destinazioni tramite una rete telematica. Questi dati vengono riprodotti man mano che arrivano a destinazione.
	
	\item \textbf{Spring}\\
	In informatica Spring è un framework open source per lo sviluppo di applicazioni su piattaforma Java.
	
	\item \textbf{Spring Boot}\\
	Spring Boot permette di creare applicazioni con un web server integrato.
\end{itemize}
	
\chapter{T} \label{T}
	\begin{itemize}
		\item \textbf{Trello}\\
		Sito internet che permette di organizzare i task per ciascun membro del gruppo di lavoro.
		Al suo interno è possibile creare delle schede contenenti il task da fare.
		Inoltre nelle schede è possibile inserire il nome della persona adibita a svolgere quel task, allegati, commenti ed etichette per differenziare la tipologia dell'attività inserita.
		Oltre a ciò, \textit{trello} dà la possibilità di poter spostare queste schede su varie colonne per poter facilitare ancora di più l'organizzazione a livello visivo.
	\end{itemize}
\input{capitoli/U}
\chapter{V} \label{V}
\begin{itemize}
	\item \textbf{Virtual Machine}\\
	Dall'inglese macchina virtuale, viene anche scritto con l'abbreviazione VM. Indica un software che, attraverso un processo di virtualizzazione, crea un ambiente virtuale che emula tipicamente il comportamento di una macchina fisica grazie all'assegnazione di risorse hardware ed in cui alcune applicazioni possono essere eseguite come se interagissero con tale macchina.
	
	\item \textbf{Vue.js}\\
	Framework$_{\scaleto{G}{3pt}}$ open-source$_{\scaleto{G}{3pt}}$ per lo sviluppo di applicazioni web, interfacce utente e applicazioni a singola pagina.
\end{itemize}
\chapter{W} \label{W}
	\begin{itemize}
		\item \textbf{wrapper}
	\end{itemize}
\chapter{X} \label{X}
\begin{itemize}
  \item \textbf{XML}\\
È un formato di file appertenente a script$_{\scaleto{G}{3pt}}$ scritti linguaggio col medesimo nome.
Si tratta di un linguaggio di markup ossia basato su un meccanismo sintattico che consente di definire e controllare il significato degli elementi contenuti in un documento o in un testo.
\end{itemize}

\input{capitoli/Y}
\input{capitoli/Z}


%COMANDO PER LA CREAZIONE DELL'INDICE

%PER RENDERE PIÙ CHIARA LA STESURA DEI DOCUMENTI È MEGLIO LASCIARE SEPARATI IN FILE DIVERSI OGNI CAPITOLO

% \input{esempio} -- esempio di codice per inserire un nuovo capitolo

\end{document}
