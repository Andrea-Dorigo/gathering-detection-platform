\input{packages}
\input{config}

\begin{document}

\makeatletter
\begin{titlepage}
	\begin{center}
		\vspace*{-5cm}
		\author{Jawa Druids}
		\title{Nome documento}
		\date{} %LASCIARE QUESTO CAMPO VUOTO, SE LO TOLGO STAMPA LA DATA CORRENTE
		\includegraphics[width=0.7\linewidth]{../immagini/DRUIDSLOGO.jpg}\\[4ex]
		{\huge \bfseries  \@title }\\[2ex]
		{\LARGE  \@author}\\[50ex]
		\vspace*{-9cm}
		\begin{table}[H]
			\renewcommand{\arraystretch}{1.4}
			\centering
			\begin{tabular}{r | l}
				\textbf{Versione} & x.x.x \\%RIGA PER INSERIRE LA VERSIONE ULTIMA DEL DOCUMENTO
				\textbf{Data approvazione} & xx-xx-xxxx\\
				\textbf{Responsabile} & Nome Cognome\\
				\textbf{Redattori} & Nome Cognome \\
				\textbf{Verificatori} & \makecell[tl]{Nome Cognome \\ Nome Cognome} \\
				%MAKECELL SERVE PER POI ANDARE A CAPO ALL'INTERNO DELLA CELLA
				\textbf{Stato} & Stato\\
				\textbf{Lista distribuzione} & \makecell[tl]{Nome Gruppo \\ Nome Professori \\ Nome Azienda Proponente}\\
				\textbf{Uso} & Esterno
			\end{tabular}
		\end{table}
		\vspace{0.1cm}
		\hfill \break
		\fontsize{17}{10}\textbf{Sommario} \\
		\vspace{0.1cm}
		\textit{Glossario}
	\end{center}
\end{titlepage}
\makeatother

\quad
\begin{center}
	\LARGE\textbf{Registro delle modifiche}
\end{center}

\def\tabularxcolumn#1{m{#1}}
{\rowcolors{2}{RawSienna!90!RawSienna!20}{RawSienna!70!RawSienna!40}


\begin{center}
	\renewcommand{\arraystretch}{1.4}
	\begin{longtable}[c]{|p{2cm-1\tabcolsep}|p{2cm}|p{3cm-2\tabcolsep}|p{2,5cm-2\tabcolsep}|p{4cm-2\tabcolsep}|p{2,5cm}|}
		\hline
		\rowcolor{airforceblue}
		\makecell[c]{\textbf{Versione}} & \makecell[c]{\textbf{Data}} & \makecell[c]{\textbf{Autore}} & \makecell[c]{\textbf{Ruolo}} & \makecell[c]{\textbf{Modifica}} & \makecell[c]{\textbf{Verificatore}} \\
		\hline
		\centering v1.0.0 & 2021-04-18 & Andrea Dorigo & \centering \textit{Responsabile} & \textit{Approvazione del documento} & \makecell[c]{-} \\
		\hline
		\centering v0.1.0 & 2021-04-17 &  \centering - & \centering - &  \textit{Revisione complessiva del documento} & Mattia Cocco \\
		\hline
		\centering v0.0.1 & 2021-04-15 & Andrea Cecchin & \centering \textit{Redattore} &\textit{Stesura del documento}  & Mattia Cocco \\
		\hline
	\end{longtable}
\end{center}


%COMANDO PER LA CREAZIONE DELL'INDICE

\tableofcontents{}
\chapter{A}
	\begin{itemize}
		\item \textbf{Attività}\\
		Composizione di più compiti semplici svolti da singoli, o gruppi, al fine di perseguire un obbiettivo comune.
	\end{itemize}
\chapter{B}
\chapter{C}
	\begin{itemize}
		\item \textbf{Capitolato (d'appalto)}\\
		Documento tecnico, allegato generalmente ad un contratto d'appalto, a cui si fa riferimento per la definizione delle specifiche tecniche dei lavori che verranno successivamente svolti per effetto dello stesso contratto.
		\item \textbf{Committente}\\
		Soggetto che ordina ad altri l'esecuzione di un lavoro.

		\item \textbf{Compito}\\
		Rappresenta la componente elementare di un'attività, viene svolta da un soggetto per senso del dovere.
		Viene anche utilizzato al plurale: compiti.
	\end{itemize}
\chapter{D}
	\begin{itemize}
		\item \textbf{Discord}\\
		Software gratuito utilizzato come canale di comunicazione da parte del gruppo di lavoro.
		Accessibile mediante smartphone, tablet e computer.
		Permette la comunicazione testuale, condivisione di file, chiamate vocali e video chiamate.
		Inoltre permette la creazione interna di canali personalizzati in modo da organizzare al meglio la comunicazione tra le persone.
	\end{itemize}
\chapter{E}
\chapter{F}
\chapter{G}
	\begin{itemize}
		\item \textbf{Git}\\
		Sistema di controllo gratuito a versione distribuita progettato per tenere traccia del lavoro svolto durante l'intero periodo di sviluppo del software.
		Utilizzato anche per tenere traccia di tutte le modifiche fatte nei file.
		I suoi punti di forza sono l'integrità dei dati e il supporto per flussi di lavoro distribuiti e non lineari.
		\item \textbf{Git-flow}\\
	\end{itemize}
\chapter{H}
\chapter{I}
\chapter{J}
\chapter{K}
\chapter{L}
\chapter{M}
\chapter{N}
	\begin{itemize}
		\item \textbf{Norma}\\
		Regola di condotta, stabilità da un singolo soggetto o da un collettivo, al fine di regolare un'attività pratica o di indicare il procedimento da seguire in determinate circostanze.
	\end{itemize}
\chapter{O}
\chapter{P}
	\begin{itemize}
		\item \textbf{Procedura}\\
		Nell'informatica s'intende una sequenza ordinata di operazioni da eseguire al fine di raggiungere un determinato scopo.

		\item \textbf{Proponente}\\
		Soggetto che propone il \textit{capitolato d'appalto} per un progetto.
	\end{itemize}
\chapter{Q}
\chapter{R}
	\begin{itemize}
		\item \textbf{Repository}
	\end{itemize}
\chapter{S}
\chapter{T}
	\begin{itemize}
		\item \textbf{Trello}\\
		Sito internet che permette di organizzare i task per ciascun membro del gruppo di lavoro.
		Al suo interno è possibile creare delle schede contenenti il task da fare.
		Inoltre nelle schede è possibile inserire il nome della persona adibita a svolgere quel task, allegati, commenti ed etichette per differenziare la tipologia dell'attività inserita.
		Oltre a ciò, \textit{trello} dà la possibilità di poter spostare queste schede su varie colonne per poter facilitare ancora di più l'organizzazione a livello visivo.
	\end{itemize}
\chapter{U}
\chapter{V}
\chapter{W}
	\begin{itemize}
		\item \textbf{wrapper}
	\end{itemize}
\chapter{X}
\chapter{Y}
\chapter{Z}

%PER RENDERE PIÙ CHIARA LA STESURA DEI DOCUMENTI È MEGLIO LASCIARE SEPARATI IN FILE DIVERSI OGNI CAPITOLO

% \input{esempio} -- esempio di codice per inserire un nuovo capitolo

\end{document}
