\chapter{P} \label{P}
	\begin{itemize}
		\item \textbf{Pandas}\\
		Strumento per la modellazione di dati per il linguaggio Python.
		\item \textbf{Prettier} \\
		È un formattatore di codice, supporta per linguaggi come: Javascript$_G$, Vue.js$_G$, CSS$_G$, HTML$_G$, JSON$_G$ etc.

		\item \textbf{Product Baseline}\\
		Tradotto: linea di base del prodotto.
		Contiene tutti i contenuti rilasciabili del progetto.

		\item \textbf{Proof of Concept}\\
		Prototipo di software che ha lo scopo di dimostrare che il progetto può essere sviluppato in modo conforme alle richieste.

		\item \textbf{Procedura}\\
		Nell'informatica s'intende una sequenza ordinata di operazioni da eseguire al fine di raggiungere un determinato scopo.

		\item \textbf{Proponente}\\
		Soggetto che propone il capitolato$_G$ d'appalto per un progetto.

		\item \textbf{Pylint} \\
		Strumento di verifica del codice sorgente per controllarne la qualità e i bug presenti all'interno del codice Python$_G$.

		\item \textbf{Python}\\
		Python è un linguaggio di programmazione ad alto livello, rilasciato pubblicamente per la prima volta nel 1991 dal suo creatore Guido van Rossum, supporta diversi paradigmi di programmazione, come quello orientato agli oggetti (con supporto all'ereditarietà multipla), quello imperativo e quello funzionale, ed offre una tipizzazione dinamica forte.
		Python è un linguaggio pseudocompilato: un interprete si occupa di analizzare il codice sorgente e, se sintatticamente corretto, di eseguirlo. Questa caratteristica rende Python un linguaggio portabile. Una volta scritto un sorgente, esso può essere interpretato ed eseguito sulla gran parte delle piattaforme attualmente utilizzate, semplicemente basta la presenza della versione corretta dell’interprete.

		\item \textbf{Pop-up}\\
		Termine inglese che indica gli elementi dell'interfaccia grafica che compaiono automaticamente durante l'uso di una applicazione.
	\end{itemize}
