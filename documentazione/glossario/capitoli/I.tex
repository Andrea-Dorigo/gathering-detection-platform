\chapter{I} \label{I}
\begin{itemize}
	\item \textbf{Indice di Gulpease}\\
	Indice di leggibilità di un testo tarato sulla lingua italiana.
	Rispetto ad altri, ha il vantaggio di utilizzare la lunghezza delle parole in lettere anziché in sillabe, semplificandone il calcolo automatico.
	
	\item \textbf{Inspection}\\
	Tecnica di verifica dove il	verificatore$_G$ eseguirà una lettura e un controllo del documento più mirato nei punti dove si sa già essere fonte di errori.
		
	\item \textbf{ISO/IEC 9126}\\
	Insieme di normative e linee guida, sviluppate dall’Organizzazione internazionale per la
	normazione (ISO) in collaborazione con la Commissione Elettrotecnica Internazionale (IEC),
	preposte a descrivere un modello di qualità del software.
	
	\item \textbf{ISO/IEC 15504}\\
	Insieme di documenti tecnici che permettono di valutare oggettivamente la qualità di un processo software a fini migliorativi. 
	Fornisce delle valutazioni sui processi in maniera ripetibile, oggettiva e comparabile. È uno degli standard internazionali dell’Organizzazione in-	ternazionale per la normalizzazione (ISO) e della Commissione elettrotecnica internazionale
	(IEC), sviluppato dal sottocomitato congiunto ISO e IEC.

	
	
	
\end{itemize}
