\chapter{J} \label{J}
\begin{itemize}
	\item \textbf{Java} \\
	 Linguaggio di programmazione ad alto livello, orientato agli oggetti e a tipizzazione statica, che si appoggia sull'omonima piattaforma software di esecuzione, specificamente progettato per essere il più possibile indipendente dalla piattaforma hardware di esecuzione. Le principali caratteristiche di Java sono la portabilità, cioè il codice sorgente e' compilato in bytecode e può essere eseguito su ogni PC che ha JVM (Java Virtual Machine), e la robustezza.

	 \item \textbf{Javascript} \\
	 Linguaggio di programmazione orientato ad oggetti ed eventi.
	 Utilizzato specificatamente per la programmazione Web lato client.
	 Aggiunge al sito effetti dinamici tramite funzioni invocate da un'azione eseguita sulla pagine Web(es. click del mouse, movimento del mouse, caricamento pagina).

	 \item \textbf{JSON} \\
	 Acronimo di "JavaScript Object Notation".
	 Si tratta di un formato file adattato all'interscambio di dati fra applicazioni client-server.
	 Basato sul linguaggio Javascript$_G$.

	 \item \textbf{JUnit} \\
	 Framework di unit testing per il linguaggio di programmazione Java$_G$.
	 Necessita di una versione Java 8 o maggiore.
\end{itemize}
