\chapter{G} \label{G}
\begin{itemize}
	\item \textbf{Git}\\
	Sistema di controllo gratuito a versione distribuita progettato per tenere traccia del lavoro svolto durante l'intero periodo di sviluppo del software.
	Utilizzato anche per tenere traccia di tutte le modifiche fatte nei file.
	I suoi punti di forza sono l'integrità dei dati e il supporto per flussi di lavoro distribuiti e non lineari.

	\item \textbf{Git-Flow}\\
	Si tratta di un ramo di Git$_G$, molto adatto alla collaborazione e scalarità del team di sviluppo.
	
	\item \textbf{Git-Hub}\\
	GitHub è un servizio di hosting per progetti software. Il nome deriva dal fatto che esso è una implementazione dello strumento di controllo versione distribuito Git$_G$.
	
	\item \textbf{Git-Kraken}\\
	Piattaforma grafica per Git$_G$.
	Utilizzata per tenere traccia graficamente della repository$_G$, vedere i branch$_G$, potendo anche scaricare il lavoro presente in locale, crearne di nuovi.
	Inoltre si può vedere lo storico di tutte le modifiche e la possibilità di fare nuovi commit.
	
	\item \textbf{GET}\\
	Il GET è una tipologia di richiesta al server che permette di ottenere informazioni riguardanti una risorsa.
	\end{itemize}