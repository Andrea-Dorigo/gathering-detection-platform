\chapter{B} \label{B}
\begin{itemize}
	\item \textbf{Back-End} \\
	Interfaccia con la quale il gestore di un sito web dinamico ne gestisce i contenuti e le funzionalità. A differenza del frontend$_G$, l'accesso al backend è riservato agli amministratori del sito che possono accedere dopo essersi autenticati.
	%%OPPURE Back End è un termine largamente utilizzato per caratterizzare le interfacce che hanno come destinatario un programma. Una applicazione back end è un programma con il quale l'utente interagisce indirettamente, in generale attraverso l'utilizzo di una applicazione front-end. In una struttura client/server il back-end è il server. Viene spesso utilizzato in contrapposizione con Front End.
	
	\item \textbf{Brainstorming} \\
	Metodo decisionale in cui, attraverso l’espressione libera delle proprie idee, mediante riunioni e confronti si ricerca una soluzione ad un dato problema.
	
	\item \textbf{Branch} \\
	Nel contesto informatico e di Git$_G$ si tratta di un ramo di lavoro.
\end{itemize}