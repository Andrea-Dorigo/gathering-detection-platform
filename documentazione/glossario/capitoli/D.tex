\chapter{D} \label{D}
	\begin{itemize}		
		\item \textbf{Design Pattern}\\
		Soluzione progettuale generale ad un problema ricorrente. Si tratta di una descrizione, o modello logico, da applicare per la risoluzione di un problema che può presentarsi in diverse situazioni durante le fasi di progettazione e sviluppo del prodotto software.
		
		\item \textbf{Diagrammi di Gantt}\\
		Strumento di supporto alla gestione dei progetti, è costruito partendo da un asse orizzontale, a rappresentazione dell'arco temporale totale del progetto, suddiviso in fasi incrementali  e da un asse verticale, a rappresentazione delle mansioni o attività che costituiscono il progetto. Permette dunque la rappresentazione grafica di un calendario di attività$_G$, utile al fine di pianificare, coordinare e tracciare specifiche attività$_{\scaleto{G}{3pt}}$ in un progetto dando una chiara illustrazione dello stato d'avanzamento del progetto rappresentato.  Ad ogni attività$_{\scaleto{G}{3pt}}$ possono essere in generale associati una serie di attributi come la durata, le risorse ed il costo.
		
		\item \textbf{Discord}\\
		Software gratuito utilizzato come canale di comunicazione da parte del gruppo di lavoro.
		Accessibile mediante smartphone, tablet e computer.
		Permette la comunicazione testuale, condivisione di file, chiamate vocali e video chiamate.
		Inoltre permette la creazione interna di canali personalizzati in modo da organizzare al meglio la comunicazione tra le persone.
		
		\item \textbf{Docker}\\
		Docker è una piattaforma software che permette di creare, testare e distribuire applicazioni con la massima rapidità. Docker raccoglie il software in unità standardizzate chiamate container che offrono tutto il necessario per la loro corretta esecuzione, incluse librerie, strumenti di sistema, codice e runtime. Con Docker, è possibile distribuire e ricalibrare le risorse per un'applicazione in qualsiasi ambiente, tenendo sempre sotto controllo il codice eseguito.
		
		\item \textbf{Drag}\\
		Parola inglese col significato di trascinare, nel contesto utilizzato indicato lo spostamento della heat map$_{\scaleto{G}{3pt}}$ negli assi orrizzontali e verticali.
	\end{itemize}