\chapter{C} \label{C}
\begin{itemize}
	\item \textbf{Capitolato (d'appalto)}\\
	Documento tecnico, allegato generalmente ad un contratto d'appalto, a cui si fa riferimento per la definizione delle specifiche tecniche dei lavori che verranno successivamente svolti per effetto dello stesso contratto.

	\item \textbf{Checkstyle} \\
	SI tratta di uno strumento per l'analisi del codice statico.
	Viene utilizzato nello sviluppo software per verificare se il codice sorgente Java$_G$ è conforme alle regole di codifica specificate.

	\item \textbf{Committente}\\
	Soggetto che ordina ad altri l'esecuzione di un lavoro.

	\item \textbf{Compito}\\
	Rappresenta la componente elementare di un'attività, viene svolta da un soggetto per senso del dovere.
	Viene anche utilizzato al plurale: compiti.

	\item \textbf{Caso d'uso}\\
	Un caso d’uso è un insieme di scenari (sequenze di azioni) che hanno in comune uno scopo finale (obiettivo) per un utente (attore).
	Viene anche utilizzato al plurale: casi d'uso.

	\item \textbf{CSS} \\
	Linguaggio utilizzato per la formattazione di documenti HTML.
	Serve per separare i contenuti delle pagine HTML dalla loro formattazione o layout, inoltre permette una programmazione più chiara e una più facile manutenibilità del codice stesso.
\end{itemize}
