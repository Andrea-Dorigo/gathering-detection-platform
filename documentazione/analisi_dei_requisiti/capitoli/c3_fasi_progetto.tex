\chapter{Fasi del progetto}
In questo capitolo verranno illustrate le fasi del progetto identificate dal capitolato d'appalto \textit{GDP-Gathering Detection Platform}. Il capitolo viene diviso nelle tre fasi generali del progetto: acquisizione, elaborazione e visualizzazione dei dati. Secondo lo standard 830-1998 verranno spiegati tutti i punti da sviluppare e nel capitolo successivo i requisiti obbligatori da implementare per la creazione del prodotto richiesto da \textit{Sync Lab}. La descrizione delle fasi è stata inserita in quanto ritenuta necessaria per il chiarimento della necessità dei requisiti individuati. %inserire descrizione requisiti da ISOIEE

\section{FC1: Acquisizione dati}%FC fase capitolato
In questa sezione vengono descritte le fasi di acquisizione dei dati.

\subsubsection{FC1.1: Acquisizione con java}

\begin{itemize}
	\item \textbf{Descrizione}: attraverso il linguaggio Java si creerà un programma che preleva informazioni da sorgenti esterne e le invia al server.
	\item \textbf{Linguaggio di programmazione}: Java.
	\item \textbf{Input}: i dati forniti saranno prelevati da siti con live-feed di webcam di Roma e simulatori di spostamenti di persone.
	\item \textbf{Output}: i dati resteranno immutati.
	\item \textbf{Risposta ad errori}: nel caso di mancanza di risposta dai siti con live-feed il programma si bloccherà ed invierà un segnale di errore al server.
\end{itemize}

%sono dubbioso su 1.2 e 1.3 dovrò modificare qualcosa

\subsubsection{FC1.2: Database}

\begin{itemize}
	\item \textbf{Descrizione}: creazione del database e archiviazione dei dati in esso;
	\item \textbf{Linguaggio}: mySQL.
\end{itemize}

\subsubsection{FC1.3: Apache Kafka$_G$}

\begin{itemize}
	\item \textbf{Descrizione}: impostazione di una piattaforma di data streaming$_G$ che consente di gestire e trasferire grandi volumi di dati in tempo reale, abbassando notevolmente i tempi di latenza;
	\item \textbf{Input}: flussi di dati dall'acquisizione con Java;
		\item \textbf{Output}: il flusso di dati rimane immutato.
\end{itemize}

%??? I dati verranno inseriti all'interno di un database, questo sarà sviluppato usando Apache Kafka un sistema distribuito che consiste di server e client i quali comunicano tra loro attraverso un protocollo di rete performante di tipo TCP. ???(non so dove inserire)

\section{FC2: Elaborazione Dati}
Ottenuti i dati, essi verranno elaborati attraverso librerie di sci-kit e tensorflow con il linguaggio di programmazione python.
Di seguito vengono individuate le fasi da seguire per l'elaborazione dei dati.

\subsubsection{FC2.1: Esplorazione Dati}

\begin{itemize}
	\item \textbf{Descrizione}: si discriminano elementi all'interno del dataset che portano a predizioni errate del modello.
	\item \textbf{Input}: i dati vengono prelevati dal database.
	\item \textbf{Output}: i dati controllati vengono aggiunti in appositi spazi per individuare la loro correttezza.
	\item \textbf{Processo}: si controlla se c'è presenza di valori mancanti, dataset non bilanciati, outliers, livello di rumore dei dati e correlazione dei dati. %???(significato: Outlier è un termine utilizzato in statistica per definire, in un insieme di osservazioni, un valore anomalo e aberrante. Un valore quindi chiaramente distante dalle altre osservazioni disponibili.) ???%
\end{itemize}

\subsubsection{FC2.2: Preprocessing}

\begin{itemize}
	\item \textbf{Descrizione}: preparazione dei dati grezzi e renderli adatti ad un modello di machine learning. 
	\item \textbf{Input}: i dati controllati.
	\item \textbf{Output}: dati pronti per l'elaborazione nel modello machine learning.
	\item \textbf{Processo}: \begin{enumerate}[leftmargin = 2cm]
		\item Cleaning: eliminazione o correzione di dati con valori invalidi o corrotti.
		\item Trasformazione dei dati: i dati vengono normalizzati, discretizzati, aggregati, si calcolano nuove variabili etc.
		\item Feature extraction: si ricavano attraverso i dati trasformati valori derivati, i quali sono più informativi e non ridondanti, facilitano le fasi successive di apprendimento e generalizzazione.
		\item Filtraggio dei dati: attraverso appositi filtri eliminare i dati ridondanti e irrilevanti al training del modello.
		\item Train / Test set splitting: si dividono i dati in due gruppi uno per il training e uno per il testing.
	\end{enumerate}

\end{itemize}

\subsubsection{FC2.3: Caso predizione}

\begin{itemize}
	\item \textbf{Descrizione}: in questa fase si effettua una scelta sull'algoritmo più adeguato da utilizzare per il training di dati.
	\item \textbf{Input}: dati controllati nella fase di preprocessing per il training.
	\item \textbf{Output}: modello di Machine Learning$_G$ allenato sui dati di input.
	\item \textbf{Tipi di algoritmi}: si dividono per classificazione e regressione.%???non so se va bene???	
\end{itemize}

\subsubsection{FC2.4: Valutazioni e validazione}

\begin{itemize}
	\item \textbf{Descrizione}: attraverso varie metriche si valuta quanto valido è il modello nella predizione dei casi.
	\item \textbf{Input}: risposta del modello Machine Learning$_{\scaleto{G}{3pt}}$ dai dati di test, dati ricavati dalle sorgenti esterne effettivi.
	\item \textbf{Output}: dati che superano la validazione.
\end{itemize}

\section{FC3: Visualizzazione dati}
In questa sezione verranno illustrate le fasi di sviluppo della parte visiva della web-app.

\subsubsection{FC3.1: Front-end}

\begin{itemize}
	\item \textbf{Descrizione}: sviluppo di una pagina web semplice ed intuitiva.
	\item \textbf{Strumenti}: si utilizzerà Angular e Spring, due librerie per framework di javascript.
	\item \textbf{Vincolo}: la web app dovrà essere costruita sia desktop che mobile friendly. 
	\item \textbf{Struttura}: la pagina sarà principalmente rivolta alla visione della mappa per la visualizzazione di aree a rischio assembramenti.
\end{itemize}

\subsubsection{FC3.2: Back-end}

\begin{itemize}
	\item \textbf{Descrizione}: sviluppo della parte di comunicazione di informazioni tra server/database e front-end.
	\item \textbf{Strumenti}: si utilizzerà Java.
\end{itemize}


%scriverla più generale è un compito non so come uscirà la pagina
%aggiungere parte caratteristiche date dal capitolato c3 sotto ogni parte ad esempio parte acquisizione dati posso scrivere come requisito il software contapersone etc
