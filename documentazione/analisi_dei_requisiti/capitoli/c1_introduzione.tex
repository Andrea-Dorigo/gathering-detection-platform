\chapter{Introduzione}\label{Introduzione}

\section{Scopo del documento}\label{IntroduzioneScopoDelDocumento}
Lo scopo del documento è quello di formalizzare i contenuti e le qualità che il prodotto sviluppato dovrà raggiungere.
I requisiti sono stati individuati attraverso lo studio del capitolato$_G$ e dagli incontri con l'azienda proponente$_G$ \textit{Sync Lab}.
Il documento inoltre è necessario a:
\begin{itemize}
	\item descrivere accuratamente tutti i requisiti proposti dal proponente;
	\item comprendere da parte del committente quali sono le richieste del cliente;
	\item definire il formato e contenuto di ogni requisito$_G$ specifico del software.
\end{itemize}
\section{Scopo del prodotto}\label{IntroduzioneScopodelProdotto}
In seguito alla pandemia del virus COVID-19 è nata l'esigenza di limitare il più possibile i
contatti fra le persone, specialmente evitando la formazione di assembramenti.
Il progetto \textit{GDP: Gathering Detection Platform} di \textit{Sync Lab} ha pertanto l'obiettivo di \textbf{creare una piattaforma in grado di rappresentare graficamente le zone potenzialmente a rischio di assembramento, al fine di prevenirlo.} Il prodotto finale è rivolto specificatamente agli
organi amministrativi delle singole città, cosicché possano gestire al meglio i punti sensibili di affollamento, come piazze o siti turistici. Lo scopo che il software intende raggiungere non è
solo quello della rappresentazione grafica real-time ma anche di poter riuscire a prevedere
assembramenti in intervalli futuri di tempo.
\\
A tal fine il gruppo \textit{Jawa Druids} si prefigge di sviluppare un prototipo software in grado di acquisire, monitorare ed analizzare i molteplici dati provenienti dai diversi sistemi e dispositivi, a scopo di identificare i possibili eventi che concorrono all'insorgere di variazioni di flussi di utenti. Il gruppo prevede inoltre lo sviluppo di un'applicazione web da interporre fra i dati elaborati e l'utente, per favorirne la consultazione.
\section{Glossario}\label{IntroduzioneGlossario}
All'interno della documentazione viene fornito un \textit{Glossario}, con l'obiettivo di assistere il lettore specificando il significato e contesto d'utilizzo di alcuni termini strettamente tecnici o ambigui, segnalati con una \textit{G} a pedice.

\section{Riferimenti}\label{IntroduzioneRiferimenti}
\subsection{Riferimenti normativi}\label{IntroduzioneRiferimentiRiferimentiNormativi}
\begin{itemize}
	\item \textit{Norme di Progetto v1.0.0;}
		\item \textit{Verbale Esterno 17-12-2020;}
		\item \textit{Capitolato d'appalto C3:} \\ \url{https://www.math.unipd.it/~tullio/IS-1/2020/Progetto/C3.pdf}
\end{itemize}
\subsection{Riferimenti informativi}\label{IntroduzioneRiferimentiRiferimentiInformativi}
\begin{itemize}
	\item \textit{Presentazione del capitolato:} \\ \url{https://www.math.unipd.it/~tullio/IS-1/2020/Progetto/C3.pdf}
		\item \textit{Materiale didattico relativo all'Analisi dei Requisiti del corso di Ingegneria del Software:}\\ \url{https://www.math.unipd.it/~tullio/IS-1/2020/Dispense/L07.pdf}
	\item \textit{IEEE Recommended Practice for Software Requirements Specifications:}\\
		\url{https://ieeexplore.ieee.org/document/720574}
	\item \textit{Seminario per approfondimenti tecnici del capitolato C3:}\\
		\url{https://www.math.unipd.it/~tullio/IS-1/2020/Progetto/ST1.pdf}
	\item \textit{Dispensa diagrammi Casi d'uso:}\\
		\url{https://www.math.unipd.it/~rcardin/swea/2021/Diagrammi\%20Use\%20Case_4x4.pdf}
\end{itemize}
