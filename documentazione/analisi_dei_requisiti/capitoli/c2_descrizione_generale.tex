\chapter{Descrizione generale}\label{descrizioneGenerale}
\section{Caratteristiche del prodotto}\label{descrizioneGeneraleCaratteristicheProdotto}
L'idea del capitolato$_G$ \textit{GDP - Gathering Detection Platform} è di creare una piattaforma che riesca a rappresentare mediante visualizzazione grafica zone potenzialmente a rischio di assembramento con l'intento di prevenirle.
La piattaforma utilizzerà dati prelevati da sensori (come telecamere, dispositivi contapersone, etc.) o sorgenti dati (come flussi di prenotazioni Uber, le tabelle degli orari di autobus/metro/treno, etc.), i quali mediante la loro elaborazione verranno rappresentati tramite una \textit{heat map$_G$}.

\section{Funzionalità del prodotto}\label{descrizioneFunzionalitàDelProdotto}
\subsection{Descrizione generale delle funzionalità}\label{}
In questa sezione verranno illustrate le funzionalità del progetto identificate nel capitolato$_{\scaleto{G}{3pt}}$ d'appalto \textit{GDP-Gathering Detection Platform}.  Secondo lo \textbf{IEEE Standard 830-1998} in questo capitolo sono descritti tutti i punti da sviluppare. La descrizione delle fasi del prodotto è stata inserita in quanto ritenuta necessaria per esplicitare le funzionalità e la necessità dei requisiti individuati. %inserire descrizione requisiti da ISOIEE
Le funzionalità sono organizzate seguendo le tre fasi generali individuate del progetto:
\begin{itemize}
	\item \textbf{Acquisizione di dati:} l'acquisizione avverrà attraverso sistemi di monitoraggio e motori software "contapersone" applicati ad immagini/stream delle videocamere o ad altre sorgenti;
	\item \textbf{Elaborazione di dati:} i dati verranno elaborati per generare valore aggiunto agli stessi e confrontare i differenti flussi di informazioni;
	\item \textbf{Rappresentazione di dati:} attraverso un sito web i dati elaborati verranno visualizzati a video mediante una \textit{heat map$_{\scaleto{G}{3pt}}$}.
\end{itemize}
Sono descritte in seguito le funzionalità di ciascuna fase.

\subsection{Funzionalità dell'acquisizione di dati}\label{}
\subsubsection{Descrizione Funzionalità dell'acquisizione di dati}\label{}
L'acquisizione dei dati avviene mediante la raccolta dati tramite sistemi di monitoraggio e motori software "contapersone" applicati ad immagini, e stream, delle videocamere o ad altre sorgenti.
\begin{itemize}
\item \textbf{Linguaggio di programmazione}: Python$_{\scaleto{G}{3pt}}$/C.
\item \textbf{Input}: i dati forniti sono prelevati da siti con live-feed$_G$ di webcam pubbliche e simulatori di spostamenti di persone.
\item \textbf{Output}: i dati resteranno immutati.
\item \textbf{Risposta ad errori}: nel caso di mancanza di risposta dai siti con live-feed il programma si bloccherà ed invierà un segnale di errore al server, con conseguente messaggio di errore visualizzabile dall'utente.
\end{itemize}

% \begin{itemize}
% 	\item \textbf{Descrizione}: attraverso il linguaggio Java$_G$ si creerà un programma che preleva informazioni da sorgenti esterne e le invia al server.
% 	\item \textbf{Linguaggio di programmazione}: Java$_{\scaleto{G}{3pt}}$.
% 	\item \textbf{Input}: i dati forniti saranno prelevati da siti con live-feed$_G$ di webcam di varie città e simulatori di spostamenti di persone.
% 	\item \textbf{Output}: i dati resteranno immutati.
% 	\item \textbf{Risposta ad errori}: nel caso di mancanza di risposta dai siti con live-feed il programma si bloccherà ed invierà un segnale di errore al server.
% \end{itemize}
%
% %sono dubbioso su 1.2 e 1.3 dovrò modificare qualcosa
%
% \subsection{FC1.2: Database}\label{fasiProgettoAquisizioneDatiDatabase}
%
% \begin{itemize}
% 	\item \textbf{Descrizione}: creazione del database e archiviazione dei dati in esso per visualizzazione future e mantenimento dei dati;
% 	\item \textbf{Linguaggio}: NoSQL.
% \end{itemize}
%
% \subsection{FC1.3: Apache Kafka$_G$}\label{fasiProgettoAquisizioneDatiApacheKafka}
%
% \begin{itemize}
% 	\item \textbf{Descrizione}: impostazione di una piattaforma di data streaming$_G$ che consente di gestire e trasferire grandi volumi di dati in tempo reale, abbassando notevolmente i tempi di latenza;
% 	\item \textbf{Input}: flussi di dati dall'acquisizione con Java$_{\scaleto{G}{3pt}}$;
% 		\item \textbf{Output}: il flusso di dati rimane immutato.
% \end{itemize}
%
% %??? I dati verranno inseriti all'interno di un database, questo sarà sviluppato usando Apache Kafka un sistema distribuito che consiste di server e client i quali comunicano tra loro attraverso un protocollo di rete performante di tipo TCP. ???(non so dove inserire)


\section{Caratteristiche utente}\label{descrizioneGeneraleCaratteristicheUtente}
Il progetto è rivolto principalmente ad utenti di tipo amministrativo, cioè i quali devono visualizzare l'intera mappa di una regione per motivi lavorativi. \\
Le conoscenze dell'utente per l'utilizzo del software sono:
\begin{itemize}
	\item Conoscenza base nell'utilizzo del motore di ricerca;
	\item Padronanza nella lettura della \textit{heat map$_{\scaleto{G}{3pt}}$}.
\end{itemize}

%specifica delle conoscenze necessarie per usare l'heat-map/ applicazione
