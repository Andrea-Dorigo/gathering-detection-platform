\chapter{Descrizione generale}\label{DescrizioneGenerale}
\section{Caratteristiche del prodotto}\label{DescrizioneGeneraleCaratteristicheProdotto}
L'idea del capitolato$_G$ \textit{GDP - Gathering Detection Platform} è di creare una piattaforma che riesca a rappresentare mediante visualizzazione grafica zone potenzialmente a rischio di assembramento con l'intento di prevenirle.
La piattaforma utilizzerà dati prelevati da sensori (come telecamere, dispositivi contapersone, etc.) o sorgenti dati (come flussi di prenotazioni Uber, le tabelle degli orari di autobus/metro/treno, etc.), i quali mediante la loro elaborazione verranno rappresentati tramite una \textit{heat map$_G$}.

\section{Funzionalità del prodotto}\label{DescrizioneGeneraleFunzionalitàDelProdotto}
La funzionalità principale identificata nel capitolato$_{\scaleto{G}{3pt}}$ d'appalto \textit{GDP-Gathering Detection Platform} è la \textbf{rappresentazione via \textit{heat map$_G$} dei dati ottenuti dalle sorgenti e della loro elaborazione}, affinché l'utente possa consultarle.

Questa funzionalità è il frutto di una serie di funzioni sottostanti, identificate e suddivise per meglio descrivere le operazioni effettuate dal back-end.
Le illustriamo nella sezione seguente.

\subsection{Sotto-funzioni della rappresentazione della heat map}\label{DescrizioneGeneraleFunzionalitàDelProdottoSottoFunzioniDellaRappresentazioneDellaHeatmap}

La descrizione delle sotto-funzioni della rappresentazione della \textit{heat map$_G$} è stata inserita in quanto ritenuta necessaria per fornire un ulteriore approfondimento riguardo tale macro-funzionalità.
Queste funzioni sono raggruppate seguendo tre funzioni generali individuate:
\begin{itemize}
	\item \textbf{Acquisizione di dati:} l'acquisizione avverrà attraverso sistemi di monitoraggio e motori software "contapersone" applicati ad immagini/stream delle videocamere o ad altre sorgenti; i dati ottenuti verranno quindi trattati con Apache Kafka$_G$ e inseriti nel db;
	\item \textbf{Elaborazione di dati:} i dati verranno elaborati per generare valore aggiunto agli stessi e confrontare i differenti flussi di informazioni;
	\item \textbf{Rappresentazione di dati:} attraverso un sito web i dati elaborati verranno visualizzati a video mediante una \textit{heat map$_{\scaleto{G}{3pt}}$}.
\end{itemize}

\subsection{Funzione di acquisizione di dati}\label{DescrizioneGeneraleFunzionalitàDelProdottoFunzioneDiAcquisizioneDiDati}
L'acquisizione dei dati avviene tramite sistemi di monitoraggio e motori software "contapersone" applicati ad immagini e/o stream, provenienti delle videocamere o ad altre sorgenti. Ne segue lo streaming di tali dati con Apache Kafka$_{\scaleto{G}{3pt}}$ e il successivo inserimento nel database.

\subsubsection{Funzione di conteggio persone}\label{DescrizioneGeneraleFunzionalitàDelProdottoFunzioneDiAcquisizioneDiDatiFunzioneDiConteggioPersone}
\begin{itemize}
\item \textbf{Linguaggio di programmazione}: Python$_{\scaleto{G}{3pt}}$/C.
\item \textbf{Input}: i dati forniti sono prelevati da siti con live-feed$_G$ di webcam pubbliche e/o simulatori di spostamenti di persone.
\item \textbf{Output}: il numero delle persone presenti in uno stream/immagine ad un preciso istante.
\item \textbf{Risposta ad errori}: nel caso di mancanza di risposta dai siti con live-feed il programma si bloccherà ed invierà un segnale di errore al server, con conseguente messaggio di errore visualizzabile dall'utente.
\end{itemize}

\subsubsection{Funzione di streaming dati con Apache Kafka}\label{DescrizioneGeneraleFunzionalitàDelProdottoFunzioneDiAcquisizioneDiDatiFunzioneDiStreamingDatiConApacheKafka}

\begin{itemize}
	\item \textbf{Descrizione}: impostazione di una piattaforma di data streaming$_G$ che consente di gestire e trasferire grandi volumi di dati in tempo reale, abbassando notevolmente i tempi di latenza;
	\item \textbf{Input}: flussi di dati dall'acquisizione con Java$_{\scaleto{G}{3pt}}$;
		\item \textbf{Output}: il flusso di dati rimane immutato.
\end{itemize}

\subsubsection{Funzione di inserimento dati nel Database}\label{DescrizioneGeneraleFunzionalitàDelProdottoFunzioneDiAcquisizioneDiDatiFunzioneDiInserimentoDatiNelDatabase}

\begin{itemize}
	\item \textbf{Descrizione}: creazione del database e archiviazione dei dati in esso per visualizzazione future e mantenimento dei dati;
	\item \textbf{Struttura}: NoSQL.
\end{itemize}


\subsection{Funzione di Elaborazione Dati}\label{DescrizioneGeneraleFunzionalitàDelProdottoFunzioneDiElaborazioneDati}
Completata la funzione precedente i dati verranno elaborati attraverso librerie di Scikit-learn e TensorFlow con il linguaggio di programmazione Python$_G$.
Di seguito vengono individuate le funzioni da seguire per l'elaborazione dei dati.

\subsubsection{Funzione di Esplorazione Dati}\label{DescrizioneGeneraleFunzionalitàDelProdottoFunzioneDiElaborazioneDatiFunzioneDiEsplorazioneDati}

\begin{itemize}
	\item \textbf{Descrizione}: si discriminano elementi all'interno del dataset che portano a predizioni errate del modello.
	\item \textbf{Input}: i dati vengono prelevati dal database.
	\item \textbf{Output}: i dati controllati vengono aggiunti in appositi spazi per individuare la loro correttezza.
	\item \textbf{Processo}: si controlla se c'è presenza di valori mancanti, dataset non bilanciati, outliers$_G$, livello di rumore dei dati e correlazione dei dati.
\end{itemize}

\subsubsection{Funzione di Preprocessing}\label{DescrizioneGeneraleFunzionalitàDelProdottoFunzioneDiElaborazioneDatiFunzioneDiPreprocessing}

\begin{itemize}
	\item \textbf{Descrizione}: preparazione dei dati grezzi per renderli adatti ad un modello di Machine Learning$_G$.
	\item \textbf{Input}: i dati controllati.
	\item \textbf{Output}: dati pronti per l'elaborazione nel modello Machine Learning$_{\scaleto{G}{3pt}}$.
	\item \textbf{Processo}: \begin{enumerate}[leftmargin = 2cm]
		\item Cleaning: eliminazione o correzione di dati con valori invalidi o corrotti.
		\item Trasformazione dei dati: i dati vengono normalizzati, discretizzati, aggregati, si calcolano nuove variabili etc.
		\item Feature extraction: si ricavano, attraverso i dati trasformati, i valori derivati, i quali sono più informativi e non ridondanti, facilitano le funzioni successive di apprendimento e generalizzazione.
		\item Filtraggio dei dati: eliminazione di dati ridondanti e irrilevanti al training del modello attraverso l'applicazione di appositi filtri.
		\item Train / Test set splitting: si dividono i dati in due gruppi uno per il training e uno per il testing.
	\end{enumerate}

\end{itemize}

\subsubsection{Funzione di predizione}\label{DescrizioneGeneraleFunzionalitàDelProdottoFunzioneDiElaborazioneDatiFunzioneDiPredizione}

\begin{itemize}
	\item \textbf{Descrizione}: in questa funzione si effettua una scelta sull'algoritmo più adeguato da utilizzare per il training di dati.
	\item \textbf{Input}: dati ottenuti dalla funzione di preprocessing per il training.
	\item \textbf{Output}: modello di Machine Learning$_{\scaleto{G}{3pt}}$ allenato sui dati di input.
	\item \textbf{Tipi di algoritmi}: si dividono per classificazione e regressione.%???non so se va bene???
\end{itemize}

\subsubsection{Funzione di Valutazione e validazione}\label{DescrizioneGeneraleFunzionalitàDelProdottoFunzioneDiElaborazioneDatiFunzioneDiValutazioneEValidazione}

\begin{itemize}
	\item \textbf{Descrizione}: attraverso varie metriche si valuta quanto valido è il modello nella predizione dei casi.
	\item \textbf{Input}: risposta del modello Machine Learning$_{\scaleto{G}{3pt}}$ dai dati di test, dati effettivi ricavati dalle sorgenti esterne.
	\item \textbf{Output}: dati che superano la validazione.
\end{itemize}

\subsection{Funzione di Visualizzazione dati}\label{DescrizioneGeneraleFunzionalitàDelProdottoFunzioneDiVisualizzazione}
In questa sezione verranno illustrate le funzioni della parte visiva della web-app.

\subsubsection{Funzione di Prelevamento dati}\label{DescrizioneGeneraleFunzionalitàDelProdottoFunzioneDiVisualizzazioneFunzioneDiPrelevamentoDati}

\begin{itemize}
	\item \textbf{Descrizione}: sviluppo della parte di comunicazione di informazioni tra server/database e front-end$_{\scaleto{G}{3pt}}$.
	\item \textbf{Strumenti}: si utilizzerà Java$_{\scaleto{G}{3pt}}$.
\end{itemize}

\subsubsection{Funzione di rappresentazione tramite web application}\label{DescrizioneGeneraleFunzionalitàDelProdottoFunzioneDiVisualizzazioneFunzioneDiRappresentazioneTramiteWebApplication}

\begin{itemize}
	\item \textbf{Descrizione}: sviluppo di una pagina web semplice ed intuitiva.
	\item \textbf{Strumenti}: si utilizzerà Vue.js e Spring$_G$, due librerie per framework$_G$ di JavaScript$_G$.
	\item \textbf{Vincolo}: la web app dovrà essere costruita sia desktop che mobile friendly.
	\item \textbf{Struttura}: la pagina sarà principalmente rivolta alla visione della mappa per la visualizzazione di aree a rischio assembramenti.
\end{itemize}


\section{Caratteristiche utente}\label{DescrizioneGeneraleCaratteristicheUtente}
Il progetto è rivolto principalmente ad utenti di tipo amministrativo, cioè i quali devono visualizzare l'intera mappa di una regione per motivi lavorativi. \\
Le conoscenze dell'utente per l'utilizzo del software sono:
\begin{itemize}
	\item Conoscenza base nell'utilizzo del motore di ricerca;
	\item Padronanza nella lettura della \textit{heat map$_{\scaleto{G}{3pt}}$}.
\end{itemize}

%specifica delle conoscenze necessarie per usare l'heat-map/ applicazione
