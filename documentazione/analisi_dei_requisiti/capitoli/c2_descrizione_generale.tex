\chapter{Descrizione generale}\label{descrizioneGenerale}
\section{Caratteristiche del prodotto}\label{descrizioneGeneraleCaratteristicheProdotto}
L'idea del capitolato$_G$ \textit{GDP - Gathering Detection Platform} è di creare una piattaforma che riesca a rappresentare mediante visualizzazione grafica zone potenzialmente a rischio di assembramento con l'intento di prevenirle.
La piattaforma utilizzerà dati prelevati da sensori (come telecamere, dispositivi contapersone, etc.) o sorgenti dati (come flussi di prenotazioni Uber, le tabelle degli orari di autobus/metro/treno, etc.), i quali mediante la loro elaborazione verranno rappresentati tramite una \textit{heat map$_G$}.

\section{Funzionalità generali}\label{descrizioneGeneraleFunzionalitàGenerali}
Il capitolato$_{\scaleto{G}{3pt}}$ \textit{GDP} individua tre principali funzionalità da sviluppare:
\begin{itemize}
	\item \textbf{Acquisizione di dati:} l'acquisizione avverrà attraverso sistemi di monitoraggio e motori software "contapersone" applicati ad immagini/stream delle videocamere;
	\item \textbf{Elaborazione di dati:} i dati verranno elaborati per generare valore aggiunto agli stessi e confrontare flussi diversi di informazioni;
	\item \textbf{Rappresentazione di dati:} attraverso un sito web i dati elaborati verranno visualizzati a video mediante una \textit{heat map$_{\scaleto{G}{3pt}}$}.
\end{itemize}

\section{Caratteristiche utente}\label{descrizioneGeneraleCaratteristicheUtente}
Il progetto è rivolto principalmente ad utenti di tipo amministrativo, cioè i quali devono visualizzare l'intera mappa di una regione per motivi lavorativi. \\
Le conoscenze dell'utente per l'utilizzo del software sono:
\begin{itemize}
	\item Conoscenza base nell'utilizzo del motore di ricerca;
	\item Padronanza nella lettura della \textit{heat map$_{\scaleto{G}{3pt}}$}.
\end{itemize}

%specifica delle conoscenze necessarie per usare l'heat-map/ applicazione
