\chapter{Requisiti}\label{Requisiti}
In questa sezione vengono illustrati attraverso una tabella tutti i requisiti$_{\scaleto{G}{3pt}}$ individuati dal proponente$_{\scaleto{G}{3pt}}$ e dal gruppo Jawa Druids. Ogni requisito viene individuato da un codice identificativo, una sua descrizione, la tipologia di requisito e codice della fase di riferimento, la spiegazione di ogni parte è descritta nel documento \textit{Norme del Progetto v1.0.0}.
\section{Requisiti funzionali}\label{RequisitiFunzionali}



\def\tabularxcolumn#1{m{#1}}
{\rowcolors{2}{Apricot!90!Bittersweet!20}{Bittersweet!70!Apricot!40}
	
	\begin{center}
		\renewcommand{\arraystretch}{1.4}
		\begin{tabularx}{\textwidth}{ |c|X|c|X| }
			\hline
			\rowcolor{Melon}
			\textbf{Codice RS} & \textbf{Descrizione} & \textbf{Tipo di requisito} & \textbf{Fonte} \\
			%fase 1
			\hline
			RSFO1 & Realizzazione di motori software ‘contapersone’  & Obbligatorio & \makecell[tcX]{Capitolato$_{\scaleto{G}{3pt}}$ \\ Verbale esterno 17-12-2020} \\
			% \shortstack{Capitolato\\Verbale esterno 17-12-2020}   \\
			\hline
			RSFF2 & Realizzazione di simulatori di altre sorgenti dati sia dei dati storici/in monitoraggio che dati previsionali & Facoltativo & Capitolato \\
			\hline
			RSFO3  & Il sistema deve visualizzare un messaggio d'errore se il flusso di dati esterno viene a mancare  & Obbligatorio & \makecell[tcX]{Interno\\FC1.1}  \\
			\hline
			RSFO4.1 & Archiviazione di tutti i dati acquisiti nel databse & Obbligatorio & Capitolato  \\
				\hline
				RSFO4.2 & Archiviazione di tutti i dati elaborati nel database & Obbligatorio & Capitolato  \\

			%fase 2
			\hline
			RSFO5 & Elaborazione in tempo reale dei dati acquisiti da flussi esterni & Obbligatorio & Capitolato  \\
			\hline
			RSFO5.1 & Identificazione di eventi che portano alla variazione del flusso di utenti & Obbligatorio & Capitolato  \\
			\hline
			RSFD6 & Previsione dell'insorgenza futura di variazioni significative di flussi di persone & Desiderabile & Capitolato  \\
			%fase 3
			\hline
			RSFO7 & Visualizzazione dei dati elaborati attraverso heat map & Obbligatorio & Capitolato  \\
		\hline
			RSFO8 & Apache Kafka$_{\scaleto{G}{3pt}}$ deve poter comunicare con il database, l'applicazione web e il modello di Machine Learning  & Obbligatorio &  Interno \\		
			\hline
	\end{tabularx}
	\end{center}

\section{Requisiti prestazionali}\label{RequisitiPrestazionali}
\begin{center}
	\renewcommand{\arraystretch}{1.4}
	\begin{tabularx}{\textwidth}{ |c|X|c|c| }
		\hline
		\rowcolor{Melon}
		\textbf{Codice RS} & \textbf{Descrizione} & \textbf{Tipo di requisito} & \textbf{Fonte} \\
		\hline
		RSPO1 & Capacità di acquisizione continuativa nel tempo dei dati da flussi esterni & Obbligatorio & Capitolato  \\
		\hline
		RSPO2 & Modalità a bassa latenza dell'aquisizione di informazioni & Obbligatorio & Capitolato \\

	\end{tabularx}
\end{center}

\section{Requisiti di qualità}\label{RequisitiQualità}
\begin{center}
	\renewcommand{\arraystretch}{1.4}
	\begin{tabularx}{\textwidth}{ |c|X|c|c| }
		\hline
		\rowcolor{Melon}
		\textbf{Codice RS} & \textbf{Descrizione} & \textbf{Tipo di requisito} & \textbf{Fonte} \\
		\hline
		RSQO1  & La progettazione e la codifica dei requisiti devono rispettare le norme e le metriche definite nel documento \textit{Norme di Progetto v1.0.0}& Obbligatorio & Interno \\
		\hline
	RSQF2  & Il codice sorgente del software deve essere disponibile in una repository$_G$ pubblica su Github$_G$  & Facoltativo & Interno \\
	\hline
	RSQF3  & Deve essere sviluppato e fornito un documento con lo schema della base di dati relazionale  & Facoltativo & Interno \\
%	\hline
%	RSQ  & Fornire una documentazione sui flussi di dati esterni.... & Facoltativo & FC3 \\
%	\hline
	\hline
	RSQF4  & Deve essere realizzato un documento contenente tutti gli errori risolti durante la realizzazione del software & Facoltativo & Interno \\
		\hline
	RSQO5  & Test che dimostrino il corretto funzionamento dei servizi e delle funzionalità previste  & Obbligatorio & Capitolato \\
	\hline
	\end{tabularx}
\end{center}


\section{Requisiti di vincolo}\label{RequisitiVincolo}

\begin{center}
	\renewcommand{\arraystretch}{1.4}
	\begin{tabularx}{\textwidth}{ |c|X|c|c| }
		\hline
		\rowcolor{Melon}
		\textbf{Codice RS} & \textbf{Descrizione} & \textbf{Tipo di requisito} & \textbf{Fonte} \\
			\hline
		RSVO1  & I dati acquisiti da telecamere in tempo reale devono avere data di riferimento associato  &Obbligatorio & Interni \\
		\hline
		RSVO1.1  & I dati acquisiti da telecamere in tempo reale devono avere un orario di riferimento associato &Obbligatorio & Interni \\
		\hline
		RSVO1.2  & I dati acquisiti da telecamere in tempo reale devono avere un luogo di riferimento associato &Obbligatorio  & Interni \\
		\hline
		RSVO2  & Il front-end$_{\scaleto{G}{3pt}}$ del prodotto viene sviluppato utilizzando tecnologie web &Obbligatorio  & \shortstack{Capitolato\\FC3.1} \\
		\hline
		RSVF2.1  & Utilizzo di leaflet.js per la creazione di heat-map & Facoltativo & \shortstack{Capitolato\\FC3.1} \\
	\hline
		RSVO2.2  & Utilizzo di angular.js per la creazione della wep-app  & Obbligatorio  & \shortstack{Capitolato\\FC3.1} \\
\hline
RSVO3  & Il sistema deve far uso dell'ecosistema Apache Kafka$_{\scaleto{G}{3pt}}$ & Obbligatorio  & \makecell[tcX]{Capitolato\\FC1.3} \\
\hline
	\end{tabularx}
\end{center}

\section{Tracciamento dei requisiti}\label{RequisitiTracciamentoDeiRequisiti}

\subsection{Requisito - fonte}\label{RequisitiTracciamentoDeiRequisitiFonte}
\begin{center}
	\renewcommand{\arraystretch}{1.4}
	\begin{tabularx}{\textwidth}{ |X|X| }
		\hline
		\rowcolor{Melon}
		\textbf{Codice RS} & \textbf{Fonte}  \\
		\hline
RSFO1 & \makecell[tcX]{Capitolato\\Verbale esterno 17-12-2020} \\
\hline
RSFF2 & Capitolato\\
\hline
RSFO3 & \makecell[tcX]{Interno\\FC1.1}\\
\hline
RSFO4.1 & Capitolato\\
\hline
RSFO4.2 & Capitolato\\
\hline
RSFO5 & Capitolato\\
\hline
RSFO5.1 & Capitolato\\
\hline
RSFD6 & Capitolato\\
\hline
RSFO7 & Capitolato\\
\hline
RSFO8 & Interno\\
\hline
RSPO1 & Capitolato\\
\hline
RSPO2 & Capitolato\\
\hline
RSQO1 & Interno\\
\hline
RSQF2 & Interno\\
\hline
RSQF3 & Interno\\
\hline
RSQF4 & Interno\\
\hline
RSQO5 & Capitolato\\
\hline
RSVO1 & Interno\\
\hline
RSVO1.1 & Interno\\
\hline
RSVO1.2 & Interno\\
\hline
RSVO2 & \makecell[tcX]{Capitolato\\FC3.1}\\
\hline
RSVF2.1 & \makecell[tcX]{Capitolato\\FC3.1}\\
\hline
	\end{tabularx}
\end{center}
\begin{center}
	\renewcommand{\arraystretch}{1.4}
	\begin{tabularx}{\textwidth}{ |X|X| }
		\hline
		\rowcolor{Melon}
		\textbf{Codice RS} & \textbf{Fonte}  \\
		\hline
RSVO2.2 & \makecell[tcX]{Capitolato\\FC3.1}\\
\hline
RSVO3 & \makecell[tcX]{Capitolato\\FC1.3}\\
\hline
	\end{tabularx}
\end{center}

\subsection{Fonte - requisito}\label{RequisitiTracciamentoDeiRequisitiFonteRequisito}
\begin{center}
	\renewcommand{\arraystretch}{1.4}
	\begin{tabularx}{\textwidth}{ |X|X| }
		\hline
		\rowcolor{Melon}
		\textbf{Fonte} & \textbf{Codice RS}  \\
		\hline
		Capitolato & \makecell[tcX]{RSFO1\\RSFF2\\RSFO4.1\\RSFO4.2\\RSFO5\\RSFO5.1\\RSFD6\\RSFO7\\RSPO1\\RSPO2\\RSQO5\\RSVO2\\RSVF2.1\\RSVO2.2\\RSVO3} \\
		\hline
		Interno &\makecell[tcX]{RSFO3\\RSFO8\\RSQO1\\RSQF2\\RSQF3\\RSQF4\\RSVO1\\RSVO1.1\\RSVO1.2} \\
		\hline
		Verbale esterno 17-12-2020 & RSFO1 \\
		\hline
		FC1.1 & RSFO3 \\
		\hline
		FC1.3 & RSVO3 \\  
		\hline
		FC3.1 & \makecell[tcX]{RSVO2\\RSVF2.1\\RSVO2.2} \\
		\hline
	\end{tabularx}
\end{center}

\section{Considerazioni}\label{requisitiConsiderazioni}
I requisiti potranno subire delle variazioni in futuro, in modo tale da apportare degli aggiornamenti alle voci presenti o delle migliorie.
Nel caso in cui le attività pianificate terminassero prima del previsto e dovessero avanzare delle ore di lavoro, potranno essere presi in carico nuovi requisiti per aggiungere del valore al prodotto. Pertanto, qualsiasi espansione è riservata solo per il futuro.