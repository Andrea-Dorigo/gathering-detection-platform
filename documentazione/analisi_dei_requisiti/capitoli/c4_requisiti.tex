\chapter{Requisiti}
In questa sezione vengono illustrati attraverso una tabella tutti i requisiti individuati dal proponente e il gruppo Jawa Druids. Ogni requisito viene individuato da un codice identificativo, una sua descrizione, la tipologia di requisito e codice della fase di riferimento, la spiegazione di ogni parte è descritta nel documento Norme del Progetto. 

\def\tabularxcolumn#1{m{#1}}
{\rowcolors{2}{Apricot!90!Bittersweet!20}{Bittersweet!70!Apricot!40}
	
	\begin{center}
		\renewcommand{\arraystretch}{1.4}
		\begin{tabularx}{\textwidth}{ |c|X|c|c| }
			\hline
			\rowcolor{Melon}
			\textbf{Codice RS} & \textbf{Descrizione} & \textbf{Tipo di requisito} & \textbf{Codice FC} \\
			%fase 1
			\hline
			RSO1 & Realizzazione di motori software ‘contapersone’  & Obbligatorio & FC1  \\
			\hline
			RSF2 & Realizzazione di simulatori di altre sorgenti dati sia dei dati storici/in monitoraggio che dati previsionali & Facoltativo & FC1 \\
			\hline
			RSO3 & Capacità di acquisizione continuativa nel tempo dei dati da flussi esterni & Obbligatorio & FC1   \\
			\hline
			RS  & I dati acquisiti da telecamere in tempo reale devono avere data di riferimento associato  &  & FC1 \\
			\hline
			RS  & I dati acquisiti da telecamere in tempo reale devono avere un orario di riferimento associato &  & FC1 \\
			\hline
			RS  & I dati acquisiti da telecamere in tempo reale devono avere un luogo di riferimento associato &  & FC1 \\
			\hline
			RS  & Il sistema deve visualizzare un messaggio d'errore se il flusso di dati esterno viene a mancare  &  & FC1 \\
			\hline
			RSO9 & Modalità a bassa latenza dell'aquisizione di informazioni & Obbligatorio & FC1 \\
			\hline
			RSO8 & Archiviazione di tutti i dati acquisiti ed elaborati & Obbligatorio & FC2  \\
			
			%fase 2
			\hline
			RSO4 & Elaborazione in tempo reale dei dati acquisiti da flussi esterni & Obbligatorio & FC2  \\
			\hline
			RSO5 & Identificazione di eventi che portano alla variazione dei dati elaborati & Obbligatorio & FC2  \\
			\hline
			RSD6 & Previsione dell'insorgenza futura di variazioni significative di flussi di persone & Desiderabile & FC2  \\
			%fase 3
			\hline
			RSO7 & Visualizzazione dei dati elaborati attraverso heat-map & Obbligatorio & FC3  \\
			\hline
			RS  & Utilizzo di leaflet.js per la creazione di heat-map &  & FC3 \\
			\hline
			RS  & Utilizzo di angular.js per la creazione della wep-app  &  & FC1 \\
			
			
			\hline
			RSO10 &  & & \\
	\end{tabularx}
	\end{center}



%\subsection{Requisiti}

%\subsubsection{RS1.1: Software Contapersone}%RS requisito specifico

%\begin{itemize}
%	\item \textbf{Descrizione}: durante l'acquisizione dei dati da sorgenti esterne relative a live webcam si svilupperà un software "contapersone" per il rilevamento del numero di persone nel frame dell'immagine catturato.
%	\item \textbf{Input}: sorgente video di una webcam.
%	\item \textbf{Output}: numero di persone rilevate per frame con orario annesso.
%	\item \textbf{Linguaggio di programmazione}: Python.
%	\item \textbf{Tipo di requisito}: Obbligatorio   
%\end{itemize}

%\subsubsection{RS1.2: Velocità di acquisizione dei dati}
%\begin{itemize}
%	\item \textbf{Descrizione}: l'acquisizione dei dati deve essere fatto in modo continuativo nel tempo e a bassa latenza.
%	\item \textbf{Tipo di requisito}: Obbligatorio.
%\end{itemize}


%\subsection{Requisiti}

%\subsubsection{RS1.3: Elaborazione in tempo reale}
%
%\begin{itemize}
%	\item \textbf{Descrizione}: sviluppato il modello dovrà essere utilizzato per l'elaborazione dei dati forniti dalle webcam in tempo reale.
%	\item \textbf{Input}: dati rilevate dalle webcam.
%	\item \textbf{Output}: informazioni sulle variazioni dei dati monitorati e correlazioni tra flussi di dati diversi.
%	\item \textbf{Tipo di requisito}: Obbligatorio.
%\end{itemize}
%
%\subsubsection{RS1.4: Archiviazione dei dati}
%\begin{itemize}
%	\item \textbf{Descrizione}: tutti i tipi di dati elaborati o grezzi dovranno essere archiviati per una possibile visione futura degli stessi.
%	\item \textbf{Strumento di archiviazione}: database e server Kafka.
%	\item \textbf{Tipo di requisito}: Obbligatorio.
%\end{itemize}
%
%\subsubsection{RS1.5: Osservazione dei dati}
%\begin{itemize}
%	\item \textbf{Descrizione}: identificazione di eventi che nel corso del tempo hanno provocato un cambiamento nel flusso di utenti in una particolare zona.
%	\item \textbf{Tipo di requisito}: Obbligatorio.
%\end{itemize}

%
%\subsection{Requisiti}
%
%\subsubsection{RS1.6: Visualizzazione mappa}
%
%\begin{itemize}
%	\item \textbf{Descrizione}: la mappa dovrà essere visualizzati in forma di heat-map. %non so come formulare bene la frase
%	\item \textbf{Strumento di sviluppo}: framework javascript Leaflet.
%	\item \textbf{Tipo di requisito}: Obbligatorio.
%\end{itemize}
