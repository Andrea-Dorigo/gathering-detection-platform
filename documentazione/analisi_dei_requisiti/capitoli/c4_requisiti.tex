\chapter{Requisiti}\label{Requisiti}
In questa sezione vengono illustrati attraverso una tabella tutti i requisiti$_{\scaleto{G}{3pt}}$ individuati dal proponente$_{\scaleto{G}{3pt}}$ e dal gruppo \textit{Jawa Druids}. Ogni requisito viene individuato da un codice identificativo, una sua descrizione, la tipologia di requisito e la fonte di riferimento, la spiegazione di ogni parte è descritta nel documento \textit{Norme del Progetto v1.0.0}. Nella sezione successiva viene illustrato attraverso una tabella il tracciamento dei requisiti alla loro fonte e viceversa.

\section{Requisiti funzionali}\label{RequisitiFunzionali}

\def\tabularxcolumn#1{m{#1}}
{\rowcolors{2}{RawSienna!90!RawSienna!20}{RawSienna!70!RawSienna!40}

	\begin{center}
		\renewcommand{\arraystretch}{1.4}
		\begin{longtable}{|p{3cm}|p{4cm}|p{4cm}|p{4cm}|}
			\hline
			\rowcolor{airforceblue}
			\makecell[c]{\textbf{Codice RS}} & \makecell[c]{\textbf{Descrizione}} & \makecell[c]{\textbf{Tipo di requisito}} & \makecell[c]{\textbf{Fonte}} \\
			%fase 1
			\hline
			\centering RSFO1 & Realizzazione di motori software ‘contapersone’  &\centering  Obbligatorio & \makecell[tc]{Capitolato$_{\scaleto{G}{3pt}}$ \\ V. esterno 17-12-2020 } \\
			% \shortstack{Capitolato\\Verbale esterno 17-12-2020}   \\
			\hline
			\centering RSFF2 & Realizzazione di simulatori di altre sorgenti dati sia dei dati storici/in monitoraggio che dati previsionali & \centering Facoltativo & \makecell[tc]{Capitolato$_{\scaleto{G}{3pt}}$ } \\
			\hline
			\centering RSFO3  & Viene visualizzato un messaggio di errore per la mancanza dati nella generazione della heat-map  &\centering  Obbligatorio & \makecell[tc]{UC2}  \\
			\hline
			\centering RSFO4 & Archiviazione di tutti i dati nel database & \centering Obbligatorio & \makecell[tc]{Capitolato$_{\scaleto{G}{3pt}}$ \\ UC5}  \\
			\hline
			\centering RSFO4.1 & Archiviazione di tutti i dati reali nel database & \centering Obbligatorio & \makecell[tc]{Capitolato$_{\scaleto{G}{3pt}}$ \\ UC5.1 \\ UC5.2}  \\
			\hline
			\centering RSFO4.2 & Archiviazione di tutti i dati elaborati dal modello ML nel database & \centering Obbligatorio & \makecell[tc]{Capitolato$_{\scaleto{G}{3pt}}$ \\ UC5.3}  \\
			%fase 2
			\hline
			\centering RSFO5 & Elaborazione in tempo reale dei dati acquisiti da flussi esterni &\centering  Obbligatorio & \makecell[tc]{Capitolato$_{\scaleto{G}{3pt}}$}  \\
			\hline
			\centering RSFD5.1 & Identificazione di eventi che portano alla variazione del flusso di utenti &\centering  Desiderabile & \makecell[tc]{Capitolato$_{\scaleto{G}{3pt}}$}  \\
			\hline
			\centering RSFD6 & Previsione dell'insorgenza futura di variazioni significative di flussi di persone & \centering Desiderabile & \makecell[tc]{Capitolato$_{\scaleto{G}{3pt}}$ }  \\
			%fase 3
			\hline
			\centering RSFO7 & Visualizzazione dei dati elaborati attraverso heat map$_{\scaleto{G}{3pt}}$ &\centering  Obbligatorio & \makecell[tc]{Capitolato$_{\scaleto{G}{3pt}}$ \\ UC1}  \\
			\hline
			\centering RSFO8 & Apache Kafka$_{\scaleto{G}{3pt}}$ deve creare una comunicazione tra il programma con il software 'contapersone' e il database  &\centering  Obbligatorio &  \makecell[tc]{Interno} 	\\
			\hline
			\centering RSFO9 & L'utente deve poter visualizzare i dati in tempo reale tramite heat map$_{\scaleto{G}{3pt}}$  &\centering  Obbligatorio &  \makecell[tc]{Interno \\ UC1} 	\\
			\hline
			\centering RSFO10 & L'utente deve poter visualizzare i dati storicizzati tramite heat map$_{\scaleto{G}{3pt}}$  &\centering  Obbligatorio &  \makecell[tc]{Interno \\ UC1} 	\\
			\hline
			\centering RSFO11 & L'utente deve poter visualizzare una previsione tramite heat map$_{\scaleto{G}{3pt}}$  &\centering  Obbligatorio &  \makecell[tc]{Interno \\ UC1} 	\\
			\hline
			\centering RSFF12 & L'utente deve poter distinguere fra i dati simulati e quelli reali  &\centering  Facoltativo &  \makecell[tc]{Interno} 	\\
			\hline
			\centering RSFD13 & L'utente deve poter visualizzare un indice di affidabilità della previsione nella mappa  &\centering  Desiderabile &  \makecell[tc]{Interno} 	\\ 
			\hline
			\centering RSFD14 & L'utente deve poter visualizzare un indice di affidabilità dei dati in tempo reale nella mappa  &\centering  Desiderabile &  \makecell[tc]{Interno} 	\\
			\hline
			\centering RSFF15 & L'utente deve poter applicare dei filtri ai dati (reali, simulati)  &\centering  Facoltativo &  \makecell[tc]{Interno } 	\\
			\hline
			\centering RSFF16 & L'utente ha la possibilità di scegliere le sorgenti dati da cui prelevare dati  &\centering  Facoltativo &  \makecell[tc]{Interno} 	\\
			\hline
			\centering RSFO17 & Il sistema deve aggiornare la mappa automaticamente ogni 10 minuti &\centering Obbligatorio & \makecell[tc]{Interno} \\
			\hline
			\centering RSFO18 & Il modello di machine learning$_{\scaleto{G}{3pt}}$ deve poter salvare i pesi e le predizioni in un file & \centering Obbligatorio &  \makecell[tc]{V. esterno 2-02-2021} \\
			\hline
			\centering RSFO18.1 & Il formato di file prodotto deve essere .h5 & \centering Obbligatorio & \makecell[tc]{V. esterno 2-02-2021} \\
			\hline
			\centering RSFO19 & Viene inviato un messaggio di errore al front end$_{\scaleto{G}{3pt}}$, dal back end, se non ci sono i dati richiesti &\centering Obbligatorio & \makecell[tc]{Interno \\ UC6} \\
			\hline
			\centering RSFO20 & L'utente può selezionare una città tra quelle disponibili &\centering Obbligatorio & \makecell[tc]{Interno \\ UC3} \\
			\hline
			\centering RSFO21 & Le zone visualizzate della città dipendono dalle sorgenti esterne utilizzate &\centering Obbligatorio & \makecell[tc]{Interno} \\
			\hline
			\centering RSFO22  & I dati acquisiti da telecamere in tempo reale devono avere data di riferimento associata  &\centering Obbligatorio & \makecell[tc]{Interno} \\
			\hline
			\centering RSFO22.1  & I dati acquisiti da telecamere in tempo reale devono avere un orario di riferimento associato &\centering Obbligatorio & \makecell[tc]{Interno} \\
			\hline
			\centering RSFO22.2  & I dati acquisiti da telecamere in tempo reale devono avere un luogo di riferimento associato &\centering Obbligatorio  & \makecell[tc]{Interno} \\
			\hline
			\centering RSFF23 & Possibilità da parte del sistema di scegliere di mostrare i dati predetti in caso di mancanza di quelli reali &\centering Facoltativo & \makecell[tc]{Interno} \\
			\hline
			\centering RSFO24 & La selezione dell'orario è effettuata su intervalli di tempo di ora in ora &\centering Obbligatorio & \makecell[tc]{UC4.1} \\
			\hline
			\centering RSFO25 & Il sistema dà priorità ai dati reali presenti nel database per la visualizzazione della mappa su periodi di tempo storici &\centering Obbligatorio & \makecell[tc]{Interno} \\
			\hline
			\centering RSFO26 & Il sistema aggiorna automaticamente la mappa alla selezione di un diverso orario &\centering Obbligatorio & \makecell[tc]{UC4.1} \\
			\hline
			\centering RSFO27 & L'utente deve poter selezionare la data del giorno di cui vuole visualizzare i dati   &\centering Obbligatorio & \makecell[tc]{UC4.2} \\
			\hline
			\centering RSFO28 & L'utente deve poter ripristinare la visione in tempo reale tramite un pulsante di ripristino &\centering Obbligatorio & \makecell[tc]{UC4.3} \\
			\hline
			\centering RSFD29 & Il sistema deve poter prelevare dati da diverse fonti e formattarle nel tipo di default &\centering Desiderabile & \makecell[tc]{Interno} \\
			\hline
			\centering RSFO30 & Il sistema deve utilizzare un software 'contapersone' già allenato &\centering Obbligatorio & \makecell[tc]{V. esterno 28-01-2021} \\
			\hline
			\rowcolor{white}
			\caption[Requisiti funzionali]{Requisiti funzionali}\label{4.1}\\
	\end{longtable}%\captionof{table}{Requisiti funzionali}

\end{center}

\section{Requisiti prestazionali}\label{RequisitiPrestazionali}
\def\tabularxcolumn#1{m{#1}}
{\rowcolors{2}{RawSienna!90!RawSienna!20}{RawSienna!70!RawSienna!40}

	\begin{center}
		\renewcommand{\arraystretch}{1.4}
		\begin{longtable}{|p{4cm}|p{4cm}|p{4cm}|p{3cm}|}
		\hline
		\rowcolor{airforceblue}
		\makecell[c]{\textbf{Codice RS}} & \makecell[c]{\textbf{Descrizione}} & \makecell[c]{\textbf{Tipo di requisito}} & \makecell[c]{\textbf{Fonte}} \\
		\hline
		\centering RSPO1 & Capacità di acquisizione continuativa nel tempo dei dati da flussi esterni, viene prelevato almeno un dato ogni 10 minuti &\centering  Obbligatorio & \makecell[tc]{Capitolato$_{\scaleto{G}{3pt}}$}  \\
		\hline
		\centering RSPO2 & Modalità a bassa latenza nell'aquisizione di informazioni, almeno un dato ogni 5 minuti assumendo una connessione con download di minimo 100kb/s & \centering Obbligatorio & \makecell[tc]{Interno} \\
		\hline
		\centering RSPO3 & Modalità a bassa latenza per l'elaborazione dei dati acquisiti, almeno una elaborazione ogni 4 minuti & \centering Obbligatorio & \makecell[tc]{Interno} \\
		\hline
		\centering RSPO4 & Modalità a bassa latenza per la visualizzazione delle informazioni, la mappa si aggiorna in massimo 30s & \centering Obbligatorio & \makecell[tc]{Interno} \\
		\hline
		\centering RSPF5 & Misurazione indice di affidabilità sui dati in tempo reale di almeno 75\% & \centering Facoltativo &\makecell[tc]{Interno} \\
		\hline
		\rowcolor{white}
		
		\caption[Requisiti prestazionali]{Requisiti prestazionali}\label{4.2}\\
			\end{longtable}
	\end{center}
\section{Requisiti di qualità}\label{RequisitiDiQualita}
\def\tabularxcolumn#1{m{#1}}
{\rowcolors{2}{RawSienna!90!RawSienna!20}{RawSienna!70!RawSienna!40}

	\begin{center}
		\renewcommand{\arraystretch}{1.4}
		\begin{longtable}{|p{4cm}|p{4cm}|p{4cm}|p{3cm}|}
			\hline
			\rowcolor{airforceblue}
			\makecell[c]{\textbf{Codice RS}} & \makecell[c]{\textbf{Descrizione}} & \makecell[c]{\textbf{Tipo di requisito}} & \makecell[c]{\textbf{Fonte}} \\
			\hline
		\centering RSQO1  & La progettazione e la codifica dei requisiti devono rispettare le norme e le metriche definite nel documento \textit{Norme di Progetto v1.0.0}&\centering  Obbligatorio & \makecell[tc]{Interno} \\
		\hline
		\centering RSQF2  & Il codice sorgente del software deve essere disponibile in una repository$_G$ pubblica su Github$_G$  &\centering  Facoltativo & \makecell[tc]{Interno} \\
		\hline
		\centering RSQF3  & Deve essere sviluppato e fornito un documento con lo schema della base di dati relazionale  & \centering Facoltativo & \makecell[tc]{Interno } \\
	%	\hline
	%	RSQ  & Fornire una documentazione sui flussi di dati esterni.... & Facoltativo & FC3 \\
	%	\hline
		\hline
		\centering RSQF4  & Deve essere realizzato un documento contenente tutti gli errori risolti durante la realizzazione del software &\centering  Facoltativo & \makecell[tc]{Interno} \\
		\hline
		\centering RSQO5  & Test che dimostrino il corretto funzionamento dei servizi e delle funzionalità previste  & \centering Obbligatorio & \makecell[tc]{Capitolato$_{\scaleto{G}{3pt}}$} \\
		\hline
		\centering RSQO6  & Dev'essere disponibile un manuale sviluppatore  & \centering Obbligatorio & \makecell[tc]{Capitolato$_{\scaleto{G}{3pt}}$} \\
		\hline
		\centering RSQO7  & Dev'essere disponibile un manuale utente  & \centering Obbligatorio & \makecell[tc]{Capitolato$_{\scaleto{G}{3pt}}$} \\
		\hline
		\rowcolor{white}

		\caption[Requisiti di qualità]{Requisiti di qualità}\label{4.3}\\
		\end{longtable}
\end{center}

\newpage
\section{Requisiti di vincolo}\label{RequisitiVincolo}
\def\tabularxcolumn#1{m{#1}}
{\rowcolors{2}{RawSienna!90!RawSienna!20}{RawSienna!70!RawSienna!40}

\begin{center}
	\renewcommand{\arraystretch}{1.4}
	\begin{longtable}{|p{3cm}|p{4cm}|p{4cm}|p{4cm}|}
		\hline
		\rowcolor{airforceblue}
		\makecell[c]{\textbf{Codice RS}} & \makecell[c]{\textbf{Descrizione}} & \makecell[c]{\textbf{Tipo di requisito}} & \makecell[c]{\textbf{Fonte}} \\
		\hline
		\centering RSVO1  & Il front-end$_{\scaleto{G}{3pt}}$ del prodotto viene sviluppato utilizzando tecnologie web &\centering Obbligatorio  & \makecell[tc]{Capitolato$_{\scaleto{G}{3pt}}$} \\
		\hline
		\centering RSVF1.1  & Utilizzo di leaflet.js$_{\scaleto{G}{3pt}}$ per la creazione di heat map$_{\scaleto{G}{3pt}}$ &\centering  Facoltativo & \makecell[tc]{Capitolato$_{\scaleto{G}{3pt}}$} \\
		\hline
		\centering RSVO1.2  & Utilizzo di vue.js$_{\scaleto{G}{3pt}}$ per la creazione della wep-app$_{\scaleto{G}{3pt}}$  &\centering  Obbligatorio  & \makecell[tc]{V. esterno 02-02-2021} \\
		\hline
		\centering RSVO3  & Il sistema deve far uso dell'ecosistema Apache Kafka$_{\scaleto{G}{3pt}}$ &\centering  Obbligatorio  & \makecell[tc]{Capitolato$_{\scaleto{G}{3pt}}$} \\
		\hline
		\centering RSVO4  & Il back end$_{\scaleto{G}{3pt}}$ del prodotto viene sviluppato utilizzando il linguaggio Java$_{\scaleto{G}{3pt}}$ &\centering  Obbligatorio  & \makecell[tc]{Capitolato$_{\scaleto{G}{3pt}}$} \\
		\hline
		\centering RSVO5  & Supporto browser Chrome, Firefox con versioni massimo di 3 anni &\centering  Obbligatorio  & \makecell[tc]{Interno} \\
		\hline
		\centering RSVF6  & Supporto browser Safari, Microsoft Edge &\centering  Facoltativo  & \makecell[tc]{Interno} \\
		\hline
		\centering RSVO7  & La web application dev'essere disponibile in un ambiente locale, di sviluppo, e di produzione & \centering  Obbligatorio  & \makecell[tc]{Capitolato$_{\scaleto{G}{3pt}}$} \\
		\hline
		\centering RSVF8 & Utilizzo di Keras per lo sviluppo del modello machine learning$_{\scaleto{G}{3pt}}$ & \centering Facoltativo & \makecell[tc]{V. esterno 02-02-2021} \\
		\hline
		\centering RSVF9 & Utilizzo di Pandas come strumento per la manipolazione dei dati & \centering Facoltativo & \makecell[tc]{V. esterno 02-02-2021} \\
		\hline
		\rowcolor{white}

		\caption[Requisiti di vincolo]{Requisiti di vincolo}\label{4.4}\\
	\end{longtable}
\end{center}

\section{Tracciamento dei requisiti}\label{RequisitiTracciamentoDeiRequisiti}

\subsection{Requisito - fonte}\label{RequisitiTracciamentoDeiRequisitiFonte}

\def\tabularxcolumn#1{m{#1}}
{\rowcolors{2}{RawSienna!90!RawSienna!20}{RawSienna!70!RawSienna!40}
	\begin{center}
		\renewcommand{\arraystretch}{1.4}
		\begin{longtable}{|p{7.5cm}|p{7.5cm}|}
		\hline
		\rowcolor{airforceblue}
		\makecell[tc]{\textbf{Codice RS}} & \makecell[c]{\textbf{Fonte}}  \\
		\hline
		\makecell[tc]{RSFO1} & \makecell[tc]{Capitolato$_{\scaleto{G}{3pt}}$\\V. esterno 17-12-2020} \\
		\hline
		\makecell[tc]{RSFF2} & \makecell[tc]{Capitolato$_{\scaleto{G}{3pt}}$}\\
		\hline
		\makecell[tc]{RSFO3} & \makecell[tc]{Interno \\ UC2}\\
		\hline
		\makecell[tc]{RSFO4} & \makecell[tc]{Capitolato$_{\scaleto{G}{3pt}}$\\UC5}\\
		\hline
		\makecell[tc]{RSFO4.1} & \makecell[tc]{Capitolato$_{\scaleto{G}{3pt}}$\\UC5.1 \\ UC5.2}\\
		\hline
		\makecell[tc]{RSFO4.2} & \makecell[tc]{Capitolato$_{\scaleto{G}{3pt}}$\\UC5.3}\\
		\hline
		\makecell[tc]{RSFO5} & \makecell[tc]{Capitolato$_{\scaleto{G}{3pt}}$}\\
		\hline
		\makecell[tc]{RSFD5.1} & \makecell[tc]{Capitolato$_{\scaleto{G}{3pt}}$}\\ 
		\hline
		\makecell[tc]{RSFD6}& \makecell[tc]{Capitolato$_{\scaleto{G}{3pt}}$}\\
		\hline
		\makecell[tc]{RSFO7} & \makecell[tc]{Capitolato$_{\scaleto{G}{3pt}}$\\UC1}\\
		\hline
		\makecell[tc]{RSFO8} & \makecell[tc]{Interno}\\ 
		\hline
		\makecell[tc]{RSFO9} & \makecell[tc]{Interno \\ UC1}\\
		\hline
		\makecell[tc]{RSFO10} & \makecell[tc]{Interno \\ UC1}\\
		\hline
		\makecell[tc]{RSFO11} & \makecell[tc]{Interno \\ UC1}\\
		\hline
		\makecell[tc]{RSFF12} & \makecell[tc]{Interno}\\
		\hline
		\makecell[tc]{RSFD13} & \makecell[tc]{Interno}\\
		\hline
		\makecell[tc]{RSFD14} & \makecell[tc]{Interno}\\
		\hline
		\makecell[tc]{RSFF15} & \makecell[tc]{Interno}\\
		\hline
		\makecell[tc]{RSFF16} & \makecell[tc]{Interno}\\
		\hline
		\makecell[tc]{RSFO17} & \makecell[tc]{Interno}\\
		\hline		
		\makecell[tc]{RSFO18} & \makecell[tc]{V. esterno 02-02-2021}\\
		\hline
		\makecell[tc]{RSFO18.1} & \makecell[tc]{V. esterno 02-02-2021}\\
		\hline
		\makecell[tc]{RSFO19} & \makecell[tc]{Interno \\ UC6}\\
		\hline
		\makecell[tc]{RSFO20} & \makecell[tc]{Interno \\ UC3}\\
		\hline
		\makecell[tc]{RSFO21} & \makecell[tc]{Interno}\\
		\hline
		\makecell[tc]{RSFO22} & \makecell[tc]{Interno}\\
		\hline
		\makecell[tc]{RSFO22.1} & \makecell[tc]{Interno}\\
		\hline
		\makecell[tc]{RSFO22.2} & \makecell[tc]{Interno}\\
		\hline
		\makecell[tc]{RSFF23} & \makecell[tc]{Interno}\\
		\hline
		\makecell[tc]{RSFO24} & \makecell[tc]{UC4.1}\\
		\hline
		\makecell[tc]{RSFO25} & \makecell[tc]{Interno}\\
		\hline
		\makecell[tc]{RSFO26} & \makecell[tc]{UC4.1}\\
		\hline
		\makecell[tc]{RSFO27} & \makecell[tc]{UC4.2}\\
		\hline
		\makecell[tc]{RSFO28} & \makecell[tc]{UC4.3}\\
		\hline
		\makecell[tc]{RSFD29} & \makecell[tc]{Interno}\\
		\hline
		\makecell[tc]{RSFO30} & \makecell[tc]{V. esterno 28-01-2021}\\
		\hline
		\makecell[tc]{RSPO1} & \makecell[tc]{Capitolato$_{\scaleto{G}{3pt}}$}\\
		\hline
		\makecell[tc]{RSPO2} & \makecell[tc]{Interno}\\
		\hline
		\makecell[tc]{RSPO3} & \makecell[tc]{Interno}\\
		\hline
		\makecell[tc]{RSPO4} & \makecell[tc]{Interno}\\
		\hline
		\makecell[tc]{RSPF5} & \makecell[tc]{Interno}\\
		\hline
		\makecell[tc]{RSQO1} & \makecell[tc]{Interno}\\
		\hline
		\makecell[tc]{RSQF2} & \makecell[tc]{Interno}\\
		\hline
		\makecell[tc]{RSQF3} & \makecell[tc]{Interno }\\
		\hline
		\makecell[tc]{RSQF4} & \makecell[tc]{Interno}\\
		\hline
		\makecell[tc]{RSQO5} & \makecell[tc]{Capitolato$_{\scaleto{G}{3pt}}$}\\
		\hline
		\makecell[tc]{RSQO6} & \makecell[tc]{Capitolato$_{\scaleto{G}{3pt}}$}\\
		\hline
		\makecell[tc]{RSQO7} & \makecell[tc]{Capitolato$_{\scaleto{G}{3pt}}$}\\
		\hline
		\makecell[tc]{RSVO1} & \makecell[tc]{Capitolato$_{\scaleto{G}{3pt}}$}\\
		\hline
		\makecell[tc]{RSVO1.1} & \makecell[tc]{Capitolato$_{\scaleto{G}{3pt}}$}\\
		\hline
		\makecell[tc]{RSVO1.2} & \makecell[tc]{V. esterno 02-02-2021}\\
		\hline
		\makecell[tc]{RSVO2} & \makecell[tc]{Capitolato$_{\scaleto{G}{3pt}}$}\\
		\hline
		\makecell[tc]{RSVF2.1} & \makecell[tc]{Capitolato$_{\scaleto{G}{3pt}}$}\\
		\hline
		\makecell[tc]{RSVO2.2} & \makecell[tc]{Capitolato$_{\scaleto{G}{3pt}}$}\\
		\hline
		\makecell[tc]{RSVO3} & \makecell[tc]{Capitolato$_{\scaleto{G}{3pt}}$}\\
		\hline
		\makecell[tc]{RSVO4} & \makecell[tc]{Capitolato$_{\scaleto{G}{3pt}}$}\\
		\hline
		\makecell[tc]{RSVO5} & \makecell[tc]{Interno}\\
		\hline
		\makecell[tc]{RSVF6} & \makecell[tc]{Interno}\\
		\hline
		\makecell[tc]{RSVO7} & \makecell[tc]{Capitolato$_{\scaleto{G}{3pt}}$}\\
		\hline
		\makecell[tc]{RSVF8} & \makecell[tc]{V. esterno 02-02-2021}\\
		\hline
		\makecell[tc]{RSVF9} & \makecell[tc]{V. esterno 02-02-2021}\\
		\hline
		\rowcolor{white}

		\caption[Tabella tracciamento requisito-fonte]{Tabella tracciamento requisito-fonte}\label{4.5}\\
	\end{longtable}
\end{center}
\clearpage
\subsection{Fonte - requisito}\label{RequisitiTracciamentoDeiRequisitiFonteRequisito}
\def\tabularxcolumn#1{m{#1}}
{\rowcolors{2}{RawSienna!90!RawSienna!20}{RawSienna!70!RawSienna!40}
	\begin{center}
		\renewcommand{\arraystretch}{1.4}
		\begin{longtable}{|p{7.5cm}|p{7.5cm}|}
		\hline
		\rowcolor{airforceblue}
		\makecell[c]{\textbf{Fonte}} & \makecell[c]{\textbf{Codice RS}}  \\
		\hline
		\makecell[c]{Capitolato$_{\scaleto{G}{3pt}}$} & \makecell[c]{RSFO1\\RSFF2\\RSFO4\\RSFO4.1\\RSFO4.2\\RSFO5\\RSFD5.1\\RSFD6\\RSFO7\\RSPO1\\RSQO5\\RSQO6\\RSQO7\\RSVO2\\RSVF2.1\\RSVO2.2\\RSVO3\\RSVO4\\RSVO7} \\
		\hline
		\makecell[c]{UC1} & \makecell[c]{RSFO7 \\ RSFO9 \\ RSFO10 \\ RSFO11} \\
		\hline
		\makecell[c]{UC2} & \makecell[c]{RSFO3} \\
		\hline
		\makecell[c]{UC3} & \makecell[c]{RSFO20 \\ RSVO3.1 } \\ % UC3.2 RSFO18, UC3.3 RSFO19
		\hline
		\makecell[c]{UC4.1} & \makecell[c]{RSFO24 \\ RSFO26} \\
		\hline
		\makecell[c]{UC4.2} & \makecell[c]{RSFO27} \\
		\hline
		\makecell[c]{UC4.3} & \makecell[c]{RSFO28} \\
		\hline
		\makecell[c]{UC5} & \makecell[c]{RSFO4} \\
		\hline
		\makecell[c]{UC5.1} & \makecell[c]{RSFO4.1 \\ RSFO26 } \\
		\hline
		\makecell[c]{UC5.2} & \makecell[c]{RSFO4.1} \\
		\hline
		\makecell[c]{UC5.3} & \makecell[c]{RSFO4.2} \\
		\hline
		\makecell[c]{UC6} & \makecell[c]{RSFO19} \\
		\hline
		\makecell[c]{Interno} &\makecell[c]{RSFO8\\RSFO9\\RSFO10\\RSFO11\\RSFF12\\RSFD13\\RSFD14\\RSFF15\\RSFF16\\RSFO17\\RSFO19\\RSFO20\\RSFO21\\RSFO22\\RSFO22.1\\RSFO22.2\\RSFF23\\RSFO25\\RSFD29\\RSPO2\\RSPO3\\RSPO4\\RSPF5\\RSQO1\\RSQF2\\RSQF3\\RSQF4\\RSVO1\\RSVO1.1\\RSVO1.2} \\
		\hline
		\makecell[c]{Verbale esterno 17-12-2020} & \makecell[c]{RSFO1} \\
		\hline
		\makecell[c]{Verbale esterno 28-01-2021} & \makecell[c]{RSFO30} \\
		\hline
		\makecell[c]{Verbale esterno 02-02-2021} & \makecell[c]{RSFO18\\RSFO18.1\\RSVO1.2} \\
		\hline
		\rowcolor{white}

		\caption[Tabella tracciamento fonte-requisito]{Tabella tracciamento fonte-requisito}\label{4.6}\\
	\end{longtable}
\end{center}

\section{Considerazioni}\label{requisitiConsiderazioni}
I requisiti potranno subire delle variazioni in futuro, in modo tale da apportare degli aggiornamenti alle voci presenti o delle migliorie.
Nel caso in cui le attività pianificate terminassero prima del previsto e dovessero avanzare delle ore di lavoro, potranno essere presi in carico nuovi requisiti per aggiungere del valore al prodotto. Pertanto, qualsiasi espansione è riservata solo per il futuro.
