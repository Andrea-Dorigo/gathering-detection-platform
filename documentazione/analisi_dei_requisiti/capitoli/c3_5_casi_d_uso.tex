\chapter{Casi d'uso}\label{CasiDUso}
In questo capitolo vengono elencati i casi d'uso$_G$ individuati per il progetto \textit{GDP: Gathering Detection Platform} in accordo con il proponente$_{\scaleto{G}{3pt}}$. Ogni caso d'uso$_{\scaleto{G}{3pt}}$ indica un'interazione tra uno o più attori e il sistema. Questa interazione genera uno scenario, cioè l'insieme delle azioni che hanno in comune uno scopo finale per un attore.

\section{Casi d'uso tra un utente e il front end}\label{CasiDUsoCasiDUsoTraUnUtenteEIlFrontEnd}
%spiegazione della sezione
\subsection{Attori dei casi d'uso}\label{CasiDUsoCasiDUsoTraUnUtenteEIlFrontEndAttoriDeiCasiDUso}
\begin{center}
	\begin{figure}[H]
		\includegraphics{../immagini/attori_casi/utente_generico.png}
		\caption{Attore: utente generico}
	\end{figure}
\end{center}
\subsubsection{Attori Primari}\label{CasiDUsoCasiDUsoTraUnUtenteEIlFrontEndAttoriDeiCasiDUsoAttoriPrimari}
\begin{itemize}
	\item \textbf{Utente generico:} definisce l'utente generico che utilizza l'applicazione web;
\end{itemize}

\subsection{Elenco casi d'uso}\label{CasiDUsoCasiDUsoTraUnUtenteEIlFrontEndElencoCasiDUso}

\subsubsection{UC1 - Visualizzazione informazioni sulla mappa}\label{CasiDUsoCasiDUsoTraUnUtenteEIlFrontEndElencoCasiDUsoUC1VisualizzazioneInformazioniSullaMappa} %parzialmente corretto
\begin{center}
	\begin{figure}[H]
		\includegraphics[scale=0.8]{../immagini/attori_casi/uc1_uc2.png}
		\caption{UC1 - Visualizzazione informazioni sulla mappa}
	\end{figure}
\end{center}
\begin{itemize}
	\item \textbf{Attori primari}: utente generico;
	\item \textbf{Descrizione}: l’utente accede all’applicazione web e visualizza la heat map$_{\scaleto{G}{3pt}}$. La mappa mostra la città impostata di default o quella selezionata tra quelle a disposizione, come definito nell’UC3(sezione \S~\ref{CasiDUsoCasiDUsoTraUnUtenteEIlFrontEndElencoCasiDUsoUC3SelezioneCittaDaVisualizzareNellaMappa}). Le informazioni vengono ricavate dall’orario e la data impostate dall’utente come indicato nel UC4.1(sezione \S~\ref{CasiDUsoCasiDUsoTraUnUtenteEIlFrontEndElencoCasiDUsoUC41SelezioneDellOrario}) e UC4.2(sezione \S~\ref{CasiDUsoCasiDUsoTraUnUtenteEIlFrontEndElencoCasiDUsoUC42ModificaDellaData}) o si utilizzano i dati in tempo reale quindi usando l’orario attuale;
	\item \textbf{Scenario principale}: L’utente accede all’applicazione web e visualizza la heat map$_{\scaleto{G}{3pt}}$ della città;
	\item \textbf{Precondizione}: il front end$_G$ può generare la mappa; la città, la data, l’ora sono state indicate dall’utente, seguendo quanto descritto rispettivamente nell'UC3 (sezione \S~\ref{CasiDUsoCasiDUsoTraUnUtenteEIlFrontEndElencoCasiDUsoUC3SelezioneCittaDaVisualizzareNellaMappa}), nell'UC4.2(sezione \S~\ref{CasiDUsoCasiDUsoTraUnUtenteEIlFrontEndElencoCasiDUsoUC42ModificaDellaData}) e nell'UC4.1(sezione \S~\ref{CasiDUsoCasiDUsoTraUnUtenteEIlFrontEndElencoCasiDUsoUC41SelezioneDellOrario}), o vengono utilizzate quelle di default, quindi data e ora sono quelle odierne di sistema per dati in tempo reale e la città è quella impostata di default;
	\item \textbf{Postcondizione}: l’utente visualizza la heat map$_{\scaleto{G}{3pt}}$ con i dati ricavati nell’istante di tempo selezionato, come definito nell’UC4 (sezione \S~\ref{CasiDUsoCasiDUsoTraUnUtenteEIlFrontEndElencoCasiDUsoUC4SelezioneDellIstanzeDiCuiVisualizzareIDatiNellaHeatmap}), e alla città scelta fra quelle disponibili come descritto nella definizione dell’UC3 (sezione \S~\ref{CasiDUsoCasiDUsoTraUnUtenteEIlFrontEndElencoCasiDUsoUC3SelezioneCittaDaVisualizzareNellaMappa});
	\item \textbf{Estensioni}: l’utente accede all’applicazione web, il front end$_{\scaleto{G}{3pt}}$, rilevando la richiesta di generazione della mappa, individua una mancanza di dati per la sua costruzione e di conseguenza viene visualizzato un messaggio relativo all’errore riscontrato (UC2, sezione \S~\ref{CasiDUsoCasiDUsoTraUnUtenteEIlFrontEndElencoCasiDUsoUC2VisualizzazioneMessaggioPerLaMancanzaDiDati});
\end{itemize}

\subsubsection{UC1.1 - Zoom-in della heat map}\label{CasiDUsoCasiDUsoTraUnUtenteEIlFrontEndElencoCasiDUsoUC11ZoomInDellaHeatMap}

\begin{itemize}
	\item \textbf{Attori primari:} utente generico;
	\item \textbf{Descrizione:} l’utente, durante la visualizzazione della heat map$_{\scaleto{G}{3pt}}$, può aumentare il livello di zoom per vedere in dettaglio la mappa della città selezionata;
	\item \textbf{Scenario principale:} l’utente aumenta il livello di zoom della heat map$_{\scaleto{G}{3pt}}$ per una visualizzazione dettagliata della città;
	\item \textbf{Precondizione:} il sistema dispone di informazioni relative alla città e il livello di zoom-in non è al massimo;
	\item \textbf{Postcondizione:} la heat map$_{\scaleto{G}{3pt}}$ si aggiorna mostrando livelli di informazioni più dettagliate in base al livello di zoom in effettuato.
\end{itemize}

\subsubsection{UC1.1.1 - Drag della heat map}\label{CasiDUsoCasiDUsoTraUnUtenteEIlFrontEndElencoCasiDUsoUC111DragDellaHeatMap}

\begin{itemize}
	\item \textbf{Attori primari:} utente generico;
	\item \textbf{Descrizione:} l’utente può spostarsi all’interno della heat map$_{\scaleto{G}{3pt}}$;
	\item \textbf{Scenario principale:} l’utente si sposta all’interno della heat map$_{\scaleto{G}{3pt}}$;
	\item \textbf{Precondizione:} il livello di zoom in è diverso da quello iniziale e si sta visualizzando un dettaglio della heat map$_{\scaleto{G}{3pt}}$;
	\item \textbf{Postcondizione:} l’utente si è spostato all’interno della heat map$_{\scaleto{G}{3pt}}$.
\end{itemize}

\subsubsection{UC1.1.2 - Visualizzazione del popup di un punto di interesse}\label{CasiDUsoCasiDUsoTraUnUtenteEIlFrontEndElencoCasiDUsoUC112VisualizzazioneDelPopupDiUnPuntoDiInteresse}

\begin{itemize}
	\item \textbf{Attori primari:} utente generico;
	\item \textbf{Descrizione:} l’utente ha la possibilità di selezionare una zona in particolare della mappa e far apparire un popup contenente le informazioni riguardanti la zona selezionata;	
	\item \textbf{Scenario principale:} l’utente seleziona una zona e appare un popup;
	\item \textbf{Precondizione:} il livello di zoom in è diverso da quello iniziale ed il sistema fa apparire un’icona apposita per il popup;	
	\item \textbf{Postcondizione:} l’utente ha premuto sull’icona ed appare il popup.
\end{itemize}

\subsubsection{UC1.2 - Zoom-out della heat map}\label{CasiDUsoCasiDUsoTraUnUtenteEIlFrontEndElencoCasiDUsoUC12ZoomOutDellaHeatMap}

\begin{itemize}
	\item \textbf{Attori primari:} utente generico;
	\item \textbf{Descrizione:} l’utente, durante la visualizzazione della heat map$_{\scaleto{G}{3pt}}$, può diminuire il livello di zoom per vedere informazioni meno dettagliate della città selezionata;
	\item \textbf{Scenario principale:} l’utente diminuisce il livello di zoom della heat map$_{\scaleto{G}{3pt}}$ per una visualizzazione meno dettagliata della città;
	\item \textbf{Precondizione:}  il sistema dispone di informazioni relative alla città e il livello di zoom-out non è al massimo;
	\item \textbf{Postcondizione:} la heat map$_{\scaleto{G}{3pt}}$ si aggiorna mostrando livelli di informazioni meno dettagliate in base al livello di zoom out effettuato.
\end{itemize}

\subsubsection{UC2 - Visualizzazione messaggio per la mancanza di dati }\label{CasiDUsoCasiDUsoTraUnUtenteEIlFrontEndElencoCasiDUsoUC2VisualizzazioneMessaggioPerLaMancanzaDiDati} %parzialmente corretto
\begin{itemize}
	\item \textbf{Attori primari}: utente generico;
	\item \textbf{Descrizione}: l’utente visualizza un messaggio d’errore per la mancanza di dati necessari alla generazione della mappa. Questo accade quando il front end$_{\scaleto{G}{3pt}}$ non ha a disposizione tutti i dati;
	\item \textbf{Scenario principale}: 
	\begin{itemize}
		\item L’operazione di generazione mappa fallisce;
		\item L’utente visualizza un messaggio di errore per la mancanza dei dati;
		\item L’utente clicca il pulsante “ok” per chiudere il messaggio.
	\end{itemize}
	\item \textbf{Precondizione}: il front end$_{\scaleto{G}{3pt}}$ effettua un controllo sui dati, non sono presenti tutti i dati;
	\item \textbf{Postcondizione}: viene visualizzato un messaggio all’utente per informarlo sul problema riscontrato e l’operazione fallisce.
\end{itemize}

\subsubsection{UC3 - Selezione città da visualizzare nella mappa}\label{CasiDUsoCasiDUsoTraUnUtenteEIlFrontEndElencoCasiDUsoUC3SelezioneCittaDaVisualizzareNellaMappa} 
\begin{center}
	\begin{figure}[H]
		\includegraphics{../immagini/attori_casi/uc3.png}
		\caption{UC3 - Selezione città da visualizzare nella mappa}
	\end{figure}
\end{center}
\begin{itemize}
	\item \textbf{Attori primari}: utente generico;
	\item \textbf{Descrizione}:  l’utente può selezionare la città di cui vuole visualizzare la heat map$_{\scaleto{G}{3pt}}$;
	\item \textbf{Scenario principale}:l’utente seleziona una città tra quelle messe a disposizione;
	\item \textbf{Precondizione}: il sistema dispone di informazioni relative a diverse città;
	\item \textbf{Postcondizione}:   l’utente ha selezionato la città che vuole visualizzare, la heat-map$_{\scaleto{G}{3pt}}$ si aggiorna in base alla scelta fatta.
\end{itemize}

\subsubsection{UC4 - Selezione dell’istante di cui visualizzare i dati nella heat map
}\label{CasiDUsoCasiDUsoTraUnUtenteEIlFrontEndElencoCasiDUsoUC4SelezioneDellIstanzeDiCuiVisualizzareIDatiNellaHeatmap}%parzialmente corretto
\begin{center}
	\begin{figure}[H]
		\includegraphics{../immagini/attori_casi/uc4.png}
		\caption{UC4 - Selezione dell’istante di cui visualizzare i dati nella heat map}
	\end{figure}
\end{center}
\begin{itemize}
	\item \textbf{Attori primari}: utente generico;
	\item \textbf{Descrizione}: l’utente, attraverso l’interfaccia del sistema, modifica l’istante di tempo di cui vuole visualizzare i dati;
	\item \textbf{Scenario principale}: attraverso l’interfaccia l’utente può decidere di:
		\begin{enumerate}
			\item Modificare l’orario dei dati da visualizzare (UC4.1, sezione  \S~\ref{CasiDUsoCasiDUsoTraUnUtenteEIlFrontEndElencoCasiDUsoUC41SelezioneDellOrario});
			\item Modificare il giorno tra quelli disponibili (UC4.2, sezione \S~\ref{CasiDUsoCasiDUsoTraUnUtenteEIlFrontEndElencoCasiDUsoUC42ModificaDellaData});
			\item Ritornare ai dati in tempo reale (UC4.3, sezione \S~\ref{CasiDUsoCasiDUsoTraUnUtenteEIlFrontEndElencoCasiDUsoUC43RipristinoDatiATempoReale}).
		\end{enumerate}
	\item \textbf{Precondizione}: il sistema dispone di informazioni su diversi istanti di tempo;
	\item \textbf{Postcondizione}: l’utente ha selezionato un istante di tempo diverso da quello attuale e visualizza i dati riguardanti ad esso.%insicuro
\end{itemize}

\subsubsection{UC4.1 - Selezione dell’orario}\label{CasiDUsoCasiDUsoTraUnUtenteEIlFrontEndElencoCasiDUsoUC41SelezioneDellOrario}
\begin{itemize}
	\item \textbf{Attori primari}: utente generico;
	\item \textbf{Descrizione}: l’utente seleziona un orario diverso da quello attuale per visualizzare i dati di quel momento;
	\item \textbf{Scenario principale}: l’utente imposta un orario utilizzando l’interfaccia dell’applicazione web;
	\item \textbf{Precondizione}: il sistema ha informazioni riguardanti tutti i diversi orari; %insicuro
	\item \textbf{Postcondizione}:  l’orario viene aggiornato e la mappa visualizza i dati della modifica fatta.
\end{itemize}

\subsubsection{UC4.2 - Modifica della data}\label{CasiDUsoCasiDUsoTraUnUtenteEIlFrontEndElencoCasiDUsoUC42ModificaDellaData}
\begin{itemize}
	\item \textbf{Attori primari}: utente generico;
	\item \textbf{Descrizione}: l’utente seleziona una data diversa da quella odierna tra quelle disponibili e visualizza la mappa della data scelta;
	\item \textbf{Scenario principale}: l’utente seleziona una data diversa da quella attuale;
	\item \textbf{Precondizione}: il sistema possiede informazioni su tutte le date fino a quella odierna;
	\item \textbf{Postcondizione}: la data viene aggiornata e l’utente visualizza l’heat map$_{\scaleto{G}{3pt}}$ aggiornata con i dati del giorno selezionato all’orario attuale o all’orario scelto dall’utente stesso, secondo quanto definito nella descrizione dell’UC4.1 (sezione \S~\ref{CasiDUsoCasiDUsoTraUnUtenteEIlFrontEndElencoCasiDUsoUC41SelezioneDellOrario}).
\end{itemize}

\subsubsection{UC4.3 - Ripristino dati a tempo reale}\label{CasiDUsoCasiDUsoTraUnUtenteEIlFrontEndElencoCasiDUsoUC43RipristinoDatiATempoReale}
\begin{itemize}
	\item \textbf{Attori primari}: utente generico;
	\item \textbf{Descrizione}:  l’utente sceglie di osservare i dati in tempo reale;
	\item \textbf{Scenario principale}: l’utente preme sul pulsante per il ripristino dei valori attuali di data e ora;
	\item \textbf{Precondizione}: l’utente ha impostato una data e/o un’ora diversa dal valore di quella attuale secondo quanto descritto nell'UC4.1 (sezione \S~\ref{CasiDUsoCasiDUsoTraUnUtenteEIlFrontEndElencoCasiDUsoUC41SelezioneDellOrario}) e nell'UC4.2 (sezione \S~\ref{CasiDUsoCasiDUsoTraUnUtenteEIlFrontEndElencoCasiDUsoUC42ModificaDellaData});
	\item \textbf{Postcondizione}: l’utente visualizza la mappa con i dati in tempo reale.
\end{itemize}

\subsubsection{UC5 - Ricerca della città da visualizzare}\label{CasiDUsoCasiDUsoTraUnUtenteEIlFrontEndElencoCasiDUsoUC5RicercaDellaCittàDaVisualizzare}

\begin{itemize}
	\item \textbf{Attori primari:} utente generico;
	\item \textbf{Descrizione:} l’utente può ricercare in una barra di ricerca la città da visualizzare;
	\item \textbf{Scenario principale:} l’utente ricerca una città tramite una barra di ricerca;
	\item \textbf{Precondizione:} l’utente ha inserito la città da ricercare;
	\item \textbf{Postcondizione:}l’utente ha inserito la città che vuole cercare e il sistema si aggiorna in base alla ricerca fatta;
	\item \textbf{Estensioni:} l’utente ha ricercato una città non presente nel database, il sistema rileva questo errore e di conseguenza viene visualizzato un messaggio relativo all’errore riscontrato (UC 6, sezione \S~\ref{CasiDUsoCasiDUsoTraUnUtenteEIlFrontEndElencoCasiDUsoUC6VisualizzazioneMessaggioDiMancanzaCittàNelDatabase}).
\end{itemize}

\subsubsection{UC5.1 - Ricerca della città da visualizzare tramite codice identificativo}\label{CasiDUsoCasiDUsoTraUnUtenteEIlFrontEndElencoCasiDUsoUC51RicercaDellaCittàDaVisualizzareTramiteCodiceIdentificativo}

\begin{itemize}
	\item \textbf{Attori primari:} utente generico;
	\item \textbf{Descrizione:} l’utente ha la possibilità di ricercare la città da visualizzare tramite codice identificativo;
	\item \textbf{Scenario principale:} l’utente ricerca la città tramite codice identificativo;
	\item \textbf{Precondizione:} l’utente ha inserito il codice identificativo della città da ricercare;
	\item \textbf{Postcondizione:} il sistema mostra all’utente il risultato della ricerca effettuata.
\end{itemize}

\subsubsection{UC5.2 - Ricerca della città da visualizzare tramite nome}\label{CasiDUsoCasiDUsoTraUnUtenteEIlFrontEndElencoCasiDUsoUC52RicercaDellaCittàDaVisualizzareTramiteNome}

\begin{itemize}
	\item \textbf{Attori primari:} utente generico;
	\item \textbf{Descrizione:} l’utente ha la possibilità di ricercare la città da visualizzare tramite nome;
	\item \textbf{Scenario principale:} l’utente ricerca la città tramite nome;
	\item \textbf{Precondizione:} l’utente ha inserito il nome della città da ricerca;
	\item \textbf{Postcondizione:} il sistema mostra all’utente il risultato della ricerca effettuata.
\end{itemize}

\subsubsection{UC6 - Visualizzazione messaggio di mancanza città nel database}\label{CasiDUsoCasiDUsoTraUnUtenteEIlFrontEndElencoCasiDUsoUC6VisualizzazioneMessaggioDiMancanzaCittàNelDatabase}

\begin{itemize}
	\item \textbf{Attori primari:} utente generico;
	\item \textbf{Descrizione:} l’utente visualizza un messaggio di errore per un inserimento nella barra di ricerca di una città non presente nel database;
	\item \textbf{Scenario principale:} 
	\begin{enumerate}
		\item l’operazione di ricerca fallisce;
		\item l’utente visualizza un messaggio di errore;
		\item l’utente preme "ok" per chiudere il messaggio.
	\end{enumerate}
	\item \textbf{Precondizione:} il front end effettua un controllo sui dati e non è presente la città ricercata;
	\item \textbf{Postcondizione:} viene visualizzato  un messaggio all’utente per informarlo sul problema.
\end{itemize}

\section{Casi d'uso tra il front end e il back end}\label{CasiDUsoCasiDUsoTraIlFrontEndEIlBackEnd}
%spiegazione sezione

\subsection{Attori dei casi d'uso}\label{CasiDUsoCasiDUsoTraIlFrontEndEIlBackEndAttoriDeiCasiDUso}
%immagine errata
\begin{center}
	\begin{figure}[H]
		\includegraphics{../immagini/attori_casi/sistema_front_end.png}
		\caption{Attore: Sistema front end}
	\end{figure}
\end{center}
\subsubsection{Attori Primari}\label{CasiDUsoCasiDUsoTraIlFrontEndEIlBackEndAttoriDeiCasiDUsoAttoriPrimari}
\begin{itemize}
	\item \textbf{Sistema front end$_{\scaleto{G}{3pt}}$:} Definisce una parte del sistema sviluppato che interagisce con il sistema back end$_{\scaleto{G}{3pt}}$;
\end{itemize}

\subsection{Elenco casi d'uso}\label{CasiDUsoCasiDUsoTraIlFrontEndEIlBackEndElencoDeiCasiDUso}


\subsubsection{UC7 - Visualizzazione delle informazioni dal back end}\label{CasiDUsoCasiDUsoTraIlFrontEndEIlBackEndElencoDeiCasiDUsoUC7VisualizzazioneDelleInformazioniDalBackEnd}
\begin{center}
	\begin{figure}[H]
		\includegraphics[scale=0.7]{../immagini/attori_casi/uc5_uc51_uc52_uc53.png}
		\caption{UC7 - Visualizzazione delle informazioni dal back end}
	\end{figure}
\end{center}
\begin{itemize}
	\item \textbf{Attori primari}: sistema front end$_{\scaleto{G}{3pt}}$;
	\item \textbf{Descrizione}: il front end$_{\scaleto{G}{3pt}}$ effettua una richiesta per il recupero dei dati al back end$_{\scaleto{G}{3pt}}$ per generare la heat-map$_{\scaleto{G}{3pt}}$. Il back end$_{\scaleto{G}{3pt}}$ esegue a richiesta ed invia le informazioni da visualizzare al front end$_{\scaleto{G}{3pt}}$;
	\item \textbf{Scenario principale}: il front end$_{\scaleto{G}{3pt}}$ richiede al back end$_{\scaleto{G}{3pt}}$ le informazioni necessarie alla generazione della heat map$_{\scaleto{G}{3pt}}$, il front end$_{\scaleto{G}{3pt}}$ visualizza le informazioni ricevute dal back end$_{\scaleto{G}{3pt}}$;
	\item \textbf{Precondizione}: il front end$_{\scaleto{G}{3pt}}$ non ha le informazioni per poter generare la mappa;
	\item \textbf{Postcondizione}: il front end$_{\scaleto{G}{3pt}}$ visualizza e riceve le nuove informazioni. 
	\item \textbf{Generalizzazioni}: il front end$_{\scaleto{G}{3pt}}$ può fare una delle seguenti richieste:
	\begin{itemize}
		\item Visualizza i dati di uno specifico istante (UC7.1, sezione \S~\ref{CasiDUsoCasiDUsoTraIlFrontEndEIlBackEndElencoDeiCasiDUsoUC71VisualizzazioneDeiDatiDiUnoSpecificoIstante});
		\item Visualizza i dati in tempo reale (UC7.2, sezione  \S~\ref{CasiDUsoCasiDUsoTraIlFrontEndEIlBackEndElencoDeiCasiDUsoUC72VisualizzazioneDeiDatiInTempoReale});
		\item Visualizza i dati predetti (UC7.3, sezione  \S~\ref{CasiDUsoCasiDUsoTraIlFrontEndEIlBackEndElencoDeiCasiDUsoUC73VisualizzazioneDeiDatiPredetti}).
	\end{itemize}
	\item \textbf{Estensione}: il front end$_{\scaleto{G}{3pt}}$ effettua la richiesta al back end$_{\scaleto{G}{3pt}}$ il quale non invia nessun dato nella risposta (UC8, sezione \S~\ref{CasiDUsoCasiDUsoTraIlFrontEndEIlBackEndElencoDeiCasiDUsoUC8VisualizzazioneMessaggioDiMancanzaDatiDalBackEnd})
\end{itemize}

\subsubsection{UC7.1 - Visualizzazione dei dati di uno specifico istante}\label{CasiDUsoCasiDUsoTraIlFrontEndEIlBackEndElencoDeiCasiDUsoUC71VisualizzazioneDeiDatiDiUnoSpecificoIstante}
\begin{itemize}
	\item \textbf{Attori primari}: sistema front end$_{\scaleto{G}{3pt}}$;
	\item \textbf{Descrizione}: il front end$_{\scaleto{G}{3pt}}$ richiede le informazioni relative ad uno specifico istante di tempo, vengono visualizzate le informazioni inviate dal back end$_{\scaleto{G}{3pt}}$;
	\item \textbf{Scenario principale}:  il front end$_{\scaleto{G}{3pt}}$ richiede al back end$_{\scaleto{G}{3pt}}$ le informazioni relative all'istante di tempo specificato, il back end$_{\scaleto{G}{3pt}}$ invia le informazioni da visualizzare al front end$_{\scaleto{G}{3pt}}$;
	\item \textbf{Precondizione}: l’utente esegue la modifica della data o dell’orario come definito rispettivamente nella descrizione di UC4.2 (sezione \S~\ref{CasiDUsoCasiDUsoTraUnUtenteEIlFrontEndElencoCasiDUsoUC42ModificaDellaData}) e UC4.1 (sezione \S~\ref{CasiDUsoCasiDUsoTraUnUtenteEIlFrontEndElencoCasiDUsoUC41SelezioneDellOrario}) ponendo un orario precedente a quello attuale;
	\item \textbf{Postcondizione}: il front end$_{\scaleto{G}{3pt}}$ visualizza e riceve le informazioni relative all'istante di tempo impostato. 
\end{itemize}

\subsubsection{UC7.2 - Visualizzazione dei dati in tempo reale}\label{CasiDUsoCasiDUsoTraIlFrontEndEIlBackEndElencoDeiCasiDUsoUC72VisualizzazioneDeiDatiInTempoReale}
\begin{itemize}
	\item \textbf{Attori primari}: sistema front end$_{\scaleto{G}{3pt}}$;
	\item \textbf{Descrizione}: il front end$_{\scaleto{G}{3pt}}$ visualizza i dati reali più recentemente aggiunti;
	\item \textbf{Scenario principale}: il front end$_{\scaleto{G}{3pt}}$ richiede al back end$_{\scaleto{G}{3pt}}$ le informazioni più recentemente aggiunte, una volta ricevute il front end$_{\scaleto{G}{3pt}}$ le visualizza;
	\item \textbf{Precondizione}: viene eseguita la visualizzazione della mappa come definito nell’UC1 (sezione \S~\ref{CasiDUsoCasiDUsoTraUnUtenteEIlFrontEndElencoCasiDUsoUC1VisualizzazioneInformazioniSullaMappa}) o avviene il ripristino dei dati in tempo reale come definito in UC4.3 (sezione \S~\ref{CasiDUsoCasiDUsoTraUnUtenteEIlFrontEndElencoCasiDUsoUC43RipristinoDatiATempoReale});
	\item \textbf{Postcondizione}: il front end$_{\scaleto{G}{3pt}}$ ha ricevuto e visualizzato i dati ed è pronto alla generazione della heat map$_{\scaleto{G}{3pt}}$. 
\end{itemize}

\subsubsection{UC7.3 - Visualizzazione dei dati predetti}\label{CasiDUsoCasiDUsoTraIlFrontEndEIlBackEndElencoDeiCasiDUsoUC73VisualizzazioneDeiDatiPredetti}
\begin{itemize}
	\item \textbf{Attori primari}: sistema front end$_{\scaleto{G}{3pt}}$;
	\item \textbf{Descrizione}: il front end$_{\scaleto{G}{3pt}}$ richiede i dati riferiti allo stesso giorno, ma ad un orario avanzato rispetto a quello attuale.
	I dati sono ricavati dall’elaborazione, attraverso un modello di machine learning$_{\scaleto{G}{3pt}}$, dei dati reali acquisti. Una volta ricevuti i dati il front end$_{\scaleto{G}{3pt}}$ li può visualizzare;
	\item \textbf{Scenario principale}: il front end$_{\scaleto{G}{3pt}}$ richiede al back end$_{\scaleto{G}{3pt}}$ i dati elaborati dal modello machine learning$_{\scaleto{G}{3pt}}$. Completata la richiesta il front end$_{\scaleto{G}{3pt}}$ visualizzerà i dati inviati dal back end$_{\scaleto{G}{3pt}}$;
	\item \textbf{Precondizione}: le informazioni vengono visualizzate sulla mappa come definito nell’UC1 (sezione \S~\ref{CasiDUsoCasiDUsoTraUnUtenteEIlFrontEndElencoCasiDUsoUC1VisualizzazioneInformazioniSullaMappa}), impostando un orario successivo a quello attuale come descritto nell’UC4.1 (sezione \S~\ref{CasiDUsoCasiDUsoTraUnUtenteEIlFrontEndElencoCasiDUsoUC41SelezioneDellOrario});
	\item \textbf{Postcondizione}: il front end$_{\scaleto{G}{3pt}}$ ha ricevuto e visualizzato i dati ed è pronto alla generazione della heat map$_{\scaleto{G}{3pt}}$. 
\end{itemize}

\subsubsection{UC8 - Visualizzazione messaggio di mancanza dati dal back end}\label{CasiDUsoCasiDUsoTraIlFrontEndEIlBackEndElencoDeiCasiDUsoUC8VisualizzazioneMessaggioDiMancanzaDatiDalBackEnd}
\begin{itemize}
	\item \textbf{Attori primari}: sistema front end$_{\scaleto{G}{3pt}}$;
	\item \textbf{Descrizione}: il front end$_{\scaleto{G}{3pt}}$ riceve un messaggio di errore per la mancanza dati rispetto alla richiesta di visualizzazione effettuata;
	\item \textbf{Scenario principale}: 
	\begin{enumerate}
		\item Il front end$_{\scaleto{G}{3pt}}$ richiede dei dati specifici al back end$_{\scaleto{G}{3pt}}$;
		\item La risposta ricevuta è un messaggio di errore;
		\item Il front end$_{\scaleto{G}{3pt}}$ ritenta la richiesta di informazioni. 
	\end{enumerate}
	\item \textbf{Precondizione}: il front end$_{\scaleto{G}{3pt}}$ effettua una richiesta di dati, il back end$_{\scaleto{G}{3pt}}$ non ha a disposizione i dati richiesti;
	\item \textbf{Postcondizione}: il front end$_{\scaleto{G}{3pt}}$ riceve un messaggio di errore per la mancanza dei dati da visualizzare. 
\end{itemize}

\section{Casi d'uso facoltativi tra un utente e il front end}\label{CasiDUsoCasiDUsoFacoltativiTraUnUtenteEIlFrontEnd}
%spiegazione della sezione
L'elenco dei casi d'uso in questa sezione individuano requisiti sviluppabili successivamente a quelli obbligatori descritti nelle sezioni precedenti.
\subsection{Attori dei casi d'uso}
\begin{center}
	\begin{figure}[H]
		\includegraphics{../immagini/attori_casi/utente_generico.png}
		\caption{Attore: utente generico}
	\end{figure}
\end{center}
\subsubsection{Attori Primari}\label{UFattoriPrimariFac}
\begin{itemize}
	\item \textbf{Utente generico:} definisce l'utente generico che utilizza l'applicazione web;
\end{itemize}

\subsection{Elenco casi d'uso}\label{CasiDUsoCasiDUsoFacoltativiTraUnUtenteEIlFrontEndElencoCasiDUso}
\subsubsection{UC9 - Visualizzazione indici di affidabilità}\label{indiciAffidabilita}
\begin{center}
	\begin{figure}[H]
		\includegraphics[scale=0.7]{../immagini/attori_casi/uc7.png}
		\caption{UC9 - Visualizzazione indici di affidabilità}
	\end{figure}
\end{center}
\begin{itemize}
	\item \textbf{Attori primari}: utente generico;
	\item \textbf{Descrizione}: l'utente può visualizzare gli indici di affidabilità dei dati reali raccolti e l'indice di affidabilità delle predizioni svolte dal modello di machine learning$_{\scaleto{G}{3pt}}$;
	\item \textbf{Scenario principale}: l'utente attraverso l'interfaccia seleziona un pulsante per visualizzare gli indici di affidabilità;
	\item \textbf{Precondizione}: il front end$_{\scaleto{G}{3pt}}$ dispone degli indici relativi ai dati reali e predetti;
	\item \textbf{Postcondizione}: l'utente visualizza correttamente gli indici di affidabilità dei dati reali e predetti. 
\end{itemize}

\subsubsection{UC10 - Impostazioni avanzate sui dati}\label{CasiDUsoCasiDUsoFacoltativiTraUnUtenteEIlFrontEndElencoCasiDUsoUC10ImpostazioniAvanzateSuiDati}
\begin{center}
	\begin{figure}[H]
		\includegraphics[scale=0.7]{../immagini/attori_casi/uc8.png}
		\caption{UC10 - Impostazioni avanzate sui dati}
	\end{figure}
\end{center}
\begin{itemize}
	\item \textbf{Attori primari}: utente generico;
	\item \textbf{Descrizione}: l'utente attraverso l'interfaccia del front end$_{\scaleto{G}{3pt}}$ può applicare filtri sui dati e modificare le sorgenti esterne da cui vengono prelevate le informazioni;
	\item \textbf{Scenario principale}: attraverso l'interfaccia l'utente può:
	\begin{itemize}
		\item Applicare filtri ai dati (sezione  \S~\ref{CasiDUsoCasiDUsoFacoltativiTraUnUtenteEIlFrontEndElencoCasiDUsoUC101ApplicazioneFiltriAiDati});
		\item Modificare le sorgenti esterne da cui vengono prelevate le informazioni (sezione \S~\ref{CasiDUsoCasiDUsoFacoltativiTraUnUtenteEIlFrontEndElencoCasiDUsoUC102SelezioneSorgentiEsterneDaCuiPrelevareIDati});
	\end{itemize}
	\item \textbf{Precondizione}: l'utente visualizza correttamente l'interfaccia e sono disponibili varie sorgenti esterne;
	\item \textbf{Postcondizione}: l'utente applica le impostazioni scelte ai dati e viene aggiornata la mappa di conseguenza. 
\end{itemize}

\subsubsection{UC10.1 - Applicazione filtri ai dati}\label{CasiDUsoCasiDUsoFacoltativiTraUnUtenteEIlFrontEndElencoCasiDUsoUC101ApplicazioneFiltriAiDati}
\begin{itemize}
	\item \textbf{Attori primari}: utente generico;
	\item \textbf{Descrizione}: l'utente attraverso l'interfaccia del front end$_{\scaleto{G}{3pt}}$ può applicare filtri sui dati reali e su quelli predetti, modificandone i colori con cui vengono visualizzati nella mappa;
	\item \textbf{Scenario principale}: 
	\begin{enumerate}
		\item L'utente può selezionare il colore per i dati reali e/o per quelli predetti;
		\item L'utente conferma i filtri da applicare alla mappa. 
	\end{enumerate}
	\item \textbf{Precondizione}: l'utente visualizza correttamente l'interfaccia;
	\item \textbf{Postcondizione}: l'utente applica i filtri ai dati e viene aggiornata la mappa di conseguenza. 
\end{itemize}

\subsubsection{UC10.2 - Selezione sorgenti esterne da cui prelevare i dati}\label{CasiDUsoCasiDUsoFacoltativiTraUnUtenteEIlFrontEndElencoCasiDUsoUC102SelezioneSorgentiEsterneDaCuiPrelevareIDati}
\begin{itemize}
	\item \textbf{Attori primari}: utente generico;
	\item \textbf{Descrizione}: l'utente attraverso l'interfaccia del front end$_{\scaleto{G}{3pt}}$ dispone di un menù in cui può selezionare le sorgenti che vuole utilizzare per il reperimento dei dati;
	\item \textbf{Scenario principale}: l'utente seleziona la modifica delle sorgenti esterne e indica quelle da cui vuole prelevare informazioni;
	\item \textbf{Precondizione}: l'utente visualizza correttamente l'interfaccia, sono disponibili varie sorgenti esterne;
	\item \textbf{Postcondizione}: l'utente visualizza la mappa con i soli dati delle sorgenti scelte. 
\end{itemize}

\subsubsection{UC11 - Recupero manuale utente}\label{CasiDUsoCasiDUsoFacoltativiTraUnUtenteEIlFrontEndElencoCasiDUsoUC11RecuperoManualeUtente}
\begin{center}
	\begin{figure}[H]
		\includegraphics[scale=0.7]{../immagini/attori_casi/uc9.png}
		\caption{UC11 - Recupero manuale utente}
	\end{figure}
\end{center}
\begin{itemize}
	\item \textbf{Attori primari}: utente generico;
	\item \textbf{Descrizione}: l'utente attraverso l'interfaccia del front end$_{\scaleto{G}{3pt}}$ può recuperare il manuale d'uso per informazioni sull'utilizzo dell'applicazione web;
	\item \textbf{Scenario principale}: l'utente seleziona il link al recupero del manuale utente;
	\item \textbf{Precondizione}: il front end$_{\scaleto{G}{3pt}}$ dispone del manuale utente;
	\item \textbf{Postcondizione}: l'utente dispone del manuale utente sul proprio dispositivo e lo può visualizzare. 
\end{itemize}

\subsubsection{UC12 -Visualizzazione di dati a confronto di due città differenti}\label{CasiDUsoCasiDUsoFacoltativiTraUnUtenteEIlFrontEndElencoCasiDUsoUC12VisualizzazioneDiDatiAConfrontoDiDueCittaDifferenti}

\begin{itemize}
	\item \textbf{Attori primari}: utente generico;
	\item \textbf{Descrizione}: l’utente può selezionare due città per poter mettere a confronto i loro dati;
	\item \textbf{Scenario principale}: l’utente seleziona le due città;
	\item \textbf{Precondizione}: il sistema dispone le informazioni riguardanti le città;
	\item \textbf{Postcondizione}: l'utente visualizza i dati di entrambe le città per poterli mettere a confronto. 
\end{itemize}

\subsubsection{UC13 -Salvataggio dei dati di una città in un file esterno}\label{CasiDUsoCasiDUsoFacoltativiTraUnUtenteEIlFrontEndElencoCasiDUsoUC13SalvataggioDeiDatiDiUnaCittaInUnFileEsterno}

\begin{itemize}
	\item \textbf{Attori primari}: utente generico;
	\item \textbf{Descrizione}: l’utente ha la possibilità di salvare localmente i dati relativi di una città in un file;
	\item \textbf{Scenario principale}: l’utente salva localmente i dati della città che sta visualizzando;
	\item \textbf{Precondizione}: il sistema dispone le informazioni riguardanti le città e  l’utente sta visualizzando la heat map$_{\scaleto{G}{3pt}}$ di una città in particolare;
	\item \textbf{Postcondizione}: il sistema ha salvato localmente i dati della città che sta visualizzando. 
\end{itemize}

\subsubsection{UC13.1 -Salvataggio dei dati di una città in un file esterno di tipo pdf}\label{CasiDUsoCasiDUsoFacoltativiTraUnUtenteEIlFrontEndElencoCasiDUsoUC131SalvataggioDeiDatiDiUnaCittaInUnFileEsternoDiTipoPdf}

\begin{itemize}
	\item \textbf{Attori primari}: utente generico;
	\item \textbf{Descrizione}: l’utente può selezionare l’estensione del file in pdf;
	\item \textbf{Scenario principale}: l’utente seleziona l’estensione del file in pdf;
	\item \textbf{Precondizione}: il sistema dispone le informazioni riguardanti le città e  l’utente sta visualizzando la heat map$_{\scaleto{G}{3pt}}$ di una città in particolare;
	\item \textbf{Postcondizione}: il sistema ha salvato localmente i dati in formato pdf.
\end{itemize}

\subsubsection{UC13.2 -Salvataggio dei dati di una città in un file esterno di tipo csv}\label{CasiDUsoCasiDUsoFacoltativiTraUnUtenteEIlFrontEndElencoCasiDUsoUC132SalvataggioDeiDatiDiUnaCittaInUnFileEsternoDiTipoCsv}

\begin{itemize}
	\item \textbf{Attori primari}: utente generico;
	\item \textbf{Descrizione}: l’utente può selezionare l’estensione del file in csv;
	\item \textbf{Scenario principale}: l’utente seleziona l’estensione del file in csv;
	\item \textbf{Precondizione}: il sistema dispone le informazioni riguardanti le città e  l’utente sta visualizzando la heat map$_{\scaleto{G}{3pt}}$ di una città in particolare;
	\item \textbf{Postcondizione}: il sistema ha salvato localmente i dati in formato csv.
\end{itemize}

\subsubsection{UC14 - Notifica via email di una città selezionata}\label{CasiDUsoCasiDUsoFacoltativiTraUnUtenteEIlFrontEndElencoCasiDUsoUC14NotificaViaEmailDiUnaCittaSelezionata}

\begin{itemize}
	\item \textbf{Attori primari}: utente generico;
	\item \textbf{Descrizione}: l'utente ha la possibilità di richiedere una notifica via email se i dati della città che sta visualizzando nella heat map$_{\scaleto{G}{3pt}}$ nel momento in cui il rischio di assembramento supera una certa soglia;
	\item \textbf{Scenario principale}: l'utente richiede la notifica via email;
	\item \textbf{Precondizione}: l'utente sta visualizzando la heat map$_{\scaleto{G}{3pt}}$ e richiede la notifica via email;
	\item \textbf{Postcondizione}: il sistema fa apparire un popup con la richiesta di inserimento dell'email a cui inviare i dati.
\end{itemize}

\subsubsection{UC14.1 - Inserimento e-mail}\label{CasiDUsoCasiDUsoFacoltativiTraUnUtenteEIlFrontEndElencoCasiDUsoUC141InserimentoEmail}

\begin{itemize}
	\item \textbf{Attori primari}: utente generico;
	\item \textbf{Descrizione}: per procedere la richiesta di notifica l'utente deve completare il form di inserimento dell'email; 
	\item \textbf{Scenario principale}: l'utente inserisce l'email;
	\item \textbf{Precondizione}: il sistema è in attesa dell'inserimento dell'email;
	\item \textbf{Postcondizione}: il sistema aggiunge nel database l'email inserita dall'utente;
	\item \textbf{Estensioni}: il front end effettua un controllo ortografico che segnala possibili errori all'utente.
\end{itemize}

\subsubsection{UC15 - Visualizzazione messaggio di errore e-mail errata}\label{CasiDUsoCasiDUsoFacoltativiTraUnUtenteEIlFrontEndElencoCasiDUsoUC15VisualizzazioneMessaggioDiErroreEmailErrata}

\begin{itemize}
	\item \textbf{Attori primari}: utente generico;
	\item \textbf{Descrizione}: Il front end$_{\scaleto{G}{3pt}}$ invia un messaggio di errore per un inserimento ortografico errato;
	\item \textbf{Scenario principale}: l'utente legge il messaggio e capisce che deve controllare l'email inserita(??)
	\item \textbf{Precondizione}: il front end$_{\scaleto{G}{3pt}}$ blocca l'invio di dati al database;
	\item \textbf{Postcondizione}: il front end$_{\scaleto{G}{3pt}}$ invia un messaggio di errore.
\end{itemize}

\subsubsection{UC16 - Visualizzazione lista delle città più cercate}\label{CasiDUsoCasiDUsoFacoltativiTraUnUtenteEIlFrontEndElencoCasiDUsoUC16VisualizzazioneListaDelleCittaPiuCercate}

\begin{itemize}
	\item \textbf{Attori primari}: utente generico;
	\item \textbf{Descrizione}: l'utente ha la possibilità di visionare la lista delle città più cercate all'interno del sito; 
	\item \textbf{Scenario principale}: l'utente visualizza la lista;
	\item \textbf{Precondizione}: il sistema è funzionante e possiede le informazioni riguardanti alle ricerche effettuate dagli utenti;
	\item \textbf{Postcondizione}: il back end$_{\scaleto{G}{3pt}}$ invia al front end$_{\scaleto{G}{3pt}}$ la lista delle città più cercate che verrà visualizzata dall'utente.
\end{itemize}

\subsubsection{UC17 - Visualizzazione lista delle città presenti in ordine crescente per rischio}\label{CasiDUsoCasiDUsoFacoltativiTraUnUtenteEIlFrontEndElencoCasiDUsoUC17VisualizzazioneListaDelleCittaPresentiInOrdineCrescentePerRischio}

\begin{itemize}
	\item \textbf{Attori primari}: utente generico;
	\item \textbf{Descrizione}: l'utente ha la possibilità di visionare la lista delle città presenti nel sito in ordine crescente per rischio; 
	\item \textbf{Scenario principale}: l'utente visualizza la lista;
	\item \textbf{Precondizione}:  il sistema è funzionante e possiede le informazioni riguardanti alle città;
	\item \textbf{Postcondizione}: il back end$_{\scaleto{G}{3pt}}$ invia al front end$_{\scaleto{G}{3pt}}$ la lista delle città in ordine crescente per rischio che verrà visualizzata dall'utente.
\end{itemize}