%COMANDO CHE DETERMINA IL TIPO DI DOCUMENTO CHE SI VUOLE CREARE

\documentclass[a4paper,12pt]{report}

%ELENCO DEI PACCHETTI UTILI PER LA STESURA DEL DOCUMENTO
\usepackage{enumitem}
\usepackage[utf8]{inputenc}
\usepackage[english, italian]{babel}
\usepackage{graphicx}
\usepackage{float}
\usepackage{tabularx}
\usepackage{makecell}
\usepackage{titlesec}
\usepackage{fancyhdr}
\usepackage{lastpage}
\usepackage{xurl}
\usepackage{hyperref}
\usepackage{geometry}
\usepackage[table,dvipsnames]{xcolor}
\setcounter{tocdepth}{5}
\setcounter{secnumdepth}{5}
\usepackage{caption}
\usepackage{etoolbox}% >= v2.1 2011-01-03
\usepackage{tikz}
\usepackage{listings}
\usepackage{scalerel}
\usepackage{longtable}
\captionsetup[table]{position=bottom}
\usepackage{comment}
\input{config}
\begin{document}

\makeatletter
\begin{titlepage}
	\begin{center}
		\vspace*{-4,0cm}
		\author{Jawa Druids} 
		\title{Analisi dei Requisiti}
		\date{} %LASCIARE QUESTO CAMPO VUOTO, SE LO TOLGO STAMPA LA DATA CORRENTE
		\includegraphics[width=0.5\linewidth]{../immagini/DRUIDSLOGO.jpg}\\[4ex]
		{\huge \bfseries  \@title }\\[2ex] 
		{\LARGE  \@author}\\[50ex]
		\vspace*{-8,0cm}
		\begin{table}[H]
			\centering
			\begin{tabular}{r | l}
				\textbf{Versione} & v1.0.0 \\%RIGA PER INSERIRE LA VERSIONE ULTIMA DEL DOCUMENTO
				\textbf{Data approvazione} & 10-01-2021\\
				\textbf{Responsabile} & Andrea Cecchin\\
				\textbf{Redattori} & \makecell[tl]{Andrea Cecchin \\ Igli Mezini} \\
				\textbf{Verificatori} & \makecell[tl]{Alfredo Graziano } \\
				%MAKECELL SERVE PER POI ANDARE A CAPO ALL'INTERNO DELLA CELLA
				\textbf{Stato} & Approvato\\
				\textbf{Lista distribuzione} & \makecell[tl]{\textit{Jawa Druids} \\ \textit{Prof. Tullio Vardanega} \\ \textit{Prof. Riccardo Cardin} \\ \textit{Sync Lab}}\\
				\textbf{Uso} & Esterno            
			\end{tabular}
		\end{table}
		\vspace{0.2cm}
		\hfill \break
		\fontsize{17}{10}\textbf{Sommario} \\
		\vspace{0.3cm}
		L'\emph{\normalsize Analisi dei Requisiti} individua tutti i requisiti da implementare nel prodotto da sviluppare.
	\end{center}
\end{titlepage}
\makeatother

\def\myformat#1{
	\centering\huge#1
}
\def\tabularxcolumn#1{m{#1}}
\quad
\section*{\myformat{Registro delle modifiche}}

{\rowcolors{2}{Apricot!90!Bittersweet!20}{Bittersweet!70!Apricot!40}
\begin{table}[H]
  \caption[Registro delle Modifiche]{}
	\vspace{-1.4cm}
  \begin{center}
    \begin{tabularx}{\textwidth}{|Y|c|c|c|c|}
    	\hline
    	\rowcolor{Melon}
    	\textbf{Modifica} & \textbf{Autore} & \textbf{Ruolo} & \textbf{Data} & \textbf{Versione}\\
      \hline
      \textit{Aggiunte sezioni §1.2, §1.3} & Andrea Dorigo & \textit{Responsabile} & 2-12-2020 & v0.0.2 \\
    	\hline
    	\textit{Aggiunta sezione §1.1} & Andrea Dorigo & \textit{Responsabile} & 30-11-2020 & v0.0.1 \\
    	\hline
    \end{tabularx}
  \end{center}
\end{table}


%COMANDO PER LA CREAZIONE DELL'INDICE

\tableofcontents{}
\listoftables{}
\chapter{Introduzione}

\section{Scopo del documento}
Lo scopo del documento è quello di formalizzare i contenuti e le qualità che il prodotto sviluppato dovrà raggiungere. 
I requisiti sono stati individuati attraverso lo studio del capitolato e incontri con l'azienda proponente \textit{Sync Lab}. 
Il documento inoltre è necessario a:
\begin{itemize}
	\item descrivere accuratamente tutti i requisiti proposti dal proponente;
	\item comprendere da parte del committente quali sono le richieste del cliente;
	\item definire il formato e contenuto di ogni requisito specifico del software.
\end{itemize} 
\section{Scopo del prodotto}
In seguito alla pandemia del virus COVID-19 è nata l'esigenza di limitare il più possibile i contatti fra le persone, specialmente evitando la formazione di assembramenti. Il progetto \textit{GDP: Gathering Detection Platform} di \textit{Sync Lab} ha pertanto l'obiettivo di \textbf{creare una piattaforma in grado di rappresentare graficamente le zone potenzialmente a rischio di assembramento, al fine di prevenirlo.}
\\
Al tal fine il gruppo \textit{Jawa Druids} si prefigge di sviluppare un prototipo software in grado di acquisire, monitorare ed analizzare i molteplici dati provenienti dai diversi sistemi e dispositivi, a scopo di identificare i possibili eventi che concorrono all'insorgere di variazioni di flussi di utenti. Il gruppo prevede inoltre lo sviluppo di un'applicazione web da interporre fra i dati elaborati e l'utente, per favorirne la consultazione.
\section{Glossario}
All'interno della documentazione viene fornito un \textit{Glossario}, con l'obiettivo di assistere il lettore specificando il significato e contesto d'utilizzo di alcuni termini strettamente tecnici o ambigui, segnalati con una \textit{G} a pedice.

\section{Riferimenti}
\subsection{Riferimenti normativi}
\begin{itemize}
	\item \textit{Norme di Progetto v1.0.0;}
		\item \textit{Verbale Esterno 17-12-2020;}
		\item \textit{Capitolato d'appalto C3:} \\ \url{https://www.math.unipd.it/~tullio/IS-1/2020/Progetto/C3.pdf}
\end{itemize}
\subsection{Riferimenti informativi}
\begin{itemize}
	\item \textit{Presentazione del capitolato:} \\ \url{https://www.math.unipd.it/~tullio/IS-1/2020/Progetto/C3.pdf}
		\item \textit{Materiale didattico relativo all'Analisi dei Requisiti del corso di Ingegneria del Software:}\\ \url{https://www.math.unipd.it/~tullio/IS-1/2020/Dispense/L07.pdf}
	\item \textit{IEEE Recommended Practice for Software Requirements Specifications:}\\
		\url{https://ieeexplore.ieee.org/document/720574}
	\item \textit{Seminario per approfondimenti tecnici del capitolato C3:}\\
		\url{https://www.math.unipd.it/~tullio/IS-1/2020/Progetto/ST1.pdf}	
		%aggiungere capitolato c3 pdf	
\end{itemize}
\chapter{Descrizione generale}\label{DescrizioneGenerale}
\section{Caratteristiche del prodotto}\label{DescrizioneGeneraleCaratteristicheProdotto}
L'idea del capitolato$_G$ \textit{GDP - Gathering Detection Platform} è di creare una piattaforma che riesca a rappresentare mediante visualizzazione grafica zone potenzialmente a rischio di assembramento con l'intento di prevenirle.
La piattaforma utilizzerà dati prelevati da sensori (come telecamere, dispositivi contapersone, etc.) o sorgenti dati (come flussi di prenotazioni Uber, le tabelle degli orari di autobus/metro/treno, etc.), i quali mediante la loro elaborazione verranno rappresentati tramite una \textit{heat map$_G$}.

\section{Funzionalità del prodotto}\label{DescrizioneGeneraleFunzionalitàDelProdotto}
La funzionalità principale identificata nel capitolato$_{\scaleto{G}{3pt}}$ d'appalto \textit{GDP-Gathering Detection Platform} è la \textbf{rappresentazione via \textit{heat map$_G$} dei dati ottenuti dalle sorgenti e della loro elaborazione}, affinché l'utente possa consultarle.

Questa funzionalità è il frutto di una serie di funzioni sottostanti, identificate e suddivise per meglio descrivere le operazioni effettuate dal back-end$_{\scaleto{G}{3pt}}$.
Le illustriamo nella sezione seguente.

\subsection{Sotto-funzioni della rappresentazione della heat map}\label{DescrizioneGeneraleFunzionalitàDelProdottoSottoFunzioniDellaRappresentazioneDellaHeatmap}

La descrizione delle sotto-funzioni della rappresentazione della \textit{heat map$_G$} è stata inserita in quanto ritenuta necessaria per fornire un ulteriore approfondimento riguardo tale macro-funzionalità.
Queste funzioni sono raggruppate seguendo tre funzioni generali individuate:
\begin{itemize}
	\item \textbf{Acquisizione di dati:} l'acquisizione avverrà attraverso sistemi di monitoraggio e motori software "contapersone" applicati ad immagini/stream delle videocamere o ad altre sorgenti; i dati ottenuti verranno quindi trattati con Apache Kafka$_G$ e inseriti nel database;
	\item \textbf{Elaborazione di dati:} i dati verranno elaborati per generare valore aggiunto agli stessi e confrontare i differenti flussi di informazioni;
	\item \textbf{Rappresentazione di dati:} attraverso un sito web i dati elaborati verranno visualizzati a video mediante una \textit{heat map$_{\scaleto{G}{3pt}}$}.
\end{itemize}

\subsection{Funzione di acquisizione di dati}\label{DescrizioneGeneraleFunzionalitàDelProdottoFunzioneDiAcquisizioneDiDati}
L'acquisizione dei dati avviene tramite sistemi di monitoraggio e motori software "contapersone" applicati ad immagini e/o stream, provenienti delle videocamere o ad altre sorgenti. Ne segue lo streaming di tali dati con Apache Kafka$_{\scaleto{G}{3pt}}$ e il successivo inserimento nel database.

\subsubsection{Funzione di conteggio persone}\label{DescrizioneGeneraleFunzionalitàDelProdottoFunzioneDiAcquisizioneDiDatiFunzioneDiConteggioPersone}
\begin{itemize}
\item \textbf{Linguaggio di programmazione}: Python$_{\scaleto{G}{3pt}}$/C.
\item \textbf{Input}: i dati forniti sono prelevati da siti con live-feed$_G$ di webcam pubbliche e/o simulatori di spostamenti di persone.
\item \textbf{Output}: il numero delle persone presenti in uno stream/immagine ad un preciso istante.
%\item \textbf{Risposta ad errori}: nel caso di mancanza di risposta dai siti con live-feed il programma si bloccherà ed invierà un segnale di errore al server, con conseguente messaggio di errore visualizzabile dall'utente.%
\end{itemize}

\subsubsection{Funzione di streaming dati con Apache Kafka}\label{DescrizioneGeneraleFunzionalitàDelProdottoFunzioneDiAcquisizioneDiDatiFunzioneDiStreamingDatiConApacheKafka}

\begin{itemize}
	\item \textbf{Descrizione}: impostazione di una piattaforma di data streaming$_G$ che consente di gestire e trasferire grandi volumi di dati in tempo reale, abbassando notevolmente i tempi di latenza;
	\item \textbf{Input}: flussi di dati dall'acquisizione con Java$_{\scaleto{G}{3pt}}$;
		\item \textbf{Output}: il flusso di dati rimane immutato.
\end{itemize}

\subsubsection{Funzione di inserimento dati nel Database}\label{DescrizioneGeneraleFunzionalitàDelProdottoFunzioneDiAcquisizioneDiDatiFunzioneDiInserimentoDatiNelDatabase}

\begin{itemize}
	\item \textbf{Descrizione}: creazione del database e archiviazione dei dati in esso per visualizzazione future e mantenimento dei dati;
	\item \textbf{Struttura}: NoSQL.
\end{itemize}


\subsection{Funzione di Elaborazione Dati}\label{DescrizioneGeneraleFunzionalitàDelProdottoFunzioneDiElaborazioneDati}
Completata la funzione precedente i dati verranno elaborati attraverso librerie di Scikit-learn e TensorFlow con il linguaggio di programmazione Python$_G$.
Di seguito vengono individuate le funzioni da seguire per l'elaborazione dei dati.

\subsubsection{Funzione di Esplorazione Dati}\label{DescrizioneGeneraleFunzionalitàDelProdottoFunzioneDiElaborazioneDatiFunzioneDiEsplorazioneDati}

\begin{itemize}
	\item \textbf{Descrizione}: si discriminano elementi all'interno del dataset$_{\scaleto{G}{3pt}}$ che portano a predizioni errate del modello.
	\item \textbf{Input}: i dati vengono prelevati dal database.
	\item \textbf{Output}: i dati controllati vengono aggiunti in appositi spazi per individuare la loro correttezza.
	\item \textbf{Processo}: si controlla se c'è presenza di valori mancanti, dataset$_{\scaleto{G}{3pt}}$ non bilanciati, outliers$_G$, livello di rumore dei dati e correlazione dei dati.
\end{itemize}

\subsubsection{Funzione di Preprocessing}\label{DescrizioneGeneraleFunzionalitàDelProdottoFunzioneDiElaborazioneDatiFunzioneDiPreprocessing}

\begin{itemize}
	\item \textbf{Descrizione}: preparazione dei dati grezzi per renderli adatti ad un modello di Machine Learning$_G$.
	\item \textbf{Input}: i dati controllati.
	\item \textbf{Output}: dati pronti per l'elaborazione nel modello Machine Learning$_{\scaleto{G}{3pt}}$.
	\item \textbf{Processo}: \begin{enumerate}[leftmargin = 2cm]
		\item Cleaning: eliminazione o correzione di dati con valori invalidi o corrotti.
		\item Trasformazione dei dati: i dati vengono normalizzati, discretizzati, aggregati, si calcolano nuove variabili etc.
		\item Feature extraction: si ricavano, attraverso i dati trasformati, i valori derivati, i quali sono più informativi e non ridondanti, facilitano le funzioni successive di apprendimento e generalizzazione.
		\item Filtraggio dei dati: eliminazione di dati ridondanti e irrilevanti al training del modello attraverso l'applicazione di appositi filtri.
		\item Train / Test set splitting: si dividono i dati in due gruppi uno per il training e uno per il testing.
	\end{enumerate}

\end{itemize}

\subsubsection{Funzione di predizione}\label{DescrizioneGeneraleFunzionalitàDelProdottoFunzioneDiElaborazioneDatiFunzioneDiPredizione}

\begin{itemize}
	\item \textbf{Descrizione}: in questa funzione si effettua una scelta sull'algoritmo più adeguato da utilizzare per il training di dati.
	\item \textbf{Input}: dati ottenuti dalla funzione di preprocessing per il training.
	\item \textbf{Output}: modello di Machine Learning$_{\scaleto{G}{3pt}}$ allenato sui dati di input.
	\item \textbf{Tipi di algoritmi}: si dividono per classificazione e regressione.%???non so se va bene???
\end{itemize}

\subsubsection{Funzione di Valutazione e validazione}\label{DescrizioneGeneraleFunzionalitàDelProdottoFunzioneDiElaborazioneDatiFunzioneDiValutazioneEValidazione}

\begin{itemize}
	\item \textbf{Descrizione}: attraverso varie metriche si valuta quanto valido è il modello nella predizione dei casi.
	\item \textbf{Input}: risposta del modello Machine Learning$_{\scaleto{G}{3pt}}$ dai dati di test, dati effettivi ricavati dalle sorgenti esterne.
	\item \textbf{Output}: dati che superano la validazione.
\end{itemize}

\subsection{Funzione di Visualizzazione dati}\label{DescrizioneGeneraleFunzionalitàDelProdottoFunzioneDiVisualizzazione}
In questa sezione verranno illustrate le funzioni della parte visiva della web-app.

\subsubsection{Funzione di Prelevamento dati}\label{DescrizioneGeneraleFunzionalitàDelProdottoFunzioneDiVisualizzazioneFunzioneDiPrelevamentoDati}

\begin{itemize}
	\item \textbf{Descrizione}: sviluppo della parte di comunicazione di informazioni tra server/database e front-end$_{\scaleto{G}{3pt}}$.
	\item \textbf{Strumenti}: si utilizzerà Java$_{\scaleto{G}{3pt}}$.
\end{itemize}

\subsubsection{Funzione di rappresentazione tramite web application}\label{DescrizioneGeneraleFunzionalitàDelProdottoFunzioneDiVisualizzazioneFunzioneDiRappresentazioneTramiteWebApplication}

\begin{itemize}
	\item \textbf{Descrizione}: sviluppo di una pagina web semplice ed intuitiva.
	\item \textbf{Strumenti}: si utilizzerà Vue.js, una libreria per framework$_G$ di JavaScript$_G$.
	\item \textbf{Vincolo}: la web app dovrà essere costruita sia desktop che mobile friendly.
	\item \textbf{Struttura}: la pagina sarà principalmente rivolta alla visione della mappa per la visualizzazione di aree a rischio assembramenti.
\end{itemize}


\section{Caratteristiche utente}\label{DescrizioneGeneraleCaratteristicheUtente}
Il progetto è rivolto principalmente ad utenti di tipo amministrativo, cioè i quali devono visualizzare l'intera mappa di una regione per motivi lavorativi. \\
Le conoscenze dell'utente per l'utilizzo del software sono:
\begin{itemize}
	\item Conoscenza base nell'utilizzo del motore di ricerca;
	\item Padronanza nella lettura della \textit{heat map$_{\scaleto{G}{3pt}}$}.
\end{itemize}

%specifica delle conoscenze necessarie per usare l'heat-map/ applicazione

\chapter{Fasi del progetto}\label{fasiProgetto}
In questo capitolo verranno illustrate le fasi del progetto identificate dal capitolato$_{\scaleto{G}{3pt}}$ d'appalto \textit{GDP-Gathering Detection Platform}. Il capitolo viene diviso nelle tre fasi generali del progetto: acquisizione, elaborazione e visualizzazione dei dati. Secondo lo \textbf{IEEE Standard 830-1998} in questo capitolo verranno spiegati tutti i punti da sviluppare. La descrizione delle fasi è stata inserita in quanto ritenuta necessaria per il chiarimento della necessità dei requisiti individuati. %inserire descrizione requisiti da ISOIEE

\section{FC1: Acquisizione dati}\label{fasiProgettoAquisizioneDati}%FC fase capitolato
In questa sezione vengono descritte le fasi di acquisizione dei dati.

\subsection{FC1.1: Acquisizione con Java}\label{fasiProgettoAquisizioneDatiJava}

\begin{itemize}
	\item \textbf{Descrizione}: attraverso il linguaggio Java$_G$ si creerà un programma che preleva informazioni da sorgenti esterne e le invia al server.
	\item \textbf{Linguaggio di programmazione}: Java$_{\scaleto{G}{3pt}}$.
	\item \textbf{Input}: i dati forniti saranno prelevati da siti con live-feed$_G$ di webcam di varie città e simulatori di spostamenti di persone.
	\item \textbf{Output}: i dati resteranno immutati.
	\item \textbf{Risposta ad errori}: nel caso di mancanza di risposta dai siti con live-feed il programma si bloccherà ed invierà un segnale di errore al server.
\end{itemize}

%sono dubbioso su 1.2 e 1.3 dovrò modificare qualcosa

\subsection{FC1.2: Database}\label{fasiProgettoAquisizioneDatiDatabase}

\begin{itemize}
	\item \textbf{Descrizione}: creazione del database e archiviazione dei dati in esso per visualizzazione future e mantenimento dei dati;
	\item \textbf{Linguaggio}: NoSQL.
\end{itemize}

\subsection{FC1.3: Apache Kafka$_G$}\label{fasiProgettoAquisizioneDatiApacheKafka}

\begin{itemize}
	\item \textbf{Descrizione}: impostazione di una piattaforma di data streaming$_G$ che consente di gestire e trasferire grandi volumi di dati in tempo reale, abbassando notevolmente i tempi di latenza;
	\item \textbf{Input}: flussi di dati dall'acquisizione con Java$_{\scaleto{G}{3pt}}$;
		\item \textbf{Output}: il flusso di dati rimane immutato.
\end{itemize}

%??? I dati verranno inseriti all'interno di un database, questo sarà sviluppato usando Apache Kafka un sistema distribuito che consiste di server e client i quali comunicano tra loro attraverso un protocollo di rete performante di tipo TCP. ???(non so dove inserire)

\section{FC2: Elaborazione Dati}\label{fasiProgettoElaborazioneDati}
Completata la fase precedente i dati verranno elaborati attraverso librerie di Scikit-learn e TensorFlow con il linguaggio di programmazione Python$_G$.
Di seguito vengono individuate le fasi da seguire per l'elaborazione dei dati.

\subsection{FC2.1: Esplorazione Dati}\label{fasiProgettoElaborazioneDatiEsplorazioneDati}

\begin{itemize}
	\item \textbf{Descrizione}: si discriminano elementi all'interno del dataset che portano a predizioni errate del modello.
	\item \textbf{Input}: i dati vengono prelevati dal database.
	\item \textbf{Output}: i dati controllati vengono aggiunti in appositi spazi per individuare la loro correttezza.
	\item \textbf{Processo}: si controlla se c'è presenza di valori mancanti, dataset non bilanciati, outliers$_G$, livello di rumore dei dati e correlazione dei dati.
\end{itemize}

\subsection{FC2.2: Preprocessing}\label{fasiProgettoElaborazioneDatiPreprocessing}

\begin{itemize}
	\item \textbf{Descrizione}: preparazione dei dati grezzi per renderli adatti ad un modello di Machine Learning$_G$.
	\item \textbf{Input}: i dati controllati.
	\item \textbf{Output}: dati pronti per l'elaborazione nel modello Machine Learning$_{\scaleto{G}{3pt}}$.
	\item \textbf{Processo}: \begin{enumerate}[leftmargin = 2cm]
		\item Cleaning: eliminazione o correzione di dati con valori invalidi o corrotti.
		\item Trasformazione dei dati: i dati vengono normalizzati, discretizzati, aggregati, si calcolano nuove variabili etc.
		\item Feature extraction: si ricavano, attraverso i dati trasformati, i valori derivati, i quali sono più informativi e non ridondanti, facilitano le fasi successive di apprendimento e generalizzazione.
		\item Filtraggio dei dati: eliminazione di dati ridondanti e irrilevanti al training del modello attraverso l'applicazione di appositi filtri.
		\item Train / Test set splitting: si dividono i dati in due gruppi uno per il training e uno per il testing.
	\end{enumerate}

\end{itemize}

\subsection{FC2.3: Caso predizione}\label{fasiProgettoElaborazioneDatiCasoPredizione}

\begin{itemize}
	\item \textbf{Descrizione}: in questa fase si effettua una scelta sull'algoritmo più adeguato da utilizzare per il training di dati.
	\item \textbf{Input}: dati controllati nella fase di preprocessing per il training.
	\item \textbf{Output}: modello di Machine Learning$_{\scaleto{G}{3pt}}$ allenato sui dati di input.
	\item \textbf{Tipi di algoritmi}: si dividono per classificazione e regressione.%???non so se va bene???
\end{itemize}

\subsubsection{FC2.4: Valutazioni e validazione}\label{fasiProgettoElaborazioneDatiValutazioniValidazione}

\begin{itemize}
	\item \textbf{Descrizione}: attraverso varie metriche si valuta quanto valido è il modello nella predizione dei casi.
	\item \textbf{Input}: risposta del modello Machine Learning$_{\scaleto{G}{3pt}}$ dai dati di test, dati effettivi ricavati dalle sorgenti esterne.
	\item \textbf{Output}: dati che superano la validazione.
\end{itemize}

\section{FC3: Visualizzazione dati}\label{fasiProgettoVisualizzazioneDati}
In questa sezione verranno illustrate le fasi di sviluppo della parte visiva della web-app.

\subsection{FC3.1: Front-end$_G$}\label{fasiProgettoVisualizzazioneDatiFrontEnd}

\begin{itemize}
	\item \textbf{Descrizione}: sviluppo di una pagina web semplice ed intuitiva.
	\item \textbf{Strumenti}: si utilizzerà Angular$_G$ e Spring$_G$, due librerie per framework$_G$ di JavaScript$_G$.
	\item \textbf{Vincolo}: la web app dovrà essere costruita sia desktop che mobile friendly.
	\item \textbf{Struttura}: la pagina sarà principalmente rivolta alla visione della mappa per la visualizzazione di aree a rischio assembramenti.
\end{itemize}

\subsection{FC3.2: Back-end$_G$}\label{fasiProgettoVisualizzazioneDatiBackEnd}

\begin{itemize}
	\item \textbf{Descrizione}: sviluppo della parte di comunicazione di informazioni tra server/database e front-end$_{\scaleto{G}{3pt}}$.
	\item \textbf{Strumenti}: si utilizzerà Java$_{\scaleto{G}{3pt}}$.
\end{itemize}


%scriverla più generale è un compito non so come uscirà la pagina
%aggiungere parte caratteristiche date dal capitolato c3 sotto ogni parte ad esempio parte acquisizione dati posso scrivere come requisito il software contapersone etc

\chapter{Requisiti}\label{Requisiti}
In questa sezione vengono illustrati attraverso una tabella tutti i requisiti$_{\scaleto{G}{3pt}}$ individuati dal proponente$_{\scaleto{G}{3pt}}$ e dal gruppo \textit{Jawa Druids}. Ogni requisito viene individuato da un codice identificativo, una sua descrizione, la tipologia di requisito e la fonte di riferimento, la spiegazione di ogni parte è descritta nel documento \textit{Norme del Progetto v1.0.0}. Nella sezione successiva viene illustrato attraverso una tabella il tracciamento dei requisiti alla loro fonte e viceversa.
\section{Requisiti funzionali}\label{RequisitiFunzionali}

\def\tabularxcolumn#1{m{#1}}
{\rowcolors{2}{RawSienna!90!RawSienna!20}{RawSienna!70!RawSienna!40}

	\begin{center}
		\renewcommand{\arraystretch}{1.4}
		\begin{longtable}{|p{3cm}|p{4cm}|p{4cm}|p{4cm}|}
			\hline
			\rowcolor{airforceblue}
			\makecell[c]{\textbf{Codice RS}} & \makecell[c]{\textbf{Descrizione}} & \makecell[c]{\textbf{Tipo di requisito}} & \makecell[c]{\textbf{Fonte}} \\
			%fase 1
			\hline
			\centering RSFO1 & Realizzazione di motori software ‘contapersone’  &\centering  Obbligatorio & \makecell[tc]{Capitolato$_{\scaleto{G}{3pt}}$ \\ V. esterno 17-12-2020 \\ FC1.1} \\
			% \shortstack{Capitolato\\Verbale esterno 17-12-2020}   \\
			\hline
			\centering RSFF2 & Realizzazione di simulatori di altre sorgenti dati sia dei dati storici/in monitoraggio che dati previsionali & \centering Facoltativo & \makecell[tc]{Capitolato$_{\scaleto{G}{3pt}}$ \\ FC1.1} \\
			\hline
			\centering RSFO3  & Il sistema deve visualizzare un messaggio d'errore se il flusso di dati esterno viene a mancare  &\centering  Obbligatorio & \makecell[tc]{Interno\\FC1.1}  \\
			\hline
			\centering RSFO4.1 & Archiviazione di tutti i dati acquisiti nel database & \centering Obbligatorio & \makecell[tc]{Capitolato$_{\scaleto{G}{3pt}}$ \\ FC1.2}  \\
			\hline
			\centering RSFO4.2 & Archiviazione di tutti i dati elaborati nel database & \centering Obbligatorio & \makecell[tc]{Capitolato$_{\scaleto{G}{3pt}}$ \\ FC1.2}  \\
			%fase 2
			\hline
			\centering RSFO5 & Elaborazione in tempo reale dei dati acquisiti da flussi esterni &\centering  Obbligatorio & \makecell[tc]{Capitolato$_{\scaleto{G}{3pt}}$}  \\
			\hline
			\centering RSFO5.1 & Identificazione di eventi che portano alla variazione del flusso di utenti &\centering  Obbligatorio & \makecell[tc]{Capitolato$_{\scaleto{G}{3pt}}$}  \\
			\hline
			\centering RSFD6 & Previsione dell'insorgenza futura di variazioni significative di flussi di persone & \centering Desiderabile & \makecell[tc]{Capitolato$_{\scaleto{G}{3pt}}$ \\ FC2}  \\
			%fase 3
			\hline
			\centering RSFO7 & Visualizzazione dei dati elaborati attraverso heat map$_{\scaleto{G}{3pt}}$ &\centering  Obbligatorio & \makecell[tc]{Capitolato$_{\scaleto{G}{3pt}}$ \\ FC3.1}  \\
			\hline
			\centering RSFO8 & Apache Kafka$_{\scaleto{G}{3pt}}$ deve poter comunicare con il database, l'applicazione web e il modello di Machine Learning$_{\scaleto{G}{3pt}}$  &\centering  Obbligatorio &  \makecell[tc]{Interno \\ FC1.3} 	\\
			\hline

			\rowcolor{white}

			\caption[Requisiti funzionali]{Requisiti funzionali}\label{4.1}\\
	\end{longtable}%\captionof{table}{Requisiti funzionali}

\end{center}

\section{Requisiti prestazionali}\label{RequisitiPrestazionali}
\def\tabularxcolumn#1{m{#1}}
{\rowcolors{2}{RawSienna!90!RawSienna!20}{RawSienna!70!RawSienna!40}

	\begin{center}
		\renewcommand{\arraystretch}{1.4}
		\begin{longtable}{|p{4cm}|p{4cm}|p{4cm}|p{3cm}|}
		\hline
		\rowcolor{airforceblue}
		\makecell[c]{\textbf{Codice RS}} & \makecell[c]{\textbf{Descrizione}} & \makecell[c]{\textbf{Tipo di requisito}} & \makecell[c]{\textbf{Fonte}} \\
		\hline
		\centering RSPO1 & Capacità di acquisizione continuativa nel tempo dei dati da flussi esterni &\centering  Obbligatorio & \makecell[tc]{Capitolato$_{\scaleto{G}{3pt}}$}  \\
		\hline
		\centering RSPO2 & Modalità a bassa latenza dell'aquisizione di informazioni & \centering Obbligatorio & \makecell[tc]{Capitolato$_{\scaleto{G}{3pt}}$} \\
		\hline
		\rowcolor{white}

		\caption[Requisiti prestazionali]{Requisiti prestazionali}\label{4.2}\\
			\end{longtable}
	\end{center}
\section{Requisiti di qualità}\label{RequisitiDiQualita}
\def\tabularxcolumn#1{m{#1}}
{\rowcolors{2}{RawSienna!90!RawSienna!20}{RawSienna!70!RawSienna!40}

	\begin{center}
		\renewcommand{\arraystretch}{1.4}
		\begin{longtable}{|p{4cm}|p{4cm}|p{4cm}|p{3cm}|}
			\hline
			\rowcolor{airforceblue}
			\makecell[c]{\textbf{Codice RS}} & \makecell[c]{\textbf{Descrizione}} & \makecell[c]{\textbf{Tipo di requisito}} & \makecell[c]{\textbf{Fonte}} \\
			\hline
		\centering RSQO1  & La progettazione e la codifica dei requisiti devono rispettare le norme e le metriche definite nel documento \textit{Norme di Progetto v1.0.0}&\centering  Obbligatorio & \makecell[tc]{Interno} \\
		\hline
		\centering RSQF2  & Il codice sorgente del software deve essere disponibile in una repository$_G$ pubblica su Github$_G$  &\centering  Facoltativo & \makecell[tc]{Interno} \\
		\hline
		\centering RSQF3  & Deve essere sviluppato e fornito un documento con lo schema della base di dati relazionale  & \centering Facoltativo & \makecell[tc]{Interno \\ FC1.2} \\
	%	\hline
	%	RSQ  & Fornire una documentazione sui flussi di dati esterni.... & Facoltativo & FC3 \\
	%	\hline
		\hline
		\centering RSQF4  & Deve essere realizzato un documento contenente tutti gli errori risolti durante la realizzazione del software &\centering  Facoltativo & \makecell[tc]{Interno} \\
		\hline
		\centering RSQO5  & Test che dimostrino il corretto funzionamento dei servizi e delle funzionalità previste  & \centering Obbligatorio & \makecell[tc]{Capitolato$_{\scaleto{G}{3pt}}$} \\
		\hline
		\rowcolor{white}

		\caption[Requisiti di qualità]{Requisiti di qualità}\label{4.3}\\
		\end{longtable}
\end{center}

\newpage
\section{Requisiti di vincolo}\label{RequisitiVincolo}
\def\tabularxcolumn#1{m{#1}}
{\rowcolors{2}{RawSienna!90!RawSienna!20}{RawSienna!70!RawSienna!40}

\begin{center}
	\renewcommand{\arraystretch}{1.4}
	\begin{longtable}{|p{4cm}|p{4cm}|p{4cm}|p{3cm}|}
		\hline
		\rowcolor{airforceblue}
		\makecell[c]{\textbf{Codice RS}} & \makecell[c]{\textbf{Descrizione}} & \makecell[c]{\textbf{Tipo di requisito}} & \makecell[c]{\textbf{Fonte}} \\
		\hline
		\centering RSVO1  & I dati acquisiti da telecamere in tempo reale devono avere data di riferimento associato  &\centering Obbligatorio & \makecell[tc]{Interno} \\
		\hline
		\centering RSVO1.1  & I dati acquisiti da telecamere in tempo reale devono avere un orario di riferimento associato &\centering Obbligatorio & \makecell[tc]{Interno} \\
		\hline
		\centering RSVO1.2  & I dati acquisiti da telecamere in tempo reale devono avere un luogo di riferimento associato &\centering Obbligatorio  & \makecell[tc]{Interno} \\
		\hline
		\centering RSVO2  & Il front-end$_{\scaleto{G}{3pt}}$ del prodotto viene sviluppato utilizzando tecnologie web &\centering Obbligatorio  & \makecell[tc]{Capitolato$_{\scaleto{G}{3pt}}$\\FC3.1} \\
		\hline
		\centering RSVF2.1  & Utilizzo di leaflet.js$_{\scaleto{G}{3pt}}$ per la creazione di heat map$_{\scaleto{G}{3pt}}$ &\centering  Facoltativo & \makecell[tc]{Capitolato$_{\scaleto{G}{3pt}}$\\FC3.1} \\
		\hline
		\centering RSVO2.2  & Utilizzo di angular.js$_{\scaleto{G}{3pt}}$ per la creazione della wep-app$_{\scaleto{G}{3pt}}$  &\centering  Obbligatorio  & \makecell[tc]{Capitolato$_{\scaleto{G}{3pt}}$\\FC3.1} \\
		\hline
		\centering RSVO3  & Il sistema deve far uso dell'ecosistema Apache Kafka$_{\scaleto{G}{3pt}}$ &\centering  Obbligatorio  & \makecell[tc]{Capitolato\\FC1.3} \\
		\hline
		\centering RSVO4  & Il back end$_{\scaleto{G}{3pt}}$ del prodotto viene sviluppato utilizzando il linguaggio Java$_{\scaleto{G}{3pt}}$ &\centering  Obbligatorio  & \makecell[tc]{Capitolato$_{\scaleto{G}{3pt}}$\\FC3.2} \\
		\hline
		\rowcolor{white}

		\caption[Requisiti di vincolo]{Requisiti di vincolo}\label{4.4}\\
	\end{longtable}
\end{center}

\section{Tracciamento dei requisiti}\label{RequisitiTracciamentoDeiRequisiti}

\subsection{Requisito - fonte}\label{RequisitiTracciamentoDeiRequisitiFonte}

\def\tabularxcolumn#1{m{#1}}
{\rowcolors{2}{RawSienna!90!RawSienna!20}{RawSienna!70!RawSienna!40}
	\begin{center}
		\renewcommand{\arraystretch}{1.4}
		\begin{longtable}{|p{7.5cm}|p{7.5cm}|}
		\hline
		\rowcolor{airforceblue}
		\makecell[tc]{\textbf{Codice RS}} & \makecell[c]{\textbf{Fonte}}  \\
		\hline
		\makecell[tc]{RSFO1} & \makecell[tc]{Capitolato$_{\scaleto{G}{3pt}}$\\V. esterno 17-12-2020 \\ FC1.1} \\
		\hline
		\makecell[tc]{RSFF2} & \makecell[tc]{Capitolato$_{\scaleto{G}{3pt}}$ \\ FC1.1}\\
		\hline
		\makecell[tc]{RSFO3} & \makecell[tc]{Interno\\FC1.1}\\
		\hline
		\makecell[tc]{RSFO4.1} & \makecell[tc]{Capitolato$_{\scaleto{G}{3pt}}$\\FC1.2}\\
		\hline
		\makecell[tc]{RSFO4.2} & \makecell[tc]{Capitolato$_{\scaleto{G}{3pt}}$\\FC1.2}\\
		\hline
		\makecell[tc]{RSFO5} & \makecell[tc]{Capitolato$_{\scaleto{G}{3pt}}$}\\
		\hline
		\makecell[tc]{RSFO5.1} & \makecell[tc]{Capitolato$_{\scaleto{G}{3pt}}$}\\
		\hline
		\makecell[tc]{RSFD6 }& \makecell[tc]{Capitolato$_{\scaleto{G}{3pt}}$ \\ FC2}\\
		\hline
		\makecell[tc]{RSFO7} & \makecell[tc]{Capitolato$_{\scaleto{G}{3pt}}$\\FC3.1}\\
		\hline
		\makecell[tc]{RSFO8} & \makecell[tc]{Interno \\FC1.3}\\
		\hline
		\makecell[tc]{RSPO1} & \makecell[tc]{Capitolato$_{\scaleto{G}{3pt}}$}\\
		\hline
		\makecell[tc]{RSPO2} & \makecell[tc]{Capitolato$_{\scaleto{G}{3pt}}$}\\
		\hline
		\makecell[tc]{RSQO1} & \makecell[tc]{Interno}\\
		\hline
		\makecell[tc]{RSQF2} & \makecell[tc]{Interno}\\
		\hline
		\makecell[tc]{RSQF3} & \makecell[tc]{Interno \\FC1.2}\\
		\hline
		\makecell[tc]{RSQF4} & \makecell[tc]{Interno}\\
		\hline
		\makecell[tc]{RSQO5} & \makecell[tc]{Capitolato$_{\scaleto{G}{3pt}}$}\\
		\hline
		\makecell[tc]{RSVO1} & \makecell[tc]{Interno}\\
		\hline
		\makecell[tc]{RSVO1.1} & \makecell[tc]{Interno}\\
		\hline
		\makecell[tc]{RSVO1.2} & \makecell[tc]{Interno}\\
		\hline
		\makecell[tc]{RSVO2} & \makecell[tc]{Capitolato$_{\scaleto{G}{3pt}}$\\FC3.1}\\
		\hline
		\makecell[tc]{RSVF2.1} & \makecell[tc]{Capitolato$_{\scaleto{G}{3pt}}$\\FC3.1}\\
		\hline
		\makecell[tc]{RSVO2.2} & \makecell[tc]{Capitolato$_{\scaleto{G}{3pt}}$\\FC3.1}\\
		\hline
		\makecell[tc]{RSVO3} & \makecell[tc]{Capitolato$_{\scaleto{G}{3pt}}$\\FC1.3}\\
		\hline
		\makecell[tc]{RSVO4} & \makecell[tc]{Capitolato$_{\scaleto{G}{3pt}}$\\FC3.2}\\
		\hline
		\rowcolor{white}

		\caption[Tabella tracciamento requisito-fonte]{Tabella tracciamento requisito-fonte}\label{4.5}\\
	\end{longtable}
\end{center}
\clearpage
\subsection{Fonte - requisito}\label{RequisitiTracciamentoDeiRequisitiFonteRequisito}
\def\tabularxcolumn#1{m{#1}}
{\rowcolors{2}{RawSienna!90!RawSienna!20}{RawSienna!70!RawSienna!40}
	\begin{center}
		\renewcommand{\arraystretch}{1.4}
		\begin{longtable}{|p{7.5cm}|p{7.5cm}|}
		\hline
		\rowcolor{airforceblue}
		\makecell[c]{\textbf{Fonte}} & \makecell[c]{\textbf{Codice RS}}  \\
		\hline
		\makecell[c]{Capitolato$_{\scaleto{G}{3pt}}$} & \makecell[c]{RSFO1\\RSFF2\\RSFO4.1\\RSFO4.2\\RSFO5\\RSFO5.1\\RSFD6\\RSFO7\\RSPO1\\RSPO2\\RSQO5\\RSVO2\\RSVF2.1\\RSVO2.2\\RSVO3} \\
		\hline
		\makecell[c]{FC1.1} & \makecell[c]{RSFO1 \\ RSFF2 \\ RSFO3} \\
		\hline
		\makecell[c]{FC1.2} & \makecell[c]{RSFO4.1 \\ RSFO4.2 \\ RSQF3} \\
		\hline
		\makecell[c]{FC1.3} & \makecell[c]{RSFO8 \\ RSVO3} \\
		\hline
		\makecell[c]{FC2} & \makecell[c]{RSFD6} \\
		\hline
		\makecell[c]{FC3.1} & \makecell[c]{RSFO7 \\ RSVO2\\RSVF2.1\\RSVO2.2} \\
		\hline
		\makecell[c]{FC3.2} & \makecell[c]{RSVO4} \\
		\hline
		\makecell[c]{Interno} &\makecell[c]{RSFO3\\RSFO8\\RSQO1\\RSQF2\\RSQF3\\RSQF4\\RSVO1\\RSVO1.1\\RSVO1.2} \\
		\hline
		\makecell[c]{Verbale esterno 17-12-2020} & \makecell[c]{RSFO1} \\
		\hline
		\rowcolor{white}

		\caption[Tabella tracciamento fonte-requisito]{Tabella tracciamento fonte-requisito}\label{4.6}\\
	\end{longtable}
\end{center}

\section{Considerazioni}\label{requisitiConsiderazioni}
I requisiti potranno subire delle variazioni in futuro, in modo tale da apportare degli aggiornamenti alle voci presenti o delle migliorie.
Nel caso in cui le attività pianificate terminassero prima del previsto e dovessero avanzare delle ore di lavoro, potranno essere presi in carico nuovi requisiti per aggiungere del valore al prodotto. Pertanto, qualsiasi espansione è riservata solo per il futuro.

%PER RENDERE PIÙ CHIARA LA STESURA DEI DOCUMENTI È MEGLIO LASCIARE SEPARATI IN FILE DIVERSI OGNI CAPITOLO

% \input{esempio} -- esempio di codice per inserire un nuovo capitolo

\end{document}
