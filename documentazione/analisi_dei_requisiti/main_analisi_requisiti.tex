%COMANDO CHE DETERMINA IL TIPO DI DOCUMENTO CHE SI VUOLE CREARE

\documentclass[a4paper,12pt]{report}

%ELENCO DEI PACCHETTI UTILI PER LA STESURA DEL DOCUMENTO

\usepackage[utf8]{inputenc}
\usepackage[english, italian]{babel}
\usepackage{graphicx}
\usepackage{float}
\usepackage{tabularx}
\usepackage{makecell}
\usepackage{titlesec}
\usepackage{fancyhdr}
\usepackage{lastpage}
\usepackage{xurl}
\usepackage{hyperref}
\usepackage{geometry}
\usepackage[table,dvipsnames]{xcolor}
\setcounter{tocdepth}{5}
\setcounter{secnumdepth}{5}
\usepackage{caption}
\usepackage{etoolbox}% >= v2.1 2011-01-03
\usepackage{tikz}
\usepackage{listings}

%COMANDO PER AVERE IL CAPITOLO CON IL NOME CHE VOGLIAMO NOI

\titleformat{\chapter}[display]
{\normalfont\bfseries}{}{0pt}{\LARGE}

%COMANDO PER LA SPAZIATURA DEI TITOLI DAL BORDO DEL FOGLIO

\titlespacing*{\chapter}{0cm}{0cm}{0.2cm}

%COMANDO PER LA SPAZIATURA DEL TESTO DAI BORDI LATERALI

\geometry{
	left=20mm,
	right=20mm,
}

%COMADNO PER AVERE L'INDICE DEL NOME CHE SI VUOLE

\renewcommand{\contentsname}{Indice}

%COMADNI PER OTTENERE SUBSUBSECTION NUMERATE E PRESENTI NELL'INDICE

\setcounter{tocdepth}{5}
\setcounter{secnumdepth}{5}

%COMANDI PER OTTENERE HEADER E FOOTER

\pagestyle{plain}

\fancypagestyle{plain}{
	\fancyhf{}
	\lhead{\includegraphics[width=3cm]{../immagini/minilogo.jpg}}
	\chead{}
	\rhead{\fontsize{12}{10}Studio di fattibilità}
	\lfoot{}
	\cfoot{\thepage\ di \pageref{LastPage}}
	\rfoot{}
}

\pagestyle{plain}

%COMANDI PER LINK

\hypersetup{
	colorlinks=true,
	linkcolor=black,
	filecolor=black,
	urlcolor=blue,
	citecolor=black,
}

%COMANDI PER SNIPPET DI CODICE

\BeforeBeginEnvironment{lstlisting}{\begin{mdframed}\vspace{-0.7em}}
	\AfterEndEnvironment{lstlisting}{\vspace{-0.5em}\end{mdframed}}

% needed for \lstcapt
\def\ifempty#1{\def\temparg{#1}\ifx\temparg\empty}

% make new caption command for listings
\usepackage{caption}
\newcommand{\lstcapt}[2][]{%
	\ifempty{#1}%
	\captionof{lstlisting}{#2}%
	\else%
	\captionof{lstlisting}[#1]{#2}%
	\fi%
	\vspace{0.75\baselineskip}%
}

\definecolor{atomlightorange}{rgb}{0.88,0.76,0.55}
\definecolor{atomdarkgrey}{RGB}{59,62,75}

% set listings
\lstset{%
	basicstyle=\footnotesize\ttfamily\color{atomlightorange},
	framesep=20pt,
	belowskip=10pt,
	aboveskip=10pt
}

% add frame environment
\usepackage[%
framemethod=tikz,
skipbelow=8pt,
skipabove=13pt
]{mdframed}
\mdfsetup{%
	leftmargin=0pt,
	rightmargin=0pt,
	backgroundcolor=atomdarkgrey,
	middlelinecolor=atomdarkgrey,
	roundcorner=6
}

%BIBLIOGRAFIA

\makeatletter
\def\thebibliography#1{\chapter*{Bibliografia\@mkboth
		{Bibliografia}{Bibliografia}}\list
	{[\arabic{enumi}]}{\settowidth\labelwidth{[#1]}\leftmargin\labelwidth
		\advance\leftmargin\labelsep
		\usecounter{enumi}}
	\def\newblock{\hskip .11em plus .33em minus .07em}
	\sloppy\clubpenalty4000\widowpenalty4000
	\sfcode`\.=1000\relax}
\makeatother

\begin{document}

\makeatletter
\begin{titlepage}
	\begin{center}
		\vspace*{-4,0cm}
		\author{Jawa Druids} 
		\title{Analisi dei Requisiti}
		\date{} %LASCIARE QUESTO CAMPO VUOTO, SE LO TOLGO STAMPA LA DATA CORRENTE
		\includegraphics[width=0.7\linewidth]{../immagini/DRUIDSLOGO.jpg}\\[4ex]
		{\huge \bfseries  \@title }\\[2ex] 
		{\LARGE  \@author}\\[50ex]
		\vspace*{-8,0cm}
		\begin{table}[H]
			\centering
			\begin{tabular}{r | l}
				\textbf{Versione} & x.x.x \\%RIGA PER INSERIRE LA VERSIONE ULTIMA DEL DOCUMENTO
				\textbf{Data approvazione} & xx-xx-xxxx\\
				\textbf{Responsabile} & Nome Cognome\\
				\textbf{Redattori} & Nome Cognome \\
				\textbf{Verificatori} & \makecell[tl]{Nome Cognome \\ Nome Cognome} \\
				%MAKECELL SERVE PER POI ANDARE A CAPO ALL'INTERNO DELLA CELLA
				\textbf{Stato} & Stato\\
				\textbf{Lista distribuzione} & \makecell[tl]{Jawa Druids \\ Nome Professori \\ Sync Lab}\\
				\textbf{Uso} & Uso del documento            
			\end{tabular}
		\end{table}
		\vspace{0.4cm}
		\hfill \break
		\fontsize{17}{10}\textbf{Sommario} \\
		\vspace{0.1cm}
		L'\emph{Analisi dei Requisiti} individua tutti i requisiti da implementare nel prodotto dal sviluppare.
	\end{center}
\end{titlepage}
\makeatother

\def\myformat#1{
	\centering\huge#1
}
\def\tabularxcolumn#1{m{#1}}
\quad
\section*{\myformat{Registro delle modifiche}}

{\rowcolors{2}{Apricot!90!Bittersweet!20}{Bittersweet!70!Apricot!40}
\begin{table}[H]
  \caption[Registro delle Modifiche]{}
	\vspace{-1.4cm}
  \begin{center}
    \begin{tabularx}{\textwidth}{|Y|c|c|c|c|}
    	\hline
    	\rowcolor{Melon}
    	\textbf{Modifica} & \textbf{Autore} & \textbf{Ruolo} & \textbf{Data} & \textbf{Versione}\\
      \hline
      \textit{Aggiunte sezioni §1.2, §1.3} & Andrea Dorigo & \textit{Responsabile} & 2-12-2020 & v0.0.2 \\
    	\hline
    	\textit{Aggiunta sezione §1.1} & Andrea Dorigo & \textit{Responsabile} & 30-11-2020 & v0.0.1 \\
    	\hline
    \end{tabularx}
  \end{center}
\end{table}

\chapter{Introduzione}

\section{Scopo del documento}
Lo scopo del documento è quello di formalizzare i contenuti e le qualità che il prodotto sviluppato dovrà raggiungere. 
I requisiti sono stati individuati attraverso lo studio del capitolato e incontri con l'azienda proponente \textit{Sync Lab}. 
Il documento inoltre è necessario a:
\begin{itemize}
	\item descrivere accuratamente tutti i requisiti proposti dal proponente;
	\item comprendere da parte del committente quali sono le richieste del cliente;
	\item definire il formato e contenuto di ogni requisito specifico del software.
\end{itemize} 
\section{Scopo del prodotto}
In seguito alla pandemia del virus COVID-19 è nata l'esigenza di limitare il più possibile i contatti fra le persone, specialmente evitando la formazione di assembramenti. Il progetto \textit{GDP: Gathering Detection Platform} di \textit{Sync Lab} ha pertanto l'obiettivo di \textbf{creare una piattaforma in grado di rappresentare graficamente le zone potenzialmente a rischio di assembramento, al fine di prevenirlo.}
\\
Al tal fine il gruppo \textit{Jawa Druids} si prefigge di sviluppare un prototipo software in grado di acquisire, monitorare ed analizzare i molteplici dati provenienti dai diversi sistemi e dispositivi, a scopo di identificare i possibili eventi che concorrono all'insorgere di variazioni di flussi di utenti. Il gruppo prevede inoltre lo sviluppo di un'applicazione web da interporre fra i dati elaborati e l'utente, per favorirne la consultazione.
\section{Glossario}
All'interno della documentazione viene fornito un \textit{Glossario}, con l'obiettivo di assistere il lettore specificando il significato e contesto d'utilizzo di alcuni termini strettamente tecnici o ambigui, segnalati con una \textit{G} a pedice.

\section{Riferimenti}
\subsection{Riferimenti normativi}
\begin{itemize}
	\item \textit{Norme di Progetto v1.0.0;}
		\item \textit{Verbale Esterno 17-12-2020;}
		\item \textit{Capitolato d'appalto C3:} \\ \url{https://www.math.unipd.it/~tullio/IS-1/2020/Progetto/C3.pdf}
\end{itemize}
\subsection{Riferimenti informativi}
\begin{itemize}
	\item \textit{Presentazione del capitolato:} \\ \url{https://www.math.unipd.it/~tullio/IS-1/2020/Progetto/C3.pdf}
		\item \textit{Materiale didattico relativo all'Analisi dei Requisiti del corso di Ingegneria del Software:}\\ \url{https://www.math.unipd.it/~tullio/IS-1/2020/Dispense/L07.pdf}
	\item \textit{IEEE Recommended Practice for Software Requirements Specifications:}\\
		\url{https://ieeexplore.ieee.org/document/720574}
	\item \textit{Seminario per approfondimenti tecnici del capitolato C3:}\\
		\url{https://www.math.unipd.it/~tullio/IS-1/2020/Progetto/ST1.pdf}	
		%aggiungere capitolato c3 pdf	
\end{itemize}
%COMANDO PER LA CREAZIONE DELL'INDICE

\tableofcontents{}

%PER RENDERE PIÙ CHIARA LA STESURA DEI DOCUMENTI È MEGLIO LASCIARE SEPARATI IN FILE DIVERSI OGNI CAPITOLO

% \input{esempio} -- esempio di codice per inserire un nuovo capitolo

\end{document}
