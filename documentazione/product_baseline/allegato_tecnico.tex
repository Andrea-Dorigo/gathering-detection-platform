%COMANDO CHE DETERMINA IL TIPO DI DOCUMENTO CHE SI VUOLE CREARE

\documentclass[a4paper,12pt]{report}

%ELENCO DEI PACCHETTI UTILI PER LA STESURA DEL DOCUMENTO
\usepackage{enumitem}
\usepackage[utf8]{inputenc}
\usepackage[english, italian]{babel}
\usepackage{graphicx}
\usepackage{float}
\usepackage{tabularx}
\usepackage{makecell}
\usepackage{titlesec}
\usepackage{fancyhdr}
\usepackage{lastpage}
\usepackage{xurl}
\usepackage{hyperref}
\usepackage{geometry}
\usepackage[table,dvipsnames]{xcolor}
\setcounter{tocdepth}{5}
\setcounter{secnumdepth}{5}
\usepackage{caption}
\usepackage{etoolbox}% >= v2.1 2011-01-03
\usepackage{tikz}
\usepackage{listings}
\usepackage{scalerel}
\usepackage{longtable}
\captionsetup[table]{position=bottom}
\usepackage{comment}
\input{config}
\begin{document}
	\makeatletter
	\begin{titlepage}
		\begin{center}
			\vspace*{-4,0cm}
			\author{Jawa Druids}
			\title{Allegato Tecnico}
			\date{}
			\includegraphics[width=0.5\linewidth]{../immagini/DRUIDSLOGO.jpg}\\[4ex]

			
			{\huge \bfseries  \@title }\\[2ex] 
			{\LARGE  \@author}\\[50ex]
			\vspace*{-9,0cm}
			\begin{table}[H]
			\renewcommand{\arraystretch}{1.4}
			\centering
			\begin{tabular}{r | l}
				\textbf{Versione} & 1.0.0 \\%RIGA PER INSERIRE LA VERSIONE ULTIMA DEL DOCUMENTO
				\textbf{Data approvazione} & 2021-04-16\\
				\textbf{Responsabile} & Andrea Cecchin \\
				\textbf{Redattori} & \makecell[tl]{Andrea Cecchin \\ Mattia Cocco} \\		
				\textbf{Verificatori} & \makecell[tl]{Emma Roveroni \\ Margherita Mitillo} \\
				%MAKECELL SERVE PER POI ANDARE A CAPO ALL'INTERNO DELLA CELLA
				\textbf{Stato} & Approvato\\
				\textbf{Lista distribuzione} & \makecell[tl]{Jawa Druids \\ Prof. Tullio Vardanega \\ Prof. Riccardo Cardin}\\
				\textbf{Uso} & Esterno     
			\end{tabular}
		\end{table}
		\vspace{0.1cm}
		\hfill \break
		\fontsize{17}{10}\textbf{Sommario} \\
		\vspace{0.1cm}
		Il presente documento contiene le scelte architetturali che il gruppo \emph{\normalsize{\textit{Jawa Druids}}} ha effettuato ai fini realizzativi del progetto. Contiene i design pattern e i diagrammi di attività, sequenza, classi e package.
		\end{center}
	\end{titlepage}
	\makeatother
	
	
	\tableofcontents{}
	\listoftables{}
	\listoffigures{}
	\chapter{Introduzione}\label{Introduzione}

\section{Scopo del documento}\label{IntroduzioneScopoDelDocumento}

Lo scopo di questo documento è fornire all'utente tutte le indicazioni per il corretto uso del software$_G$ da noi prodotto.

\section{Scopo del prodotto}\label{IntroduzioneScopoDelProdotto}

In seguito alla pandemia del virus COVID-19 è nata l'esigenza di limitare il più possibile i
contatti fra le persone, specialmente evitando la formazione di assembramenti. 
Il progetto \textit{GDP: Gathering Detection Platform} di \textit{Sync Lab} ha pertanto l'obiettivo di \textbf{creare una piattaforma in grado di rappresentare graficamente le zone potenzialmente a rischio di assembramento, al fine di prevenirlo.}Il prodotto finale è rivolto specificatamente agli
organi amministrativi delle singole città, cosicché possano gestire al meglio i punti sensibili di
affollamento, come piazze o siti turistici. Lo scopo che il software$_{\scaleto{G}{3pt}}$ intende raggiungere non è
solo quello della rappresentazione grafica real-time ma anche di poter riuscire a prevedere
assembramenti in intervalli futuri di tempo.
\\
A tal fine il gruppo \textit{Jawa Druids} si prefigge di sviluppare un prototipo software$_{\scaleto{G}{3pt}}$ in grado di acquisire, monitorare ed analizzare i molteplici dati provenienti dai diversi sistemi e dispositivi, a scopo di identificare i possibili eventi che concorrono all'insorgere di variazioni di flussi di utenti. Il gruppo prevede inoltre lo sviluppo di un'applicazione web$_G$ da interporre fra i dati elaborati e l'utente, per favorirne la consultazione.

\section{Glossario}\label{IntroduzioneGlossario}

All'interno della documentazione viene fornito un \textit{Glossario}, con l'obiettivo di assistere il lettore specificando il significato e contesto d'utilizzo di alcuni termini strettamente tecnici o ambigui, segnalati con una \textit{G} a pedice.

%\section{Riferimenti}\label{IntroduzioneRiferimenti}

%\subsection{Riferimenti informativi}\label{IntroudioneRiferimentiRiferimentiInformativi}
	\chapter{Architettura del prodotto}\label{ArchitetturaDelProdotto}

\section{Descrizione generale}
In fase di progettazione, il gruppo \textit{Jawa Druids} ha deciso di suddividere la modellazione architetturale di \textit{Gathering-Detection-Platform} in tre distinti moduli, tutti indipendenti tra loro.
Il primo modulo si occupa solamente di leggere, tramite file JSON$_G$, tutte le webcam disponibili per poi effettuare il riconoscimento persone tramite i frame scaricati. Successivamente i dati estrapolati verranno invitati al database.
Il secondo modulo, il machine-learning$_G$, si occupa di recuperare questi dati dal database per lavorarli producendo predizioni per le ore future.
Infine il terzo modulo, la web-app$_G$ vera e propria, si occuperà di rappresentare graficamente i dati all'interno del database mediante una heat-map$_G$ e farli visualizzare all'utente.

\section{Architettura Acquisition}\label{ArchitetturaDelProdottoAcquisition}
L'architettura riguardante il modulo di acquisizione, ovvero il primo modulo del software, è basata sul fatto che è creata sul paradigma della codifica procedurale.
Inoltre non presenta alcuna classe in quanto non crea oggetti, crea esclusivamente un array con i dati che estrapola dai frame e dalle informazioni del tempo.
%Inserire
%Grafici

\section{Architettura Prediction}\label{ArchitetturaDelProdottoPrediction}
L'architettura del modulo del machine-learning si può semplificare ad un modulo unico con all'interno i metodi necessari per prelevare dati dal database per poi reinviarli da lavorati.
Non necessita classi interne in quanto svolge esclusivamente operazioni funzionali
%Inserire
%Grafici


\section{Architettura Web-App}\label{ArchitetturaDelProdottoWebApp}
Per il modulo relativo al front-end$_G$, si è deciso di utilizzare il pattern \textit{Model-View-Controller}(MVC).
Questa scelta è dovuta al fatto che, essendo la web-app sviluppata con spring, il pattern è quello che più si adatta alla tipologia sia di modellazione sia di scopo. 
Lo scambio di dati tra fron-end e back-end avviene attraverso il pattern architetturale REST. Si è deciso questo pattern per avere un oggetto di scambio tra le due parti unico strutturalmente e quindi non è essere vincolato dalla struttura presente nel database.
Questo permette di aumentare la portabilità dell'applicazione web potendo applicare il back-end, possibilmente, a diversi front-end. Le richieste che il front-end effettua al back-end sono HTTP Request GET, quindi sono sempre richieste di visione di informazioni per poi essere utilizzate per aggiornare la parte grafica visibile dal client. 
%Inserire
%Grafici

	\input{capitoli/c4_Requisiti_soddisfatti}
	% bibliography, glossary and index would go here.
	
\end{document}