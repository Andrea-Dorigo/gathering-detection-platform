\chapter{Introduzione}

\section{Scopo del documento}
Lo scopo del documento è quello di elencare e motivare le scelte architetturali fatte dal gruppo Jawa Druids, per quanto riguarda il progetto GDP: Gathering Detection Platform. 
\section{Scopo del prodotto}
In seguito alla pandemia del virus COVID-19 è nata l'esigenza di limitare il più possibile i contatti fra le persone, specialmente evitando la formazione di assembramenti. Il progetto \textit{GDP: Gathering Detection Platform} di \textit{Sync Lab} ha pertanto l'obiettivo di \textbf{creare una piattaforma in grado di rappresentare graficamente le zone potenzialmente a rischio di assembramento, al fine di prevenirlo.}
Il prodotto finale è rivolto specificatamente agli organi amministrativi delle singole città, cosicché possano gestire al meglio i punti sensibili di affollamento, come piazze o siti turistici.
Lo scopo che il software intende raggiungere non è solo quello della rappresentazione grafica real-time ma anche quella di poter riuscire sia di prevedere assembramenti in intervalli futuri di tempo sia di offrire una storicizzazione dei dati passati. \\
Al tal fine il gruppo \textit{Jawa Druids} si prefigge di sviluppare un prototipo software in grado di acquisire, monitorare ed analizzare i molteplici dati provenienti dai diversi sistemi e dispositivi, a scopo di identificare i possibili eventi che concorrono all’insorgere di variazioni di flussi di utenti. Il gruppo prevede inoltre lo sviluppo di un'applicazione web da interporre fra i dati elaborati e l'utente, per favorirne la consultazione.
\section{Glossario}
All'interno della documentazione viene fornito un \textit{Glossario v3.0.0}, con l'obiettivo di assistere il lettore specificando il significato e contesto d'utilizzo di alcuni termini strettamente tecnici o ambigui, segnalati con una \textit{G} a pedice.
\section{Riferimenti}\label{IntroduzioneRiferimenti}
\subsection{Riferimenti normativi}\label{IntroduzioneRiferimentiRiferimentiNormativi}
\begin{itemize}
	\item \textit{Norme di Progetto 3.0.0;}
	\item \textit{Capitolato d'appalto;}
	\url{https://www.math.unipd.it/~tullio/IS-1/2020/Progetto/C3.pdf}
\end{itemize}
\subsection{Riferimenti informativi}\label{IntroduzioneRiferimentiRiferimentiInformativi}
\begin{itemize}
	\item \textit{Dispensa diagrammi delle classi:}\\
	\url{https://www.math.unipd.it/~rcardin/swea/2021/Diagrammi\%20delle\%20Classi_4x4.pdf}
	\item \textit{Dispensa diagrammi dei package:}\\
	\url{https://www.math.unipd.it/~rcardin/swea/2021/Diagrammi\%20dei\%20Package_4x4.pdf}
	\item \textit{Dispensa diagrammi di sequenza:}\\
	\url{https://www.math.unipd.it/~rcardin/swea/2021/Diagrammi\%20di\%20Sequenza_4x4.pdf}
	\item \textit{Dispensa diagrammi di attività:}\\
	\url{https://www.math.unipd.it/~rcardin/swea/2021/Diagrammi\%20di\%20Attivit\%c3\%a0_4x4.pdf}
	\item \textit{Dispensa principi SOLID:}\\
	\url{https://www.math.unipd.it/~rcardin/swea/2021/SOLID\%20Principles\%20of\%20Object-Oriented\%20Design_4x4.pdf}
	\item \textit{Dispensa REST-based API:}\\
	\url{https://www.math.unipd.it/~rcardin/sweb/2021/L03.pdf}	
\end{itemize}