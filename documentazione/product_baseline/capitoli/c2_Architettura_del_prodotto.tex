\chapter{Architettura del prodotto}\label{ArchitetturaDelProdotto}

\section{Descrizione generale}
In fase di progettazione, il gruppo \textit{Jawa Druids} ha deciso di suddividere la modellazione architetturale di \textit{Gathering-Detection-Platform} in tre distinti moduli, tutti indipendenti tra loro.
Il primo modulo si occupa solamente di leggere, tramite file JSON$_G$, tutte le webcam disponibili per poi effettuare il riconoscimento persone tramite i frame scaricati. Successivamente i dati estrapolati verranno invitati al database.
Il secondo modulo, il machine-learning$_G$, si occupa di recuperare questi dati dal database per lavorarli producendo predizioni per le ore future.
Infine il terzo modulo, la web-app$_G$ vera e propria, si occuperà di rappresentare graficamente i dati all'interno del database mediante una heat-map$_G$ e farli visualizzare all'utente.

\section{Architettura Acquisition}\label{ArchitetturaDelProdottoAcquisition}
L'architettura riguardante il modulo di acquisizione, ovvero il primo modulo del software, è molto semplice ed intuitiva.
Non vi è alcuna classe e si basa su una programmazione procedurale, in cui nel \textit{detect.py}, ovvero lo script principale del modulo, vengono richiamate le funzioni, in maniera sequenziale, per manipolare i dati scaricati dalle webcam.
La scelta dell'utilizzo di un paradigma procedurale risiede nel fatto che la creazione di oggetti e il loro utilizzo risultavano, nell'insieme, più complicati mentre chiamando delle semplici funzioni esterne il programma risultava più leggibile e efficiente.
Gli unici oggetti presenti in \textit{detect.py} sono quelli di tipo \textit{data}, necessari per la giusta esecuzione dello script.
%Inserire
%Grafici

\section{Architettura Prediction}\label{ArchitetturaDelProdottoPrediction}
L'architettura del modulo del machine-learning si può semplificare ad un modulo unico con all'interno i metodi necessari per prelevare dati dal database per poi reinviarli da lavorati.
Non necessita classi interne in quanto svolge esclusivamente operazioni funzionali
%Inserire
%Grafici


\section{Architettura Web-App}\label{ArchitetturaDelProdottoWebApp}
Per il modulo relativo al front-end$_G$, si è deciso di utilizzare il pattern \textit{Model-View-Controller}(MVC).
Questa scelta è dovuta al fatto che, essendo la web-app sviluppata con spring, il pattern è quello che più si adatta alla tipologia sia di modellazione sia di scopo.
%Inserire
%Grafici
