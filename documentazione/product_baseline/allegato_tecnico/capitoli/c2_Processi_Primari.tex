\chapter{Processi Primari}
\section{Fornitura}
\subsection{Scopo}
La fornitura secondo lo standard ISO/IEC 12207:1995 descrive tutte le attività e compiti svolte dal fornitore al fine di sviluppare un prodotto soddisfacente e che rispetti appieno le richieste del committente.
Durante questa fase si prevede la compilazione di diversi documenti, i quali verranno poi inviati al committente per guadagnare la possibilità di lavorare al progetto offerto dall'azienda \emph{Sync Lab}.
Il fornitore esegue un'attività di analisi e stesura dello Studio di Fattibilità, questo documento rileva i rischi e le criticità riscontrate riscontrate nella richiesta di appalto.
Si definisce inoltre un accordo contrattuale con il proponente mediante il quale si regolano i rapporti con l'azienda, la consegna e la manutenzione del prodotto sviluppato. 

\subsection{Studio di Fattibilità}
Lo \emph{Studio di Fattibilità} consiste nell'analisi e nella valutazione sistematica delle caratteristiche, dei costi, e dei possibili risultati di un progetto sulla base di una preliminare idea di massima.
A seguire della presentazione dei capitolati d'appalto da parte di ogni proponente avvenuta il 05-11-2020, il \emph{Responsabile di Progetto} si è impegnato a programmare incontri con tutti i componenti del gruppo \emph{JawaDruids} per valutare le scelte di ogni membro e attuare così un primo scambio di idee. Una volta individuato il capitolato d'interesse ogni \emph{Analista} provvederà alla stesura dello \emph{Studio di Fattibilità}, il quale fornirà un'analisi accurata di ogni capitolato.
Nella stesura dello \emph{Studio di Fattibilità} per ogni capitolato si riporterà anche:
\begin{itemize}
	\item informazioni generali: informazioni riguardanti il proponente.
	\item descrizione del capitolato: una sintesi del progetto da sviluppare. 
	\item finalità del progetto: le finalità richieste dal capitolato d'appalto.
	\item tecnologie interessate: le tecnologie che verranno utilizzate nello svolgimento del capitolato
	\item aspetti positivi: aspetti favorevoli alla scelta del capitolato.
	\item criticità e fattori di rischio: problematiche che potrebbero sorgere nello svolgimento del capitolato.
	\item conclusioni: accettazione o rifiuto del capitolato in base alle informazioni illustrate precedentemente e anche all'interesse dimostrato da ogni membro nel gruppo.
\end{itemize}
\subsection{Altra documentazione da fornire}
Oltre allo \emph{Studio di Fattibilità} vengono consegnati altri documenti all'azienda \emph{Sync Lab} ed ai committenti \emph{Prof. Tullio Vardanega}. Questi documenti sono necessari al fine di tracciare le attività di Analisi, Pianificazione, Verifica, Validazione e Controllo di Qualità per assicurare una completa trasparenza durante tutta la durata del ciclo di vita del progetto.
I documenti sono:
\begin{itemize}
	\item Analisi dei Requisiti: identifica e dettaglia in modo completo ed esaustivo i requisiti del sistema descritto nel capitolato che il fornitore si impegna a soddisfare
	\item Piano di Qualifica: illustra la strategia complessiva di verifica e validazione proposta dal fornitore per pervenire al collaudo del sistema con la massima efficienza ed efficacia
	\item Piano di Progetto: presenta l'organigramma dettagliato del fornitore, lo schema proposto per l'assegnazione e la rotazione dei ruoli di progetto, l'impegno complessivo previsto per ogni ruolo e per ogni individuo, l'analisi dei rischi, la pianificazione di massima per la realizzazione del prodotto, e il corrispondente conto economico preventivo
\end{itemize} 
Alla documentazione appena illustrata il gruppo \emph{JawaDruids} allegherà inoltre una lettera di presentazione con la quale si formalizza l'impegno nel portare al termine il capitolato prescelto entro i termini definiti nella lettera e rispettandone i requisiti minimi.
\subsection{Strumenti}
Di seguito sono riportati gli strumenti impiegati dal gruppo durante il progetto per il processo di fornitura.
\subsubsection{Vuoto per ora}


\section{Sviluppo}
\subsection{Scopo}
Il processo di sviluppo contiene tutte le attività che riguardano la produzione del software richiesto dal cliente, in particolare analisi dei requisiti, design, codifica, integrazione, test e installazione.
\subsection{Descrizione}

