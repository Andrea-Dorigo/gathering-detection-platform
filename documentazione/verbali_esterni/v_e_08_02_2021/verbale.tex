\newpage
\section{Informazioni Generali}
\begin{enumerate}
	\item \textbf{Luogo:} \normalfont Zoom;
	\item \textbf{Data:} \normalfont 08-02-2021;
	\item \textbf{Orario inizio:} \normalfont 12.30;
	\item \textbf{Orario fine:} \normalfont 13.00;
	\item \textbf{Partecipanti:}
	\begin{itemize}
		\item Prof. Riccardo Cardin;
		\item Mattia Cocco;
		\item Andrea Dorigo;
		\item Andrea Cecchin;
		\item Roveroni Emma.
	\end{itemize}
	\item \textbf{Assenti:}
	\begin{itemize}
		\item Margherita Mitillo;
		\item Igli Mezini;
		\item Graziano Alfredo;
	\end{itemize}
\end{enumerate}
\section{Ordine del giorno}
Appena i componenti del gruppo si sono connessi alla riunione, l'incontro con il Prof. Riccardo Cardin è iniziato.
Il meeting si è concentrato sui seguenti punti:
\begin{enumerate}
	\item Risoluzione di dubbi riguardo l'inclusione e/o esclusione di alcuni casi d'uso nell'\textit{Analisi dei Requisiti};
	\item Chiarimenti sull'introduzione di un nuovo attore nell'\textit{Analisi dei Requisiti};
\end{enumerate}
\section{Risoluzione dubbi sui casi d'uso}
Il gruppo ha esposto al Prof. Riccardo Cardin alcune incertezze sulla possibile inclusione, all'interno dell'\textit{Analisi dei Requisiti}, di alcuni casi d'uso indivduati. Il Prof. Cardin ha chiarito le idee del gruppo riguardo quali casi d'uso è meglio escludere dal documento sopra citato.
\section{Introduzione di un nuovo attore}
Il Prof. Cardin ha consigliato al gruppo di vedere il front-end sia come parte del sistema che come attore. Ciò comporta l'individuazione di casi d'uso che riguardano l'interazione fra il front end$_{\scaleto{G}{3pt}}$, visto come attore, ed il back end$_{\scaleto{G}{3pt}}$. Il Prof. Cardin ha suggerito, inoltre, una divisione del capitolo riguardante i casi d'uso nel documento \textit{Analisi dei Requisiti}, suddividendolo, quindi, in due paragrafi, ciascuno per ogni attore, utente e front end$_{\scaleto{G}{3pt}}$.
\section{Decisioni derivate dal colloquio}
\begin{itemize}
	\item Introdurre nei casi d'uso la relazione tra front end$_{\scaleto{G}{3pt}}$ e back end$_{\scaleto{G}{3pt}}$ e di conseguenza i loro relativi casi d'uso;
	\item Aggiunta dei casi d'uso tra utente e front end$_{\scaleto{G}{3pt}}$;
	\item Aggiunta dei casi d'uso tra front end$_{\scaleto{G}{3pt}}$ e back end$_{\scaleto{G}{3pt}}$.
\end{itemize}
