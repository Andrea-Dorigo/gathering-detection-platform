\newpage
\section{Informazioni Generali}
\begin{enumerate}
	\item \textbf{Luogo:} \normalfont Discord;
	\item \textbf{Data:} \normalfont 2021-03-25;
	\item \textbf{Orario inizio:} \normalfont 10.00;
	\item \textbf{Orario fine:} \normalfont 11.00;
	\item \textbf{Partecipanti:}
	\begin{itemize}
		\item Dario Stagnitto;
		\item Mattia Cocco;
		\item Andrea Dorigo;
		\item Andrea Cecchin.
	\end{itemize}
	\item \textbf{Assenti:}
	\begin{itemize}
		\item Margherita Mitillo;
		\item Igli Mezini;
		\item Alfredo Graziano;
		\item Emma Roveroni.
	\end{itemize}
\end{enumerate}
\section{Ordine del giorno}
Il colloquio ha trattato gli argomenti riguardanti il modulo di predizione del prodotto da sviluppare. Dario Stagnitto ha esposto le motivazioni sul perché non fosse ideale utilizzare la libreria Keras$_G$.
Le motivazioni principali riguardavano la complessità della libreria e la necessità di una comprensione più elevata del funzionamento del machine learning$_G$.
Inoltre ha spiegato come impostare l'ambiente di sviluppo locale per il machine learning e ci ha introdotto all'utilizzo della libreria Sci-kit Learn$_G$ per le predizioni sui dati.
\section{Decisioni derivate dal colloquio}
\begin{itemize}
	\item \textbf{E\_2021\_03\_25.1}: utilizzo della libreria Sci-kit Learn$_G$ per la predizione dei dati.
	\item \textbf{E\_2021\_03\_25.2}: utilizzare Jupiter$_G$ come interfaccia di sviluppo con ambiente virtuale Anaconda$_G$.
	\item \textbf{E\_2021\_03\_25.3}: utilizzare più tipi di algoritmi per il calcolo delle predizioni e individuare la migliore.
	\item \textbf{E\_2021\_03\_25.4}: utilizzare Pandas$_G$ per modellare i dati.
\end{itemize}