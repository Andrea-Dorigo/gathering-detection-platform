\newpage
\section{Informazioni Generali}
\begin{enumerate}
	\item \textbf{Luogo:} \normalfont Google Meet;
	\item \textbf{Data:} \normalfont 17-12-2020;
	\item \textbf{Orario inizio:} \normalfont 15.00;
	\item \textbf{Orario fine:} \normalfont 16.10;
	\item \textbf{Partecipanti:}
	\begin{itemize}
		\item il Dott. Fabio Pallaro e il Dott.Fabio Scettro come rappresentanti dell'azienda \textit{Sync Lab};
		\item Mattia Cocco; 
		\item Andrea Dorigo;
		\item Margherita Mitillo;
		\item Igli Mezini;
		\item Andrea Cecchin;
		\item Roveroni Emma
		\item Graziano Alfredo.
	\end{itemize}
	\item \textbf{Assenti:}
	\begin{itemize}
		\item Nessuno
	\end{itemize}
\end{enumerate}
\section{Oridne del giorno}
Appena tutti i componenti del gruppo si sono connessi alla riunione, l'incontro con l'azienda proponente è iniziato. Il gruppo ha presentato i seguenti dubbi:
\begin{enumerate}
	\item Chiarimenti sulla scelta fatta dal gruppo per il software legato al Machine Learnig;
	\item Chiarimenti sulla tipologia di utenti richiesta dal proponente e sulla rappresentazione dei dati;
	\item Delucidazioni sulla raccolta dati e la loro elaborazione.
\end{enumerate}
\section{Software legato al Machine Learnig}
Il gruppo ha studiato due software proposti dall'azienda legati al Machine Learning: Tensor Flow e Scikit Learn.
Dopo un attento studio il gruppo ha indicato all'azienda la sua preferenza nell'utilizzare il secondo in quanto più intuitivo per neofiti dell'argomento e quindi con una curva di apprendimento meno ripida rispetto al primo che risulta più facilmente accessibile ad esperti della materia. \\
Riguardo a questo argomento Fabio Pallaro ci ha consigliato di mandare un messaggio a Cristoforo sul canale Discord in quanto più informato sulla materia.
\section{Tipologia di utenti e rappresentazione dei dati}
Il dubbio sorto all'interno del gruppo era legato alla tipologia di utenti che avrebbero utilizzato il prodotto software per poter svolgere una migliore analisi dei vari casi d'uso. Siamo arrivati alla conclusione che l'applicazione che dovremmo produrre è rivolta ad un utente esperto come dipendenti della pubblica amministrazione o del ministero della salute. \\
Puntualizzando questo argomento abbiamo anche chiarito il dubbio legato della rappresentazione dei dati che poteva risultare un problema se l'applicazione fosse indirizzata anche ad utenti meno esperti.
\section{Raccolta dati}
Questo argomento è risultato essere la questione più difficile da risolvere. Abbiamo discusso su quale soluzione sarebbe migliore per raccogliere dati da poter utilizzare in maniera efficiente nel Machine Learning.\\
Le possibili soluzioni sono:
\begin{itemize}
	\item raccogliere dati tramite webcam e programma conta-persone;
	\item integrare i dati presi da webcam e programma conta-persone con dati legati a possibili eventi e simulare dove non fossero sufficienti le informazioni ricavate;
	\item simulare interamente i dati.
\end{itemize}
In accordo con Fabio abbiamo deciso di studiare ulteriormente il problema per trovare la soluzione migliore che ci porti a produrre un codice efficiente lato Machine Learning.