\newpage
\section{Informazioni Generali}
\begin{enumerate}
	\item \textbf{Luogo:} \normalfont Zoom;
	\item \textbf{Data:} \normalfont 02-02-2021;
	\item \textbf{Orario inizio:} \normalfont 15.15;
	\item \textbf{Orario fine:} \normalfont 16.00;
	\item \textbf{Partecipanti:}
	\begin{itemize}
		\item Fabio Pallaro (Sync Lab);
		\item Cristoforo Decaro (Sync Lab);
		\item Mattia Cocco;
		\item Andrea Dorigo;
		\item Andrea Cecchin;
		\item Emma Roveroni;
		\item Margherita Mitillo;
		\item Igli Mezini;
		\item Alfredo Graziano;
	\end{itemize}
	\item \textbf{Assenti:}
	\item Nessuno
\end{enumerate}
\section{Ordine del giorno}
Appena i componenti del gruppo si sono connessi alla riunione, l'incontro con l'azienda \textit{Sync Lab}$_{\scaleto{G}{3pt}}$ è iniziato.
Il meeting si è concentrato sui seguenti punti:
\begin{enumerate}
	\item tecnologie per lo sviluppo del prodotto;
	\item la realizzazione del Proof of Concept$_{\scaleto{G}{3pt}}$ non richiederà l'utilizzo di Apache Kafka$_{\scaleto{G}{3pt}}$;
	\item definizione di requisiti prestazionali.
\end{enumerate}
\section{Tecnologie per lo sviluppo del prodotto}
Durante l'incontro con con il proponente sono state esposte nuove tecnologie per lo sviluppo del prodotto che possano semplificarne la codifica. Il proponente ha consigliato l'uso di:
\begin{itemize}
	\item Keras$_{\scaleto{G}{3pt}}$, una libreria di python$_{\scaleto{G}{3pt}}$ per la realizzazione del modello machine learning$_{\scaleto{G}{3pt}}$;
	\item Pandas$_{\scaleto{G}{3pt}}$, una libreria di python$_{\scaleto{G}{3pt}}$ per manipolazione di dati.
\end{itemize}
\section{Definizione di requisiti prestazionali}
I requisiti prestazionali delineati sono:
\begin{itemize}
	\item il software 'contapersone' ha un indice di affidabilità di almeno 75\%;
	\item l'aggiornamento della mappa con i nuovi dati deve essere effettuato in meno di 30s;
	\item il modello machine learning$_{\scaleto{G}{3pt}}$ elabora almeno un'immagine per la predizione ogni 4 minuti.
\end{itemize}
\section{Decisioni derivate dal colloquio}
\begin{itemize}
	\item \textbf{E\_02-02-2021.1}: le tecnologie proposte in questa riunione verranno confrontate con quelle scelte in precedenza per valutare quale sia la migliore;
	\item \textbf{E\_02-02-2021.2}: introdurre nell'Analisi dei Requisiti$_{\scaleto{G}{3pt}}$ nuovi requisiti prestazionali.
\end{itemize}
