\newpage
\section{Informazioni Generali}
\begin{enumerate}
	\item \textbf{Luogo:} \normalfont Discord;
	\item \textbf{Data:} \normalfont 2021-03-29;
	\item \textbf{Orario inizio:} \normalfont 10.00;
	\item \textbf{Orario fine:} \normalfont 10.40;
	\item \textbf{Partecipanti:}
	\begin{itemize}
		\item Dario Stagnitto;
		\item Mattia Cocco;
		\item Andrea Dorigo;
		\item Andrea Cecchin.
	\end{itemize}
	\item \textbf{Assenti:}
	\begin{itemize}
		\item Margherita Mitillo;
		\item Igli Mezini;
		\item Alfredo Graziano;
		\item Emma Roveroni.
	\end{itemize}
\end{enumerate}
\section{Ordine del giorno}
Il colloquio ha trattato gli argomenti riguardanti il modulo di predizione del prodotto da sviluppare. Dario Stagnitto ha controllato gli sviluppi effettuati dai componenti del gruppo \textit{Jawa Druids}. Successivamente ha illustrato come visualizzare i dati attraverso i grafici e osservare l'andamento degli stessi. Questo permette uno studio più approfondito dell'affidabilità del modulo "contapersone" e l'individuazione di valori spuri nel dataset$_G$ utilizzato.
Infine Dario ha esposto vari algoritmi da utilizzare sui dati a disposizione nel database. I vari algoritmi proposti sono: Lasso, Ridge e Random Forest Regressor.
\section{Decisioni derivate dal colloquio}
\begin{itemize}
	\item \textbf{E\_2021\_03\_29.1}: utilizzo di grafici per la visualizzazione dei dataset$_{\scaleto{G}{3pt}}$.
	\item \textbf{E\_2021\_03\_29.2}: utilizzare degli algoritmi Lasso, Ridge e Random Forest Regressor per elaborare predizioni.
\end{itemize}