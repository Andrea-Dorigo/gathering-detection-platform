\newpage
\section{Informazioni Generali}
\begin{enumerate}
	\item \textbf{Luogo:} \normalfont Discord;
	\item \textbf{Data:} \normalfont 2021-04-16;
	\item \textbf{Orario inizio:} \normalfont 15.00;
	\item \textbf{Orario fine:} \normalfont 18.00;
	\item \textbf{Partecipanti:}
	\begin{itemize}
		\item team dell'azienda \textit{Sync Lab};
		\item Mattia Cocco;
		\item Andrea Dorigo;
		\item Andrea Cecchin;
		\item Margherita Mitillo;
		\item Igli Mezini;
		\item Alfredo Graziano;
		\item Emma Roveroni.
	\end{itemize}
	\item \textbf{Assenti:}
	\begin{itemize}
		\item nessuno.
	\end{itemize}
\end{enumerate}
\section{Ordine del giorno}
Il gruppo \textit{Jawa Druids} ha svolto un dibattito, tramite messaggi all'interno del server Discord$_G$ dell'azienda \textit{Sync Lab}, riguardante lo schema delle classi sviluppato per il modulo back-end del prodotto. 
Dopo aver esposto lo schema delle classi realizzato dal gruppo \textit{Jawa Druids}, il team dell'azienda proponente ha sollevato qualche perplessità su alcune funzionalità delle classi e illustrato le proprie idee su come modificare e migliorare alcuni metodi.
\section{Decisioni derivate dal colloquio}
\begin{itemize}
	\item \textbf{E\_2021\_04\_16.1}: modifica del diagramma delle classi del modulo back-end;
	\item \textbf{E\_2021\_04\_16.2}: modifica del tipo di ritorno del metodo getLastValue della classe DetectionCustomRepository;
	\item \textbf{E\_2021\_04\_16.3}: aggiunta, in futuro, di un campo dati per l'indice di affidabilità nella classe Detection;
\end{itemize}