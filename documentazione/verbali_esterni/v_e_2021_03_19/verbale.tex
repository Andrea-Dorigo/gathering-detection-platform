\newpage
\section{Informazioni Generali}
\begin{enumerate}
	\item \textbf{Luogo:} \normalfont Google Meet;
	\item \textbf{Data:} \normalfont 2021-03-19;
	\item \textbf{Orario inizio:} \normalfont 17.00;
	\item \textbf{Orario fine:} \normalfont 18.00;
	\item \textbf{Partecipanti:}
	\begin{itemize}
		\item Fabio Pallaro;
		\item Membri gruppo Sync Lab
		\item Mattia Cocco;
		\item Andrea Dorigo;
		\item Andrea Cecchin;
		\item Margherita Mitillo;
		\item Igli Mezini;
		\item Alfredo Graziano;
		\item Emma Roveroni.
	\end{itemize}
	\item \textbf{Assenti:}
	\begin{itemize}
		\item Nessuno.
	\end{itemize}
\end{enumerate}
\section{Ordine del giorno}
La presentazione è avvenuta in due parti a seguito di ognuna delle quali i componenti del team dell'azienda \textit{Sync Lab} hanno posto domande su quanto era stato esposto. Nella prima parte si è descritto attraverso una presentazione di slide il POC sviluppato, esponendo il prodotto in modo teorico. Nella seconda parte invece si è mostrato una demo del prodotto al momento sviluppato presentandone le sue funzionalità. 
\section{Decisioni derivate dal colloquio}
\begin{itemize}
	\item \textbf{E\_2021\_03\_19.1}: aggiungere dei campi al nostro dataset per poter avere previsioni più accurate in particolare le previsioni del meteo.
\end{itemize}
\begin{itemize}
	\item \textbf{E\_2021\_03\_19.2}: utilizzare Google Colab per effettuare i test degli algoritmi di Machine Learning$_G$ da utilizzare.
\end{itemize}
\begin{itemize}
	\item \textbf{E\_2021\_03\_19.3}: aggiungere webcam da città diverse e non necessariamente nella stessa città.
\end{itemize}