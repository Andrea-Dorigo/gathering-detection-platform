\input{packages}
\input{config}

\begin{document}

\makeatletter
\begin{titlepage}
	\begin{center}
		\vspace*{-4cm}
		\author{Jawa Druids}
		\title{Verbale Esterno 25-02-2021}
		\date{} %LASCIARE QUESTO CAMPO VUOTO, SE LO TOLGO STAMPA LA DATA CORRENTE
		\includegraphics[width=0.7\linewidth]{../../immagini/DRUIDSLOGO.jpg}\\[4ex]
		{\huge \bfseries  \@title }\\[2ex]
		{\LARGE  \@author}\\[50ex]
		\vspace*{-9cm}
		\begin{table}[H]
			\renewcommand{\arraystretch}{1.4}
			\centering
			\begin{tabular}{r | l}
				\textbf{Versione} & v1.0.0 \\%RIGA PER INSERIRE LA VERSIONE ULTIMA DEL DOCUMENTO
				\textbf{Data approvazione} & 27-02-2021\\
				\textbf{Responsabile} & Andrea Dorigo\\
				\textbf{Redattori} & Emma Roveroni \\
				\textbf{Verificatori} & \makecell[tl]{Mattia Cocco} \\
				%MAKECELL SERVE PER POI ANDARE A CAPO ALL'INTERNO DELLA CELLA
				\textbf{Stato} & Approvato\\
				\textbf{Lista distribuzione} & \makecell[tl]{Jawa Druids \\ Prof. Tullio Vardanega \\ Prof. Riccardo Cardin}\\
				\textbf{Uso} & Esterno
			\end{tabular}
		\end{table}
		\vspace{0.1cm}
		\hfill \break
		\fontsize{17}{10}\textbf{Sommario} \\
		\vspace{0.1cm}
		Verbale esterno del giorno 25-02-2021 riguardo la conversazione avuta su Discord con Fabio Pallaro di Sync Lab.
	\end{center}
\end{titlepage}
\makeatother

\quad
\begin{center}
	\LARGE\textbf{Registro delle modifiche}
\end{center}

\def\tabularxcolumn#1{m{#1}}
{\rowcolors{2}{RawSienna!90!RawSienna!20}{RawSienna!70!RawSienna!40}


\begin{center}
	\renewcommand{\arraystretch}{1.4}
	\begin{longtable}[c]{|p{2cm-1\tabcolsep}|p{2cm}|p{3cm-2\tabcolsep}|p{2,5cm-2\tabcolsep}|p{4cm-2\tabcolsep}|p{2,5cm}|}
		\hline
		\rowcolor{airforceblue}
		\makecell[c]{\textbf{Versione}} & \makecell[c]{\textbf{Data}} & \makecell[c]{\textbf{Autore}} & \makecell[c]{\textbf{Ruolo}} & \makecell[c]{\textbf{Modifica}} & \makecell[c]{\textbf{Verificatore}} \\
		\hline
		\centering v1.0.0 & 2021-04-18 & Andrea Dorigo & \centering \textit{Responsabile} & \textit{Approvazione del documento} & \makecell[c]{-} \\
		\hline
		\centering v0.1.0 & 2021-04-17 &  \centering - & \centering - &  \textit{Revisione complessiva del documento} & Mattia Cocco \\
		\hline
		\centering v0.0.1 & 2021-04-15 & Andrea Cecchin & \centering \textit{Redattore} &\textit{Stesura del documento}  & Mattia Cocco \\
		\hline
	\end{longtable}
\end{center}


%COMANDO PER LA CREAZIONE DELL'INDICE

\newpage
\tableofcontents{}

%PER RENDERE PIÙ CHIARA LA STESURA DEI DOCUMENTI È MEGLIO LASCIARE SEPARATI IN FILE DIVERSI OGNI CAPITOLO

\newpage
	\section{Informazioni Generali}
	\begin{enumerate}
		\item \textbf{Luogo:} \normalfont Server Discord \textbf{Jawa Druids};
		\item \textbf{Data:} \normalfont 06-01-2021;
		\item \textbf{Orario inizio:} \normalfont 16.00;
		\item \textbf{Orario fine:} \normalfont 17.20;
		\item \textbf{Partecipanti:}
		\begin{itemize}
			\item Mattia Cocco;
			\item Andrea Dorigo;
			\item Igli Mezini;
			\item Andrea Cecchin;
			\item Emma Roveroni;
			\item Alfredo Graziano.
		\end{itemize}
		\item \textbf{Assenti:}
		\begin{itemize}
			\item Margherita Mitillo.
		\end{itemize}
	\end{enumerate}
	\section{Ordine del giorno}
	Una volta che tutti membri del gruppo di lavoro erano pronti è iniziata la riunione con i seguenti punti del giorno:
	\begin{itemize}
		\item Resoconto lavoro svolto;
		\item Assegnazione dei nuovi task.
	\end{itemize}

	\section{Resoconto lavoro svolto}
	Ogni membro del gruppo ha presentato il lavoro svolto fino ad ora, ponendo eventuali domande su dubbi sorti durante la stesura del documento.

	\section{Assegnazione dei nuovi task}
	Ogni membro del gruppo ha preso in carico un nuovo task da portare a termine. La suddivisione dei nuovi task è la seguente:
	\begin{enumerate}
		\item Scrittura del \textbf{Verbale Interno} della riunione avvenuta in data 06-01-2021: \textbf{Emma Roveroni};
		\item Completamento degli ultimi paragrafi delle \textbf{Norme di Progetto} : \textbf{Mattia Cocco};
		\item Inizio verifica di \textbf{Analisi dei Requisiti} : \textbf{Alfredo Graziano}.
	\end{enumerate}
	Inoltre ciascun membro del gruppo completerà i task già assegnati in precedenza, ancora in attesa di completamento.


% \input{esempio} -- esempio di codice per inserire un nuovo capitolo

\end{document}
