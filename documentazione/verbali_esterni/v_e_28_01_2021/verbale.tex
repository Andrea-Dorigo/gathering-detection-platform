\newpage
\section{Informazioni Generali}
\begin{enumerate}
	\item \textbf{Luogo:} \normalfont Zoom;
	\item \textbf{Data:} \normalfont 28-01-2021;
	\item \textbf{Orario inizio:} \normalfont 17.00;
	\item \textbf{Orario fine:} \normalfont 18.00;
	\item \textbf{Partecipanti:}
	\begin{itemize}
		\item Sync Lab;
		\item Mattia Cocco;
		\item Andrea Dorigo;
		\item Margherita Mitillo;
		\item Igli Mezini;
		\item Alfredo Graziano;
		\item Andrea Cecchin.
	\end{itemize}
	\item \textbf{Assenti:}
	\begin{itemize}
		\item Emma Roveroni.
	\end{itemize}
\end{enumerate}
\section{Ordine del giorno}
Appena i componenti del gruppo si sono connessi alla riunione, l'incontro con l'azienda Sync Lab è iniziato.
Il meeting si è concentrato sui seguenti punti:
\begin{enumerate}
	\item nuovi requisiti e casi d'uso da introdurre nel progetto proposto;
	\item tecnologie per il frontend${\scaleto{G}{3pt}}$.
\end{enumerate}
\section{Nuovi requisiti e casi d'uso da introdurre nel progetto proposto}
Il gruppo ha proposto all'azienda l'aggiunta di nuovi requisiti e l'implementazione di altri casi d'uso da inserire all'interno del progetto da loro proposto.
Concordi con l'azienda sono stati quindi aggiunti alcuni requisiti prestazionali.

\section{Tecnologie per il frontend}
Dopo un confronto sulle tecnologie da utilizzare per il frontend, in accordo col proponente si è deciso di sostituire la tecnologia Angular${\scaleto{G}{3pt}}$ con Vue.js${\scaleto{G}{3pt}}$.

\section{Decisioni derivate dal colloquio}
\begin{itemize}
	\item inserimento di nuovi requisiti prestazionali quali:
		\begin{enumerate}
			\item tempo di latenza per l'invio dei dati da parte del 'conta persone' al database;
			\item tempo di latenza per la visualizzazione dei dati in tempo reale e nel tempo futuro.
		\end{enumerate}
	\item Vue.js$_{\scaleto{G}{3pt}}$ prenderà il posto di Angular$_{\scaleto{G}{3pt}}$;
	\item utilizzo di Spring$_{\scaleto{G}{3pt}}$ come tecnologia per il collegamento tra backend$_{\scaleto{G}{3pt}}$ e frontend$_{\scaleto{G}{3pt}}$.
\end{itemize}
