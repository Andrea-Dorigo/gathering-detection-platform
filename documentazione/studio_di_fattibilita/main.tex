\input{packages}
\input{config}

\begin{document}
	
	\makeatletter
	\begin{titlepage}
		\begin{center}
			\vspace*{-5cm}
			\author{Jawa Druids} 
			\title{Nome documento}
			\date{} %LASCIARE QUESTO CAMPO VUOTO, SE LO TOLGO STAMPA LA DATA CORRENTE
			\includegraphics[width=0.5\linewidth]{../immagini/DRUIDSLOGO.jpg}\\[4ex]
			{\huge \bfseries  \@title }\\[2ex] 
			{\LARGE  \@author}\\[50ex]
			\vspace*{-9cm}
			\begin{table}[H]
				\renewcommand{\arraystretch}{1.4}
				\centering
				\begin{tabular}{r | l}
					\textbf{Versione} & x.x.x \\%RIGA PER INSERIRE LA VERSIONE ULTIMA DEL DOCUMENTO
					\textbf{Data approvazione} & xx-xx-xxxx\\
					\textbf{Responsabile} & Nome Cognome\\
					\textbf{Redattori} & \makecell[tl]{Margherita Mitillo \\ Emma Roveroni} \\
					\textbf{Verificatori} & \makecell[tl]{Igli Mezini \\ Nome Cognome} \\
					%MAKECELL SERVE PER POI ANDARE A CAPO ALL'INTERNO DELLA CELLA
					\textbf{Stato} & Stato\\
					\textbf{Lista distribuzione} & \makecell[tl]{JawaDruids \\ Prof. Tullio vardanega \\ Prof. Riccardo Cardin \\ Sync Lab}\\
					\textbf{Uso} & Interno
				\end{tabular}
			\end{table}
			\vspace{0.1cm}
			\hfill \break
			\fontsize{17}{10}\textbf{Sommario} \\
			\vspace{0.1cm}
			Studio di Fattibilità dei capitolati proposti 
		\end{center}
	\end{titlepage}
	\makeatother
	
	\quad
\begin{center}
	\LARGE\textbf{Registro delle modifiche}
\end{center}

\def\tabularxcolumn#1{m{#1}}
{\rowcolors{2}{RawSienna!90!RawSienna!20}{RawSienna!70!RawSienna!40}

\begin{center}
	\renewcommand{\arraystretch}{1.4}
\begin{tabularx}{\textwidth}{|X|c|c|c|c|}
	\hline
	\rowcolor{airforceblue}
	\textbf{Modifica} & \textbf{Autore} & \textbf{Ruolo} & \textbf{Data} & \textbf{Versione}\\
	\hline
	\textit{Verificato Capitolato C7} & Igli Mezini & \textit{Verificatore} & 03-12-2020 & v0.4.0 \\
	\hline
	\textit{Verificati Capitolati C5 e C6} & Igli Mezini & \textit{Verificatore} & 01-12-2020 & v0.3.0 \\
	\hline
	\textit{Aggiunto Capitolato C6} & Emma Roveroni & \textit{Analista} & 30-11-2020 & v0.2.2 \\
	\hline
	\textit{Aggiunto Capitolato C5 e C7} & Margherita Mitillo & \textit{Analista} & 29-11-2020 & v0.2.1 \\
	\hline
	\textit{Verificati Capitolati C2 e C4} & Igli Mezini & \textit{Verificatore} & 27-11-2020 & v0.2.0 \\
	\hline
	\textit{Aggiunto Capitolato C4} & Emma Roveroni & \textit{Analista} & 26-11-2020 & v0.1.1 \\
	\hline
	\textit{Verificata Introduzione e Capitolati C1 e C3} & Igli Mezini & \textit{Verificatore} & 25-11-2020 & v0.1.0 \\
	\hline
	\textit{Aggiunto Capitolato C1} & Emma Roveroni & \textit{Analista} & 24-11-2020 & v0.0.4 \\
	\hline
	\textit{Aggiunti Capitolati C3 e C2} & Margherita Mitillo & \textit{Analista} & 23-11-2020 & v0.0.3 \\
	\hline
	\textit{Aggiunta Introduzione} & Margherita Mitillo & \textit{Analista} & 20-11-2020 & v0.0.2 \\
	\hline
	\textit{Prima stesura del documento} & Margherita Mitillo & \textit{Analista} & 19-11-2020 & v0.0.1 \\
	\hline
\end{tabularx}
\end{center}
	
	%COMANDO PER LA CREAZIONE DELL'INDICE
	
	\tableofcontents{}
	
	%PER RENDERE PIÙ CHIARA LA STESURA DEI DOCUMENTI È MEGLIO LASCIARE SEPARATI IN FILE DIVERSI OGNI CAPITOLO
	
	% \input{esempio} -- esempio di codice per inserire un nuovo capitolo
\chapter{Introduzione}

\section{Scopo del documento}
Lo scopo del documento è quello di formalizzare tutte le regole e procedure fondamentali che ciascun membro di JawaDruids si impegna a rispettare per tutta la durata dello sviluppo del progetto. 
Le norme verranno aggiunte passo dopo passo a seguito di un'attenta analisi e concordate all'interno del gruppo. 
\section{Scopo del prodotto}

\section{Glossario}
All'interno del documento sono presenti termini che possono risultare ambigui o incongruenti a seconda del contesto in cui si trovano. Per evitare il sorgere di incomprensioni viene fornito un glossario individuabile nel file \textit{Glossario} contenente i suddetti termini con la loro relativa spiegazione.
\\Nella seguente documentazione per favorire chiarezza ed evitare inutili ridondanze tali termini verranno indicati mettendo la lettera "G" come pedice ad ogni prima ricorrenza che si incontra ad inizio di ogni sezione.

\section{Riferimenti}
\subsection{Riferimenti normativi}
\begin{itemize}
	\item \textit{Norme di Progetto}
\end{itemize}
\subsection{Riferimenti informativi}
\begin{itemize}
	
\end{itemize}
\chapter{Capitolato scelto: C3 - GDP, Gathering Detection Platform} \label{CapitolatoC3}

Il capitolato$_{\scaleto{G}{3pt}}$ C3 è stato presentato dall'azienda \textit{Sync Lab}, Software House  nata nel 2002 che propone nel mercato prodotti software nei settori mobile, videosorveglianza e sicurezza nelle infrastrutture e informatiche aziendali. L’obiettivo di \textit{Sync Lab} è la realizzazione, messa in opera e governance di soluzioni IT. Inoltre è molto sensibile all'innovazione attraverso attività di ricerca  e sviluppo.

\section{Informazioni generali} \label{C3InformazioniGenerali}
\begin{itemize}
	\item \textbf{Nome} - GDP: Gathering Detection Platform;
	\item \textbf{Proponente}$_G$ - \textit{Sync Lab};
	\item \textbf{Committente}$_G$ - Prof. Tullio Vardanega e Prof. Riccardo Cardin.
\end{itemize}
\section{Descrizione del capitolato} \label{C3DescrizioneDelCapitolato}
La pandemia dovuta al  virus Covid-19 ha portato noi cittadini inizialmente ad una quarantena forzata, successivamente ad una parziale circolazione. Questo ha comportato alla creazione di situazioni di rischio assembramento e di conseguenza ad un aumento del  pericolo di contagio. L'idea dell'azienda è quella di creare quindi una piattaforma che possa aiutare i cittadini a vivere più serenamente e con sicurezza questa situazione. Infatti, attraverso tale servizio si potrà individuare quali zone sono più a rischio assembramento rispetto ad altre. \\
Per realizzare ciò, è necessario utilizzare elementi raccolti da sensoristica e da altre sorgenti, in modo tale da poter fare una stima ed ottenere indicazioni sui potenziali aggregamenti e conseguentemente fornire un supporto per l'ottimizzazione del traffico pedonale.
\section{Finalità del progetto} \label{C3FinalitàDelProgetto}
La soluzione al problema precedentemente descritto deve essere un prototipo software in grado di acquisire e monitorare dati per poi estrapolarne le informazioni da sfruttare in modo da identificare le zone e/o  eventi che presentino un rischioso flusso di persone. I fruitori della piattaforma devono poter quindi visualizzare dati in tempo reale tramite heat map$_G$ oppure predizioni future di una determinata zona. \\
In particolare gli obiettivi tecnologici che si vogliono raggiungere sono:
\begin{itemize}
	\item Realizzazione di un software atto a contare le persone nei mezzi pubblici;
	\item Realizzazione di simulazione dei dati per poter monitorare dati storici e previsionali;
	\item Capacità di acquisire informazioni a bassa latenza ed in modo continuativo;
	\item Elaborazione in tempo reale dei dati;
	\item Identificazione di eventi concorrenti;
	\item Previsione di variazione del flusso di utenti.
\end{itemize}
Riguardo al lato predittivo si dovrà istruire il software tramite Machine Learning$_G$ a riconoscere dati di momenti passati ed elaborare predizioni future. Inoltre bisogna aggiornare automaticamente le previsioni sulla base dei nuovi dati che vengono osservati.
\section{Tecnologie interessate} \label{C3TecnologieInteressate}
Il proponente$_{\scaleto{G}{3pt}}$ ha interesse nell'esplorare nuove tecnologie quindi preferisce non imporne di specifiche, affidandosi alle proposte dei fornitori. Ha specificato comunque alcune scelte tecnologiche da considerare per lo svolgimento del progetto:
\begin{itemize}
	\item \textit{Java}$_G$ e \textit{Angular}$_G$ per lo sviluppo delle parti di Back-End$_G$ e Front-End$_G$;
	\item Il framework$_G$ \textit{Leaflet}$_G$ per la gestione delle mappe.
\end{itemize}
\section{Aspetti positivi} \label{C3AspettiPositivi}
\begin{itemize}
	\item La possibilità di lavorare ad un progetto legato a tematiche contemporanee;
	\item Anche se il progetto nasce da un determinato tema attuale, il gruppo ha concluso che un prodotto software di questo tipo ha applicazione in diversi campi, quindi vi è una possibilità di utilizzo dei ragionamenti e dei metodi di sviluppo in futuro;
	\item La possibilità di non avere vincoli tecnologici;
	\item Il proponente$_{\scaleto{G}{3pt}}$ è aperto al confronto e disponibile a creare un ambiente caratterizzato da una forte collaborazione;
	\item La possibilità di esplorare tecnologie non presenti nel percorso di studi universitario;
	\item Interesse da parte di tutti i componenti del gruppo a lavorare con il Machine Learning$_{\scaleto{G}{3pt}}$.
\end{itemize}
\section{Criticità e fattori di rischio} \label{C3CriticitàEFattoriDiRischio}
\begin{itemize}
	\item L'apprendimento delle nuove tecnologie o delle strumentazioni previste potrebbe risultare lento per i membri del gruppo che non le hanno mai utilizzate.
\end{itemize}
\section{Conclusioni} \label{C3Conclusioni}
Il capitolato$_{\scaleto{G}{3pt}}$ ha attirato l'attenzione del gruppo fin da subito grazie alla chiarezza e alla linearità della presentazione. La possibilità di avere libertà tecnologica e la propensione alla ricerca di tecnologie innovative da parte del proponente$_{\scaleto{G}{3pt}}$ potrebbe permetterci di imparare argomenti non presenti negli insegnamenti del percorso di studi. La presenza del Machine Learning$_{\scaleto{G}{3pt}}$ è stato un altro fattore decisivo per la scelta del capitolato$_{\scaleto{G}{3pt}}$, settore interessante che non viene toccato nel percorso di studi. A seguito di queste affermazioni, il gruppo \textit{Jawa Druids} ha eletto questo capitolato$_{\scaleto{G}{3pt}}$ come prima scelta, fiduciosi del fatto di riuscire a colmare le lacune e affrontare in sinergia ogni possibile difficoltà che si presenterà nel processo di creazione del prodotto software richiesto.
\chapter{C1 - BlockCOVID: supporto digitale al contrasto della pandemia} \label{CapitolatoC1}

Il capitolato$_{\scaleto{G}{3pt}}$ C1 è stato presentato da \textit{Imola Informatica}, azienda che si occupa di consulenza IT.

\section{Informazioni generali} \label{C1InformazioniGenerali}
\begin{itemize}
	\item \textbf{Nome} - BlockCOVID: supporto digitale al contrasto della pandemia;
	\item \textbf{Proponente}$_{\scaleto{G}{3pt}}$ - \textit{Imola Informatica};
	\item \textbf{Committente}$_{\scaleto{G}{3pt}}$ - Prof. Tullio Vardanega e Prof. Riccardo Cardin.
\end{itemize}
\section{Descrizione del capitolato} \label{C1DescrizioneDelCapitolato}
L'azienda proponente$_{\scaleto{G}{3pt}}$ ha deciso di trattare la tematica della pandemia contemporanea in ambito alla sicurezza al lavoro. Infatti ogni azienda dovrebbe assicurare ai propri dipendenti un luogo di lavoro sicuro dal rischio di contagio e, perciò, sanificato correttamente. In particolare bisogna poter segnalare le postazioni in uso e , successivamente, comunicare quando vengono liberate in modo che gli addetti possano procedere con una sanificazione corretta, rendendo la postazione pronta per un nuovo utilizzo.\\
Questa procedura, oltre a tutelare il dipendente, tutela anche il datore di lavoro nel caso in cui avvenga un caso di contagio.
\section{Finalità del progetto} \label{C1FinalitàDelProgetto}
L'obiettivo è quello di sviluppare un'applicazione in grado di indicare quando una postazione viene occupata da un determinato dipendente. In particolare, tramite l'applicazione, si deve poter sapere se una postazione è libera, occupata oppure prenotata, sapere lo stato di avanzamento della sanificazione e prenotare una postazione. Gli utenti che possono usare questa applicazione sono divisi in tre categorie: amministratore, utente ed addetto alle pulizie. Il primo deve poter gestire le postazioni di lavoro, i dipendenti presenti ed estrapolare un report legato alle postazioni utilizzate da un singolo utente ed uno legato alle sanificazioni effettuate. Il secondo, invece, deve poter prenotare e segnalare l'occupazione della postazione e quando la pulisce con il kit aziendale. Il terzo, infine, deve poter ricevere un elenco delle postazioni che necessitano di sanificazione e spuntare quelle sanificate.
\section{Tecnologie interessate} \label{C1TecnologieInteressate}
Il proponente$_{\scaleto{G}{3pt}}$ preferisce non imporre particolari tecnologie da utilizzare per svolgere il progetto in quanto sempre interessato alla ricerca di nuove soluzioni tecnologiche. L'azienda si sente comunque di consigliare al fornitore le seguenti tecnologie: 
\begin{itemize}
	\item \textit{Java}$_{\scaleto{G}{3pt}}$, \textit{Python} o \textit{nodeJS} per lo sviluppo del Back-End$_{\scaleto{G}{3pt}}$.;
	\item \textit{IAAS Kubernetes} oppure un \textit{PAAS} per il rilascio delle componenti.
\end{itemize}
\section{Aspetti positivi} \label{C1AspettiPositivi}
\begin{itemize}
	\item La possibilità di scegliere autonomamente le tecnologie da utilizzare.
\end{itemize}
\section{Criticità e fattori di rischio} \label{C1CriticitàEFattoriDiRischio}
\begin{itemize}
	\item Il prodotto software di questo capitolato$_{\scaleto{G}{3pt}}$ è legato ad una tematica ristretta e che quindi difficilmente potrà essere applicato in campi diversi da quello in cui nasce;
	\item Block Covid è un capitolato$_{\scaleto{G}{3pt}}$ che sviluppa un'applicazione con idee già esistenti e nella teoria semplici da capire che non porta il gruppo ad un percorso di autoformazione.
\end{itemize}
\section{Conclusioni} \label{C1Conclusioni}
Questo capitolato$_{\scaleto{G}{3pt}}$ non ha suscitato particolare interesse nel gruppo in quanto legato ad un tema troppo ristretto. Inoltre non si è riusciti ad evidenziare un possibile percorso di autofromazione riguardo a argomenti diversi da quelli del proprio percorso di studi. Per questo motivo, il gruppo ha preferito orientarsi verso un'alternativa più stimolante.
\chapter{C2 - EmporioLambda: piattaforma di e-commerce in stile Serverless}
\section{Informazioni generali}
\begin{itemize}
	\item \textbf{Nome} - EmporioLambda: piattaforma di e-commerce in stile Serverless
	\item \textbf{Proponente} - Red Babel
	\item \textbf{Committente} - Prof. Tullio Vardanega e Prof. Riccardo Cardin
\end{itemize}
\section{Descrizione del capitolato}

\section{Finalità del progetto}
\section{Tecnologie interessate}
\section{Aspetti positivi}
\section{Criticità e fattori di rischio}
\section{Conclusioni}
\chapter{C4 - HD Viz: visualizzazione di dati multidimensionali} \label{CapitolatoC4}
Il capitolato$_{\scaleto{G}{3pt}}$ C4 è stato presentato dalla \textit{Zucchetti}, società che produce soluzioni software ed hardware e servizi per aziende, assicurazioni e banche.
\section{Informazioni generali} \label{C4InformazioniGenerali}
\begin{itemize}
	\item \textbf{Nome} - HD Viz: visualizzazione di dati multidimensionali;
	\item \textbf{Proponente}$_{\scaleto{G}{3pt}}$ - \textit{Zucchetti};
	\item \textbf{Committente}$_{\scaleto{G}{3pt}}$ - Prof. Tullio Vardanega e Prof. Riccardo Cardin.
\end{itemize}
\section{Descrizione del capitolato} \label{C4DescrizioneDelCapitolato}
Con l'evoluzione della tecnologia le applicazioni moderne riescono a memorizzare volumi di dati molto elevati e con molte dimensioni. Anche i programmi tradizionali si sono evoluti in questo campo. L'esposizione grafica di questi dati, però, in certe situazioni, potrebbe risultare inefficiente e causa di errori. \\
Per questo motivo l'azienda richiedere di creare una piattaforma che permetta di visualizzare chiaramente i dati multidimensionali così da poter individuare subito i possibili errori.
\section{Finalità del progetto} \label{C4FinalitàDelProgetto}
In particolare il progetto da realizzare è una piattaforma web dove è possibile visualizzare dati multidimensionali con almeno 15 dimensioni. Questi dati si devono poter fornire sia tramite query che tramite file formato csv. La piattaforma dovrà garantire almeno quattro tipologie di visualizzazioni:
\begin{itemize}
	\item Scatter Plot Matrix, visualizzazione a riquadri posti a matrice;
	\item Force Field, visualizzazione che traduce le distanze tra i punti;
	\item Heat Map$_{\scaleto{G}{3pt}}$, visualizzazione che mostra la distanza tra punti con colori più o meno intensi;
	\item Proiezione Lineare Multi Asse, visualizzazione dei punti in un piano cartesiano.
\end{itemize}
\section{Tecnologie interessate} \label{C4TecnologieInteressate}
Le tecnologie richieste per lo sviluppo di questo progetto sono le seguenti:
\begin{itemize}
	\item HTML, CS e Javascript per lo sviluppo della piattaforma web;
	\item Java$_{\scaleto{G}{3pt}}$ con server Tomcat o in alternativa Javascript con server Node.js per lo sviluppo lato server.
\end{itemize}
\section{Aspetti positivi} \label{C4AspettiPositivi}
\begin{itemize}
	\item La presenza di tecnologie già note potrebbe facilitare la realizzazione del progetto.
\end{itemize}
\section{Criticità e fattori di rischio} \label{C4CriticitàEFattoriDiRischio}
\begin{itemize}
	\item Dalla documentazione sorge che si dovranno utilizzare tecnologie già note ai componenti del gruppo stimolando poco il nostro interesse;
	\item Le tematiche del capitolato$_{\scaleto{G}{3pt}}$ non hanno suscitato interesse particolare in buona parte dei componenti del gruppo.
\end{itemize}
\section{Conclusioni} \label{C4Conclusioni}
Questo capitolato$_{\scaleto{G}{3pt}}$ non ha suscitato interesse nella maggior parte del gruppo, in particolare per via della presenza di tecnologie già note, in quanto già studiate nel percorso di laurea. Per questo motivo il gruppo ha deciso di escluderlo nella scelta del progetto da svolgere.
\chapter{C5 - PORTACS: piattaforma di controllo mobilità autonoma}
\section{Informazioni generali}
\begin{itemize}
	\item \textbf{Nome} - PORTACS: piattaforma di controllo mobilità autonoma
	\item \textbf{Proponente} - SanMarco Informatica
	\item \textbf{Committente} - Prof. Tullio Vardanega e Prof. Riccardo Cardin
\end{itemize}
\section{Descrizione del capitolato}
\section{Finalità del progetto}
\section{Tecnologie interessate}
\section{Aspetti positivi}
\section{Criticità e fattori di rischio}
\section{Conclusioni}
\chapter{C6 - RGP: Realtime Gaming Platform}
\section{Informazioni generali}
\begin{itemize}
	\item \textbf{Nome} - RGP: Realtime Gaming Platform
	\item \textbf{Proponente} - zero12
	\item \textbf{Committente} - Prof. Tullio Vardanega e Prof. Riccardo Cardin
\end{itemize}
\section{Descrizione del capitolato}
\section{Finalità del progetto}
\section{Tecnologie interessate}
\section{Aspetti positivi}
\section{Criticità e fattori di rischio}
\section{Conclusioni}
\chapter{C7 - SSD: soluzioni di sincronizzazione desktop}
\label{CapitolatoC7}
Il capitolato$_{\scaleto{G}{3pt}}$ C7 è stato presentato dall'azienda \textit{Zextras}, nata come Service Provider Zimbra, cioè un sistema email on-premise
\section{Informazioni generali} \label{C7InformazioniGenerali}
\begin{itemize}
	\item \textbf{Nome} - SSD: soluzioni di sincronizzazione desktop;
	\item \textbf{Proponente}$_{\scaleto{G}{3pt}}$ - \textit{Zextras};
	\item \textbf{Committente}$_{\scaleto{G}{3pt}}$ - Prof. Tullio Vardanega e Prof. Riccardo Cardin.
\end{itemize}
\section{Descrizione del capitolato} \label{C7DescrizioneDelCapitolato}
Al giorno d'oggi l'utente necessita di poter accedere ai propri contenuti da diversi dispositivi, sia web che mobile. Il proponente$_{\scaleto{G}{3pt}}$ vorrebbe approfondire la sincronizzazione desktop per poter permettere all'utente il salvataggio in cloud del proprio lavoro.
\section{Finalità del progetto} \label{C7FinalitàDelProgetto}
Il progetto ha come obiettivo finale lo sviluppo di un'interfaccia multipiattaforma che sfrutti un algoritmo efficiente, capace di garantire il salvataggio in cloud del lavoro e la sincronizzazione dei cambiamenti presenti in cloud. Tale interfaccia deve poter essere disponibile per i sistemi operativi più usati (Windows, Mac, Linux), senza domandare all'utente di installare prodotti aggiuntivi per il suo funzionamento. 
\section{Tecnologie interessate} \label{C7TecnologieInteressate}
Il proponente$_{\scaleto{G}{3pt}}$ ha consigliato diverse tecnologie utili per lo svolgimento del progetto e sono le seguenti.
\begin{itemize}
	\item Qt Framework per lo sviluppo dell'interfaccia e del controller dell'architettura;
	\item Python per lo sviluppo del Back-End$_{\scaleto{G}{3pt}}$.
\end{itemize}
\section{Aspetti positivi} \label{C7AspettiPositivi}
\begin{itemize}
	\item Tematica ritenuta interessante per alcuni membri del gruppo.
\end{itemize}
\section{Criticità e fattori di rischio} \label{C7CriticitàEFattoriDiRischio}
\begin{itemize}
	\item La presenza di Qt Framework tra le tecnologie proposte, IDE non apprezzato da alcuni componenti del gruppo.
\end{itemize}
\section{Conclusioni} \label{C7Conclusioni}
Il gruppo ha deciso di escludere dalla scelta del progetto da svolgere tale capitolato$_{\scaleto{G}{3pt}}$ in quanto una parte del gruppo riteneva poco interessante la tematica e per la presenza  del framework$_{\scaleto{G}{3pt}}$ di Qt.

% bibliography, glossary and index would go here.

\end{document}