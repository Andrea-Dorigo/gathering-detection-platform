\chapter{C7 - SSD: soluzioni di sincronizzazione desktop}
\section{Informazioni generali}
\begin{itemize}
	\item \textbf{Nome} - SSD: soluzioni di sincronizzazione desktop
	\item \textbf{Proponente} - Zextras
	\item \textbf{Committente} - Prof. Tullio Vardanega e Prof. Riccardo Cardin
\end{itemize}
\section{Descrizione del capitolato}
Al giorno d'oggi l'utente necessita di poter accedere ai propri contenuti da diversi dispositivi, sia web che mobile. Il proponente vorrebbe approfondire la sincronizzazione desktop per poter permettere all'utente il salvataggio in cloud del proprio lavoro.
\section{Finalità del progetto}
Il progetto ha come obiettivo finale lo sviluppo di un'interfaccia multipiattaforma che sfrutti un algoritmo efficiente capace di garantire il salataggio in cloud del lavoro e la sincronizzazione dei cambiamenti presenti in cloud. Tale interfacci deve poter essere disponibile per i sistemi operativi più usati (Windows, Mac, Linux), senza domandare all'utente di installare prodotti aggiuntivi per il suo funzionamento. 
\section{Tecnologie interessate}
Il proponente ha consigliato diverse tecnologie utili per lo svolgimento del progetto e sono le seguenti.
\begin{itemize}
	\item Qt Framework per lo sviluppo dell'interfaccia e del controller dell'architettura;
	\item Python per lo sviluppo del Back End.
\end{itemize}
\section{Aspetti positivi}
\begin{itemize}
	\item Tematica ritenuta interessante per alcuni membri del gruppo.
\end{itemize}
\section{Criticità e fattori di rischio}
\begin{itemize}
	\item La presenza di Qt Framework tra le tecnologie proposte, IDE non apprezzato da alcuni componenti del gruppo;
	\item Alta complessità per la realizzazione del progetto.
\end{itemize}
\section{Conclusioni}
Una parte del gruppo riteneva poco interessante la tematica di questo capitolato, inoltre l'alta complessità e la presenza del framework di Qt hanno portato a scartare questo progetto.