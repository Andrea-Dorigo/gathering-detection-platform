\chapter{C7 - SSD: soluzioni di sincronizzazione desktop}
\label{CapitolatoC7}
Il capitolato$_{\scaleto{G}{3pt}}$ C7 è stato presentato dall'azienda Zextras, nata come Service Provider Zimbra, cioè un sistema email on-premise
\section{Informazioni generali} \label{C7InformazioniGenerali}
\begin{itemize}
	\item \textbf{Nome} - SSD: soluzioni di sincronizzazione desktop
	\item \textbf{Proponente}$_{\scaleto{G}{3pt}}$ - Zextras
	\item \textbf{Committente}$_{\scaleto{G}{3pt}}$ - Prof. Tullio Vardanega e Prof. Riccardo Cardin
\end{itemize}
\section{Descrizione del capitolato} \label{C7DescrizioneDelCapitolato}
Al giorno d'oggi l'utente necessita di poter accedere ai propri contenuti da diversi dispositivi, sia web che mobile. Il proponente$_{\scaleto{G}{3pt}}$ vorrebbe approfondire la sincronizzazione desktop per poter permettere all'utente il salvataggio in cloud del proprio lavoro.
\section{Finalità del progetto} \label{C7FinalitàDelProgetto}
Il progetto ha come obiettivo finale lo sviluppo di un'interfaccia multipiattaforma che sfrutti un algoritmo efficiente, capace di garantire il salvataggio in cloud del lavoro e la sincronizzazione dei cambiamenti presenti in cloud. Tale interfaccia deve poter essere disponibile per i sistemi operativi più usati (Windows, Mac, Linux), senza domandare all'utente di installare prodotti aggiuntivi per il suo funzionamento. 
\section{Tecnologie interessate} \label{C7TecnologieInteressate}
Il proponente$_{\scaleto{G}{3pt}}$ ha consigliato diverse tecnologie utili per lo svolgimento del progetto e sono le seguenti.
\begin{itemize}
	\item Qt Framework per lo sviluppo dell'interfaccia e del controller dell'architettura;
	\item Python per lo sviluppo del Back-End$_{\scaleto{G}{3pt}}$.
\end{itemize}
\section{Aspetti positivi} \label{C7AspettiPositivi}
\begin{itemize}
	\item Tematica ritenuta interessante per alcuni membri del gruppo.
\end{itemize}
\section{Criticità e fattori di rischio} \label{C7CriticitàEFattoriDiRischio}
\begin{itemize}
	\item La presenza di Qt Framework tra le tecnologie proposte, IDE non apprezzato da alcuni componenti del gruppo;
\end{itemize}
\section{Conclusioni} \label{C7Conclusioni}
Il gruppo ha deciso di escludere dalla scelta del progetto da svolgere tale capitolato$_{\scaleto{G}{3pt}}$ in quanto una parte del gruppo riteneva poco interessante la tematica e per la presenza  del framework$_{\scaleto{G}{3pt}}$ di Qt.