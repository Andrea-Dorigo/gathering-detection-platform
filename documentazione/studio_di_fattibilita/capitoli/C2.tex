\chapter{C2 - EmporioLambda: piattaforma di e-commerce in stile Serverless} \label{CapitolatoC2}
Il capitolato$_{\scaleto{G}{3pt}}$ C2 è stato presentato dall'azienda Red Babel, azienda di consulenza per la progettazione di siti web.
\section{Informazioni generali} \label{C2InformazioniGenerali}
\begin{itemize}
	\item \textbf{Nome} - EmporioLambda: piattaforma di e-commerce in stile Serverless
	\item \textbf{Proponente} - Red Babel
	\item \textbf{Committente} - Prof. Tullio Vardanega e Prof. Riccardo Cardin
\end{itemize}
\section{Descrizione del capitolato} \label{C2DescrizioneDelCapitolato}
L'azienda proponente ha deciso di proporre un capitolato legato ad un argomento recente e sempre in crescita, ossia l'e-commerce. Per questo motivo chiede di creare una demo di un possibile sito di questa tipologia tramite la tecnologia serverless di Amazon Web Services. \\
I prodotti presenti nel catalogo del sito dovranno essere riguardanti una categoria scelta dal gruppo che il proponente dovrà poi accettatare.
\section{Finalità del progetto} \label{C2FinalitàDelProgetto}
Il sito richiesto deve presentare delle componenti principali legate a due tipologie di utenti: i dipendenti dell'attività e i clienti. I dipendenti devono poter gestire in modo manuale oppure automatico elementi come:
\begin{itemize}
	\item Contabilità;
	\item Finanza;
	\item Inventario;
	\item Completamento degli ordini;
	\item Distribuzione;
	\item Spedizione.
\end{itemize}
Mentre il lato utente deve avere le seguenti funzionalità:
\begin{itemize}
	\item Home page;
	\item Elenco dei prodotti;
	\item Pagina descrittiva del singolo prodotto;
	\item Carrello;
	\item Pagina di checkout;
	\item Pagina dell'account.
\end{itemize}
\section{Tecnologie interessate} \label{C2TecnologieInteressate}
Le tecnologie richieste per lo sviluppo del progetto sono le seguenti:
\begin{itemize}
	\item Typescript come linguaggio di programmazione principale;
	\item Serverless Framework, Amazon CloudWatch e Amazon Web Service framework$_{\scaleto{G}{3pt}}$ e servizi di Amazon per la produzione di un'architettura serverless;
	\item Next.js per le connessioni tra il Back-End$_{\scaleto{G}{3pt}}$ ed il Front-End$_{\scaleto{G}{3pt}}$;
	\item HTML per lo sviluppo delle pagine internet.
\end{itemize}
\section{Aspetti positivi} \label{C2AspettiPositivi}
\begin{itemize}
	\item Il progetto è legato ad un ambito accattivante, attuale ed in costante sviluppo,
	con possibili sbocchi lavorativi interessanti;
	\item Lo sviluppo di un sito web potrebbe risultare un argomento più facile per alcuni componenti del gruppo che hanno già lavorato in questo campo.
\end{itemize}
\section{Criticità e fattori di rischio} \label{C2CriticitàEFattoriDiRischio}
\begin{itemize}
	\item La comprensione dello sviluppo del servizio ServerLess potrebber rivelarsi complessa;
	\item L'utilizzo della tecnologia Amazon Web Service può dimostrarsi complicata.
\end{itemize}
\section{Conclusioni} \label{C2Conclusioni}
Nonostante la tematica interessante ed innovativa di questo capitolato, il gruppo ha deciso di escluderlo poichè non ha suscitato interesse in tutti i componenti del gruppo.