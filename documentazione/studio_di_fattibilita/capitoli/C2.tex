\chapter{C2 - EmporioLambda: piattaforma di e-commerce in stile Serverless}
Il capitolato C2 è stato presentato dall'azienda Red Babel, azienda di consulenza per la progettazione di siti web.
\section{Informazioni generali}
\begin{itemize}
	\item \textbf{Nome} - EmporioLambda: piattaforma di e-commerce in stile Serverless
	\item \textbf{Proponente} - Red Babel
	\item \textbf{Committente} - Prof. Tullio Vardanega e Prof. Riccardo Cardin
\end{itemize}
\section{Descrizione del capitolato}
L'azienda proponente ha deciso di proporre un capitolato legato ad una tematica recente e sempre in crescita, ossia l'e-commerce. Per questo motivo chiede di creare una demo di un possibile sito di questo genere tramite la tecnologia serverless di Amazon Web Services. \\
I prodotti presenti nel catalogo del sito dovranno essere riguardanti un argomento scelto dal proponente che venga accettato da loro.
\section{Finalità del progetto}
Il sito richiesto deve presentare delle componenti principali legate a due tipologie di utenti: i dipendenti dell'attività e i clienti. I dipendenti devono poter gestire in modo manuale oppure automatico elementi come:
\begin{itemize}
	\item  contabilità;
	\item finanza;
	\item inventario;
	\item completamento degli ordini;
	\item distribuzione;
	\item spedizione.
\end{itemize}
Mentre lato utente devono essere presenti le seguenti funzionalità:
\begin{itemize}
	\item Home page;
	\item elenco dei prodotti;
	\item pagina descrittiva del singolo prodotto;
	\item carrello;
	\item pagina di checkout;
	\item pagina dell'account.
\end{itemize}
\section{Tecnologie interessate}
\begin{itemize}
	\item Typescript come linguaggio di programmazione principale;
	\item Serverless Framework, Amazon CloudWatch e Amazon Web Service framework e servizi di Amazon per la produzione di un'architettura serverless;
	\item Next.js per le connessioni tra il Back end ed il Front end;
	\item HTML per lo sviluppo delle pagine internet.
\end{itemize}
\section{Aspetti positivi}
\begin{itemize}
	\item Il progetto è legato ad una tematica interessante ed unica con possibili sbocchi lavorativi interessanti;
	\item Lo sviluppo di un sito web potrebbe risultare un argomento più facile per alcuni componenti del gruppo che hanno già lavorato in questo campo.
\end{itemize}
\section{Criticità e fattori di rischio}
\begin{itemize}
	\item La mole di lavoro per produrre un prodotto soddisfacente potrebbe risultare pensate;
	\item La comprensione dello sviluppo del servizio ServerLess potrebber risultare complessa;
	\item L'utilizzo della tecnologia Amazon Web Service può risultare molto complessa.
\end{itemize}
\section{Conclusioni}
Nonostante la tematica interessante ed innovativa di questo capitolato, il gruppo ha deciso di escluderlo a seguito dell'elevata complessità percepita per lo sviluppo e l'autoapprendimento di nuove tecnologie.