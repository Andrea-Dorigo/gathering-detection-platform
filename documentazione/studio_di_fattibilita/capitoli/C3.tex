\chapter{Capitolato scelto: C3 - GDP, Gathering Detection Platform}

Il capitolato C3 è stato presentato dall'azienda Sync Lab, Software House  nata dal 2002 che propone nel mercato prodotti software nei settori mobile, videosorveglianza e molti altri. L’obiettivo di Sync Lab è la realizzazione, messa in opera e governance di soluzioni IT ed inoltre è molto sensibile all'innovazione attraverso attività di ricerca  e sviluppo.

\section{Informazioni generali}
\begin{itemize}
	\item \textbf{Nome} - GDP: Gathering Detection Platform
	\item \textbf{Proponente} - Sync Lab S.r.l.
	\item \textbf{Committente} - Prof. Tullio Vardanega e Prof. Riccardo Cardin
\end{itemize}
\section{Descrizione del capitolato}
Il periodo storico che stiamo vivendo ha portato noi cittadini inizialmente ad una quarantena forzata e successivamente è stata concessa la circolazione, ma questo porta alla creazione di situazioni di rischio assembramento e di conseguenza un aumento nel rischio di contagio. L'idea dell'azienda è creare quindi una piattaforma che possa aiutare i cittadini a vivere più serenamente e con sicurezza questa situazione con la quale si possa scoprire quali zone sono più a rischio assembramento rispetto ad altre. \\
Ossia bisogna utilizzare elementi raccolti da sensoristica e da altre sorgenti; in questo modo si può fare una stima ed ottenere indicazioni sui potenziali assembramenti e conseguentemente fornire un supporto per l'ottimizzazione del traffico.
\section{Finalità del progetto}
La soluzione al problema precedentemente descritto deve essere un prototipo software in grado di acquisire e monitorare dati per poi estrapolare informazioni da utilizzare per identificare le possibili zone o i possibili eventi che presentano un alto flusso di utenti. Gli utilizzatori della piattaforma devono poter quindi visualizzare dati in tempo reale tramite heatmap oppure predizioni future di una determinata zona. \\
In particolare gli obiettivi tecnologici che si vogliono raggiungere sono:
\begin{itemize}
	\item realizzazione di un software atto a contare le persone nei mezzi pubblici;
	\item realizzazione di simulazione dei dati per poter monitorare dati storici e previsionali;
	\item capacità di acquisire informazioni a bassa latenza ed in modo continuativo;
	\item elaborazione in tempo reale dei dati;
	\item identificazione di eventi concorrenti;
	\item previsione di variazione del flusso di utenti.
\end{itemize}
Riguardo al lato predittivo si dovrà istruire il software tramite Machine Learnig a riconoscere dati di momenti passati ed elaborare predizioni di future. Inoltre bisogna aggiornare automaticamente le previsioni sulla base dei nuovi dati che vengono osservati.
\section{Tecnologie interessate}
Il proponente è interessato ad esplorare nuove tecnologie quindi predilige non imporre tecnologie specifiche affidandosi alle proposte dei fornitori, ma ha specificato comunque alcune scelte tecnologiche da considerare per lo svolgimento del progetto. Le tecnologie consigliate sono le seguenti:
\begin{itemize}
	\item \textit{Java} e \textit{Angular} per lo sviluppo delle parti di Back-End e Front-End;
	\item il framework \textit{Leaflet} per la gestione delle mappe;
\end{itemize}
\section{Aspetti positivi}
\begin{itemize}
	\item La possibilità di lavorare ad un progetto legato a tematiche contemporanee;
	\item Anche se il progetto nasce da una determinata tematica il gruppo ha concluso che un prodotto software del genere potrebbe essere applicato in diversi campi, quindi vi è una possibilità di riutilizzo dei ragionamenti e dei metodi di sviluppo futuri;
	\item La possibilità di non avere vincoli tecnologici;
	\item Il proponente si è dimostrato aperto al confronto e disponibile a creare un ambiente caratterizzato da una forte collaborazione;
	\item La possibilità di esplorare tecnologie e tematiche non presenti nel percorso di studi universitario;
	\item Interesse da parte di tutti i componenti del gruppo a lavorare con il Machine Learning.
\end{itemize}
\section{Criticità e fattori di rischio}
\begin{itemize}
	\item L'apprendimento delle nuove tecnologie o delle strumentazioni previste potrebbe risultare lento per i membri del gruppo che non le hanno mai utilizzate.
\end{itemize}
\section{Conclusioni}
Il capitolato ha attirato l'attenzione del gruppo fin da subito grazie alla chiarezza e alla linearità della presentazione. Inoltre la possibilità di avere libertà tecnologica e la propensione alla ricerca di tecnologie innovative da parte del proponente potrebbe dare la possibilità di poter imparare ad utilizzare tecnologie non presenti negli insegnamenti del percorso di studi. La presenza del Machine Learning è stato un altro fattore decisivo per la scelta del capitolato, settore interessante che non viene toccato nel percorso di studi. A seguito di queste affermazioni, il gruppo Jawa Druids ha eletto questo capitolato come prima scelta fiduciosi del fatto di riuscire a colmare le lacune e affrontare in sinergia ogni possibile difficoltà che si presenterà nel percorso di creazione del prodotto software richiesto.