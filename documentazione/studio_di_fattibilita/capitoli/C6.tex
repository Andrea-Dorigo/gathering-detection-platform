\chapter{C6 - RGP: Realtime Gaming Platform}
\section{Informazioni generali}
\begin{itemize}
	\item \textbf{Nome} - RGP: Realtime Gaming Platform
	\item \textbf{Proponente} - zero12
	\item \textbf{Committente} - Prof. Tullio Vardanega e Prof. Riccardo Cardin
\end{itemize}
\section{Descrizione del capitolato}
Il progetto prevede la realizzazione di un videogioco a scorrimento verticale da utilizzare su dispositivi mobile, con possibilità di giocare sia in modalità single player che multiplayer. La sfida tra più giocatori deve essere la parte centrale dello svolgimento del progetto.
\section{Finalità del progetto}
La parte fondamentale della realizzazione del progetto è lo sviluppo di un'architettura cloud-based  per dare la possibilità di far comunicare diversi dispositivi. La modalità multiplayer deve avere le seguenti caratteristiche:
\begin{itemize}
	\item la sfida deve essere ad eliminazione;
	\item durante la partita deve essere possibile vedere in tempo reale anche gli altri giocatori;
	\item i nemici e il power up devono essere sincronizzati in tempo reale per rendere la sfida equa.
\end{itemize}
Per quanto riguarda la modalità single player invece, questa dovrà essere infinita con livelli di difficoltà sempre crescenti.
\section{Tecnologie interessate}
Il proponente ha esplicitamente indicato le scelte di tecnologiche di seguito elencate.
\begin{itemize}
	\item Swift oppure SwiftUI per lo sviluppo in sistemi operativi IOS;
	\item Kotlin per lo sviluppo in sistemi operativi Android;
	\item NodeJs servizio preferibile per lo sviluppo;
	\item Amazon Web Services per gestire la comunicazione tra diversi device.
\end{itemize}
\section{Aspetti positivi}
\begin{itemize}
	\item Formazione su argomenti al di fuori del percorso di studi.
\end{itemize}
\section{Criticità e fattori di rischio}
\begin{itemize}
	\item Alta complessità di realizzazione del progetto;
	\item Tematica ritenuta interessante solo da una parte del gruppo.
\end{itemize}
\section{Conclusioni}
Nonostante la possibilità di autoformazione riguardo ad argomenti al di fuori del percorso di studi, la tematica non ha suscitato interesse nella maggior parte del gruppo e di conseguenza si è deciso di escluderlo nella scelta del progetto da svolgere.