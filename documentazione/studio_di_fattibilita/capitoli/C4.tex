\chapter{C4 - HD Viz: visualizzazione di dati multidimensionali} \label{CapitolatoC4}
Il capitolato$_{\scaleto{G}{3pt}}$ C4 è stato presentato dalla \textit{Zucchetti}, società che produce soluzioni software ed hardware e servizi per aziende, assicurazioni e banche.
\section{Informazioni generali} \label{C4InformazioniGenerali}
\begin{itemize}
	\item \textbf{Nome} - HD Viz: visualizzazione di dati multidimensionali;
	\item \textbf{Proponente}$_{\scaleto{G}{3pt}}$ - \textit{Zucchetti};
	\item \textbf{Committente}$_{\scaleto{G}{3pt}}$ - Prof. Tullio Vardanega e Prof. Riccardo Cardin.
\end{itemize}
\section{Descrizione del capitolato} \label{C4DescrizioneDelCapitolato}
Con l'evoluzione della tecnologia le applicazioni moderne riescono a memorizzare volumi di dati molto elevati e con molte dimensioni. Anche i programmi tradizionali si sono evoluti in questo campo. L'esposizione grafica di questi dati, però, in certe situazioni, potrebbe risultare inefficiente e causa di errori. \\
Per questo motivo l'azienda richiedere di creare una piattaforma che permetta di visualizzare chiaramente i dati multidimensionali così da poter individuare subito i possibili errori.
\section{Finalità del progetto} \label{C4FinalitàDelProgetto}
In particolare il progetto da realizzare è una piattaforma web dove è possibile visualizzare dati multidimensionali con almeno 15 dimensioni. Questi dati si devono poter fornire sia tramite query che tramite file formato csv. La piattaforma dovrà garantire almeno quattro tipologie di visualizzazioni:
\begin{itemize}
	\item Scatter Plot Matrix, visualizzazione a riquadri posti a matrice;
	\item Force Field, visualizzazione che traduce le distanze tra i punti;
	\item Heat Map$_{\scaleto{G}{3pt}}$, visualizzazione che mostra la distanza tra punti con colori più o meno intensi;
	\item Proiezione Lineare Multi Asse, visualizzazione dei punti in un piano cartesiano.
\end{itemize}
\section{Tecnologie interessate} \label{C4TecnologieInteressate}
Le tecnologie richieste per lo sviluppo di questo progetto sono le seguenti:
\begin{itemize}
	\item HTML, CS e Javascript per lo sviluppo della piattaforma web;
	\item Java$_{\scaleto{G}{3pt}}$ con server Tomcat o in alternativa Javascript con server Node.js per lo sviluppo lato server.
\end{itemize}
\section{Aspetti positivi} \label{C4AspettiPositivi}
\begin{itemize}
	\item La presenza di tecnologie già note potrebbe facilitare la realizzazione del progetto.
\end{itemize}
\section{Criticità e fattori di rischio} \label{C4CriticitàEFattoriDiRischio}
\begin{itemize}
	\item Dalla documentazione sorge che si dovranno utilizzare tecnologie già note ai componenti del gruppo stimolando poco il nostro interesse;
	\item Le tematiche del capitolato$_{\scaleto{G}{3pt}}$ non hanno suscitato interesse particolare in buona parte dei componenti del gruppo.
\end{itemize}
\section{Conclusioni} \label{C4Conclusioni}
Questo capitolato$_{\scaleto{G}{3pt}}$ non ha suscitato interesse nella maggior parte del gruppo, in particolare per via della presenza di tecnologie già note, in quanto già studiate nel percorso di laurea. Per questo motivo il gruppo ha deciso di escluderlo nella scelta del progetto da svolgere.