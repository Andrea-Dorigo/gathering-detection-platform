\chapter{C4 - HD Viz: visualizzazione di dati multidimensionali}
Il capitolato C4 è stato presentato dalla Zucchetti, società che produce soluzioni software ed hardware e servizi per aziende, assicurazioni e banche.
\section{Informazioni generali}
\begin{itemize}
	\item \textbf{Nome} - HD Viz: visualizzazione di dati multidimensionali
	\item \textbf{Proponente} - Zucchetti SpA
	\item \textbf{Committente} - Prof. Tullio Vardanega e Prof. Riccardo Cardin
\end{itemize}
\section{Descrizione del capitolato}
Con l'evoluzione della tecnologia le applicazioni moderne riescono a memorizzare volumi di dati molto elevati e con molte dimensioni, anche i programmi tradizionali sono riusciti ad evolversi riguardo a questa tematica. L'esposizione grafica di questi dati però in certe situazioni potrebbero risultare inefficienti e causa di errori. \\
Per questo motivo l'azienda richiedere di creare una piattaforma in modo tale da poter visualizzare chiaramente i dati multidimensionali in modo da poter scorgere subito possibili errori conseguentemente da dover verificare.
\section{Finalità del progetto}
In particolare il progetto che si deve realizzare è una piattaforma web dove è possibile visualizzare dati multidimensionali con almeno 15 dimensioni. Questi dati devo poter essere forniti sia tramite query che tramite file formato csv. La piattaforma dovrà fornire almeno quattro tipologie di visualizzazioni:
\begin{itemize}
	\item Scatter Plot Matrix, visualizzazione a riquadri posti a matrice;
	\item Force Field, visualizzazione che traduce le distanze tra i punti;
	\item Heat Map, visualizzazione che mostra la distanza tra punti con colori più o meno intensi;
	\item Proiezione Lineare Multi Asse, visualizzazione dei punti in un piano cartesiano.
\end{itemize}
\section{Tecnologie interessate}
Le tecnologie richieste per lo sviluppo di questo progetto sono le seguenti.
\begin{itemize}
	\item HTML, CSS e Javascript per lo sviluppo della piattaforma web;
	\item Java con server Tomcat o in alternativa Javascript con server Node.js per lo sviluppo lato server.
\end{itemize}
\section{Aspetti positivi}
\begin{itemize}
	\item La presenza di tecnologie già note potrebbe facilitare la realizzazione del progetto.
\end{itemize}
\section{Criticità e fattori di rischio}
\begin{itemize}
	\item Dalla documentazione sorge che si dovranno utilizzare tecnologie già note ai componenti del gruppo stimolando poco il nostro interesse;
	\item La documentazione è sembrata ai componenti del gruppo in certi punti poco chiara e troppo generalista;
	\item Le tematiche del capitolato non hanno suscitato interesse particolare in buona parte dei componenti del gruppo.
\end{itemize}
\section{Conclusioni}
Questo capitolato non ha suscitato interesse nella maggior parte del gruppo a causa della documentazione poco chiara, della tematica non interessante e della presenza di tecnologie già note essendo state studiate per un corso presente nel percorso di studio. Per questo motivo il gruppo ha deciso di escluderlo nella scelta del progetto da svolgere.