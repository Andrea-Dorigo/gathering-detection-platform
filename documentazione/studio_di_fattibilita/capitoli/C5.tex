\chapter{C5 - PORTACS: piattaforma di controllo mobilità autonoma} \label{CapitolatoC5}
Il capitolato$_{\scaleto{G}{3pt}}$ C5 è stato presentato dalla \textit{SanMarco Informatica}, azienda che offre soluzioni tecnologiche e consulenza digital.
\section{Informazioni generali} \label{C5InformazioniGenerali}
\begin{itemize}
	\item \textbf{Nome} - PORTACS: piattaforma di controllo mobilità autonoma;
	\item \textbf{Proponente}$_{\scaleto{G}{3pt}}$ - \textit{SanMarco Informatica};
	\item \textbf{Committente}$_{\scaleto{G}{3pt}}$ - Prof. Tullio Vardanega e Prof. Riccardo Cardin.
\end{itemize}
\section{Descrizione del capitolato} \label{C5DescrizioneDelCapitolato}
L'azienda richiede la realizzazione di un prodotto dove si potranno visualizzare tutte le unità, ossia robot, muletto o automobile, che avranno un punto di partenza in una griglia ed una lista di punti di interesse. Nel lato utente si dovranno vedere quattro frecce direzionali, un pulsante stop/start e un indicatore di velocità. Il sistema dovrà indicare la prossima direzione da prendere in base alla posizione dei punti di interesse e delle altre unità per evitare collisioni.
\section{Finalità del progetto} \label{C5FinalitàDelProgetto}
Il software finito dovrà essere in grado di accettare in input la scacchiera (o mappa), con le percorrenze e i relativi vincoli, e la definizione delle unità, con identificativo di sistema, velocità massima, posizione iniziale e lista di punti di interesse da dover raggiungere già ordinati. \\
Inoltre le unità dovranno inviare costantemente la propria posizione in modo tale che il sistema centrale riesca a pilotare tutte le unità presenti sulla scacchiera.
\section{Tecnologie interessate} \label{C5TecnologieInteressate}
Il proponente$_{\scaleto{G}{3pt}}$ non ha specificato nulla riguardo alle tecnologie da utilizzare per lo svolgimento del progetto.
\section{Aspetti positivi} \label{C5AspettiPositivi}
\begin{itemize}
	\item La possibilità di lavorare per un'azienda grande potrebbe essere vantaggioso per quanto riguarda il curriculum.
\end{itemize}
\section{Criticità e fattori di rischio} \label{C5CriticitàEFattoriDiRischio}
\begin{itemize}
	\item L'assenza di linee guida riguardanti le tecnologie da utilizzare;
	\item La tematica del capitolato$_{\scaleto{G}{3pt}}$ non ha interessato nessuno dei componenti del gruppo;
	\item La documentazione poco chiara e generale.
\end{itemize}
\section{Conclusioni} \label{C5Conclusioni}
Dato che il capitolato$_{\scaleto{G}{3pt}}$ non ha suscitato interesse e la documentazione è risultata poco esplicativa, il gruppo ha deciso di escluderlo nella scelta del progetto da svolgere.