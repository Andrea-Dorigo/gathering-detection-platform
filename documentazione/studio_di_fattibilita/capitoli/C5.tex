\chapter{C5 - PORTACS: piattaforma di controllo mobilità autonoma}
Il capitolato C5 è stato presentato dalla SanMarco Informatica, azienda che offre soluzioni tecnologiche e consulenza digital.
\section{Informazioni generali}
\begin{itemize}
	\item \textbf{Nome} - PORTACS: piattaforma di controllo mobilità autonoma
	\item \textbf{Proponente} - SanMarco Informatica SpA
	\item \textbf{Committente} - Prof. Tullio Vardanega e Prof. Riccardo Cardin
\end{itemize}
\section{Descrizione del capitolato}
L'azienda richiede la realizzazione di un prodotto dove si potranno visualizzare ogni unità, ossia robot, muletto o automobile, che avranno un punto di partenza in una griglia ed una lista di punti di interesse. Nel lato utente dovranno essere visibili quattro frecce direzionali, un pulsante stop/start e un indicatore di velocità. Il sistema dovrà indicare la prossima direzione da prendere in base alla posizione dei punti di interesse e delle altre unità per evitare le collisioni.
\section{Finalità del progetto}
Il software finito dovrà essere in grado di accettare in input la scacchiera (o mappa), con le percorrenze e i relativi vincoli, e la definizione delle unità, con identificativo, velocità massima, posizione iniziale e lista di punti di interesse da dover raggiungere già ordinati. \\
Inoltre le unità dovranno inviare costantemente al sistema la propria posizione in modo tale che il sistema centrale riesca a pilotare tutte le unità presenti sulla scacchiera.
\section{Tecnologie interessate}
Il proponente non ha specificato nulla riguardo alle tecnologie da utilizzare per lo svolgimento del progetto.
\section{Aspetti positivi}
\begin{itemize}
	\item La possibilità di lavorare per un'azienda grande potrebbe essere vantaggioso per quanto riguarda il curriculum.
\end{itemize}
\section{Criticità e fattori di rischio}
\begin{itemize}
	\item L'assenza di linee guida riguardanti le tecnologie da utilizzare;
	\item La tematica del capitolato non ha interessato nessuno dei componenti del gruppo;
	\item La documentazione poco chiara e generalista.
\end{itemize}
\section{Conclusioni}
Dato che il capitolato non ha suscitato interesse e la documentazione è risultata poco esplicativa ed inconsistente, il gruppo ha deciso di escluderlo nella scelta del progetto da svolgere.