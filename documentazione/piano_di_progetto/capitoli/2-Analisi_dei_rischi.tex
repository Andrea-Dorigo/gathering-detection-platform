\chapter{Analisi dei rischi}\label{AnalisiDeiRischi}

\section{Piano per la gestione dei rischi}\label{AnalisiDeiRischiPianoPerLaGestioneDeiRischi}
Con l'intento di prevenire il naturale insorgere di problemi durante lo svolgimento del progetto è stato elaborato un'approfondito piano per la gestione dei rischi. Quest'ultimo è suddiviso in quattro attività$_{\scaleto{G}{3pt}}$:
\begin{itemize}
  \item \textbf{Individuazione dei rischi:} attività$_{\scaleto{G}{3pt}}$ di identificazione e documentazione di possibili elementi problematici che possano ostacolare il naturale percorso del progetto;
  \item \textbf{Analisi dei rischi:} attività$_{\scaleto{G}{3pt}}$ di analisi dei fattori di rischio, che si articola in probabilità di occorrenza, indice di gravità e conseguente impatto sul progetto;
  \item \textbf{Pianificazione di controllo:} attività$_{\scaleto{G}{3pt}}$ di pianificazione delle misure da adottare per la prevenzione e contenimento del problema;
  \item \textbf{Monitoraggio dei rischi:} attività$_{\scaleto{G}{3pt}}$ di controllo dei rischi che accompagna tutto lo svolgimento del progetto, al fine di evitarli o agire tempestivamente alla loro occorrenza per contenerne i danni.
\end{itemize}
Le principali tipologie di rischio sono state quindi codificate e categorizzate come segue:
\begin{itemize}
  \item \textbf{RT:} Rischi legati alle tecnologie;
  \item \textbf{RO:} Rischi legati all'organizzazione;
  \item \textbf{RI:} Rischi interpersonali, ovvero legati alle relazioni personali interne ed esterne o alla disponibilità e risorse dei componenti.
\end{itemize}

\quad
\begin{center}
	\LARGE\textbf{Rischi legati alle tecnologie}
\end{center}

\def\tabularxcolumn#1{m{#1}}
{\rowcolors{2}{RawSienna!90!RawSienna!20}{RawSienna!70!RawSienna!40}

	\begin{center}
		\renewcommand{\arraystretch}{1.4}
		\begin{longtable}{|p{5cm}|p{12cm}|}
			\hline
			\rowcolor{airforceblue}
			\multicolumn{2}{|c|}{\textit{Inesperienza tecnologica}}\\
			\hline
			\textit{Codice} & RT1 \\
			\hline
			\textit{Descrizione} & Alcune tecnologie utilizzate in questo progetto sono nuove per tutti i membri del gruppo di lavoro. \\
			\hline
			\textit{Conseguenza} & Lo studio e l'apprendimento di tali tecnologie potrebbero richiedere un intervallo di tempo difficile da quantificare, maggiore del previsto e variabile da membro a membro con conseguenti difficoltà operative. \\
			\hline
			\textit{Possibilità di occorrenza} & Alta. \\
			\hline
			\textit{Pericolosità} & Alta. \\
			\hline
			\textit{Precauzioni} & Il \textit{Responsabile di Progetto} dovrà suddividere i compiti$_G$ nel modo più congruo possibile, considerando le conoscenze preliminari di ciascun componente; prevederà inoltre un tempo di Slack$_G$ maggiore per i compiti$_{\scaleto{G}{3pt}}$ assegnati ad un componente senza particolare famigliarità con la relativa tecnologia. Il \textit{Responsabile di Progetto} assegnerà i task$_G$ di maggiore complessità a più membri ove necessario.  \\
			\hline
			\textit{Piano di contingenza} & Ciascun membro comunicherà il prima possibile al \textit{Responsabile di progetto} la previsione di un eventuale ritardo o mancanza; egli provvederà a ridistribuire i compiti$_{\scaleto{G}{3pt}}$ se necessario in modo da sanare eventuali lacune o sottostime. \\
			\hline
		\end{longtable}
	\captionof{table}{\textbf{Analisi dei rischi delle tecnologie utilizzate}}
	\end{center}


\def\tabularxcolumn#1{m{#1}}
{\rowcolors{2}{RawSienna!90!RawSienna!20}{RawSienna!70!RawSienna!40}

	\begin{center}
		\renewcommand{\arraystretch}{1.4}
		\begin{longtable}{|p{5cm}|p{12cm}|}
			\hline
			\rowcolor{airforceblue}
			\multicolumn{2}{|c|}{\textit{Software terze parti}}\\
			\hline
			\textit{Codice} & RT2 \\
			\hline
			\textit{Descrizione} & Eventuali problematiche con software di terze parti, quali la mancanza di documentazione o problemi tecnici, sono indipendenti dai membri del gruppo. \\
			\hline
			\textit{Conseguenza} & Ciò causerebbe ritardi pesanti sul proseguo del lavoro e anche possibili ritardi sulla consegna.
			La necessità di cambiare tecnologia potrebbe richiedere molto tempo e risorse per la ricerca di una sostituzione. \\
			\hline
			\textit{Possibilità di occorrenza} & Bassa. \\
			\hline
			\textit{Pericolosità} & Alta. \\
			\hline
			\textit{Precauzioni} & Il gruppo sceglierà i software più stabili e documentati per evitare questi tipi di problemi.  \\
			\hline
			\textit{Piano di contingenza} & Assieme al \textit{Responsabile di progetto} il gruppo di lavoro si attiverà al fine di tentare di risolvere il problema. Se ciò non è possibile sarà necessario un cambio di tecnologia, anche tramite l'aiuto del proponente$_{\scaleto{G}{3pt}}$.  \\
			\hline
		\end{longtable}
		\captionof{table}{\textbf{Analisi dei rischi dei software di terze parti}}
	\end{center}


\def\tabularxcolumn#1{m{#1}}
{\rowcolors{2}{RawSienna!90!RawSienna!20}{RawSienna!70!RawSienna!40}

	\begin{center}
		\renewcommand{\arraystretch}{1.4}
		\begin{longtable}{|p{5cm}|p{12cm}|}
			\hline
			\rowcolor{airforceblue}
			\multicolumn{2}{|c|}{\textit{Validità dei dati}}\\
			\hline
			\textit{Codice} & RT3 \\
			\hline
			\textit{Descrizione} & Problemi legati alla validità e all'elaborazione dei dati. \\
			\hline
			\textit{Conseguenza} & Arresto obbligato del lavoro in corso, con possibilità di invalidazione del lavoro svolto fino a quel momento. \\
			\hline
			\textit{Possibilità di occorrenza} & Medio/Alta. \\
			\hline
			\textit{Pericolosità} & Molto alta. \\
			\hline
			\textit{Precauzioni} & Prima dell'inizio della raccolta dati il gruppo si assicurerà che la fonte sia affidabile e coerente.
			Questa operazione sarà svolta per prima in quanto critica per l'intero sviluppo.  \\
			\hline
			\textit{Piano di contingenza} & Il gruppo, insieme al proponente$_{\scaleto{G}{3pt}}$, valuterà se sarà necessario cambiare solo la fonte di provenienza dei dati oppure simularli in maniera consona. \\
			\hline
		\end{longtable}
		\captionof{table}{\textbf{Analisi dei rischi della validità dei dati}}
	\end{center}
\clearpage
\quad
\begin{center}
	\LARGE\textbf{Rischi legati all'organizzazione}
\end{center}

\def\tabularxcolumn#1{m{#1}}
{\rowcolors{2}{RawSienna!90!RawSienna!20}{RawSienna!70!RawSienna!40}

	\begin{center}
		\renewcommand{\arraystretch}{1.4}
		\begin{longtable}{|p{5cm}|p{12cm}|}
			\hline
			\rowcolor{airforceblue}
			\multicolumn{2}{|c|}{\textit{Problemi organizzativi}}\\
			\hline
			\textit{Codice} & RO1 \\
			\hline
			\textit{Descrizione} & I problemi organizzativi possono scaturire da vari motivi, sia da parte dei membri che dal proponente$_{\scaleto{G}{3pt}}$, così come dagli impegni personali e dai periodi vacanzieri.  \\
			\hline
			\textit{Conseguenza} & Questi problemi possono far ritardare il completamento dei task$_{\scaleto{G}{3pt}}$ di un tempo più o meno definito, rendendo l'avanzamento più lento o addirittura bloccandolo. \\
			\hline
			\textit{Possibilità di occorrenza} & Alta. \\
			\hline
			\textit{Pericolosità} & Alta. \\
			\hline
			\textit{Precauzioni} & Ogni membro del gruppo di lavoro dovrà avvisare il \textit{Responsabile di progetto} nel caso in cui, per cause di forza maggiore, non si riesca a completare il task$_{\scaleto{G}{3pt}}$ nel tempo deciso oppure non si riesca proprio a farlo. \\
			\hline
			\textit{Piano di contingenza} & Il \textit{Responsabile di progetto} avrà l'incarico di riassegnare i compiti$_{\scaleto{G}{3pt}}$ in modo da riuscire a completare i task$_{\scaleto{G}{3pt}}$ nel tempo stimato, così da non avere ritardi nel portarli a termine.
			Nel caso in cui sia il proponente$_{\scaleto{G}{3pt}}$ a creare questi disagi organizzativi, sarà sempre premura del \textit{Responsabile di progetto} risolvere il problema mediante i canali di comunicazione adatti.  \\
			\hline
		\end{longtable}
	\captionof{table}{\textbf{Analisi dei rischi dei problemi organizzativi}}
	\end{center}


\def\tabularxcolumn#1{m{#1}}
{\rowcolors{2}{RawSienna!90!RawSienna!20}{RawSienna!70!RawSienna!40}

	\begin{center}
		\renewcommand{\arraystretch}{1.4}
		\begin{longtable}{|p{5cm}|p{12cm}|}
			\hline
			\rowcolor{airforceblue}
			\multicolumn{2}{|c|}{\textit{Problemi dei sistemi operativi e configurazioni software}}\\
			\hline
			\textit{Codice} & RO2 \\
			\hline
			\textit{Descrizione} & Problemi dovuti alle differenze degli standard utilizzati dai software in base al sistema operativo in cui sono installati. \\
			\hline
			\textit{Conseguenza} & Possibili incongruenze nella visualizzazione o funzionalità del prodotto software.\\
			\hline
			\textit{Possibilità di occorrenza} & Media. \\
			\hline
			\textit{Pericolosità} & Media. \\
			\hline
			\textit{Precauzioni} & Il gruppo cercherà di trovare una configurazione software adatta per ogni sistema operativo in modo da ridurre al minimo il pericolo. \\
			\hline
			\textit{Piano di contingenza} & Il gruppo cercherà di trovare una soluzione nel minor tempo possibile. \\
			\hline
		\end{longtable}
		\captionof{table}{\textbf{Analisi dei rischi su software e sistemi operativi}}
	\end{center}


\quad
\begin{center}
	\LARGE\textbf{Rischi interpersonali}
\end{center}

\def\tabularxcolumn#1{m{#1}}
{\rowcolors{2}{RawSienna!90!RawSienna!20}{RawSienna!70!RawSienna!40}

	\begin{center}
		\renewcommand{\arraystretch}{1.4}

		\begin{longtable}{|p{5cm}|p{12cm}|}
			\hline
			\rowcolor{airforceblue}
			\multicolumn{2}{|c|}{\textit{Problemi di relazione tra i membri}}\\
			\hline
			\textit{Codice} & RI1 \\
			\hline
			\textit{Descrizione} & Problemi legati ai contrasti che potrebbero intercorrere tra i membri. \\
			\hline
			\textit{Conseguenza} & Difficoltà di avanzamento nel lavoro, poca collaborazione tra i membri in contrasto, malumore nel gruppo. \\
			\hline
			\textit{Possibilità di occorrenza} & Media \\
			\hline
			\textit{Pericolosità} & Alta. \\
			\hline
			\textit{Precauzioni} & In caso di contrasti tra i membri, questi dovranno immediatamente coinvolgere il \textit{Responsabile di progetto} in modo da poter risolvere subito la diatriba.
			In caso non riesca a risolvere la controversia,  comunicherà col \textit{prof. Tullio Vardanega} per la risoluzione dei problemi. \\
			\hline
			\textit{Piano di contingenza} & I membri dovranno impegnarsi nel ridurre al minimo eventuali tensioni tra di loro per favorire l'avanzamento dei lavori e per realizzare al meglio il progetto.\\
			\hline
		\end{longtable}
	\captionof{table}{\textbf{Analisi dei rischi dei problemi relazionali}}
	\end{center}
