\chapter{Analisi dei rischi}\label{AnalisiDeiRischi}

\section{Piano per la gestione dei rischi}\label{AnalisiDeiRischiPianoPerLaGestioneDeiRischi}
Con l'intento di prevenire il naturale insorgere di problemi durante lo svolgimento del progetto è stato elaborato un'approfondito piano per la gestione dei rischi. Quest'ultimo è suddiviso in quattro attività:
\begin{itemize}
  \item \textbf{Individuazione dei rischi:} attività di identificazione e documentazione di possibili elementi problematici che possano ostacolare il naturale percorso del progetto;
  \item \textbf{Analisi dei rischi:} attività di analisi dei fattori di rischio, che si articola in probabilità di occorrenza, indice di gravità e conseguente impatto sul progetto;
  \item \textbf{Pianificazione di controllo:} attività di pianificazione delle misure da adottare per la prevenzione e contenimento del problema;
  \item \textbf{Monitoraggio dei rischi:} attività di controllo dei rischi che accompagna tutto lo svolgimento del progetto, al fine di evitarli o agire tempestivamente alla loro occorrenza per contenerne i danni.
\end{itemize}
Le principali tipologie di rischio sono state quindi codificate e categorizzate come segue:
\begin{itemize}
  \item \textbf{RT:} Rischi legati alle tecnologie;
  \item \textbf{RO:} Rischi legati all'organizzazione;
  \item \textbf{RI:} Rischi interpersonali, ovvero rischi legati alle relazioni personali interne ed esterne o alla disponibilità e risorse dei componenti.
\end{itemize}

\quad
\begin{center}
	\LARGE\textbf{Rischi legati alle tecnologie}
\end{center}

\def\tabularxcolumn#1{m{#1}}
{\rowcolors{2}{RawSienna!90!RawSienna!20}{RawSienna!70!RawSienna!40}
	
	\begin{center}
		\renewcommand{\arraystretch}{1.4}
		\begin{tabularx}{\textwidth}{|c|X|}
			\hline
			\rowcolor{airforceblue}
			\multicolumn{2}{|c|}{\textit{Inesperienza tecnologica}}\\
			\hline
			\textit{Codice} & RT1 \\
			\hline
			\textit{Descrizione} & Le tecnologie utilizzate in questo progetto sono nuove per tutti i membri del gruppo di lavoro. \\
			\hline
			\textit{Conseguenza} & Lo studio e l'apprendimento di tali tecnologie potrebbe richiedere tempi diversi da membro a membro, in quanto non tutti avranno la stessa velocità nell'impararle, potendo causare ritardi sul lavoro da svolgere. \\
			\hline
			\textit{Possibilità di occorrenza} & Alta. \\
			\hline
			\textit{Pericolosità} & Alta. \\
			\hline
			\textit{Precauzioni} & Dichiarando le conoscenze preliminari riguardo le tecnologie al \textit{Responsabile di Progetto}, quest'ultimo cercherà di spartire i compiti in modo più congruo.
			I membri del gruppo dovranno tempestivamente avvisare i colleghi in caso di difficoltà.  \\
			\hline
			\textit{Linee Guida} & Dopo che ciascun membro si sarà documentato autonomamente, tramite il materiale fornito dal committente, assieme al \textit{Responsabile di progetto} si distribuiranno i compiti e si cercherà di far lavorare fra loro tutti i membri in modo da sanare le eventuali lacune.
			Inoltre, per i tasks più complessi, si cercherà di distribuirgli a più membri in modo da favorire l'avanzamento del lavoro.  \\
			\hline
		\end{tabularx}
	\captionof{table}{\textbf{Analisi dei rischi delle tecnologie utilizzate}}	
	\end{center}

\quad
\begin{center}
	\LARGE\textbf{Rischi legati all'organizzazione}
\end{center}

\def\tabularxcolumn#1{m{#1}}
{\rowcolors{2}{RawSienna!90!RawSienna!20}{RawSienna!70!RawSienna!40}
	
	\begin{center}
		\renewcommand{\arraystretch}{1.4}
		\begin{tabularx}{\textwidth}{|c|X|}
			\hline
			\rowcolor{airforceblue}
			\multicolumn{2}{|c|}{\textit{Problemi organizzativi}}\\
			\hline
			\textit{Codice} & RO1 \\
			\hline
			\textit{Descrizione} & I problemi organizzativi possono essere scaturiti da vari motivi. Sia dai membri che dal committente, così come dagli impegni personali e dai periodi vacanzieri.  \\
			\hline
			\textit{Conseguenza} & Questi problemi possono far ritardare il completamento dei tasks di un tempo più o meno definito, rendendo l'avanzamento più lento o addirittura bloccandolo. \\
			\hline
			\textit{Possibilità di occorrenza} & Alta. \\
			\hline
			\textit{Pericolosità} & Alta. \\
			\hline
			\textit{Precauzioni} & Ogni membro del gruppo di lavoro dovrà avvisare il \textit{Responsabile di progetto} nel caso in cui, per cause di forza maggiore, non si riesca a completare il task nel tempo deciso oppure non si riesca proprio a farlo. \\
			\hline
			\textit{Linee Guida} & Il \textit{Responsabile di progetto} avrà il compito di riassegnare i compiti in modo da riuscire a completare i tasks nel tempo stimato, in modo da non avere ritardi nel concluderli.
			Nel caso in cui sia il committente a creare questi disagi organizzativi, sarà sempre premura del \textit{Responsabile di progetto} di risolvere il problema mediante i canali di comunicazione adatti.  \\
			\hline
		\end{tabularx}
	\captionof{table}{\textbf{Analisi dei rischi dei problemi organizzativi}}
	\end{center}

\quad
\begin{center}
	\LARGE\textbf{Rischi interpersonali}
\end{center}

\def\tabularxcolumn#1{m{#1}}
{\rowcolors{2}{RawSienna!90!RawSienna!20}{RawSienna!70!RawSienna!40}
	
	\begin{center}
		\renewcommand{\arraystretch}{1.4}
		 
		\begin{tabularx}{\textwidth}{|c|X|}
			\hline
			\rowcolor{airforceblue}
			\multicolumn{2}{|c|}{\textit{Problemi di relazione tra i membri}}\\
			\hline
			\textit{Codice} & RI1 \\
			\hline
			\textit{Descrizione} & Problemi legati ai contrasti che potrebbero intercorrere tra i membri. \\
			\hline
			\textit{Conseguenza} & Difficoltà di avanzamento nel lavoro, poca collaborazione tra i membri in contrasto, malumore nel gruppo. \\
			\hline
			\textit{Possibilità di occorrenza} & Media \\
			\hline
			\textit{Pericolosità} & Alta. \\
			\hline
			\textit{Precauzioni} & In caso di nascite di contrasti tra i membri, questi dovranno immediatamente coinvolgere il \textit{Responsabile di progetto} in modo da poter subito risolvere la diatriba.
			In caso non riesca a placare gli animi comunicherà col \textit{prof. Vardanega} per la risoluzione dei problemi. \\
			\hline
			\textit{Linee guida} & I membri dovranno impegnarsi nel ridurre al minimo eventuali contrasti tra loro per favorire l'avanzamento dei lavori e per realizzare al meglio il progetto.\\
			\hline
		\end{tabularx}
	\captionof{table}{\textbf{Analisi dei rischi dei problemi relazionali}}
	\end{center}


