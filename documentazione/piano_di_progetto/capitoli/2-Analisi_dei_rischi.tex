\chapter{Analisi dei rischi}

\section{Piano per la gestione dei rischi}
Con l'intento di prevenire il naturale insorgere di problemi durante lo svolgimento del progetto è stato elaborato un'approfondito piano per la gestione dei rischi. Quest'ultimo è suddiviso in quattro attività:
\begin{itemize}
  \item \textbf{Individuazione dei rischi:} attività di identificazione e documentazione di possibili elementi problematici che possano ostacolare il naturale percorso del progetto;
  \item \textbf{Analisi dei rischi:} attività di analisi dei fattori di rischio, che si articola in probabilità di occorrenza, indice di gravità e conseguente impatto sul progetto;
  \item \textbf{Pianificazione di controllo:} attività di pianificazione delle misure da adottare per la prevenzione e contenimento del problema;
  \item \textbf{Monitoraggio dei rischi:} attività di controllo dei rischi che accompagna tutto lo svolgimento del progetto, al fine di evitarli o agire tempestivamente alla loro occorrenza per contenerne i danni.
\end{itemize}
Le principali tipologie di rischio sono state quindi codificate e categorizzate come segue:
\begin{itemize}
  \item \textbf{RT:} Rischi legati alle tecnologie;
  \item \textbf{RO:} Rischi legati all'organizzazione;
  \item \textbf{RI:} Rischi interpersonali, ovvero rischi legati alle relazioni personali interne ed esterne o alla disponibilità e risorse dei componenti.
\end{itemize}
