\chapter{Analisi dei rischi}\label{AnalisiDeiRischi}

\section{Piano per la gestione dei rischi}\label{AnalisiDeiRischiPianoPerLaGestioneDeiRischi}
Con l'intento di prevenire il naturale insorgere di problemi durante lo svolgimento del progetto è stato elaborato un'approfondito piano per la gestione dei rischi. Quest'ultimo è suddiviso in quattro attività$_{\scaleto{G}{3pt}}$:
\begin{itemize}
  \item \textbf{Individuazione dei rischi:} attività$_{\scaleto{G}{3pt}}$ di identificazione e documentazione di possibili elementi problematici che possano ostacolare il naturale percorso del progetto;
  \item \textbf{Analisi dei rischi:} attività$_{\scaleto{G}{3pt}}$ di analisi dei fattori di rischio, che si articola in probabilità di occorrenza, indice di gravità e conseguente impatto sul progetto;
  \item \textbf{Pianificazione di controllo:} attività$_{\scaleto{G}{3pt}}$ di pianificazione delle misure da adottare per la prevenzione e contenimento del problema;
  \item \textbf{Monitoraggio dei rischi:} attività$_{\scaleto{G}{3pt}}$ di controllo dei rischi che accompagna tutto lo svolgimento del progetto, al fine di evitarli o agire tempestivamente alla loro occorrenza per contenerne i danni.
\end{itemize}
Le principali tipologie di rischio sono state quindi codificate e categorizzate come segue:
\begin{itemize}
  \item \textbf{RT:} Rischi legati alle tecnologie;
  \item \textbf{RO:} Rischi legati all'organizzazione;
  \item \textbf{RI:} Rischi interpersonali, ovvero legati alle relazioni personali interne ed esterne o alla disponibilità e risorse dei componenti.
\end{itemize}

\quad
\begin{center}
	\LARGE\textbf{Rischi legati alle tecnologie}
\end{center}

\def\tabularxcolumn#1{m{#1}}
{\rowcolors{2}{RawSienna!90!RawSienna!20}{RawSienna!70!RawSienna!40}

	\begin{center}
		\renewcommand{\arraystretch}{1.4}
		\begin{longtable}{|p{5cm}|p{12cm}|}
			\hline
			\rowcolor{airforceblue}
			\multicolumn{2}{|c|}{\textit{Inesperienza tecnologica}}\\
			\hline
			\textit{Codice} & RT1 \\
			\hline
			\textit{Descrizione} & Alcune tecnologie utilizzate in questo progetto sono nuove per tutti i membri del gruppo di lavoro. \\
			\hline
			\textit{Conseguenza} & Lo studio e l'apprendimento di tali tecnologie potrebbero richiedere un intervallo di tempo difficile da quantificare, maggiore del previsto e variabile da membro a membro con conseguenti difficoltà operative. \\
			\hline
			\textit{Possibilità di occorrenza} & Alta. \\
			\hline
			\textit{Pericolosità} & Alta. \\
			\hline
			\textit{Precauzioni} & Il \textit{Responsabile di Progetto} dovrà suddividere i compiti$_G$ nel modo più congruo possibile, considerando le conoscenze preliminari di ciascun componente; prevederà inoltre un tempo di Slack$_G$ maggiore per i compiti$_{\scaleto{G}{3pt}}$ assegnati ad un componente senza particolare famigliarità con la relativa tecnologia. Il \textit{Responsabile di Progetto} assegnerà i task$_G$ di maggiore complessità a più membri ove necessario.  \\
			\hline
			\textit{Piano di contingenza} & Ciascun membro comunicherà il prima possibile al \textit{Responsabile di progetto} la previsione di un eventuale ritardo o mancanza; egli provvederà a ridistribuire i compiti$_{\scaleto{G}{3pt}}$ se necessario in modo da sanare eventuali lacune o sottostime. \\
			\hline
		\end{longtable}
	\captionof{table}{\textbf{Analisi dei rischi delle tecnologie utilizzate}}
	\end{center}


\def\tabularxcolumn#1{m{#1}}
{\rowcolors{2}{RawSienna!90!RawSienna!20}{RawSienna!70!RawSienna!40}

	\begin{center}
		\renewcommand{\arraystretch}{1.4}
		\begin{longtable}{|p{5cm}|p{12cm}|}
			\hline
			\rowcolor{airforceblue}
			\multicolumn{2}{|c|}{\textit{Software terze parti}}\\
			\hline
			\textit{Codice} & RT2 \\
			\hline
			\textit{Descrizione} & Eventuali problematiche con software di terze parti, quali la mancanza di documentazione o problemi tecnici, sono indipendenti dai membri del gruppo. \\
			\hline
			\textit{Conseguenza} & Ciò causerebbe ritardi pesanti sul proseguo del lavoro e anche possibili ritardi sulla consegna.
			La necessità di cambiare tecnologia potrebbe richiedere molto tempo e risorse per la ricerca di una sostituzione. \\
			\hline
			\textit{Possibilità di occorrenza} & Bassa. \\
			\hline
			\textit{Pericolosità} & Alta. \\
			\hline
			\textit{Precauzioni} & Il gruppo sceglierà i software più stabili e documentati per evitare questi tipi di problemi.  \\
			\hline
			\textit{Piano di contingenza} & Assieme al \textit{Responsabile di progetto} il gruppo di lavoro si attiverà al fine di tentare di risolvere il problema. Se ciò non è possibile sarà necessario un cambio di tecnologia, anche tramite l'aiuto del proponente$_{\scaleto{G}{3pt}}$.  \\
			\hline
		\end{longtable}
		\captionof{table}{\textbf{Analisi dei rischi dei software di terze parti}}
	\end{center}


\def\tabularxcolumn#1{m{#1}}
{\rowcolors{2}{RawSienna!90!RawSienna!20}{RawSienna!70!RawSienna!40}

	\begin{center}
		\renewcommand{\arraystretch}{1.4}
		\begin{longtable}{|p{5cm}|p{12cm}|}
			\hline
			\rowcolor{airforceblue}
			\multicolumn{2}{|c|}{\textit{Validità dei dati}}\\
			\hline
			\textit{Codice} & RT3 \\
			\hline
			\textit{Descrizione} & Problemi legati alla validità e all'elaborazione dei dati. \\
			\hline
			\textit{Conseguenza} & Arresto obbligato del lavoro in corso, con possibilità di invalidazione del lavoro svolto fino a quel momento. \\
			\hline
			\textit{Possibilità di occorrenza} & Medio/Alta. \\
			\hline
			\textit{Pericolosità} & Molto alta. \\
			\hline
			\textit{Precauzioni} & Prima dell'inizio della raccolta dati il gruppo si assicurerà che la fonte sia affidabile e coerente.
			Questa operazione sarà svolta per prima in quanto critica per l'intero sviluppo.  \\
			\hline
			\textit{Piano di contingenza} & Il gruppo, insieme al proponente$_{\scaleto{G}{3pt}}$, valuterà se sarà necessario cambiare solo la fonte di provenienza dei dati oppure simularli in maniera consona. \\
			\hline
		\end{longtable}
		\captionof{table}{\textbf{Analisi dei rischi della validità dei dati}}
	\end{center}

\def\tabularxcolumn#1{m{#1}}
{\rowcolors{2}{RawSienna!90!RawSienna!20}{RawSienna!70!RawSienna!40}
	
	\begin{center}
		\renewcommand{\arraystretch}{1.4}
		\begin{longtable}{|p{5cm}|p{12cm}|}
			\hline
			\rowcolor{airforceblue}
			\multicolumn{2}{|c|}{\textit{Malfunzionamenti hardware o software dei pc dei membri del gruppo}}\\
			\hline
			\textit{Codice} & RT4 \\
			\hline
			\textit{Descrizione} & Può accadere che qualcuno abbia il PC non funzionante o in assistenza e non possa contribuire attivamente alla realizzazione del prodotto. \\
			\hline
			\textit{Conseguenza} & Ciò causerebbe una mole di lavoro maggiore per gli altri componenti del gruppo ed eventuali ritardi.\\
			\hline
			\textit{Possibilità di occorrenza} & Bassa. \\
			\hline
			\textit{Pericolosità} & Media. \\
			\hline
			\textit{Precauzioni} & Ogni membro del team deve monitorare il corretto funzionamento del proprio PC.  \\
			\hline
			\textit{Piano di contingenza} & A seconda della gravità del problema si provvederà alla reinstallazione del software, del sistema operativo o della sostituzione della propria macchina. \\
			\hline
		\end{longtable}
		\captionof{table}{\textbf{Analisi dei rischi del malfunzionamento del PC}}
	\end{center}
\clearpage
\quad
\begin{center}
	\LARGE\textbf{Rischi legati all'organizzazione}
\end{center}

\def\tabularxcolumn#1{m{#1}}
{\rowcolors{2}{RawSienna!90!RawSienna!20}{RawSienna!70!RawSienna!40}

	\begin{center}
		\renewcommand{\arraystretch}{1.4}
		\begin{longtable}{|p{5cm}|p{12cm}|}
			\hline
			\rowcolor{airforceblue}
			\multicolumn{2}{|c|}{\textit{Contrasti tra i componenti}}\\
			\hline
			\textit{Codice} & RO1 \\
			\hline
			\textit{Descrizione} & I componenti del gruppo devono cooperare con professionalità. \\
			\hline
			\textit{Conseguenza} & Tensioni o contrasti tra i componenti, sfavorendo il corretto proseguimento del progetto. \\
			\hline
			\textit{Possibilità di occorrenza} & Bassa. \\
			\hline
			\textit{Pericolosità} & Alta. \\
			\hline
			\textit{Precauzioni} & Ogni membro del gruppo di lavoro cercherà di essere più comprensibile e limitare così eventuali tensioni a favore del collettivo. \\
			\hline
			\textit{Piano di contingenza} & Il \textit{Responsabile di Progetto} avrà l'incarico di mediatore in tali controversie.
			Eventualmente, insieme al resto del gruppo si cercherà di sanare le discordie e solamente in casi estremi verrà chiamato in causa il Prof. Tullio Vardanega.  \\
			\hline
		\end{longtable}
	\captionof{table}{\textbf{Analisi dei rischi per i contrasti tra i componenti}}
	\end{center}


\def\tabularxcolumn#1{m{#1}}
{\rowcolors{2}{RawSienna!90!RawSienna!20}{RawSienna!70!RawSienna!40}

	\begin{center}
		\renewcommand{\arraystretch}{1.4}
		\begin{longtable}{|p{5cm}|p{12cm}|}
			\hline
			\rowcolor{airforceblue}
			\multicolumn{2}{|c|}{\textit{Impegni personali}}\\
			\hline
			\textit{Codice} & RO2 \\
			\hline
			\textit{Descrizione} & Può presentarsi la possibilità che in alcuni momenti uno o più componenti del gruppo abbiano degli impegni accademici o personali. \\
			\hline
			\textit{Conseguenza} & Rallentamento del lavoro.\\
			\hline
			\textit{Possibilità di occorrenza} & Media. \\
			\hline
			\textit{Pericolosità} & Media. \\
			\hline
			\textit{Precauzioni} & Essenziale sarà il comunicare gli impegni al Responsabile appena il componente ne viene a conoscenza. \\
			\hline
			\textit{Piano di contingenza} & Il Responsabile provvederà ad approvare delle modifiche organizzative per evitare o limitare rallentamenti ai lavori. \\
			\hline
		\end{longtable}
		\captionof{table}{\textbf{Analisi dei rischi sugli impegni personali}}
	\end{center}

\def\tabularxcolumn#1{m{#1}}
{\rowcolors{2}{RawSienna!90!RawSienna!20}{RawSienna!70!RawSienna!40}
	
	\begin{center}
		\renewcommand{\arraystretch}{1.4}
		\begin{longtable}{|p{5cm}|p{12cm}|}
			\hline
			\rowcolor{airforceblue}
			\multicolumn{2}{|c|}{\textit{Calcolo dei tempi e dei costi}}\\
			\hline
			\textit{Codice} & RO3 \\
			\hline
			\textit{Descrizione} & E' possibile che i tempi e i costi preventivati si rivelino imprecisi con l'avanzamento del progetto. \\
			\hline
			\textit{Conseguenza} & Costi preventivati sbagliati. \\
			\hline
			\textit{Possibilità di occorrenza} & Alta. \\
			\hline
			\textit{Pericolosità} & Alta. \\
			\hline
			\textit{Precauzioni} & Nel caso in cui un componente riscontri una differenza dalle ore di lavoro preventivate, dovrà farlo presente al Responsabile.  \\
			\hline
			\textit{Piano di contingenza} & Nel caso in cui una stima oraria risulti non sufficiente per portare a termine la consegna, il Responsabile provvederà ad assegnare più risorse in modo da limitare rallentamenti. Eventualmente se ci dovessero essere lo stesso variazioni al preventivo, allora il Responsabile provvederà a comunicarlo al Committente$_{\scaleto{G}{3pt}}$ \\
			\hline
		\end{longtable}
		\captionof{table}{\textbf{Analisi dei rischi del calcolo dei tempi e dei costi}}
	\end{center}

\def\tabularxcolumn#1{m{#1}}
{\rowcolors{2}{RawSienna!90!RawSienna!20}{RawSienna!70!RawSienna!40}
	
	\begin{center}
		\renewcommand{\arraystretch}{1.4}
		\begin{longtable}{|p{5cm}|p{12cm}|}
			\hline
			\rowcolor{airforceblue}
			\multicolumn{2}{|c|}{\textit{Inesperienza nel coordinamento}}\\
			\hline
			\textit{Codice} & RO4 \\
			\hline
			\textit{Descrizione} & I membri non hanno esperienza di lavoro che richieda il coordinamento di sette persone.\\
			\hline
			\textit{Conseguenza} & Problematiche o ritardi a causa di una mancata o scarsa organizzazione del team con tempi di latenza e compiti svolti più volte da membri differenti. \\
			\hline
			\textit{Possibilità di occorrenza} & Alta. \\
			\hline
			\textit{Pericolosità} & Alta. \\
			\hline
			\textit{Precauzioni} & Il Responsabile di Progetto deve, insieme al resto del gruppo, pianificare le mansioni. Si cercherà di avere una rotazione dei ruoli in modo da far collaborare tutti in modo che ciascuna attività venga svolta dai componenti con più esperienza, insieme a quelli che ancora non ne hanno.  \\
			\hline
			\textit{Piano di contingenza} & Qualunque difficoltà sarà notificata al Responsabile di Progetto, che dopo essersi consultato con il gruppo, provvederà eventualmente ad assegnare un compito più semplice all'interessato. \\
			\hline
		\end{longtable}
		\captionof{table}{\textbf{Analisi dei rischi per inesperienza nel coordinamento}}
	\end{center}

\def\tabularxcolumn#1{m{#1}}
{\rowcolors{2}{RawSienna!90!RawSienna!20}{RawSienna!70!RawSienna!40}
	
	\begin{center}
		\renewcommand{\arraystretch}{1.4}
		\begin{longtable}{|p{5cm}|p{12cm}|}
			\hline
			\rowcolor{airforceblue}
			\multicolumn{2}{|c|}{\textit{Scarsa comunicazione}}\\
			\hline
			\textit{Codice} & RO5 \\
			\hline
			\textit{Descrizione} & Per avanzare nelle attività pianificate e rispettare le scadenze, è necessaria una comunicazione costante tra tutti i membri del gruppo.\\
			\hline
			\textit{Conseguenza} & Problematiche o ritardi a causa di una scarsa comunicazione del team. \\
			\hline
			\textit{Possibilità di occorrenza} & Medio. \\
			\hline
			\textit{Pericolosità} & Alta. \\
			\hline
			\textit{Precauzioni} & Il Responsabile di Progetto provvederà a promuovere un adeguato livello di comunicazione tra i vari componenti del gruppo.  \\
			\hline
			\textit{Piano di contingenza} & Nel caso si rilevi una scarsa comunicazione sarà compito del Responsabile di Progetto provvedere a risolverlo, aiutandosi attraverso una riunione interna al gruppo per discuterne la situazione. \\
			\hline
		\end{longtable}
		\captionof{table}{\textbf{Analisi dei rischi per scarsa comunicazione}}
	\end{center}


\def\tabularxcolumn#1{m{#1}}
{\rowcolors{2}{RawSienna!90!RawSienna!20}{RawSienna!70!RawSienna!40}
	
	\begin{center}
		\renewcommand{\arraystretch}{1.4}
		\begin{longtable}{|p{5cm}|p{12cm}|}
			\hline
			\rowcolor{airforceblue}
			\multicolumn{2}{|c|}{\textit{Approvazione errata dei documenti}}\\
			\hline
			\textit{Codice} & RO6 \\
			\hline
			\textit{Descrizione} & E' possibile che il Responsabile durante l'approvazione non si accorga o commetta alcuni errori.\\
			\hline
			\textit{Conseguenza} & Approvazione e possibile consegna di documenti errati. \\
			\hline
			\textit{Possibilità di occorrenza} & Bassa. \\
			\hline
			\textit{Pericolosità} & Alta. \\
			\hline
			\textit{Precauzioni} & Per ogni documento devono essere eseguiti controlli costanti, in modo che sia possibile identificare in maniera tempestiva gli eventuali errori.  \\
			\hline
			\textit{Piano di contingenza} & Il responsabile si dovrà occupare di controllare che i documenti da approvare siano effettivamente validi. \\
			\hline
		\end{longtable}
		\captionof{table}{\textbf{Analisi dei rischi per approvazione errata dei documenti}}
	\end{center}

\def\tabularxcolumn#1{m{#1}}
{\rowcolors{2}{RawSienna!90!RawSienna!20}{RawSienna!70!RawSienna!40}
	
	\begin{center}
		\renewcommand{\arraystretch}{1.4}
		\begin{longtable}{|p{5cm}|p{12cm}|}
			\hline
			\rowcolor{airforceblue}
			\multicolumn{2}{|c|}{\textit{Analisi dei requisiti imperfetta}}\\
			\hline
			\textit{Codice} & RO7 \\
			\hline
			\textit{Descrizione} & E' possibile che a causa dell'inesperienza del gruppo venga prodotta un'Analisi dei Requisiti insoddisfacente.\\
			\hline
			\textit{Conseguenza} & La proposta potrebbe risultare inadeguata alle aspettative. \\
			\hline
			\textit{Possibilità di occorrenza} & Media. \\
			\hline
			\textit{Pericolosità} & Alta. \\
			\hline
			\textit{Precauzioni} & Ogni dubbio verrà discusso con il proponente$_G$.  \\
			\hline
			\textit{Piano di contingenza} & Qualsiasi errore verrà corretto con la massima priorità. \\
			\hline
		\end{longtable}
		\captionof{table}{\textbf{Analisi dei rischi per l'analisi dei requisiti imperfetta}}
	\end{center}

\quad
\begin{center}
	\LARGE\textbf{Rischi interpersonali}
\end{center}

\def\tabularxcolumn#1{m{#1}}
{\rowcolors{2}{RawSienna!90!RawSienna!20}{RawSienna!70!RawSienna!40}

	\begin{center}
		\renewcommand{\arraystretch}{1.4}

		\begin{longtable}{|p{5cm}|p{12cm}|}
			\hline
			\rowcolor{airforceblue}
			\multicolumn{2}{|c|}{\textit{Comunicazione interna}}\\
			\hline
			\textit{Codice} & RI1 \\
			\hline
			\textit{Descrizione} & Potrebbero esserci momenti nei quali uno o più componenti non sono reperibili. \\
			\hline
			\textit{Conseguenza} & Rallentamenti del lavoro qualora non si riuscisse a comunicare con la persona interessata. \\
			\hline
			\textit{Possibilità di occorrenza} & Bassa. \\
			\hline
			\textit{Pericolosità} & Alta. \\
			\hline
			\textit{Precauzioni} & E' necessario che ciascun componente riferisca tempestivamente al Responsabile eventuali momenti nei quali potrebbe non essere reperibile. \\
			\hline
			\textit{Piano di contingenza} & E' stato concordato con il gruppo di svolgere riunioni frequenti per comunicare l'avanzamento del lavoro.\\
			\hline
		\end{longtable}
	\captionof{table}{\textbf{Analisi dei rischi della comunicazione interna}}
	\end{center}

\def\tabularxcolumn#1{m{#1}}
{\rowcolors{2}{RawSienna!90!RawSienna!20}{RawSienna!70!RawSienna!40}
	
	\begin{center}
		\renewcommand{\arraystretch}{1.4}
		
		\begin{longtable}{|p{5cm}|p{12cm}|}
			\hline
			\rowcolor{airforceblue}
			\multicolumn{2}{|c|}{\textit{Comunicazione esterna}}\\
			\hline
			\textit{Codice} & RI2 \\
			\hline
			\textit{Descrizione} & Potrebbero esserci momenti nel quale l'azienda Proponente$_G$ non è reperibile qualora avessimo necessita di contattarla. \\
			\hline
			\textit{Conseguenza} & Rallentamenti del lavoro qualora non si riuscisse a comunicare. \\
			\hline
			\textit{Possibilità di occorrenza} & Bassa. \\
			\hline
			\textit{Pericolosità} & Media. \\
			\hline
			\textit{Precauzioni} & E' stato creato un canale sulla piattaforma Discord$_G$ per poter comunicare con il Proponente$_G$ in maniera facile e rapida. \\
			\hline
			\textit{Piano di contingenza} & Qualora si presentasse la necessità di organizzare un incontro con il Proponente$_G$ è sufficiente richiederlo ed accordarsi con la disponibilità.\\
			\hline
		\end{longtable}
		\captionof{table}{\textbf{Analisi dei rischi della comunicazione esterna}}
	\end{center}

\def\tabularxcolumn#1{m{#1}}
{\rowcolors{2}{RawSienna!90!RawSienna!20}{RawSienna!70!RawSienna!40}
	
	\begin{center}
		\renewcommand{\arraystretch}{1.4}
		
		\begin{longtable}{|p{5cm}|p{12cm}|}
			\hline
			\rowcolor{airforceblue}
			\multicolumn{2}{|c|}{\textit{Stato di malattia}}\\
			\hline
			\textit{Codice} & RI3 \\
			\hline
			\textit{Descrizione} & Uno o più membri del gruppo possono ammalarsi. \\
			\hline
			\textit{Conseguenza} & Può influire sui task$_G$ assegnati. \\
			\hline
			\textit{Possibilità di occorrenza} & Medio/Alta. \\
			\hline
			\textit{Pericolosità} & Media. \\
			\hline
			\textit{Precauzioni} & Il membro del gruppo comunica il proprio stato di poca salute. \\
			\hline
			\textit{Piano di contingenza} & Se la malattia impedisce di lavorare, il componente del gruppo è tenuto a riprendere uno stato di salute ottimale, ed il suo lavoro è ridistribuito tra gli altri componenti del gruppo.\\
			\hline
		\end{longtable}
		\captionof{table}{\textbf{Analisi dei rischi dello stato di malattia}}
	\end{center}

\def\tabularxcolumn#1{m{#1}}
{\rowcolors{2}{RawSienna!90!RawSienna!20}{RawSienna!70!RawSienna!40}
	
	\begin{center}
		\renewcommand{\arraystretch}{1.4}
		
		\begin{longtable}{|p{5cm}|p{12cm}|}
			\hline
			\rowcolor{airforceblue}
			\multicolumn{2}{|c|}{\textit{Stress mentale}}\\
			\hline
			\textit{Codice} & RI4 \\
			\hline
			\textit{Descrizione} & Gli orari di lavoro eccessivi e la situazione di pandemia possono portare a frustazione e disagio. \\
			\hline
			\textit{Conseguenza} & Sintomi fisici di stress ed emotività instabile. \\
			\hline
			\textit{Possibilità di occorrenza} & Media. \\
			\hline
			\textit{Pericolosità} & Media. \\
			\hline
			\textit{Precauzioni} & Il gruppo collabora per creare un clima e delle relazioni all'interno per aiutare a gestire eventuali problematiche. L'appoggio anche se virtuale aiuta a sentirsi parte di un gruppo. \\
			\hline
			\textit{Piano di contingenza} & Per rilasciare la tensione si cerca di fare attività fisica costante.\\
			\hline
		\end{longtable}
		\captionof{table}{\textbf{Analisi dei rischi dello stress mentale}}
	\end{center}

