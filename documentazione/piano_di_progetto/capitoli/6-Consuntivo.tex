\chapter{Consuntivo}\label{Consuntivo}
Di seguito vengono indicate le spese sostenute dal gruppo confrontandole con quanto preventivato. Il bilancio potrà essere:
\begin{itemize}
	\item positivo: la spesa effettiva è minore di quanto preventivato;
	\item pari: la spesa effettiva è uguale a quanto preventivato;
	\item negativo: la spesa effettiva è maggiore di quanto preventivato.
\end{itemize}
\section{Periodo di analisi}\label{ConsuntivoPeriodoDiAnalisi}
Le ore di lavoro che sono state sostenute durante la fase di analisi sono considerate come ore di investimento per questo motivo esse non vengono rendicontate.
\quad
\def\tabularxcolumn#1{m{#1}}
{\rowcolors{2}{RawSienna!90!RawSienna!20}{RawSienna!70!RawSienna!40}
	\begin{center}
		\renewcommand{\arraystretch}{1.4}
		\begin{tabularx}{10cm}{|X|c|c|}
			\hline
			\rowcolor{airforceblue}
			\textbf{Ruolo} & \textbf{Ore} & \textbf{Costo}\\
			\hline
			Responsabile & 26(+0) & 780\euro(+0\euro)\\
			\hline
			Amministratore & 42(+0) & 840\euro(+0\euro)\\
			\hline
			Analista & 49(+15) & 1225\euro(+375\euro)\\
			\hline
			Progettista & 0(+0) & 0\euro(+0\euro)\\
			\hline
			Programmatore & 0(+0) & 0\euro(+0\euro)\\
			\hline
			Verificatore & 33(+10) & 495\euro(+150\euro)\\
			\hline
			\textbf{Totale Preventivo} & \textbf{150} & \textbf{3340\euro}\\
			\hline
			\textbf{Totale Consultivo} & \textbf{175} & \textbf{3865\euro}\\
			\hline
			\textbf{Differenza} & \textbf{25} & \textbf{525\euro}
		\end{tabularx}
		\captionof{table}{\textbf{Consuntivo della fase di Analisi}}
	\end{center}

\subsection{Conclusioni}\label{ConsuntivoPeriodoDiAnalisiConclusioni}
Come emerso dalla tabella precedente il bilancio risulta negativo in quanto il gruppo ha ritenuto necessario impiegare più tempo del previsto nei ruoli di \textit{Analista} e \textit{Verificatore}. Il motivo di tale ritardo è:
\begin{itemize}
	\item l'individuazione dei requisiti é risultata più complessa del previsto;
	\item la grande quantità di lavoro nel revisionare i documenti in quanto essendo un processo nuovo ed ogni componente ha dovuto imparare come farlo in maniera corretta, efficacie ed efficiente.
\end{itemize}

\subsection{Preventivo a finire}\label{ConsuntivoPeriodoDiAnalisiPreventivoAFinire}
Il preventivo a finire, nonostante in questa fase siano state necessarie più ore del previsto, è in linea con quanto descritto nella sezione precedente. Il gruppo non ritiene il surplus di 500\euro{} un problema in quanto le ore lavorative e i costi sostenuti in questa fase non verranno rendicontati. Per questo motivo il gruppo ha deciso di non prendere alcuna contromisura nella pianificazione futura.

\section{Periodo di consolidamento dei requisiti}\label{ConsuntivoPeriodoDiConsolidamentoDeiRequisiti}
Le ore di lavoro calcolate per questo periodo sono considerate come ore di investimento e per tale motivo non vengono rendicontate.

\quad
\def\tabularxcolumn#1{m{#1}}
{\rowcolors{2}{RawSienna!90!RawSienna!20}{RawSienna!70!RawSienna!40}
	\begin{center}
		\renewcommand{\arraystretch}{1.4}
		\begin{tabularx}{10cm}{|X|c|c|}
			\hline
			\rowcolor{airforceblue}
			\textbf{Ruolo} & \textbf{Ore} & \textbf{Costo}\\
			\hline
			Responsabile & 4(+0) & 120\euro(+0\euro)\\
			\hline
			Amministratore & 8(+0) & 160\euro(+0\euro)\\
			\hline
			Analista & 4(+0) & 100\euro(+0\euro)\\
			\hline
			Progettista & 0(+0) & 0\euro(+0\euro)\\
			\hline
			Programmatore & 0(+0) & 0\euro(+0\euro)\\
			\hline
			Verificatore & 8(0) & 120\euro(+0\euro)\\
			\hline
			\textbf{Totale Preventivo} & \textbf{24} & \textbf{500\euro}\\
			\hline
			\textbf{Totale Consultivo} & \textbf{24} & \textbf{500\euro}\\
			\hline
			\textbf{Differenza} & \textbf{0} & \textbf{0\euro}
		\end{tabularx}
		\captionof{table}{\textbf{Consuntivo della fase di Consolidamento dei requisiti}}
	\end{center}

\subsection{Conclusioni}\label{ConsuntivoPeriodoDiConsolidamentoDeiRequisitiConclusioni}
Grazie al minor carico di lavoro le ore preventivate sono state rispettate quindi non è presente alcuna differenza rispetto alle ore effettive di lavoro. Inoltre il gruppo è riuscito a procedere senza alcun problema con lo studio personale per lo svolgimento della fase successiva del lavoro.

\subsection{Preventivo a finire}\label{ConsuntivoPeriodoDiConsolidamentoDeiRequisitiPreventivoAFinire}
In quanto le ore di lavoro previste sono state rispettate il preventivo a finire risulta coerente con quello previsto.

\section{Periodo di progettazione architetturale}\label{ConsuntivoPeriodoDiProgettazioneArchitetturale}

Il gruppo ha suddiviso questo periodo in diverse parti per organizzare meglio il lavoro, di conseguenza il consuntivo viene analizzato in funzione di ogni sua parte.

\subsection{Periodo di progettazione architetturale - Incremento e Verifica}\label{ConsuntivoPeriodoDiProgettazioneArchitetturaleIncrementoEVerifica}

Le ore dedicate in questo periodo sono atte al completamento della fase di Incremento e Verifica descritta nella sezione \S~\ref{PianificazioneProgettazioneArchitetturale}.

\quad
\def\tabularxcolumn#1{m{#1}}
{\rowcolors{2}{RawSienna!90!RawSienna!20}{RawSienna!70!RawSienna!40}
	\begin{center}
		\renewcommand{\arraystretch}{1.4}
		\begin{tabularx}{10cm}{|X|c|c|}
			\hline
			\rowcolor{airforceblue}
			\textbf{Ruolo} & \textbf{Ore} & \textbf{Costo}\\
			\hline
			Responsabile & 7(+0) & 210\euro(+0\euro)\\
			\hline
			Amministratore & 7(+0) & 140\euro(+0\euro)\\
			\hline
			Analista & 8(+0) & 200\euro(+0\euro)\\
			\hline
			Progettista & 2(+0) & 44\euro(+0\euro)\\
			\hline
			Programmatore & 0(+0) & 0\euro(+0\euro)\\
			\hline
			Verificatore & 15(+15) & 225\euro(+225\euro)\\
			\hline
			\textbf{Totale Preventivo} & \textbf{39} & \textbf{819\euro}\\
			\hline
			\textbf{Totale Consultivo} & \textbf{54} & \textbf{1044\euro}\\
			\hline
			\textbf{Differenza} & \textbf{35} & \textbf{225\euro}
		\end{tabularx}
		\captionof{table}{\textbf{Consuntivo della fase di Incremento e Verifica}}
	\end{center}

\subsubsection{Conclusioni}
Come emerso dalla tabella precedente il bilancio risulta negativo in quanto il gruppo ha ritenuto necessario impiegare più tempo del previsto nel ruolo di \textit{Verificatore} poiché ha dovuto procedere nella correzione di errori sollevati alla prima consegna. (voglio dire che abbiamo corretto i difetti gravi segnalati da tullio ed è per questo che il verificatore ha queste ore in più ma non so come dirlo in modo carino e formale)

\subsubsection{Preventivo a finire}
Il preventivo a finire presenta un surplus di 225\euro. Per questo motivo il gruppo ha deciso di tamponare il problema cercando di rispettare le ore preventivate nelle fasi successive in modo tale da non sforare troppo dal preventivo inizialmente previsto.

\subsection{Periodo di progettazione architetturale - Technology Baseline (Primo Incremento)}\label{ConsuntivoPeriodoDiProgettazioneArchitetturaleTechnologyBaselinePrimoIncremento}

Le ore dedicate in questo periodo sono atte al completamento del primo incremento della fase di Technology Baseline$_{\scaleto{G}{3pt}}$ descritta nella sezione \S~\ref{PianificazioneProgettazioneArchitetturale}.

\quad
\def\tabularxcolumn#1{m{#1}}
{\rowcolors{2}{RawSienna!90!RawSienna!20}{RawSienna!70!RawSienna!40}
	\begin{center}
		\renewcommand{\arraystretch}{1.4}
		\begin{tabularx}{10cm}{|X|c|c|}
			\hline
			\rowcolor{airforceblue}
			\textbf{Ruolo} & \textbf{Ore} & \textbf{Costo}\\
			\hline
			Responsabile & 6(+0) & 180\euro(+0\euro)\\
			\hline
			Amministratore & 7(+0) & 140\euro(+0\euro)\\
			\hline
			Analista & 7(+0) & 175\euro(+0\euro)\\
			\hline
			Progettista & 42(+10) & 924\euro(+220\euro)\\
			\hline
			Programmatore & 4(+0) & 60\euro(+0\euro)\\
			\hline
			Verificatore & 14(+0) & 210\euro(+0\euro)\\
			\hline
			\textbf{Totale Preventivo} & \textbf{80} & \textbf{1689\euro}\\
			\hline
			\textbf{Totale Consultivo} & \textbf{90} & \textbf{1909\euro}\\
			\hline
			\textbf{Differenza} & \textbf{35} & \textbf{220\euro}
		\end{tabularx}
		\captionof{table}{\textbf{Consuntivo della fase di Technology Baseline (Primo Incremento)}}
	\end{center}

\subsubsection{Conclusioni}
Come emerso dalla tabella precedente il bilancio risulta negativo in quanto il gruppo ha ritenuto necessario impiegare più tempo del previsto nel ruolo di \textit{Progettista}. Il motivo di tale ritardo è la mole di lavoro inaspettata che ha dovuto ricoprire questo ruolo per l'acerbità dei componenti riguardo alle tecnologie da utilizzare.

\subsubsection{Preventivo a finire}
Il preventivo a finire presenta un surplus di 220\euro. Per cercare di rientrare nelle ore prestabilite per le consegne successive il gruppo ha deciso di organizzare meglio lo studio individuale di ognuno e di migliorare la comunicazione interna così che quanto emerge un problema questo venga risolto non dal singolo impiegando troppo tempo, ma dal gruppo.

\subsection{Periodo di progettazione architetturale - Technology Baseline (Secondo Incremento)}\label{ConsuntivoPeriodoDiProgettazioneArchitetturaleTechnologyBaselineSecondoIncremento}

Le ore dedicate in questo periodo sono atte al completamento del secondo incremento della fase di Technology Baseline$_{\scaleto{G}{3pt}}$ descritta nella sezione \S~\ref{PianificazioneProgettazioneArchitetturale}.

\quad
\def\tabularxcolumn#1{m{#1}}
{\rowcolors{2}{RawSienna!90!RawSienna!20}{RawSienna!70!RawSienna!40}
	\begin{center}
		\renewcommand{\arraystretch}{1.4}
		\begin{tabularx}{10cm}{|X|c|c|}
			\hline
			\rowcolor{airforceblue}
			\textbf{Ruolo} & \textbf{Ore} & \textbf{Costo}\\
			\hline
			Responsabile & 6(+0) & 180\euro(+0\euro)\\
			\hline
			Amministratore & 6(+0) & 120\euro(+0\euro)\\
			\hline
			Analista & 8(+0) & 200\euro(+0\euro)\\
			\hline
			Progettista & 24(+10) & 528\euro(+220\euro)\\
			\hline
			Programmatore & 3(+0) & 45\euro(+0\euro)\\
			\hline
			Verificatore & 12(+0) & 180\euro(+0\euro)\\
			\hline
			\textbf{Totale Preventivo} & \textbf{59} & \textbf{1253\euro}\\
			\hline
			\textbf{Totale Consultivo} & \textbf{69} & \textbf{1473\euro}\\
			\hline
			\textbf{Differenza} & \textbf{35} & \textbf{220\euro}
		\end{tabularx}
		\captionof{table}{\textbf{Consuntivo della fase di Technology Baseline (Secondo Incremento)}}
	\end{center}

\subsubsection{Conclusioni}
Come emerso dalla tabella precedente il bilancio risulta negativo come per l'incremento precedentemente analizzato. Il ruolo di \textit{Progettista} ha dovuto impiegare più ore del previsto a causa dell'inesperienza tecnologica.

\subsubsection{Preventivo a finire}
Il preventivo a finire presenta un surplus di 220\euro, come l'incremento precedentemente descritto. Per questo motivo la soluzione proposta è identica alla precedente: organizzare meglio lo studio individuale e migliorare la comunicazione interna per una risoluzione dei possibili problemi più efficiente ed efficace.

\subsection{Periodo di progettazione architetturale - Technology Baseline (Terzo Incremento)}\label{ConsuntivoPeriodoDiProgettazioneArchitetturaleTechnologyBaselineTerzoIncremento}

Le ore dedicate in questo periodo sono atte al completamento del terzo incremento della fase di Technology Baseline$_{\scaleto{G}{3pt}}$ descritta nella sezione \S~\ref{PianificazioneProgettazioneArchitetturale}.

\quad
\def\tabularxcolumn#1{m{#1}}
{\rowcolors{2}{RawSienna!90!RawSienna!20}{RawSienna!70!RawSienna!40}
	\begin{center}
		\renewcommand{\arraystretch}{1.4}
		\begin{tabularx}{10cm}{|X|c|c|}
			\hline
			\rowcolor{airforceblue}
			\textbf{Ruolo} & \textbf{Ore} & \textbf{Costo}\\
			\hline
			Responsabile & 5(+0) & 150\euro(+0\euro)\\
			\hline
			Amministratore & 6(+0) & 120\euro(+0\euro)\\
			\hline
			Analista & 6(+0) & 150\euro(+0\euro)\\
			\hline
			Progettista & 18(+0) & 396\euro(+0\euro)\\
			\hline
			Programmatore & 5(+0) & 75\euro(+0\euro)\\
			\hline
			Verificatore & 13(+0) & 195\euro(+0\euro)\\
			\hline
			\textbf{Totale Preventivo} & \textbf{53} & \textbf{1086\euro}\\
			\hline
			\textbf{Totale Consultivo} & \textbf{53} & \textbf{1086\euro}\\
			\hline
			\textbf{Differenza} & \textbf{0} & \textbf{0\euro}
		\end{tabularx}
		\captionof{table}{\textbf{Consuntivo della fase di Technology Baseline (Terzo Incremento)}}
	\end{center}

\subsubsection{Conclusioni}
Le ore preventivate sono state rispettate quindi non è presente alcuna differenza rispetto alle ore effettive di lavoro ovvero il gruppo è riuscito a procedere senza alcun problema con lo sviluppo dei moduli presenti in questo incremento.

\subsubsection{Preventivo a finire}
In quanto le ore di lavoro previste sono state rispettate il preventivo a finire risulta coerente con quello previsto.

\subsection{Consuntivo complessivo delle fasi}\label{ConsuntivoPeriodoDiProgettazioneArchitetturaleConsuntivoComplessivoDelleFasi}

Nella tabella successiva viene descritto il calcolo delle ore totali di tutte le parti precedentemente descritte.

\quad
\def\tabularxcolumn#1{m{#1}}
{\rowcolors{2}{RawSienna!90!RawSienna!20}{RawSienna!70!RawSienna!40}
	\begin{center}
		\renewcommand{\arraystretch}{1.4}
		\begin{tabularx}{10cm}{|X|c|c|}
			\hline
			\rowcolor{airforceblue}
			\textbf{Ruolo} & \textbf{Ore} & \textbf{Costo}\\
			\hline
			Responsabile & 24(+0) & 720\euro(+0\euro)\\
			\hline
			Amministratore & 26(+0) & 520\euro(+0\euro)\\
			\hline
			Analista & 29(+0) & 725\euro(+0\euro)\\
			\hline
			Progettista & 86(+20) & 1892\euro(+440\euro)\\
			\hline
			Programmatore & 12(+0) & 180\euro(+0\euro)\\
			\hline
			Verificatore & 54(+15) & 810\euro(+225\euro)\\
			\hline
			\textbf{Totale Preventivo} & \textbf{231} & \textbf{4847\euro}\\
			\hline
			\textbf{Totale Consultivo} & \textbf{266} & \textbf{5512\euro}\\
			\hline
			\textbf{Differenza} & \textbf{35} & \textbf{665\euro}
		\end{tabularx}
		\captionof{table}{\textbf{Consuntivo complessivo delle fasi}}
	\end{center}

\subsection{Conclusioni}\label{ConsuntivoPeriodoDiProgettazioneArchitetturaleConclusioni}
Il bilancio, come emerge dalla tabella precedente, risulta negativo poiché il gruppo ha ritenuto necessario impiegare più ore nei ruoli di \textit{Progettista} e \textit{Verificatore}. I motivi di tale ritardo sono:
\begin{itemize}
	\item il tempo impiegato per la correzione e l'aggiornamento dei documenti si è rivelato essere più di quello preventivato;
	\item essendo un progetto complesso con tecnologie nuove ad ogni componente del gruppo la parte di progettazione si è rivelata molto più complessa del previsto.
\end{itemize}

\subsection{Preventivo a finire}\label{ConsuntivoPeriodoDiProgettazioneArchitetturalePreventivoAFinire}
Il preventivo a finire risulta quindi con un surplus di 665\euro. Dalle analisi fatte riguardo ai preventivi a finire di ogni parte di questo periodo il gruppo ha rilevato che:
\begin{itemize}
	\item deve essere presente un'organizzazione migliore nella verifica e validazione dei documenti;
	\item lo studio personale delle tecnologie deve essere più efficiente in modo da poter sviluppare in maniera efficace;
	\item il miglioramento delle comunicazioni interne al gruppo può essere una parte fondamentale nella risoluzione di possibili problemi che emergono sviluppando il progetto.
\end{itemize}
Se ogni componente cerca di perseguire questi tre obiettivi il surplus presente in questo preventivo non risulterà un problema per l'ideazione del progetto.