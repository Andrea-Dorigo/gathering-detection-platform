\chapter{Consuntivo}\label{Consuntivo}
Di seguito vengono indicate le spese sostenute dal gruppo confrontandole con quanto preventivato. Il bilancio potrà essere:
\begin{itemize}
	\item positivo: la spesa effettiva è minore di quanto preventivato;
	\item pari: la spesa effettiva è uguale a quanto preventivato;
	\item negativo: la spesa effettiva è maggiore di quanto preventivato.
\end{itemize}
\section{Periodo di analisi}\label{ConsuntivoPeriodoDiAnalisi}
Le ore di lavoro che sono state sostenute durante la fase di analisi sono considerate come ore di investimento per questo motivo esse non vengono rendicontate.
\quad
\def\tabularxcolumn#1{m{#1}}
{\rowcolors{2}{RawSienna!90!RawSienna!20}{RawSienna!70!RawSienna!40}
	\begin{center}
		\renewcommand{\arraystretch}{1.4}
		\begin{tabularx}{10cm}{|X|c|c|}
			\hline
			\rowcolor{airforceblue}
			\textbf{Ruolo} & \textbf{Ore} & \textbf{Costo}\\
			\hline
			Responsabile & 26(+0) & 780\euro(+0\euro)\\
			\hline
			Amministratore & 42(+0) & 840\euro(+0\euro)\\
			\hline
			Analista & 49(+15) & 1225\euro(+375\euro)\\
			\hline
			Progettista & 0(+0) & 0\euro(+0\euro)\\
			\hline
			Programmatore & 0(+0) & 0\euro(+0\euro)\\
			\hline
			Verificatore & 33(+10) & 495\euro(+150\euro)\\
			\hline
			\textbf{Totale Preventivo} & \textbf{150} & \textbf{3340\euro}\\
			\hline
			\textbf{Totale Consultivo} & \textbf{175} & \textbf{3865\euro}\\
			\hline
			\textbf{Differenza} & \textbf{25} & \textbf{525\euro}
		\end{tabularx}
		\captionof{table}{\textbf{Consuntivo della fase di Analisi}}
	\end{center}

\subsection{Conclusioni}\label{ConsuntivoPeriodoDiAnalisiConclusioni}
Come emerso dalla tabella precedente il bilancio risulta negativo in quanto il gruppo ha ritenuto necessario impiegare più tempo del previsto nei ruoli di \textit{Analista} e \textit{Verificatore}. Il motivo di tale ritardo è:
\begin{itemize}
	\item l'individuazione dei requisiti é risultata più complessa del previsto;
	\item la grande quantità di lavoro nel revisionare i documenti in quanto essendo un processo nuovo ed ogni componente ha dovuto imparare come farlo in maniera corretta, efficacie ed efficiente.
\end{itemize}

\subsection{Preventivo a finire}\label{ConsuntivoPeriodoDiAnalisiPreventivoAFinire}
Il preventivo a finire, nonostante in questa fase siano state necessarie più ore del previsto, è in linea con quanto descritto nella sezione precedente. Il gruppo non ritiene il surplus di 500\euro{} un problema in quanto le ore lavorative e i costi sostenuti in questa fase non verranno rendicontati. Per questo motivo il gruppo ha deciso di non prendere alcuna contromisura nella pianificazione futura.

\section{Periodo di consolidamento dei requisiti}\label{ConsuntivoPeriodoDiConsolidamentoDeiRequisiti}
Le ore di lavoro calcolate per questo periodo sono considerate come ore di investimento e per tale motivo non vengono rendicontate.

\quad
\def\tabularxcolumn#1{m{#1}}
{\rowcolors{2}{RawSienna!90!RawSienna!20}{RawSienna!70!RawSienna!40}
	\begin{center}
		\renewcommand{\arraystretch}{1.4}
		\begin{tabularx}{10cm}{|X|c|c|}
			\hline
			\rowcolor{airforceblue}
			\textbf{Ruolo} & \textbf{Ore} & \textbf{Costo}\\
			\hline
			Responsabile & 4(+0) & 120\euro(+0\euro)\\
			\hline
			Amministratore & 8(+0) & 160\euro(+0\euro)\\
			\hline
			Analista & 4(0) & 100\euro(+0\euro)\\
			\hline
			Progettista & 0(+0) & 0\euro(+0\euro)\\
			\hline
			Programmatore & 0(+0) & 0\euro(+0\euro)\\
			\hline
			Verificatore & 8(0) & 120\euro(+0\euro)\\
			\hline
			\textbf{Totale Preventivo} & \textbf{24} & \textbf{500\euro}\\
			\hline
			\textbf{Totale Consultivo} & \textbf{24} & \textbf{500\euro}\\
			\hline
			\textbf{Differenza} & \textbf{0} & \textbf{0\euro}
		\end{tabularx}
		\captionof{table}{\textbf{Consuntivo della fase di Consolidamento dei requisiti}}
	\end{center}

\subsection{Conclusioni}\label{ConsuntivoPeriodoDiConsolidamentoDeiRequisitiConclusioni}
Grazie al minor carico di lavoro le ore preventivate sono state rispettate quindi non è presente alcuna differenza rispetto alle ore effettivo di lavoro. Inoltre il gruppo è riuscito a procedere senza alcun problema con lo studio personale per lo svolgimento della fase successiva del lavoro.

\subsection{Preventivo a finire}\label{ConsuntivoPeriodoDiConsolidamentoDeiRequisitiPreventivoAFinire}
In quanto le ore di lavoro previste sono state rispettate il preventivo a finire risulta coerente con quello previsto.

\section{Periodo di progettazione architetturale}\label{ConsuntivoPeriodoDiProgettazioneArchitetturale}

Il gruppo ha suddiviso questo periodo in diverse parti, di conseguenza il consuntivo viene analizzato in funzione di ogni fase(??).

\subsection{Periodo di progettazione architetturale - Incremento e Verifica}\label{ConsuntivoPeriodoDiProgettazioneArchitetturaleIncrementoEVerifica}

Le ore dedicate in questo periodo sono atte al completamento della fase di Incremento e Verifica descritta nella sezione \S~\ref{PianificazioneProgettazioneArchitetturale}.

\quad
\def\tabularxcolumn#1{m{#1}}
{\rowcolors{2}{RawSienna!90!RawSienna!20}{RawSienna!70!RawSienna!40}
	\begin{center}
		\renewcommand{\arraystretch}{1.4}
		\begin{tabularx}{10cm}{|X|c|c|}
			\hline
			\rowcolor{airforceblue}
			\textbf{Ruolo} & \textbf{Ore} & \textbf{Costo}\\
			\hline
			Responsabile & 24(+0) & 720\euro(+0\euro)\\
			\hline
			Amministratore & 26(+0) & 520\euro(+0\euro)\\
			\hline
			Analista & 29(0) & 725\euro(+0\euro)\\
			\hline
			Progettista & 86(+20) & 1892\euro(+440\euro)\\
			\hline
			Programmatore & 12(+0) & 180\euro(+0\euro)\\
			\hline
			Verificatore & 54(+15) & 810\euro(+225\euro)\\
			\hline
			\textbf{Totale Preventivo} & \textbf{231} & \textbf{4847\euro}\\
			\hline
			\textbf{Totale Consultivo} & \textbf{266} & \textbf{5512\euro}\\
			\hline
			\textbf{Differenza} & \textbf{35} & \textbf{665\euro}
		\end{tabularx}
		\captionof{table}{\textbf{Consuntivo della fase di Incremento e Verifica}}
	\end{center}

\subsubsection{Conclusioni}

\subsubsection{Preventivo a finire}

\subsection{Periodo di progettazione architetturale - Technology Baseline (Primo Incremento)}\label{ConsuntivoPeriodoDiProgettazioneArchitetturaleTechnologyBaselinePrimoIncremento}

Le ore dedicate in questo periodo sono atte al completamento del primo incremento della fase di Technology Baseline$_{\scaleto{G}{3pt}}$ descritta nella sezione \S~\ref{PianificazioneProgettazioneArchitetturale}.

\quad
\def\tabularxcolumn#1{m{#1}}
{\rowcolors{2}{RawSienna!90!RawSienna!20}{RawSienna!70!RawSienna!40}
	\begin{center}
		\renewcommand{\arraystretch}{1.4}
		\begin{tabularx}{10cm}{|X|c|c|}
			\hline
			\rowcolor{airforceblue}
			\textbf{Ruolo} & \textbf{Ore} & \textbf{Costo}\\
			\hline
			Responsabile & 24(+0) & 720\euro(+0\euro)\\
			\hline
			Amministratore & 26(+0) & 520\euro(+0\euro)\\
			\hline
			Analista & 29(0) & 725\euro(+0\euro)\\
			\hline
			Progettista & 86(+20) & 1892\euro(+440\euro)\\
			\hline
			Programmatore & 12(+0) & 180\euro(+0\euro)\\
			\hline
			Verificatore & 54(+15) & 810\euro(+225\euro)\\
			\hline
			\textbf{Totale Preventivo} & \textbf{231} & \textbf{4847\euro}\\
			\hline
			\textbf{Totale Consultivo} & \textbf{266} & \textbf{5512\euro}\\
			\hline
			\textbf{Differenza} & \textbf{35} & \textbf{665\euro}
		\end{tabularx}
		\captionof{table}{\textbf{Consuntivo della fase di Technology Baseline (Primo Incremento)}}
	\end{center}

\subsubsection{Conclusioni}

\subsubsection{Preventivo a finire}

\subsection{Periodo di progettazione architetturale - Technology Baseline (Secondo Incremento)}\label{ConsuntivoPeriodoDiProgettazioneArchitetturaleTechnologyBaselineSecondoIncremento}

Le ore dedicate in questo periodo sono atte al completamento del secondo incremento della fase di Technology Baseline$_{\scaleto{G}{3pt}}$ descritta nella sezione \S~\ref{PianificazioneProgettazioneArchitetturale}.

\quad
\def\tabularxcolumn#1{m{#1}}
{\rowcolors{2}{RawSienna!90!RawSienna!20}{RawSienna!70!RawSienna!40}
	\begin{center}
		\renewcommand{\arraystretch}{1.4}
		\begin{tabularx}{10cm}{|X|c|c|}
			\hline
			\rowcolor{airforceblue}
			\textbf{Ruolo} & \textbf{Ore} & \textbf{Costo}\\
			\hline
			Responsabile & 24(+0) & 720\euro(+0\euro)\\
			\hline
			Amministratore & 26(+0) & 520\euro(+0\euro)\\
			\hline
			Analista & 29(0) & 725\euro(+0\euro)\\
			\hline
			Progettista & 86(+20) & 1892\euro(+440\euro)\\
			\hline
			Programmatore & 12(+0) & 180\euro(+0\euro)\\
			\hline
			Verificatore & 54(+15) & 810\euro(+225\euro)\\
			\hline
			\textbf{Totale Preventivo} & \textbf{231} & \textbf{4847\euro}\\
			\hline
			\textbf{Totale Consultivo} & \textbf{266} & \textbf{5512\euro}\\
			\hline
			\textbf{Differenza} & \textbf{35} & \textbf{665\euro}
		\end{tabularx}
		\captionof{table}{\textbf{Consuntivo della fase di Technology Baseline (Secondo Incremento)}}
	\end{center}

\subsubsection{Conclusioni}

\subsubsection{Preventivo a finire}

\subsection{Periodo di progettazione architetturale - Technology Baseline (Terzo Incremento)}\label{ConsuntivoPeriodoDiProgettazioneArchitetturaleTechnologyBaselineTerzoIncremento}

Le ore dedicate in questo periodo sono atte al completamento del terzo incremento della fase di Technology Baseline$_{\scaleto{G}{3pt}}$ descritta nella sezione \S~\ref{PianificazioneProgettazioneArchitetturale}.

\quad
\def\tabularxcolumn#1{m{#1}}
{\rowcolors{2}{RawSienna!90!RawSienna!20}{RawSienna!70!RawSienna!40}
	\begin{center}
		\renewcommand{\arraystretch}{1.4}
		\begin{tabularx}{10cm}{|X|c|c|}
			\hline
			\rowcolor{airforceblue}
			\textbf{Ruolo} & \textbf{Ore} & \textbf{Costo}\\
			\hline
			Responsabile & 24(+0) & 720\euro(+0\euro)\\
			\hline
			Amministratore & 26(+0) & 520\euro(+0\euro)\\
			\hline
			Analista & 29(0) & 725\euro(+0\euro)\\
			\hline
			Progettista & 86(+20) & 1892\euro(+440\euro)\\
			\hline
			Programmatore & 12(+0) & 180\euro(+0\euro)\\
			\hline
			Verificatore & 54(+15) & 810\euro(+225\euro)\\
			\hline
			\textbf{Totale Preventivo} & \textbf{231} & \textbf{4847\euro}\\
			\hline
			\textbf{Totale Consultivo} & \textbf{266} & \textbf{5512\euro}\\
			\hline
			\textbf{Differenza} & \textbf{35} & \textbf{665\euro}
		\end{tabularx}
		\captionof{table}{\textbf{Consuntivo della fase di Technology Baseline (Terzo Incremento)}}
	\end{center}

\subsubsection{Conclusioni}

\subsubsection{Preventivo a finire}

\subsection{Consutivo complessivo delle fasi}\label{ConsuntivoPeriodoDiProgettazioneArchitetturaleConsuntivoComplessivoDelleFasi}

Nella tabella successiva viene descritto il calcolo delle ore totali di tutte le fasi precedentemente descritte.

\quad
\def\tabularxcolumn#1{m{#1}}
{\rowcolors{2}{RawSienna!90!RawSienna!20}{RawSienna!70!RawSienna!40}
	\begin{center}
		\renewcommand{\arraystretch}{1.4}
		\begin{tabularx}{10cm}{|X|c|c|}
			\hline
			\rowcolor{airforceblue}
			\textbf{Ruolo} & \textbf{Ore} & \textbf{Costo}\\
			\hline
			Responsabile & 24(+0) & 720\euro(+0\euro)\\
			\hline
			Amministratore & 26(+0) & 520\euro(+0\euro)\\
			\hline
			Analista & 29(0) & 725\euro(+0\euro)\\
			\hline
			Progettista & 86(+20) & 1892\euro(+440\euro)\\
			\hline
			Programmatore & 12(+0) & 180\euro(+0\euro)\\
			\hline
			Verificatore & 54(+15) & 810\euro(+225\euro)\\
			\hline
			\textbf{Totale Preventivo} & \textbf{231} & \textbf{4847\euro}\\
			\hline
			\textbf{Totale Consultivo} & \textbf{266} & \textbf{5512\euro}\\
			\hline
			\textbf{Differenza} & \textbf{35} & \textbf{665\euro}
		\end{tabularx}
		\captionof{table}{\textbf{Consuntivo della fase di Technology Baseline (Terzo Incremento)}}
	\end{center}

\subsection{Conclusioni}\label{ConsuntivoPeriodoDiProgettazioneArchitetturaleConclusioni}
Il bilancio, come emerge dalla tabella precedente, risulta negativo poiché il gruppo ha ritenuto necessario impiegare più ore nei ruoli di \textit{Progettista} e \textit{Verificatore}. I motivi di tale ritardo sono:
\begin{itemize}
	\item il tempo impiegato per la correzione e l'aggiornamento dei documenti si è rivelato essere più di quello preventivato;
	\item essendo un progetto complesso con tecnologie nuove ad ogni componente del gruppo la parte di progettazione si è rivelata molto più complessa del previsto.
\end{itemize}

\subsection{Preventivo a finire}\label{ConsuntivoPeriodoDiProgettazioneArchitetturalePreventivoAFinire}
Il preventivo a finire risulta quindi con un surplus di 665\euro. AGGIUNGI PARTE DOVE PENSIAMO AD UNA SOLUZIONE