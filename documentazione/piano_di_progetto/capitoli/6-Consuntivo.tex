\chapter{Consuntivo}\label{Consuntivo}
Di seguito vengono indicate le spese sostenute dal gruppo confrontandole con quanto preventivato. Il bilancio potrà essere:
\begin{itemize}
	\item positivo: la spesa effettiva è minore di quanto preventivato;
	\item pari: la spesa effettiva è uguale a quanto preventivato;
	\item negativo: la spesa effettiva è maggiore di quanto preventivato.
\end{itemize}
\section{Periodo di analisi}\label{ConsuntivoPeriodoDiAnalisi}
Le ore di lavoro che sono state sostenute durante la fase di analisi sono considerate come ore di investimento e per questo motivo esse non vengono rendicontate.
\quad
\def\tabularxcolumn#1{m{#1}}
{\rowcolors{2}{RawSienna!90!RawSienna!20}{RawSienna!70!RawSienna!40}
	\begin{center}
		\renewcommand{\arraystretch}{1.4}
		\begin{tabularx}{10cm}{|X|c|c|}
			\hline
			\rowcolor{airforceblue}
			\textbf{Ruolo} & \textbf{Ore} & \textbf{Costo}\\
			\hline
			Responsabile & 26(+0) & 780\euro(+0\euro)\\
			\hline
			Amministratore & 42(+0) & 840\euro(+0\euro)\\
			\hline
			Analista & 49(+15) & 1225\euro(+375\euro)\\
			\hline
			Progettista & 0(+0) & 0\euro(+0\euro)\\
			\hline
			Programmatore & 0(+0) & 0\euro(+0\euro)\\
			\hline
			Verificatore & 33(+10) & 495\euro(+150\euro)\\
			\hline
			\textbf{Totale Preventivo} & \textbf{150} & \textbf{3340\euro}\\
			\hline
			\textbf{Totale Consuntivo} & \textbf{175} & \textbf{3865\euro}\\
			\hline
			\textbf{Differenza} & \textbf{25} & \textbf{525\euro}
		\end{tabularx}
		\captionof{table}{\textbf{Consuntivo della fase di Analisi}}
	\end{center}

\subsection{Conclusioni}\label{ConsuntivoPeriodoDiAnalisiConclusioni}
Come emerso dalla tabella precedente, il bilancio risulta negativo in quanto il gruppo ha ritenuto necessario impiegare più tempo del previsto nei ruoli di \textit{Analista} e \textit{Verificatore}. I motivi di tale ritardo sono:
\begin{itemize}
	\item la complessità nell'individuazione dei requisiti;
	\item la grande quantità di lavoro nel revisionare i documenti. Infatti, trattandosi di un processo nuovo, ogni componente ha dovuto imparare a svolgerlo in maniera corretta, efficacie ed efficiente.
\end{itemize}

\subsection{Preventivo a finire}\label{ConsuntivoPeriodoDiAnalisiPreventivoAFinire}
Il preventivo a finire, nonostante in questa fase siano state necessarie più ore del previsto, è in linea con quanto descritto nella sezione precedente. Il gruppo non ritiene il surplus di 500\euro{} un problema in quanto le ore lavorative e i costi sostenuti in questa fase non verranno rendicontati. Per questo motivo il gruppo ha deciso di non prendere alcuna contromisura nella pianificazione futura.

\section{Periodo di consolidamento dei requisiti}\label{ConsuntivoPeriodoDiConsolidamentoDeiRequisiti}
Le ore di lavoro calcolate per questo periodo sono considerate come ore di investimento e, per tale motivo, non vengono rendicontate.

\quad
\def\tabularxcolumn#1{m{#1}}
{\rowcolors{2}{RawSienna!90!RawSienna!20}{RawSienna!70!RawSienna!40}
	\begin{center}
		\renewcommand{\arraystretch}{1.4}
		\begin{tabularx}{10cm}{|X|c|c|}
			\hline
			\rowcolor{airforceblue}
			\textbf{Ruolo} & \textbf{Ore} & \textbf{Costo}\\
			\hline
			Responsabile & 4(+0) & 120\euro(+0\euro)\\
			\hline
			Amministratore & 8(+0) & 160\euro(+0\euro)\\
			\hline
			Analista & 4(+0) & 100\euro(+0\euro)\\
			\hline
			Progettista & 0(+0) & 0\euro(+0\euro)\\
			\hline
			Programmatore & 0(+0) & 0\euro(+0\euro)\\
			\hline
			Verificatore & 8(0) & 120\euro(+0\euro)\\
			\hline
			\textbf{Totale Preventivo} & \textbf{24} & \textbf{500\euro}\\
			\hline
			\textbf{Totale Consuntivo} & \textbf{24} & \textbf{500\euro}\\
			\hline
			\textbf{Differenza} & \textbf{0} & \textbf{0\euro}
		\end{tabularx}
		\captionof{table}{\textbf{Consuntivo della fase di Consolidamento dei requisiti}}
	\end{center}

\subsection{Conclusioni}\label{ConsuntivoPeriodoDiConsolidamentoDeiRequisitiConclusioni}
Grazie al minor carico di lavoro, le ore preventivate sono state rispettate quindi non è presente alcuna differenza rispetto alle ore effettive di lavoro. Inoltre il gruppo è riuscito a procedere senza alcun problema con lo studio personale per lo svolgimento della fase successiva del lavoro.

\subsection{Preventivo a finire}\label{ConsuntivoPeriodoDiConsolidamentoDeiRequisitiPreventivoAFinire}
Poichè le ore di lavoro previste sono state rispettate, il preventivo a finire risulta coerente con quello previsto.

\section{Periodo di progettazione architetturale}\label{ConsuntivoPeriodoDiProgettazioneArchitetturale}

Il gruppo ha suddiviso questa fase in diversi periodi per organizzare al meglio il lavoro, di conseguenza il consuntivo viene analizzato in funzione di ogni sua parte.

\subsection{Primo periodo - dal 19-01-2021 al 15-02-2021}\label{ConsuntivoPeriodoDiProgettazioneArchitetturaleIncrementoEVerifica}

Le ore dedicate in questo periodo sono atte al completamento della fase di Incremento e Verifica descritta nella \S~\ref{PianificazioneProgettazioneArchitetturale} e a formazione personale sulle tecnologie da utilizzare per lo sviluppo della Technology Baseline$_{\scaleto{G}{3pt}}$.

\quad
\def\tabularxcolumn#1{m{#1}}
{\rowcolors{2}{RawSienna!90!RawSienna!20}{RawSienna!70!RawSienna!40}
	\begin{center}
		\renewcommand{\arraystretch}{1.4}
		\begin{tabularx}{10cm}{|X|c|c|}
			\hline
			\rowcolor{airforceblue}
			\textbf{Ruolo} & \textbf{Ore} & \textbf{Costo}\\
			\hline
			Responsabile & 7(+0) & 210\euro(+0\euro)\\
			\hline
			Amministratore & 7(+0) & 140\euro(+0\euro)\\
			\hline
			Analista & 8(+10) & 200\euro(+250\euro)\\
			\hline
			Progettista & 2(+0) & 44\euro(+0\euro)\\
			\hline
			Programmatore & 0(+0) & 0\euro(+0\euro)\\
			\hline
			Verificatore & 15(+10) & 225\euro(+150\euro)\\
			\hline
			\textbf{Totale Preventivo} & \textbf{39} & \textbf{819\euro}\\
			\hline
			\textbf{Totale Consuntivo} & \textbf{59} & \textbf{1219\euro}\\
			\hline
			\textbf{Differenza} & \textbf{20} & \textbf{400\euro}
		\end{tabularx}
		\captionof{table}{\textbf{Consuntivo del primo periodo}}
	\end{center}

\subsubsection{Conclusioni}
Come emerso dalla tabella precedente, il bilancio risulta negativo in quanto il gruppo ha ritenuto necessario impiegare più tempo del previsto nel ruolo di \textit{Verificatore} e di \textit{Analista}, per via l'esigenza di correggere alcuni errori sollevati in seguito alla Revisione dei Requisiti. 
%(voglio dire che abbiamo corretto i difetti gravi segnalati da tullio ed è per questo che il verificatore ha queste ore in più ma non so come dirlo in modo carino e formale)

\subsubsection{Preventivo a finire}
Il preventivo a finire presenta un surplus di 400\euro. Per questo motivo il gruppo ha deciso di tamponare il problema cercando di rispettare le ore preventivate nelle fasi successive in modo tale da non sforare troppo dal preventivo inizialmente previsto.

\subsection{Secondo periodo - dal 16-02-2021 al 04-03-2021 }\label{ConsuntivoPeriodoDiProgettazioneArchitetturaleTechnologyBaselinePrimoIncremento}

Le ore dedicate in questo periodo sono atte al completamento del primo incremento descritto nella  \S~\ref{PianificazioneProgettazioneArchitetturale}.

\quad
\def\tabularxcolumn#1{m{#1}}
{\rowcolors{2}{RawSienna!90!RawSienna!20}{RawSienna!70!RawSienna!40}
	\begin{center}
		\renewcommand{\arraystretch}{1.4}
		\begin{tabularx}{10cm}{|X|c|c|}
			\hline
			\rowcolor{airforceblue}
			\textbf{Ruolo} & \textbf{Ore} & \textbf{Costo}\\
			\hline
			Responsabile & 9(+0) & 270\euro(+0\euro)\\
			\hline
			Amministratore & 10(+0) & 200\euro(+0\euro)\\
			\hline
			Analista & 11(+0) & 275\euro(+0\euro)\\
			\hline
			Progettista & 54(+15) & 1188\euro(+330\euro)\\
			\hline
			Programmatore & 7(+0) & 105\euro(+0\euro)\\
			\hline
			Verificatore & 20(+0) & 300\euro(+0\euro)\\
			\hline
			\textbf{Totale Preventivo} & \textbf{111} & \textbf{2338\euro}\\
			\hline
			\textbf{Totale Consuntivo} & \textbf{126} & \textbf{2668\euro}\\
			\hline
			\textbf{Differenza} & \textbf{15} & \textbf{330\euro}
		\end{tabularx}
		\captionof{table}{\textbf{Consuntivo del secondo periodo}}
	\end{center}

\subsubsection{Conclusioni}
Come emerso dalla tabella precedente, il bilancio risulta negativo in quanto il gruppo ha ritenuto necessario impiegare più tempo del previsto nel ruolo di \textit{Progettista}. Il motivo di ciò è la mole di lavoro inaspettata che ha dovuto ricoprire questo ruolo per l'acerbità dei componenti riguardo alle tecnologie da utilizzare.

\subsubsection{Preventivo a finire}
Il preventivo a finire presenta un surplus di 330\euro. Per cercare di rientrare nelle ore prestabilite per le consegne successive, il gruppo ha deciso di organizzare meglio lo studio individuale di ognuno e di migliorare la comunicazione interna: in questo modo, quando emerge un problema, questo può essere risolto non dal singolo, che potrebbe impiegarci troppo tempo, ma dal gruppo.

\subsection{Terzo periodo - dal 05-03-2021 al 15-03-2021}\label{ConsuntivoPeriodoDiProgettazioneArchitetturaleTechnologyBaselineTerzoIncremento}

Le ore dedicate in questo periodo sono atte al completamento del secondo incremento descritto nella \S~\ref{PianificazioneProgettazioneArchitetturale}.

\quad
\def\tabularxcolumn#1{m{#1}}
{\rowcolors{2}{RawSienna!90!RawSienna!20}{RawSienna!70!RawSienna!40}
	\begin{center}
		\renewcommand{\arraystretch}{1.4}
		\begin{tabularx}{10cm}{|X|c|c|}
			\hline
			\rowcolor{airforceblue}
			\textbf{Ruolo} & \textbf{Ore} & \textbf{Costo}\\
			\hline
			Responsabile & 8(+0) & 240\euro(+0\euro)\\
			\hline
			Amministratore & 9(+0) & 180\euro(+0\euro)\\
			\hline
			Analista & 10(+0) & 250\euro(+0\euro)\\
			\hline
			Progettista & 30(+5) & 660\euro(+150\euro)\\
			\hline
			Programmatore & 5(+0) & 75\euro(+0\euro)\\
			\hline
			Verificatore & 19(+0) & 285\euro(+0\euro)\\
			\hline
			\textbf{Totale Preventivo} & \textbf{81} & \textbf{1690\euro}\\
			\hline
			\textbf{Totale Consuntivo} & \textbf{86} & \textbf{1840\euro}\\
			\hline
			\textbf{Differenza} & \textbf{5} & \textbf{150\euro}
		\end{tabularx}
		\captionof{table}{\textbf{Consuntivo del terzo periodo}}
	\end{center}

\subsubsection{Conclusioni}
Come emerso dalla tabella il bilancio risulta essere negativo in quanto il gruppo ha ritenuto necessario impiegare più tempo del previsto nel ruolo di \textit{Progettista} dovuto alla mole di lavoro legata al secondo incremento e alla conclusione dei documenti da presentare per la Revisione di Progettazione.

\subsubsection{Preventivo a finire}
Il preventivo a finire presenta un surplus di 150\euro. Per cercare di rientrare nelle ore prestabilite il gruppo ha lavorato in modo tale da avvantaggiarsi con il lavoro della fase successiva.

\subsection{Consuntivo complessivo delle fasi}\label{ConsuntivoPeriodoDiProgettazioneArchitetturaleConsuntivoComplessivoDelleFasi}

Nella tabella successiva viene descritto il calcolo delle ore totali di tutte le parti precedentemente descritte.

\quad
\def\tabularxcolumn#1{m{#1}}
{\rowcolors{2}{RawSienna!90!RawSienna!20}{RawSienna!70!RawSienna!40}
	\begin{center}
		\renewcommand{\arraystretch}{1.4}
		\begin{tabularx}{10cm}{|X|c|c|}
			\hline
			\rowcolor{airforceblue}
			\textbf{Ruolo} & \textbf{Ore} & \textbf{Costo}\\
			\hline
			Responsabile & 24(+0) & 720\euro(+0\euro)\\
			\hline
			Amministratore & 26(+0) & 520\euro(+0\euro)\\
			\hline
			Analista & 29(+10) & 725\euro(+250\euro)\\
			\hline
			Progettista & 86(+20) & 1892\euro(+440\euro)\\
			\hline
			Programmatore & 12(+0) & 180\euro(+0\euro)\\
			\hline
			Verificatore & 54(+10) & 810\euro(+150\euro)\\
			\hline
			\textbf{Totale Preventivo} & \textbf{231} & \textbf{4847\euro}\\
			\hline
			\textbf{Totale Consuntivo} & \textbf{271} & \textbf{5687\euro}\\
			\hline
			\textbf{Differenza} & \textbf{40} & \textbf{840\euro}
		\end{tabularx}
		\captionof{table}{\textbf{Consuntivo complessivo delle fasi}}
	\end{center}

\subsection{Conclusioni}\label{ConsuntivoPeriodoDiProgettazioneArchitetturaleConclusioni}
Il bilancio, come emerge dalla tabella precedente, risulta negativo poiché il gruppo ha ritenuto necessario impiegare più ore nei ruoli di \textit{Progettista},\textit{Verificatore} e \textit{Analista}. I motivi di tale ritardo sono:
\begin{itemize}
	\item il tempo impiegato per la correzione e l'aggiornamento dei documenti si è rivelato essere più di quello preventivato;
	\item trattandosi di un progetto complesso ed articolato, con tecnologie nuove ad ogni componente del gruppo, la parte di progettazione si è rivelata molto più complicata del previsto.
\end{itemize}

\subsection{Preventivo a finire}\label{ConsuntivoPeriodoDiProgettazioneArchitetturalePreventivoAFinire}
Il preventivo a finire risulta quindi con un surplus di 840\euro. Dalle analisi fatte riguardo ai preventivi a finire di ogni parte di questa fase il gruppo ha rilevato che:
\begin{itemize}
	\item deve essere presente un'organizzazione migliore nella verifica e validazione dei documenti;
	\item lo studio personale delle tecnologie deve essere più efficiente in modo da poter sviluppare in maniera più proficua;
	\item il miglioramento delle comunicazioni interne al gruppo può essere una parte fondamentale nella risoluzione di possibili problemi che emergono nel corso dello sviluppo del progetto.
\end{itemize}
Se ogni componente cerca di perseguire questi tre obiettivi il surplus presente in questo preventivo non risulterà un problema per l'ideazione del progetto. Inoltre per cercare di sopperire le ore aggiuntive il gruppo ha cercato di avanzare il più possibile con lo sviluppo dell'applicazione per non gravare troppo nelle fasi successive.