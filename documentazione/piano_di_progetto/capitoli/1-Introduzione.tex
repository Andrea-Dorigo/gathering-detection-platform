\chapter{Introduzione}

\section{Scopo del documento}
Lo scopo del documento è presentare un prospetto della pianificazione tenuta dal gruppo Jawa Druids al fine di soddisfare gli obiettivi prefissati dal capitolato \textit{GDP: Gathering Detection Platform} di Sync Lab. Il documento tratta le seguenti tematiche:
\begin{itemize}
\item analisi dei rischi e riflessioni;
\item riassunto del modello di sviluppo adottato e relative motivazioni;
\item pianificazione delle attività e suddivisione dei ruoli;
\item preventivi e stima delle risorse necessarie.
\end{itemize}

\section{Scopo del prodotto}
In seguito alla pandemia del virus COVID-19 è nata l'esigenza di limitare il più possibile i contatti fra le persone, specialmente evitando la formazione di assembramenti. Il progetto \textit{GDP: Gathering Detection Platform} di Sync Lab ha pertanto l'obiettivo di \textbf{creare una piattaforma in grado di rappresentare graficamente le zone potenzialmente a rischio di assembramento, al fine di prevenirlo.}
% il paragrafo sottostante richiede particolare attenzione durante la revisione
% e probabilmente non dovrebbe essere qui sotto gli Scopi del prodotto
% La raffigurazione di queste aree può portare a molteplici benefici. Il più evidente è che un utente cittadino, consapevole delle zone più a rischio, possa evitarle. Un altro vantaggio risiede nella possibilità di fornire ai progettisti urbani una mappa delle aree di maggior affluenza per poter...

Al tal fine il gruppo Jawa Druids si prefigge di sviluppare un prototipo software in grado di acquisire, monitorare ed analizzare i molteplici dati provenienti dai diversi sistemi e dispositivi, a scopo di identificare i possibili eventi che concorrono all’insorgere di variazioni di flussi di utenti. Il gruppo prevende inoltre lo sviluppo di un'applicazione web da interporre fra i dati elaborati e l'utente, per favorirne la consultazione.

\section{Glossario}

All'interno della documentazione viene fornito un \textit{Glossario}, con l'obiettivo di assistere il lettore specificando il significato e contesto d'utilizzo di alcuni termini strettamente tecnici o ambigui, segnalati con una \textit{G} a pedice.
