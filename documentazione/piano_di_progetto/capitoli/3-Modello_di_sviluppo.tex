\chapter{Modello di sviluppo}\label{ModelloDiSviluppo}
Per lo sviluppo del progetto abbiamo deciso di adottare il \textbf{modello incrementale}.
\section{Modello incrementale}\label{3.1}
Per una buona esecuzione del progetto e della pianificazione, è necessario adottare un modello di sviluppo che, in base alle sue caratteristiche, lomiti la progettazione stessa. Una prerogativa del gruppo è qualità, la quale deve riflettersi anche nel modello di sviluppo al fine di raggiungere gli obiettivi delineati dal modello stesso e realizzare così lo sviluppo in modo corretto e coerente.
Sulla base di queste considerazioni e sulla valutazione della natura del progetto, è stato adottato il modello di sviluppo incrementale, il quale prevede lo sviluppo del prodotto tramite incrementi multipli e sucessivi, ossia dei rilasci che realizzano ciascuno una nuova funzionalità che viene integrata nel sistema.
Nel modello di sviluppo incrementale i requisiti vengono classificati in base alla loro importanza strategica a livello di sistema. I requisiti più importanti sono trattati dai primi incrementi, in modo da renderli chiari e stabili nel minor tempo possibile per poterli poi soddisfare con maggiore facilità.
Gli incrementi successivi coprono, quindi, requisiti meno importanti che hanno quindi più tempo per integrarsi con il sistema.
Sebbene il modello di sviluppo non lo preveda, considerando il numero di componenti e di funzionalità che realizzano il sistema, sono consentite modifiche, aggiunte e rimozioni di requisiti.
Tali operazioni sono possibili solamente previa valutazione ed approvazione da parte del proponente. Queste modifiche che non possono essere discusse durante lo sviluppo di un incremento, è necessario prima effettuare il rilascio e poi valutare il cambiamento dei requisiti.
I vantaggi del modello di sviluppo incrementale sono i seguenti:
\begin{itemize}
	\item ogni incremento produce valore aggiunto, rendendo disponibili delle nuove funzionalità e chiarendo meglio i requisiti per gli incrementi successivi;
	\item ad ogni incremento è possibile ricevere in tempi brevi un feedback da parte del proponente sull'insieme delle funzionalità sviluppate;
	\item le funzionalità principali vengono sviluppate all'inizio con i primi incrementi, in quanto relative ai requisiti più importanti;
	\item ad ogni incremento vengono svolte attività di verifica come aggiunte e modifiche, rendendo l'intera verifica più semplice ed economica, in quanto il resto del prodotto era già stato testato con gli incrementi precedenti e gli errori sono limitati all'incremento annuale;
	\item gli errori in un singolo incremento sono più facili da individuare e correggere, in quanto relativi solo alle modifiche apportate all'incremento;
	\item ogni incremento riduce il rischio di fallimento.
\end{itemize}