\chapter{Modello di sviluppo}\label{ModelloDiSviluppo}
La scelta di un modello di comprovata efficacia è fondamentale per il corretto svolgimento del progetto: l'adozione di uno standard garantisce sicurezza e avanzamento sia al fornitore$_G$ che al proponente$_{\scaleto{G}{3pt}}$.
\section{Modello incrementale}\label{ModelloDiSviluppoModelloIncrementale}
Per lo sviluppo del progetto il gruppo ha deciso di adottare il \textbf{modello incrementale}.
Una prerogativa del gruppo è la qualità, la quale deve riflettersi anche nel modello di sviluppo al fine di raggiungere gli obiettivi delineati dal modello stesso e realizzare così il progetto in modo corretto e coerente.
Sulla base di queste considerazioni e sulla valutazione della natura del progetto, si è deciso di adottare il modello di sviluppo \textbf{incrementale}. Esso prevede lo sviluppo del prodotto tramite incrementi multipli e successivi, ossia dei rilasci che realizzano ciascuno una nuova funzionalità integrata nel sistema.

Nel modello di sviluppo incrementale i requisiti$_G$ vengono classificati in base alla loro importanza strategica a livello di sistema. I requisiti$_{\scaleto{G}{3pt}}$ più importanti sono trattati dai primi incrementi, in modo da renderli chiari e stabili nel minor tempo possibile per poterli poi soddisfare con maggiore facilità.
Gli incrementi successivi coprono, quindi, requisiti$_{\scaleto{G}{3pt}}$ meno importanti e perciò che hanno più tempo per integrarsi con il sistema.
Sebbene il modello di sviluppo non lo preveda, considerando il numero di componenti e di funzionalità che realizzano il sistema, sono consentite modifiche, aggiunte e rimozioni di requisiti$_{\scaleto{G}{3pt}}$.
Tali operazioni sono possibili solamente previa valutazione ed approvazione da parte del proponente$_{\scaleto{G}{3pt}}$. Per queste modifiche, che non possono essere discusse durante lo sviluppo di un incremento, è necessario prima effettuare il rilascio e poi valutare il cambiamento dei requisiti$_{\scaleto{G}{3pt}}$.

Abbiamo scelto il modello incrementale in quanto:
\begin{itemize}
	\item ogni incremento produce un valore aggiunto, rendendo disponibili delle nuove funzionalità e chiarendo meglio i requisiti$_{\scaleto{G}{3pt}}$ per gli incrementi successivi;
	\item ad ogni incremento è possibile ricevere in tempi brevi un feedback da parte del proponente$_{\scaleto{G}{3pt}}$ sull'insieme delle funzionalità sviluppate;
	\item le funzionalità principali vengono sviluppate all'inizio con i primi incrementi, in quanto relative ai requisiti$_{\scaleto{G}{3pt}}$ più importanti;
	\item ad ogni incremento vengono svolte attività di verifica come aggiunte e modifiche, rendendo l'intera verifica più semplice ed economica, in quanto il resto del prodotto era già stato testato con gli incrementi precedenti;
	\item gli errori in un singolo incremento sono più facili da individuare e correggere, in quanto relativi solo alle modifiche apportate all'incremento;
	\item ogni incremento riduce il rischio di fallimento.
\end{itemize}

\section{Confronto con il modello iterativo}\label{ModelloDiSviluppoConfrontoConIlModelloIterativo}
Durante la scelta del modello da adottare, il gruppo ha valutato attentamente anche il \textbf{modello iterativo}.
L'elasticità data da tale modello comporta un'elevata capacità di adattamento all'insorgere di eventuali problemi legati alle nuove tecnologie e ai requisiti$_{\scaleto{G}{3pt}}$, fattore molto rilevante nello sviluppo del capitolato$_{\scaleto{G}{3pt}}$ \textit{GDP: Gathering Detection Platform}.
Tuttavia per una buona esecuzione del progetto e della pianificazione, è necessario adottare un modello di sviluppo che, in base alle sue caratteristiche, limiti la progettazione stessa.

\section{Incrementi}\label{ModelloDiSviluppoIncrementi}
In questa sezione viene riportata una tabella contenente i dettagli di sviluppo di ogni incremento, facendo riferimento agli obiettivi, ai casi d'uso$_{\scaleto{G}{3pt}}$ e ai requisiti$_{\scaleto{G}{3pt}}$ di ognuno di essi.
\begin{center}
	\renewcommand{\arraystretch}{1.4}
	\begin{longtable}[c]{p{4cm}|p{4cm}|p{4cm}|p{3cm}}
		\hline
		\rowcolor{airforceblue}
		\makecell[c]{\textbf{Incremento}} & \makecell[c]{\textbf{Obiettivi}} & \makecell[c]{\textbf{Casi d'uso}} &  \makecell[c]{\textbf{Requisiti}}\\
		\hline
		\multicolumn{4}{|c|}{Fase di progettazione}\\
		\hline
		\centering Incremento 0 & \centering Sviluppo di un
		Proof of Concept$_{\scaleto{G}{3pt}}$ che implementi un software conta persone funzionante che salvi i dati nel database e li visualizzi graficamente in una heat map$_{\scaleto{G}{3pt}}$. & \centering  UC1, UC2, UC3, UC5.1, UC5.3, UC8.1, UC9 & \makecell[c]{RSFO5 \\ RSFO7 \\ RSFO9 \\ RSFO24 \\ RSFO26 \\ RSFO28 \\ RSFO32 \\ RSFO32.1 \\ RSFO32.1.1 \\ RSFO32.1.2 \\ RSFO32.2} \\
		\hline
		\centering Incremento 1 & \centering Incremento della documentazione e preparazione alle attività di progettazione e codifica di dettaglio tramite studio e approfondimenti. & \centering - & \makecell[c]{-} \\
		\hline
		\centering Incremento 2 & \centering Sviluppo e impostazione programma per la raccolta dati e invio informazioni al database; inizio stesura del manuale utente. & \centering UC8.1, UC8.2 & \makecell[c]{RSFO1 \\ RSFO4.1 \\ RSFO22 \\ RSFO22.1 \\ RSFO22.2 \\ RSFO30} \\
		\hline
		\centering Incremento 3 & \centering Sviluppo e impostazione front end$_{\scaleto{G}{3pt}}$ relativo a impianto grafico e richiesta informazioni attraverso uno Spring$_{\scaleto{G}{3pt}}$ controller; correzione della documentazione in base alle segnalazioni ricevute dai committenti$_{\scaleto{G}{3pt}}$. & \centering UC1, UC2, UC3, UC5.1, UC5.3, UC8, UC9 & \makecell[c]{RSFO3 \\ RSFO5\\ RSFO7 \\  RSFO9 \\ RSFO10 \\ RSFO17 \\ RSFO19 \\ RSFO21 \\ RSFO24 \\ RSFO26 \\ RSFO28 \\ RSFO32 \\ RSFO32.1 \\ RSFO32.1.1 \\ RSFO32.1.2 \\ RSFO32.2} \\
		\hline
		\centering Incremento 4 & \centering Implementazione di  un modello machine learning$_{\scaleto{G}{3pt}}$ in grado di elaborare i dati per effettuare predizioni e di salvarli, in modo che siano visualizzabili dall'utente nella heat map$_{\scaleto{G}{3pt}}$. & \centering UC1, UC8.3 & \makecell[c]{RSFO4.2 \\ RSFO11 \\ RSFO18 \\	RSFO18.1} \\
		\hline
		\centering Incremento 5 & \centering Implementazione della funzionalità di selezione e ricerca 
		della città di cui visualizzare i dati; implementazione della possibilità di visualizzare dati di giorni passati. & \centering UC4, UC5.2, UC6, UC6.1, UC6.2, UC7 & \makecell[c]{RSFO20 \\ RSFO27 \\ RSFD33 \\ RSFD33.1 \\ RSFD33.2 \\ RSFD34} \\
		\hline
		\centering Incremento 6 & \centering Completamento manuale utente ed altra documentazione da corredare al prodotto software; controllo del codice e correzione in base alle indicazioni ricevute dal committente$_{\scaleto{G}{3pt}}$. & \centering - & \makecell[c]{-} \\
		\hline
		\centering Incremento 7 & \centering Incremento e verifica finale di tutti i documenti da consegnare in Revisione di Qualifica e preparazione all'esposiozione. & \centering - & \makecell[c]{-} \\
		\hline
		\centering & \centering & \centering & \makecell[c]{} \\
		\hline
		\rowcolor{white}
		\caption[Nome caption]{Tabella degli incrementi}\label{qua va in base alle label di altre tabelle mi sa}
	\end{longtable}
\end{center}
