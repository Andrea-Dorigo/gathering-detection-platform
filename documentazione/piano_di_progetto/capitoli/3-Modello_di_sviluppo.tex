\chapter{Modello di sviluppo}\label{ModelloDiSviluppo}
Per lo sviluppo del progetto il gruppo ha deciso di adottare il \textbf{modello incrementale}.
La scelta di un modello è fondamentale per il corretto svolgimento del progetto: l'adozione di uno standard garantisce sicurezza e avanzamento sia al fornitore$_G$ che al proponente$_{\scaleto{G}{3pt}}$.
\section{Modello incrementale}\label{ModelloDiSviluppoModelloIncrementale}
Una prerogativa del gruppo è la qualità, la quale deve riflettersi anche nel modello di sviluppo al fine di raggiungere gli obiettivi delineati dal modello stesso e realizzare così il progetto in modo corretto e coerente.
Sulla base di queste considerazioni e sulla valutazione della natura del progetto, si è deciso di adottare il modello di sviluppo \textbf{incrementale}. Esso prevede lo sviluppo del prodotto tramite incrementi multipli e successivi, ossia dei rilasci che realizzano ciascuno una nuova funzionalità integrata nel sistema.

Nel modello di sviluppo incrementale i requisiti$_G$ vengono classificati in base alla loro importanza strategica a livello di sistema. I requisiti$_{\scaleto{G}{3pt}}$ più importanti sono trattati dai primi incrementi, in modo da renderli chiari e stabili nel minor tempo possibile per poterli poi soddisfare con maggiore facilità.
Gli incrementi successivi coprono, quindi, requisiti$_{\scaleto{G}{3pt}}$ meno importanti e perciò che hanno più tempo per integrarsi con il sistema.
Sebbene il modello di sviluppo non lo preveda, considerando il numero di componenti e di funzionalità che realizzano il sistema, sono consentite modifiche, aggiunte e rimozioni di requisiti$_{\scaleto{G}{3pt}}$.
Tali operazioni sono possibili solamente previa valutazione ed approvazione da parte del proponente$_{\scaleto{G}{3pt}}$. Per queste modifiche, che non possono essere discusse durante lo sviluppo di un incremento, è necessario prima effettuare il rilascio e poi valutare il cambiamento dei requisiti$_{\scaleto{G}{3pt}}$.

Abbiamo scelto il modello incrementale in quanto:
\begin{itemize}
	\item ogni incremento produce un valore aggiunto, rendendo disponibili delle nuove funzionalità e chiarendo meglio i requisiti$_{\scaleto{G}{3pt}}$ per gli incrementi successivi;
	\item ad ogni incremento è possibile ricevere in tempi brevi un feedback da parte del proponente$_{\scaleto{G}{3pt}}$ sull'insieme delle funzionalità sviluppate;
	\item le funzionalità principali vengono sviluppate all'inizio con i primi incrementi, in quanto relative ai requisiti$_{\scaleto{G}{3pt}}$ più importanti;
	\item ad ogni incremento vengono svolte attività di verifica come aggiunte e modifiche, rendendo l'intera verifica più semplice ed economica, in quanto il resto del prodotto era già stato testato con gli incrementi precedenti;
	\item gli errori in un singolo incremento sono più facili da individuare e correggere, in quanto relativi solo alle modifiche apportate all'incremento;
	\item ogni incremento riduce il rischio di fallimento.
\end{itemize}

\section{Confronto con il modello iterativo}\label{ModelloDiSviluppoConfrontoConIlModelloIterativo}
Durante la scelta del modello da adottare, il gruppo ha valutato attentamente anche il \textbf{modello iterativo}.
L'elasticità data da tale modello comporta un'elevata capacità di adattamento all'insorgere di eventuali problemi legati alle nuove tecnologie e ai requisiti$_{\scaleto{G}{3pt}}$, fattore molto rilevante nello sviluppo del capitolato$_{\scaleto{G}{3pt}}$ \textit{GDP: Gathering Detection Platform}.
Tuttavia per una buona esecuzione del progetto e della pianificazione, è necessario adottare un modello di sviluppo che, in base alle sue caratteristiche, limiti la progettazione stessa.
