\input{packages}
\input{config}

\begin{document}

\makeatletter
\begin{titlepage}
	\begin{center}
		\vspace*{-5cm}
		\author{Jawa Druids}
		\title{Piano di Progetto}
		\date{} %LASCIARE QUESTO CAMPO VUOTO, SE LO TOLGO STAMPA LA DATA CORRENTE
		\includegraphics[width=0.5\linewidth]{../immagini/DRUIDSLOGO.jpg}\\[4ex]
		{\huge \bfseries  \@title }\\[2ex]
		{\LARGE  \@author}\\[50ex]
		\vspace*{-9cm}
		\begin{table}[H]
			\renewcommand{\arraystretch}{1.4}
			\centering
			\begin{tabular}{r | l}
				\textbf{Versione} & x.x.x \\%RIGA PER INSERIRE LA VERSIONE ULTIMA DEL DOCUMENTO
				\textbf{Data approvazione} & xx-xx-xxxx\\
				\textbf{Responsabile} & Nome Cognome\\
				\textbf{Redattori} & \makecell[tl]{Andrea Dorigo \\ Margherita Mitillo \\ Mattia Cocco} \\
				\textbf{Verificatori} & Nome Cognome \\
		%MAKECELL SERVE PER POI ANDARE A CAPO ALL'INTERNO DELLA CELLA
				\textbf{Stato} & Redazione in corso\\
				\textbf{Lista distribuzione} & \makecell[tl]{Jawa Druids \\ prof. Tullio Vardanega \\ prof. Riccardo Cardin \\ Sync Lab}\\
				\textbf{Uso} & Esterno
			\end{tabular}
		\end{table}
		\vspace{0.1cm}
		\hfill \break
		\fontsize{17}{10}\textbf{Sommario} \\
		\vspace{0.1cm}
    Il presente documento contiene la pianificazione delle attività del gruppo Jawa Druids atte al soddisfacimento del capitolato \normalsize\textit{GDP: Gathering Detection Platform} di Sync Lab.
	\end{center}
\end{titlepage}
\makeatother

\quad
\begin{center}
	\LARGE\textbf{Registro delle modifiche}
\end{center}

\def\tabularxcolumn#1{m{#1}}
{\rowcolors{2}{RawSienna!90!RawSienna!20}{RawSienna!70!RawSienna!40}


\begin{center}
	\renewcommand{\arraystretch}{1.4}
	\begin{longtable}[c]{|p{2cm-1\tabcolsep}|p{2cm}|p{3cm-2\tabcolsep}|p{2,5cm-2\tabcolsep}|p{4cm-2\tabcolsep}|p{2,5cm}|}
		\hline
		\rowcolor{airforceblue}
		\makecell[c]{\textbf{Versione}} & \makecell[c]{\textbf{Data}} & \makecell[c]{\textbf{Autore}} & \makecell[c]{\textbf{Ruolo}} & \makecell[c]{\textbf{Modifica}} & \makecell[c]{\textbf{Verificatore}} \\
		\hline
		\centering v1.0.0 & 2021-04-18 & Andrea Dorigo & \centering \textit{Responsabile} & \textit{Approvazione del documento} & \makecell[c]{-} \\
		\hline
		\centering v0.1.0 & 2021-04-17 &  \centering - & \centering - &  \textit{Revisione complessiva del documento} & Mattia Cocco \\
		\hline
		\centering v0.0.1 & 2021-04-15 & Andrea Cecchin & \centering \textit{Redattore} &\textit{Stesura del documento}  & Mattia Cocco \\
		\hline
	\end{longtable}
\end{center}


%COMANDO PER LA CREAZIONE DELL'INDICE

\tableofcontents{}
\listoffigures
\listoftables
\chapter{Introduzione}

Lo scopo di questo documento è la stesura di un elenco di linee guida ed esempi che i componenti sono incoraggiati a seguire per migliorare l'efficacia della collaborazione.
A differenza delle Norme di Progetto, questo documento è redatto in un linguaggio più informale per facilitarne la comprensione, con spezzati di codice a scopo esemplificativo e riferimenti esterni sulle best practices da seguire.\\
Il documento può essere soggetto a modifiche ed aggiunte per tutta la durata del progetto.

\chapter{Analisi dei rischi}\label{AnalisiDeiRischi}

\section{Piano per la gestione dei rischi}\label{AnalisiDeiRischiPianoPerLaGestioneDeiRischi}
Con l'intento di prevenire il naturale insorgere di problemi durante lo svolgimento del progetto è stato elaborato un approfondito piano per la gestione dei rischi. Quest'ultimo è suddiviso in quattro attività$_{\scaleto{G}{3pt}}$:
\begin{itemize}
  \item \textbf{Individuazione dei rischi:} attività$_{\scaleto{G}{3pt}}$ di identificazione e documentazione di possibili elementi problematici che possano ostacolare il naturale percorso del progetto;
  \item \textbf{Analisi dei rischi:} attività$_{\scaleto{G}{3pt}}$ di analisi dei fattori di rischio, che si articola in probabilità di occorrenza, indice di gravità e conseguente impatto sul progetto;
  \item \textbf{Pianificazione di controllo:} attività$_{\scaleto{G}{3pt}}$ di pianificazione delle misure da adottare per la prevenzione e contenimento del problema;
  \item \textbf{Monitoraggio dei rischi:} attività$_{\scaleto{G}{3pt}}$ di controllo dei rischi che accompagna tutto lo svolgimento del progetto, al fine di evitarli o agire tempestivamente alla loro occorrenza per contenerne i danni.
\end{itemize}
Le principali tipologie di rischio sono state quindi codificate e categorizzate come segue:
\begin{itemize}
  \item \textbf{RT:} Rischi legati alle tecnologie;
  \item \textbf{RO:} Rischi legati all'organizzazione;
  \item \textbf{RI:} Rischi interpersonali, ovvero legati alle relazioni personali interne ed esterne o alla disponibilità e alle risorse dei componenti.
\end{itemize}

\clearpage
\quad
\begin{center}
	\LARGE\textbf{Rischi legati alle tecnologie}
\end{center}

\def\tabularxcolumn#1{m{#1}}
{\rowcolors{2}{RawSienna!90!RawSienna!20}{RawSienna!70!RawSienna!40}

	\begin{center}
		\renewcommand{\arraystretch}{1.4}
		\begin{longtable}{|p{5cm}|p{12cm}|}
			\hline
			\rowcolor{airforceblue}
			\multicolumn{2}{|c|}{\textit{Inesperienza tecnologica}}\\
			\hline
			\textit{Codice} & RT1 \\
			\hline
			\textit{Descrizione} & Alcune tecnologie utilizzate in questo progetto sono nuove per tutti i membri del gruppo di lavoro. \\
			\hline
			\textit{Conseguenza} & Lo studio e l'apprendimento di tali tecnologie potrebbero richiedere un intervallo di tempo difficile da quantificare, maggiore del previsto e variabile da membro a membro con conseguenti difficoltà operative. \\
			\hline
			\textit{Possibilità di occorrenza} & Alta. \\
			\hline
			\textit{Pericolosità} & Alta. \\
			\hline
			\textit{Precauzioni} & Il \textit{Responsabile di Progetto} dovrà suddividere i compiti$_G$ nel modo più congruo possibile, considerando le conoscenze preliminari di ciascun componente; prevederà inoltre un tempo di Slack$_G$ maggiore per i compiti$_{\scaleto{G}{3pt}}$ assegnati ad un componente senza particolare familiarità con la relativa tecnologia. Il \textit{Responsabile di Progetto} assegnerà i task$_G$ di maggiore complessità a più membri ove necessario.  \\
			\hline
			\textit{Piano di contingenza} & Ciascun membro comunicherà il prima possibile al \textit{Responsabile di progetto} la previsione di un eventuale ritardo o mancanza; egli provvederà a ridistribuire i compiti$_{\scaleto{G}{3pt}}$ se necessario in modo da sanare eventuali lacune o sottostime. \\
			\hline
		\end{longtable}
	\captionof{table}{\textbf{Analisi dei rischi delle tecnologie utilizzate}}
	\end{center}


\def\tabularxcolumn#1{m{#1}}
{\rowcolors{2}{RawSienna!90!RawSienna!20}{RawSienna!70!RawSienna!40}

	\begin{center}
		\renewcommand{\arraystretch}{1.4}
		\begin{longtable}{|p{5cm}|p{12cm}|}
			\hline
			\rowcolor{airforceblue}
			\multicolumn{2}{|c|}{\textit{Software terze parti}}\\
			\hline
			\textit{Codice} & RT2 \\
			\hline
			\textit{Descrizione} & Eventuali problematiche con software di terze parti, quali la mancanza di documentazione o problemi tecnici, sono indipendenti dai membri del gruppo. \\
			\hline
			\textit{Conseguenza} & Ciò causerebbe ritardi pesanti sul proseguo del lavoro e anche possibili ritardi sulla consegna.
			La necessità di cambiare tecnologia potrebbe richiedere molto tempo e risorse per la ricerca di una sostituzione. \\
			\hline
			\textit{Possibilità di occorrenza} & Bassa. \\
			\hline
			\textit{Pericolosità} & Alta. \\
			\hline
			\textit{Precauzioni} & Il gruppo sceglierà i software più stabili e documentati per evitare questi tipi di problemi.  \\
			\hline
			\textit{Piano di contingenza} & Assieme al \textit{Responsabile di progetto} il gruppo di lavoro si attiverà al fine di tentare di risolvere il problema. Se ciò non è possibile sarà necessario un cambio di tecnologia, anche tramite l'aiuto del proponente$_{\scaleto{G}{3pt}}$.  \\
			\hline
		\end{longtable}
		\captionof{table}{\textbf{Analisi dei rischi dei software di terze parti}}
	\end{center}


\def\tabularxcolumn#1{m{#1}}
{\rowcolors{2}{RawSienna!90!RawSienna!20}{RawSienna!70!RawSienna!40}

	\begin{center}
		\renewcommand{\arraystretch}{1.4}
		\begin{longtable}{|p{5cm}|p{12cm}|}
			\hline
			\rowcolor{airforceblue}
			\multicolumn{2}{|c|}{\textit{Validità dei dati}}\\
			\hline
			\textit{Codice} & RT3 \\
			\hline
			\textit{Descrizione} & Problemi legati alla validità e all'elaborazione dei dati. \\
			\hline
			\textit{Conseguenza} & Arresto obbligato del lavoro in corso, con possibilità di invalidazione del lavoro svolto fino a quel momento. \\
			\hline
			\textit{Possibilità di occorrenza} & Medio/Alta. \\
			\hline
			\textit{Pericolosità} & Molto alta. \\
			\hline
			\textit{Precauzioni} & Prima dell'inizio della raccolta dati il gruppo si assicurerà che la fonte sia affidabile e coerente.
			Questa operazione sarà svolta per prima in quanto critica per l'intero sviluppo.  \\
			\hline
			\textit{Piano di contingenza} & Il gruppo, insieme al proponente$_{\scaleto{G}{3pt}}$, valuterà se sarà necessario cambiare solo la fonte di provenienza dei dati oppure simularli in maniera consona. \\
			\hline
		\end{longtable}
		\captionof{table}{\textbf{Analisi dei rischi della validità dei dati}}
	\end{center}

\def\tabularxcolumn#1{m{#1}}
{\rowcolors{2}{RawSienna!90!RawSienna!20}{RawSienna!70!RawSienna!40}
	
	\begin{center}
		\renewcommand{\arraystretch}{1.4}
		\begin{longtable}{|p{5cm}|p{12cm}|}
			\hline
			\rowcolor{airforceblue}
			\multicolumn{2}{|c|}{\textit{Malfunzionamenti hardware o software dei pc dei membri del gruppo}}\\
			\hline
			\textit{Codice} & RT4 \\
			\hline
			\textit{Descrizione} & Può accadere che qualcuno abbia il PC non funzionante o in assistenza e non possa contribuire attivamente alla realizzazione del prodotto. \\
			\hline
			\textit{Conseguenza} & Ciò causerebbe una mole di lavoro maggiore per gli altri componenti del gruppo ed eventuali ritardi.\\
			\hline
			\textit{Possibilità di occorrenza} & Bassa. \\
			\hline
			\textit{Pericolosità} & Media. \\
			\hline
			\textit{Precauzioni} & Ogni membro del team deve monitorare il corretto funzionamento del proprio PC.  \\
			\hline
			\textit{Piano di contingenza} & A seconda della gravità del problema si provvederà alla reinstallazione del software, del sistema operativo o della sostituzione della propria macchina. \\
			\hline
		\end{longtable}
		\captionof{table}{\textbf{Analisi dei rischi del malfunzionamento del PC}}
	\end{center}
\clearpage
\quad
\begin{center}
	\LARGE\textbf{Rischi legati all'organizzazione}
\end{center}

\def\tabularxcolumn#1{m{#1}}
{\rowcolors{2}{RawSienna!90!RawSienna!20}{RawSienna!70!RawSienna!40}

	\begin{center}
		\renewcommand{\arraystretch}{1.4}
		\begin{longtable}{|p{5cm}|p{12cm}|}
			\hline
			\rowcolor{airforceblue}
			\multicolumn{2}{|c|}{\textit{Contrasti tra i componenti}}\\
			\hline
			\textit{Codice} & RO1 \\
			\hline
			\textit{Descrizione} & Può presentarsi la possibilità che durante le varie attività possano emergere contrasti e tensioni tra i componenti. \\
			\hline
			\textit{Conseguenza} & Tensioni o contrasti tra i componenti sfavoriscono il corretto proseguimento del progetto. \\
			\hline
			\textit{Possibilità di occorrenza} & Bassa. \\
			\hline
			\textit{Pericolosità} & Alta. \\
			\hline
			\textit{Precauzioni} & Ogni membro del gruppo di lavoro cercherà di essere più comprensibile e limitare così eventuali tensioni a favore del collettivo. \\
			\hline
			\textit{Piano di contingenza} & Il \textit{Responsabile di Progetto} avrà l'incarico di mediatore in tali controversie.
			Eventualmente, insieme al resto del gruppo si cercherà di sanare le discordie e solamente in casi estremi verrà chiamato in causa il Prof. Tullio Vardanega.  \\
			\hline
		\end{longtable}
	\captionof{table}{\textbf{Analisi dei rischi per i contrasti tra i componenti}}
	\end{center}


\def\tabularxcolumn#1{m{#1}}
{\rowcolors{2}{RawSienna!90!RawSienna!20}{RawSienna!70!RawSienna!40}

	\begin{center}
		\renewcommand{\arraystretch}{1.4}
		\begin{longtable}{|p{5cm}|p{12cm}|}
			\hline
			\rowcolor{airforceblue}
			\multicolumn{2}{|c|}{\textit{Impegni personali}}\\
			\hline
			\textit{Codice} & RO2 \\
			\hline
			\textit{Descrizione} & Può presentarsi la possibilità che in alcuni momenti uno o più componenti del gruppo abbiano degli impegni accademici o personali. \\
			\hline
			\textit{Conseguenza} & Rallentamento del lavoro.\\
			\hline
			\textit{Possibilità di occorrenza} & Media. \\
			\hline
			\textit{Pericolosità} & Media. \\
			\hline
			\textit{Precauzioni} & Essenziale sarà il comunicare gli impegni al \textit{Responsabile di Progetto} appena il componente ne viene a conoscenza. \\
			\hline
			\textit{Piano di contingenza} & Il \textit{Responsabile di Progetto} provvederà ad approvare delle modifiche organizzative per evitare o limitare rallentamenti ai lavori. \\
			\hline
		\end{longtable}
		\captionof{table}{\textbf{Analisi dei rischi sugli impegni personali}}
	\end{center}

\def\tabularxcolumn#1{m{#1}}
{\rowcolors{2}{RawSienna!90!RawSienna!20}{RawSienna!70!RawSienna!40}
	
	\begin{center}
		\renewcommand{\arraystretch}{1.4}
		\begin{longtable}{|p{5cm}|p{12cm}|}
			\hline
			\rowcolor{airforceblue}
			\multicolumn{2}{|c|}{\textit{Calcolo dei tempi e dei costi}}\\
			\hline
			\textit{Codice} & RO3 \\
			\hline
			\textit{Descrizione} & E' possibile che i tempi e i costi preventivati si rivelino imprecisi con l'avanzamento del progetto. \\
			\hline
			\textit{Conseguenza} & Costi preventivati sbagliati. \\
			\hline
			\textit{Possibilità di occorrenza} & Alta. \\
			\hline
			\textit{Pericolosità} & Alta. \\
			\hline
			\textit{Precauzioni} & Nel caso in cui un componente riscontri una differenza dalle ore di lavoro preventivate, dovrà farlo presente al \textit{Responsabile di Progetto}.  \\
			\hline
			\textit{Piano di contingenza} & Nel caso in cui una stima oraria risulti non sufficiente per portare a termine la consegna, il \textit{Responsabile di Progetto} provvederà ad assegnare più risorse in modo da limitare rallentamenti. Eventualmente se ci dovessero essere lo stesso variazioni al preventivo, allora il \textit{Responsabile di Progetto} provvederà a comunicarlo al Committente$_{\scaleto{G}{3pt}}$ \\
			\hline
		\end{longtable}
		\captionof{table}{\textbf{Analisi dei rischi del calcolo dei tempi e dei costi}}
	\end{center}

\def\tabularxcolumn#1{m{#1}}
{\rowcolors{2}{RawSienna!90!RawSienna!20}{RawSienna!70!RawSienna!40}
	
	\begin{center}
		\renewcommand{\arraystretch}{1.4}
		\begin{longtable}{|p{5cm}|p{12cm}|}
			\hline
			\rowcolor{airforceblue}
			\multicolumn{2}{|c|}{\textit{Inesperienza nel coordinamento}}\\
			\hline
			\textit{Codice} & RO4 \\
			\hline
			\textit{Descrizione} & I membri non hanno esperienza di lavoro che richieda il coordinamento di sette persone.\\
			\hline
			\textit{Conseguenza} & Problematiche o ritardi a causa di una mancata o scarsa organizzazione del team con tempi di latenza e compiti svolti più volte da membri differenti. \\
			\hline
			\textit{Possibilità di occorrenza} & Alta. \\
			\hline
			\textit{Pericolosità} & Alta. \\
			\hline
			\textit{Precauzioni} & Il \textit{Responsabile di Progetto} deve, insieme al resto del gruppo, pianificare le mansioni. Si cercherà di avere una rotazione dei ruoli in modo da far collaborare tutti in modo che ciascuna attività venga svolta dai componenti con più esperienza, insieme a quelli che ancora non ne hanno.  \\
			\hline
			\textit{Piano di contingenza} & Qualunque difficoltà sarà notificata al \textit{Responsabile di Progetto}, che dopo essersi consultato con il gruppo, provvederà eventualmente ad assegnare un compito più semplice all'interessato. \\
			\hline
		\end{longtable}
		\captionof{table}{\textbf{Analisi dei rischi per inesperienza nel coordinamento}}
	\end{center}

\def\tabularxcolumn#1{m{#1}}
{\rowcolors{2}{RawSienna!90!RawSienna!20}{RawSienna!70!RawSienna!40}
	
	\begin{center}
		\renewcommand{\arraystretch}{1.4}
		\begin{longtable}{|p{5cm}|p{12cm}|}
			\hline
			\rowcolor{airforceblue}
			\multicolumn{2}{|c|}{\textit{Scarsa comunicazione}}\\
			\hline
			\textit{Codice} & RO5 \\
			\hline
			\textit{Descrizione} & Per avanzare nelle attività pianificate e rispettare le scadenze, è necessaria una comunicazione costante tra tutti i membri del gruppo.\\
			\hline
			\textit{Conseguenza} & Problematiche o ritardi a causa di una scarsa comunicazione del team. \\
			\hline
			\textit{Possibilità di occorrenza} & Medio. \\
			\hline
			\textit{Pericolosità} & Alta. \\
			\hline
			\textit{Precauzioni} & Il \textit{Responsabile di Progetto} provvederà a promuovere un adeguato livello di comunicazione tra i vari componenti del gruppo.  \\
			\hline
			\textit{Piano di contingenza} & Nel caso si rilevi una scarsa comunicazione sarà compito del \textit{Responsabile di Progetto} provvedere a risolverlo, aiutandosi attraverso una riunione interna al gruppo per discuterne la situazione. \\
			\hline
		\end{longtable}
		\captionof{table}{\textbf{Analisi dei rischi per scarsa comunicazione}}
	\end{center}

\clearpage
\def\tabularxcolumn#1{m{#1}}
{\rowcolors{2}{RawSienna!90!RawSienna!20}{RawSienna!70!RawSienna!40}
	
	\begin{center}
		\renewcommand{\arraystretch}{1.4}
		\begin{longtable}{|p{5cm}|p{12cm}|}
			\hline
			\rowcolor{airforceblue}
			\multicolumn{2}{|c|}{\textit{Approvazione errata dei documenti}}\\
			\hline
			\textit{Codice} & RO6 \\
			\hline
			\textit{Descrizione} & E' possibile che il \textit{Responsabile di Progetto} durante l'approvazione non si accorga o commetta alcuni errori.\\
			\hline
			\textit{Conseguenza} & Approvazione e possibile consegna di documenti errati. \\
			\hline
			\textit{Possibilità di occorrenza} & Bassa. \\
			\hline
			\textit{Pericolosità} & Alta. \\
			\hline
			\textit{Precauzioni} & Per ogni documento devono essere eseguiti controlli costanti, in modo che sia possibile identificare in maniera tempestiva gli eventuali errori.  \\
			\hline
			\textit{Piano di contingenza} & Il \textit{Responsabile di Progetto} si dovrà occupare di controllare che i documenti da approvare siano effettivamente validi. \\
			\hline
		\end{longtable}
		\captionof{table}{\textbf{Analisi dei rischi per approvazione errata dei documenti}}
	\end{center}

\def\tabularxcolumn#1{m{#1}}
{\rowcolors{2}{RawSienna!90!RawSienna!20}{RawSienna!70!RawSienna!40}
	
	\begin{center}
		\renewcommand{\arraystretch}{1.4}
		\begin{longtable}{|p{5cm}|p{12cm}|}
			\hline
			\rowcolor{airforceblue}
			\multicolumn{2}{|c|}{\textit{Analisi dei requisiti imperfetta}}\\
			\hline
			\textit{Codice} & RO7 \\
			\hline
			\textit{Descrizione} & E' possibile che a causa dell'inesperienza del gruppo venga prodotta un'Analisi dei Requisiti insoddisfacente.\\
			\hline
			\textit{Conseguenza} & La proposta potrebbe risultare inadeguata alle aspettative. \\
			\hline
			\textit{Possibilità di occorrenza} & Media. \\
			\hline
			\textit{Pericolosità} & Alta. \\
			\hline
			\textit{Precauzioni} & Ogni dubbio verrà discusso con il proponente$_{\scaleto{G}{3pt}}$.  \\
			\hline
			\textit{Piano di contingenza} & Qualsiasi errore verrà corretto con la massima priorità. \\
			\hline
		\end{longtable}
		\captionof{table}{\textbf{Analisi dei rischi per l'analisi dei requisiti imperfetta}}
	\end{center}
\clearpage
\quad
\begin{center}
	\LARGE\textbf{Rischi interpersonali}
\end{center}

\def\tabularxcolumn#1{m{#1}}
{\rowcolors{2}{RawSienna!90!RawSienna!20}{RawSienna!70!RawSienna!40}

	\begin{center}
		\renewcommand{\arraystretch}{1.4}

		\begin{longtable}{|p{5cm}|p{12cm}|}
			\hline
			\rowcolor{airforceblue}
			\multicolumn{2}{|c|}{\textit{Comunicazione interna}}\\
			\hline
			\textit{Codice} & RI1 \\
			\hline
			\textit{Descrizione} & Potrebbero esserci momenti nei quali uno o più componenti non sono reperibili. \\
			\hline
			\textit{Conseguenza} & Rallentamenti del lavoro qualora non si riuscisse a comunicare con la persona interessata. \\
			\hline
			\textit{Possibilità di occorrenza} & Bassa. \\
			\hline
			\textit{Pericolosità} & Alta. \\
			\hline
			\textit{Precauzioni} & E' necessario che ciascun componente riferisca tempestivamente al \textit{Responsabile di Progetto} eventuali momenti nei quali potrebbe non essere reperibile. \\
			\hline
			\textit{Piano di contingenza} & E' stato concordato con il gruppo di svolgere riunioni frequenti per comunicare l'avanzamento del lavoro.\\
			\hline
		\end{longtable}
	\captionof{table}{\textbf{Analisi dei rischi della comunicazione interna}}
	\end{center}

\def\tabularxcolumn#1{m{#1}}
{\rowcolors{2}{RawSienna!90!RawSienna!20}{RawSienna!70!RawSienna!40}
	
	\begin{center}
		\renewcommand{\arraystretch}{1.4}
		
		\begin{longtable}{|p{5cm}|p{12cm}|}
			\hline
			\rowcolor{airforceblue}
			\multicolumn{2}{|c|}{\textit{Comunicazione esterna}}\\
			\hline
			\textit{Codice} & RI2 \\
			\hline
			\textit{Descrizione} & Potrebbero esserci momenti nel quale l'azienda proponente$_{\scaleto{G}{3pt}}$ non è reperibile qualora avessimo necessita di contattarla. \\
			\hline
			\textit{Conseguenza} & Rallentamenti del lavoro qualora non si riuscisse a comunicare. \\
			\hline
			\textit{Possibilità di occorrenza} & Bassa. \\
			\hline
			\textit{Pericolosità} & Media. \\
			\hline
			\textit{Precauzioni} & E' stato creato un canale sulla piattaforma Discord$_G$ per poter comunicare con il proponente$_{\scaleto{G}{3pt}}$ in maniera facile e rapida. \\
			\hline
			\textit{Piano di contingenza} & Qualora si presentasse la necessità di organizzare un incontro con il proponente$_{\scaleto{G}{3pt}}$ è sufficiente richiederlo ed accordarsi con la disponibilità.\\
			\hline
		\end{longtable}
		\captionof{table}{\textbf{Analisi dei rischi della comunicazione esterna}}
	\end{center}

\def\tabularxcolumn#1{m{#1}}
{\rowcolors{2}{RawSienna!90!RawSienna!20}{RawSienna!70!RawSienna!40}
	
	\begin{center}
		\renewcommand{\arraystretch}{1.4}
		
		\begin{longtable}{|p{5cm}|p{12cm}|}
			\hline
			\rowcolor{airforceblue}
			\multicolumn{2}{|c|}{\textit{Stato di malattia}}\\
			\hline
			\textit{Codice} & RI3 \\
			\hline
			\textit{Descrizione} & Uno o più membri del gruppo possono ammalarsi. \\
			\hline
			\textit{Conseguenza} & Può influire sui task$_G$ assegnati. \\
			\hline
			\textit{Possibilità di occorrenza} & Medio/Alta. \\
			\hline
			\textit{Pericolosità} & Media. \\
			\hline
			\textit{Precauzioni} & Il membro del gruppo comunica il proprio stato di poca salute. \\
			\hline
			\textit{Piano di contingenza} & Se la malattia impedisce di lavorare, il componente del gruppo è tenuto a riprendere uno stato di salute ottimale, ed il suo lavoro è ridistribuito tra gli altri componenti del gruppo.\\
			\hline
		\end{longtable}
		\captionof{table}{\textbf{Analisi dei rischi dello stato di malattia}}
	\end{center}

\def\tabularxcolumn#1{m{#1}}
{\rowcolors{2}{RawSienna!90!RawSienna!20}{RawSienna!70!RawSienna!40}
	
	\begin{center}
		\renewcommand{\arraystretch}{1.4}
		
		\begin{longtable}{|p{5cm}|p{12cm}|}
			\hline
			\rowcolor{airforceblue}
			\multicolumn{2}{|c|}{\textit{Stress mentale}}\\
			\hline
			\textit{Codice} & RI4 \\
			\hline
			\textit{Descrizione} & Gli orari di lavoro eccessivi e la situazione di pandemia possono portare a frustazione e disagio. \\
			\hline
			\textit{Conseguenza} & Sintomi fisici di stress ed emotività instabile. \\
			\hline
			\textit{Possibilità di occorrenza} & Media. \\
			\hline
			\textit{Pericolosità} & Media. \\
			\hline
			\textit{Precauzioni} & Il gruppo collabora per creare un clima e delle relazioni all'interno per aiutare a gestire eventuali problematiche. L'appoggio anche se virtuale aiuta a sentirsi parte di un gruppo. \\
			\hline
			\textit{Piano di contingenza} & Per rilasciare la tensione si cerca di fare attività fisica costante.\\
			\hline
		\end{longtable}
		\captionof{table}{\textbf{Analisi dei rischi dello stress mentale}}
	\end{center}


\chapter{Modello di sviluppo}\label{ModelloDiSviluppo}
La scelta di un modello di comprovata efficacia è fondamentale per il corretto svolgimento del progetto: l'adozione di uno standard garantisce sicurezza e avanzamento sia al fornitore$_G$ che al proponente$_{\scaleto{G}{3pt}}$.
\section{Modello incrementale}\label{ModelloDiSviluppoModelloIncrementale}
Per lo sviluppo del progetto il gruppo ha deciso di adottare il \textbf{modello incrementale}.
Una prerogativa del gruppo è la qualità, la quale deve riflettersi anche nel modello di sviluppo al fine di raggiungere gli obiettivi delineati dal modello stesso e realizzare così il progetto in modo corretto e coerente.
Sulla base di queste considerazioni e sulla valutazione della natura del progetto, si è deciso di adottare il modello di sviluppo \textbf{incrementale}. Esso prevede lo sviluppo del prodotto tramite incrementi multipli e successivi, ossia dei rilasci che realizzano ciascuno una nuova funzionalità integrata nel sistema.

Nel modello di sviluppo incrementale i requisiti$_G$ vengono classificati in base alla loro importanza strategica a livello di sistema. I requisiti$_{\scaleto{G}{3pt}}$ più importanti sono trattati dai primi incrementi, in modo da renderli chiari e stabili nel minor tempo possibile per poterli poi soddisfare con maggiore facilità.
Gli incrementi successivi coprono, quindi, requisiti$_{\scaleto{G}{3pt}}$ meno importanti e perciò che hanno più tempo per integrarsi con il sistema.
Sebbene il modello di sviluppo non lo preveda, considerando il numero di componenti e di funzionalità che realizzano il sistema, sono consentite modifiche, aggiunte e rimozioni di requisiti$_{\scaleto{G}{3pt}}$.
Tali operazioni sono possibili solamente previa valutazione ed approvazione da parte del proponente$_{\scaleto{G}{3pt}}$. Per queste modifiche, che non possono essere discusse durante lo sviluppo di un incremento, è necessario prima effettuare il rilascio e poi valutare il cambiamento dei requisiti$_{\scaleto{G}{3pt}}$.

Abbiamo scelto il modello incrementale in quanto:
\begin{itemize}
	\item ogni incremento produce un valore aggiunto, rendendo disponibili delle nuove funzionalità e chiarendo meglio i requisiti$_{\scaleto{G}{3pt}}$ per gli incrementi successivi;
	\item ad ogni incremento è possibile ricevere in tempi brevi un feedback da parte del proponente$_{\scaleto{G}{3pt}}$ sull'insieme delle funzionalità sviluppate;
	\item le funzionalità principali vengono sviluppate all'inizio con i primi incrementi, in quanto relative ai requisiti$_{\scaleto{G}{3pt}}$ più importanti;
	\item ad ogni incremento vengono svolte attività di verifica come aggiunte e modifiche, rendendo l'intera verifica più semplice ed economica, in quanto il resto del prodotto era già stato testato con gli incrementi precedenti;
	\item gli errori in un singolo incremento sono più facili da individuare e correggere, in quanto relativi solo alle modifiche apportate all'incremento;
	\item ogni incremento riduce il rischio di fallimento.
\end{itemize}

\section{Confronto con il modello iterativo}\label{ModelloDiSviluppoConfrontoConIlModelloIterativo}
Durante la scelta del modello da adottare, il gruppo ha valutato attentamente anche il \textbf{modello iterativo}.
L'elasticità data da tale modello comporta un'elevata capacità di adattamento all'insorgere di eventuali problemi legati alle nuove tecnologie e ai requisiti$_{\scaleto{G}{3pt}}$, fattore molto rilevante nello sviluppo del capitolato$_{\scaleto{G}{3pt}}$ \textit{GDP: Gathering Detection Platform}.
Tuttavia per una buona esecuzione del progetto e della pianificazione, è necessario adottare un modello di sviluppo che, in base alle sue caratteristiche, limiti la progettazione stessa.

\section{Incrementi}\label{ModelloDiSviluppoIncrementi}
In questa sezione viene riportata una tabella contenente i dettagli di sviluppo di ogni incremento, facendo riferimento agli obiettivi, ai casi d'uso$_{\scaleto{G}{3pt}}$ e ai requisiti$_{\scaleto{G}{3pt}}$ di ognuno di essi.
\begin{center}
	\renewcommand{\arraystretch}{1.4}
	\begin{longtable}[c]{p{4cm}|p{4cm}|p{4cm}|p{3cm}}
		\hline
		\rowcolor{airforceblue}
		\makecell[c]{\textbf{Incremento}} & \makecell[c]{\textbf{Obiettivi}} & \makecell[c]{\textbf{Casi d'uso}} &  \makecell[c]{\textbf{Requisiti}}\\
		\hline
		\multicolumn{4}{|c|}{Fase di progettazione}\\
		\hline
		\centering Incremento 0 & \centering Sviluppo di un
		Proof of Concept$_{\scaleto{G}{3pt}}$ che implementi un software conta persone funzionante che salvi i dati nel database e li visualizzi graficamente in una heat map$_{\scaleto{G}{3pt}}$. & \centering  UC1, UC2, UC3, UC5.1, UC5.3, UC8.1, UC9 & \makecell[c]{RSFO5 \\ RSFO7 \\ RSFO9 \\ RSFO24 \\ RSFO26 \\ RSFO28 \\ RSFO32 \\ RSFO32.1 \\ RSFO32.1.1 \\ RSFO32.1.2 \\ RSFO32.2} \\
		\hline
		\centering Incremento 1 & \centering Incremento della documentazione e preparazione alle attività di progettazione e codifica di dettaglio tramite studio e approfondimenti. & \centering - & \makecell[c]{-} \\
		\hline
		\centering Incremento 2 & \centering Sviluppo e impostazione programma per la raccolta dati e invio informazioni al database; inizio stesura del manuale utente. & \centering UC8.1, UC8.2 & \makecell[c]{RSFO1 \\ RSFO4.1 \\ RSFO22 \\ RSFO22.1 \\ RSFO22.2 \\ RSFO30} \\
		\hline
		\centering Incremento 3 & \centering Sviluppo e impostazione front end$_{\scaleto{G}{3pt}}$ relativo a impianto grafico e richiesta informazioni attraverso uno Spring$_{\scaleto{G}{3pt}}$ controller; correzione della documentazione in base alle segnalazioni ricevute dai committenti$_{\scaleto{G}{3pt}}$. & \centering UC1, UC2, UC3, UC5.1, UC5.3, UC8, UC9 & \makecell[c]{RSFO3 \\ RSFO5\\ RSFO7 \\  RSFO9 \\ RSFO10 \\ RSFO17 \\ RSFO19 \\ RSFO21 \\ RSFO24 \\ RSFO26 \\ RSFO28 \\ RSFO32 \\ RSFO32.1 \\ RSFO32.1.1 \\ RSFO32.1.2 \\ RSFO32.2} \\
		\hline
		\centering Incremento 4 & \centering Implementazione di  un modello machine learning$_{\scaleto{G}{3pt}}$ in grado di elaborare i dati per effettuare predizioni e di salvarli, in modo che siano visualizzabili dall'utente nella heat map$_{\scaleto{G}{3pt}}$. & \centering UC1, UC8.3 & \makecell[c]{RSFO4.2 \\ RSFO11 \\ RSFO18 \\	RSFO18.1} \\
		\hline
		\centering Incremento 5 & \centering Implementazione della funzionalità di selezione e ricerca 
		della città di cui visualizzare i dati; implementazione della possibilità di visualizzare dati di giorni passati. & \centering UC4, UC5.2, UC6, UC6.1, UC6.2, UC7 & \makecell[c]{RSFO20 \\ RSFO27 \\ RSFD33 \\ RSFD33.1 \\ RSFD33.2 \\ RSFD34} \\
		\hline
		\centering Incremento 6 & \centering Completamento manuale utente ed altra documentazione da corredare al prodotto software; controllo del codice e correzione in base alle indicazioni ricevute dal committente$_{\scaleto{G}{3pt}}$. & \centering - & \makecell[c]{-} \\
		\hline
		\centering Incremento 7 & \centering Incremento e verifica finale di tutti i documenti da consegnare in Revisione di Qualifica e preparazione all'esposiozione. & \centering - & \makecell[c]{-} \\
		\hline
		\centering & \centering & \centering & \makecell[c]{} \\
		\hline
		\rowcolor{white}
		\caption[Nome caption]{Tabella degli incrementi}\label{qua va in base alle label di altre tabelle mi sa}
	\end{longtable}
\end{center}

\chapter{Pianificazione}\label{Pianificazione}
Il gruppo \textit{Jawa Druids} ha pianificato le attività di progetto seguendo le scadenze riportate nel capitolo \ref{IntroduzioneScadenze}. Il progetto è dunque suddiviso nelle seguenti fasi:
\begin{itemize}
	\item Analisi;
	\item Consolidamento dei requisiti$_{\scaleto{G}{3pt}}$;
	\item Progettazione architetturale;
	\item Progettazione di dettaglio e codifica;
	\item Validazione e collaudo.
\end{itemize}
Ognuna di queste fasi è formata da attività$_{\scaleto{G}{3pt}}$ illustrate nei diagrammi di Gantt$_G$, che permettono la rappresentazione grafica di un calendario, utile al fine di pianificare, coordinare e tracciare specifiche attività dando una chiara illustrazione del suo stato di avanzamento.
\section{Analisi}\label{PianificazioneAnalisi}
\textbf{Periodo:} dal 22-11-2020 al 11-01-2021.\\
Questo periodo ha inizio con la formazione dei gruppi e termina con la scadenza per la consegna dei documenti relativi alla Revisione dei Requisiti.
Le principali attività$_{\scaleto{G}{3pt}}$ svolte in questo periodo riguardano l'analisi dei requisiti$_{\scaleto{G}{3pt}}$ posti dal proponente, la pianificazione, la scelta di metriche adeguate per il \textit{Piano di Qualifica} e la stesura della documentazione necessaria a supporto del progetto stesso, come riportato in seguito:
\begin{itemize}
	\item \textbf{Studio di Fattibilità:} attività$_{\scaleto{G}{3pt}}$ di studio di tutti i capitolati$_{\scaleto{G}{3pt}}$, elencando per ciascuno i punti positivi e negativi che li caratterizzano. Si specificano inoltre le motivazioni riguardanti la scelta del capitolato$_{\scaleto{G}{3pt}}$ \textit{GDP: Gathering Detection Platform}.
	Questa attività$_{\scaleto{G}{3pt}}$ è bloccante per l'inizio dell'\textit{Analisi dei Requisiti};
	\item \textbf{Norme di Progetto:} definisce tutte le regole, convenzioni e tecnologie che il gruppo \textit{Jawa Druids} deve rispettare ed utilizzare durante lo sviluppo dell'intero progetto;
	\item \textbf{Glossario:} racchiude termini che possono risultare ambigui durante lo svolgimento del progetto, con annessa una breve descrizione;
	\item \textbf{Piano di Progetto:} il presente documento in cui le attività$_{\scaleto{G}{3pt}}$, i compiti$_{\scaleto{G}{3pt}}$, e le risorse precedentemente analizzate vengono distribuite tra i componenti di \textit{Jawa Druids}. Presenta inoltre il calcolo del preventivo e le scadenze che il gruppo intende rispettare;
	\item \textbf{Lettera di Presentazione:} lettera in cui il gruppo \textit{Jawa Druids} si candida ufficialmente come fornitore$_{\scaleto{G}{3pt}}$ del prodotto software richiesto;
	\item \textbf{Analisi dei requisiti:} studio ed analisi dei requisiti$_{\scaleto{G}{3pt}}$ del capitolato$_{\scaleto{G}{3pt}}$ scelto nello \textit{Studio di Fattibilità};
	\item \textbf{Piano di qualifica:} documento in cui vengono indicate le strategie di verifica e validazione che il gruppo adotta per garantire la qualità del prodotto software.
\end{itemize}
\subsection{Diagramma di Gantt: Analisi}\label{PianificazioneDiagrammaDiGanttAnalisi}
\begin{figure}[!h]
	\begin{center}
		\includegraphics[width=1\linewidth]{../immagini/pdp/gantt_analisi.png}
		\caption{Diagramma di Gantt dell'attività di analisi}
	\end{center}
\end{figure}

\section{Consolidamento dei requisiti}\label{PianificazioneConsolidamentoDeiRequisiti}
\textbf{Periodo:} dal 11-01-2021 al 18-01-2021
Questo periodo ha inizio subito dopo il termine del precedente e finisce con la presentazione della Revisione dei Requisiti.
Il gruppo \textit{Jawa Druids} si dedicherà ai seguenti compiti$_{\scaleto{G}{3pt}}$:
\begin{itemize}
	\item avanzamento con lo studio individuale relativo a:
	\begin{itemize}
		\item acquisizione dei dati;
		\item simulazione dei dati;
		\item machine learning$_G$;
		\item web app.
	\end{itemize}
	\item preparazione del materiale necessario alla presentazione.
\end{itemize}
\subsection{Diagramma di Gantt: consolidamento dei requisiti}\label{PianificazioneDiagrammaDiGanttConsolidamento}
\begin{figure}[!h]
	\begin{center}
		\includegraphics[width=1\linewidth]{../immagini/pdp/gantt_consolidamento_requisiti.png}
		\caption{Diagramma di Gantt del consolidamento dei requisiti}
	\end{center}
\end{figure}
\clearpage
\section{Progettazione architetturale}\label{PianificazioneProgettazioneArchitetturale}
\textbf{Periodo:} dal 18-01-2021 al 08-03-2021. \\
Questo periodo ha inizio subito dopo conclusione del precedente e termina con la Revisione di Progettazione.
Esso porta all'individuazione di una soluzione architetturale che permetta il soddisfacimento dei requisiti$_{\scaleto{G}{3pt}}$ obbligatori. È suddivisa in:
\begin{itemize}
	\item \textbf{Incremento e verifica:} analizzando l'esito della Revisione dei Requisiti, vengono svolte attività$_{\scaleto{G}{3pt}}$ di Incremento e Verifica sui vari documenti redatti, dove necessario;
	\item \textbf{Technology Baseline$_G$:} viene redatta la documentazione di supporto, contenente la descrizione delle tecnologie individuate e il tracciamento della relazione tra le componenti e i requisiti$_{\scaleto{G}{3pt}}$ che vanno a soddisfare.
	Viene realizzato un Proof of Concept$_G$ che verrà condiviso col proponente$_{\scaleto{G}{3pt}}$ per verificare il corretto sviluppo del software. In particolare il gruppo ha suddiviso ulteriormente questa fase in tre incrementi:
	\begin{itemize}
		\item \textbf{Primo incremento:} Fase di acquisizione dei dati nella quale vengono implementati tre moduli. Il primo serve per la corretta acquisizione dei video delle live-webcam per l'estrapolazione dei frame che verranno utilizzati nel secondo modulo. Quest'ultimo analizza i frame e ritorna il numero di persone presenti in ciascuno di essi ed invia questi dati al terzo modulo caratterizzato dal database.
		\item \textbf{Secondo incremento:} Fase in cui si istruisce tramite machine learning$_G$ la macchina implementando un modulo. Questo modulo prende i dati salvati grazie alla fase precedente e li utilizza per il training e l'elaborazione ed infine restituisce i dati per la previsione. 
		\item \textbf{Terzo incremento:} Fase di creazione della web app dove viene implementato l'ultimo modulo. La web app riceve i dati elaborati tramite il primo ed il secondo incremento e, attraverso essi, mostra una heat map$_G$ del periodo scelto dall'utente.
	\end{itemize}
\end{itemize}
\subsection{Diagramma di Gantt: progettazione architetturale}\label{PianificazioneDiagrammaDiGanttProgettazioneArchitetturale}
\begin{figure}[h]
	\begin{center}
		\includegraphics[width=0.9\linewidth]{../immagini/pdp/gantt_progettazione_architetturale.png}
		\caption{Diagramma di Gantt della progettazione architetturale}
	\end{center}
\end{figure}
\clearpage
\section{Progettazione di dettaglio e codifica}\label{PianificazioneProgettazioneDettaglio}
\textbf{Periodo:} dal 15-03-2021 al 09-04-2021
Questo periodo inizia appena concluso il precedente e termina con la Revisione di Qualifica.
Le principali attività$_{\scaleto{G}{3pt}}$ svolte in questo periodo sono
\begin{itemize}
	\item \textbf{Incremento e verifica:} alcuni dei documenti già prodotti vengono migliorati e aggiornati;
	\item \textbf{Product Baseline$_G$:} segue la Technology Baseline$_{\scaleto{G}{3pt}}$, dove vengono studiati meglio design pattern$_G$, classi e attività$_{\scaleto{G}{3pt}}$ necessarie alla codifica;
	\item \textbf{Specifica Tecnica:} è un documento contenente tutte le caratteristiche del prodotto e le motivazioni che hanno portato alla loro scelta;
	\item \textbf{Codifica:} attività$_{\scaleto{G}{3pt}}$ nella quale viene prodotto e verificato il codice;
	\item \textbf{Manuale utente:} attività$_{\scaleto{G}{3pt}}$ nella quale viene redatto il documento contenente le informazioni su come funziona e su come si utilizza il prodotto.
\end{itemize}
\subsection{Diagramma di Gantt: progettazione di dettaglio e codifica}\label{PianificazioneDiagrammaDiGanttProgettazioneDettaglio}
\begin{figure}[!h]
	\begin{center}
		\includegraphics[width=0.8\linewidth]{../immagini/pdp/gantt_progettazione_dettaglio.png}
		\caption{Diagramma di Gantt dell'attività di progettazione di dettaglio e codifica}
	\end{center}
\end{figure}
\section{Validazione e Collaudo}\label{PianificazioneValidazione}
\textbf{Periodo:} dal 16-04-2021 al 10-05-2021
Questo periodo inizia appena concluso il precedente e termina con la Revisione di Accettazione.
Le principali attività$_{\scaleto{G}{3pt}}$ svolte in questo periodo sono:
\begin{itemize}
	\item \textbf{Incremento e verifica:} analizzando l'esito della Revisione di Qualifica vengono svolte attività$_{\scaleto{G}{3pt}}$ di Incremento e Verifica sui vari documenti redatti;
	\item \textbf{Validazione e Collaudo:} vengono realizzati gli ultimi test, con i dovuti controlli finali, in modo da garantire un buon livello di qualità e correttezza.
\end{itemize}
\subsection{Diagramma di Gantt: validazione e collaudo}\label{PianificazioneDiagrammaDiGanttValidazione}
\begin{figure}[!h]
	\begin{center}
		\includegraphics[width=0.8\linewidth]{../immagini/pdp/gantt_validazione.png}
		\caption{Diagramma di Gantt dell'attività di validazione e collaudo}
	\end{center}
\end{figure}

\chapter{Preventivo}\label{Preventivo}

In questa sezione il gruppo \textit{Jawa Druids} descrive come userà le risorse a sua disposizione. Per identificare i ruoli nelle tabelle vengono indicati i ruoli con le seguenti sigle:
\begin{itemize}
	\item \textbf{Re}: \textit{Responsabile};
	\item \textbf{Am}: \textit{Amministratore};
	\item \textbf{An}: \textit{Analista};
	\item \textbf{Pt}: \textit{Progettista};
	\item \textbf{Pr}: \textit{Programmatore};
	\item \textbf{Ve}: \textit{Verificatore}.
\end{itemize}

\section{Fase di Analisi}\label{5.1}

\subsection{Prospetto orario}\label{5.1.1}
In questa fase la distribuzione oraria è la seguente:
\quad
\def\tabularxcolumn#1{m{#1}}
{\rowcolors{2}{RawSienna!90!RawSienna!20}{RawSienna!70!RawSienna!40}
	
	\begin{center}
		\renewcommand{\arraystretch}{1.4}
		\begin{tabularx}{\textwidth}{|X|c|c|c|c|c|c|c|}
			\hline
			\rowcolor{airforceblue}
			\textbf{Nominativo} & \textbf{Re} & \textbf{Am} & \textbf{An} & \textbf{Pt} & \textbf{Pr} & \textbf{Ve} & \textbf{Totale ore}\\
			\hline
			\textit{Andrea Dorigo} & x & x & x & x & x & x & x\\
			\hline
			\textit{Margherita Mitillo} & x & x & x & x & x & x & x\\
			\hline
			\textit{Igli Mezini} & x & x & x & x & x & x & x\\
			\hline
			\textit{Andrea Cecchin} & x & x & x & x & x & x & x\\
			\hline
			\textit{Emma Roveroni} & x & x & x & x & x & x & x\\
			\hline
			\textit{Alfredo Graziano} & x & x & x & x & x & x & x\\
			\hline
			\textit{Mattia Cocco} & x & x & x & x & x & x & x\\
			\hline
		\end{tabularx}
	\captionof{table}{\textbf{Distribuzione delle ore nel periodo di Analisi}}
	\end{center}

Il seguente istogramma riassume i dati ottenuti:

\subsection{Prospetto economico}\label{5.1.2}
\quad
\def\tabularxcolumn#1{m{#1}}
{\rowcolors{2}{RawSienna!90!RawSienna!20}{RawSienna!70!RawSienna!40}	
	\begin{center}
		\renewcommand{\arraystretch}{1.4}
		\begin{tabularx}{7cm}{|X|c|c|}
			\hline
			\rowcolor{airforceblue}
			\textbf{Ruolo} & \textbf{Ore} & \textbf{Costo}\\
			\hline
			Responsabile & x & x\\
			\hline
			Amministratore & x & x\\
			\hline
			Analista & x & x\\
			\hline
			Progettista & x & x\\
			\hline
			Programmatore & x & x\\
			\hline
			Verificatore & x & x\\
			\hline
			Totale & x & x\\
			\hline
		\end{tabularx}
	\captionof{table}{\textbf{Prospetto dei costi per ruolo nel periodo di Analisi}}
	\end{center}

Il seguente grafico a torta riassume i dati ottenuti:

\section{Fase di Consolidamento dei requisiti}\label{5.2}

\subsection{Prospetto orario}\label{5.2.1}
In questa fase la distribuzione oraria è la seguente:
\quad
\def\tabularxcolumn#1{m{#1}}
{\rowcolors{2}{RawSienna!90!RawSienna!20}{RawSienna!70!RawSienna!40}
	
	\begin{center}
		\renewcommand{\arraystretch}{1.4}
		\begin{tabularx}{\textwidth}{|X|c|c|c|c|c|c|c|}
			\hline
			\rowcolor{airforceblue}
			\textbf{Nominativo} & \textbf{Re} & \textbf{Am} & \textbf{An} & \textbf{Pt} & \textbf{Pr} & \textbf{Ve} & \textbf{Totale ore}\\
			\hline
			\textit{Andrea Dorigo} & x & x & x & x & x & x & x\\
			\hline
			\textit{Margherita Mitillo} & x & x & x & x & x & x & x\\
			\hline
			\textit{Igli Mezini} & x & x & x & x & x & x & x\\
			\hline
			\textit{Andrea Cecchin} & x & x & x & x & x & x & x\\
			\hline
			\textit{Emma Roveroni} & x & x & x & x & x & x & x\\
			\hline
			\textit{Alfredo Graziano} & x & x & x & x & x & x & x\\
			\hline
			\textit{Mattia Cocco} & x & x & x & x & x & x & x\\
			\hline
		\end{tabularx}
	\captionof{table}{\textbf{Distribuzione delle ore nel periodo di Consolidamento dei requisiti}}
	\end{center}

Il seguente istogramma riassume i dati ottenuti:

\subsection{Prospetto economico}\label{5.2.2}
\quad
\def\tabularxcolumn#1{m{#1}}
{\rowcolors{2}{RawSienna!90!RawSienna!20}{RawSienna!70!RawSienna!40}	
	\begin{center}
		\renewcommand{\arraystretch}{1.4}
		\begin{tabularx}{7cm}{|X|c|c|}
			\hline
			\rowcolor{airforceblue}
			\textbf{Ruolo} & \textbf{Ore} & \textbf{Costo}\\
			\hline
			Responsabile & x & x\\
			\hline
			Amministratore & x & x\\
			\hline
			Analista & x & x\\
			\hline
			Progettista & x & x\\
			\hline
			Programmatore & x & x\\
			\hline
			Verificatore & x & x\\
			\hline
			Totale & x & x\\
			\hline
		\end{tabularx}
	\captionof{table}{\textbf{Prospetto dei costi per ruolo nel periodo di Consolidamento dei requisiti}}
	\end{center}

Il seguente grafico a torta riassume i dati ottenuti:

\section{Fase di Progettazione architetturale}\label{5.3}

\subsection{Prospetto orario}\label{5.3.1}
In questa fase la distribuzione oraria è la seguente:
\quad
\def\tabularxcolumn#1{m{#1}}
{\rowcolors{2}{RawSienna!90!RawSienna!20}{RawSienna!70!RawSienna!40}
	
	\begin{center}
		\renewcommand{\arraystretch}{1.4}
		\begin{tabularx}{\textwidth}{|X|c|c|c|c|c|c|c|}
			\hline
			\rowcolor{airforceblue}
			\textbf{Nominativo} & \textbf{Re} & \textbf{Am} & \textbf{An} & \textbf{Pt} & \textbf{Pr} & \textbf{Ve} & \textbf{Totale ore}\\
			\hline
			\textit{Andrea Dorigo} & x & x & x & x & x & x & x\\
			\hline
			\textit{Margherita Mitillo} & x & x & x & x & x & x & x\\
			\hline
			\textit{Igli Mezini} & x & x & x & x & x & x & x\\
			\hline
			\textit{Andrea Cecchin} & x & x & x & x & x & x & x\\
			\hline
			\textit{Emma Roveroni} & x & x & x & x & x & x & x\\
			\hline
			\textit{Alfredo Graziano} & x & x & x & x & x & x & x\\
			\hline
			\textit{Mattia Cocco} & x & x & x & x & x & x & x\\
			\hline
		\end{tabularx}
	\captionof{table}{\textbf{Distribuzione delle ore nel periodo di Progettazione architetturale}}
	\end{center}

Il seguente istogramma riassume i dati ottenuti:

\subsection{Prospetto economico}\label{5.3.2}
\quad
\def\tabularxcolumn#1{m{#1}}
{\rowcolors{2}{RawSienna!90!RawSienna!20}{RawSienna!70!RawSienna!40}	
	\begin{center}
		\renewcommand{\arraystretch}{1.4}
		\begin{tabularx}{7cm}{|X|c|c|}
			\hline
			\rowcolor{airforceblue}
			\textbf{Ruolo} & \textbf{Ore} & \textbf{Costo}\\
			\hline
			Responsabile & x & x\\
			\hline
			Amministratore & x & x\\
			\hline
			Analista & x & x\\
			\hline
			Progettista & x & x\\
			\hline
			Programmatore & x & x\\
			\hline
			Verificatore & x & x\\
			\hline
			Totale & x & x\\
			\hline
		\end{tabularx}
	\captionof{table}{\textbf{Prospetto dei costi per ruolo nel periodo di Progettazione architetturale}}
	\end{center}

Il seguente grafico a torta riassume i dati ottenuti:

\section{Fase di Progettazione di dettaglio e codifica}\label{5.4}

\subsection{Prospetto orario}\label{5.4.1}
In questa fase la distribuzione oraria è la seguente:
\quad
\def\tabularxcolumn#1{m{#1}}
{\rowcolors{2}{RawSienna!90!RawSienna!20}{RawSienna!70!RawSienna!40}
	
	\begin{center}
		\renewcommand{\arraystretch}{1.4}
		\begin{tabularx}{\textwidth}{|X|c|c|c|c|c|c|c|}
			\hline
			\rowcolor{airforceblue}
			\textbf{Nominativo} & \textbf{Re} & \textbf{Am} & \textbf{An} & \textbf{Pt} & \textbf{Pr} & \textbf{Ve} & \textbf{Totale ore}\\
			\hline
			\textit{Andrea Dorigo} & x & x & x & x & x & x & x\\
			\hline
			\textit{Margherita Mitillo} & x & x & x & x & x & x & x\\
			\hline
			\textit{Igli Mezini} & x & x & x & x & x & x & x\\
			\hline
			\textit{Andrea Cecchin} & x & x & x & x & x & x & x\\
			\hline
			\textit{Emma Roveroni} & x & x & x & x & x & x & x\\
			\hline
			\textit{Alfredo Graziano} & x & x & x & x & x & x & x\\
			\hline
			\textit{Mattia Cocco} & x & x & x & x & x & x & x\\
			\hline
		\end{tabularx}
	\captionof{table}{\textbf{Distribuzione delle ore nel periodo di Progettazione di dettaglio e codifica}}
	\end{center}

Il seguente istogramma riassume i dati ottenuti:

\subsection{Prospetto economico}\label{5.4.2}
\quad
\def\tabularxcolumn#1{m{#1}}
{\rowcolors{2}{RawSienna!90!RawSienna!20}{RawSienna!70!RawSienna!40}	
	\begin{center}
		\renewcommand{\arraystretch}{1.4}
		\begin{tabularx}{7cm}{|X|c|c|}
			\hline
			\rowcolor{airforceblue}
			\textbf{Ruolo} & \textbf{Ore} & \textbf{Costo}\\
			\hline
			Responsabile & x & x\\
			\hline
			Amministratore & x & x\\
			\hline
			Analista & x & x\\
			\hline
			Progettista & x & x\\
			\hline
			Programmatore & x & x\\
			\hline
			Verificatore & x & x\\
			\hline
			Totale & x & x\\
			\hline
		\end{tabularx}
	\captionof{table}{\textbf{Prospetto dei costi per ruolo nel periodo di Progettazione di dettaglio e codifica}}
	\end{center}

Il seguente grafico a torta riassume i dati ottenuti:

\section{Fase di Progettazione di Validazione e collaudo}\label{5.5}

\subsection{Prospetto orario}\label{5.5.1}
In questa fase la distribuzione oraria è la seguente:
\quad
\def\tabularxcolumn#1{m{#1}}
{\rowcolors{2}{RawSienna!90!RawSienna!20}{RawSienna!70!RawSienna!40}
	
	\begin{center}
		\renewcommand{\arraystretch}{1.4}
		\begin{tabularx}{\textwidth}{|X|c|c|c|c|c|c|c|}
			\hline
			\rowcolor{airforceblue}
			\textbf{Nominativo} & \textbf{Re} & \textbf{Am} & \textbf{An} & \textbf{Pt} & \textbf{Pr} & \textbf{Ve} & \textbf{Totale ore}\\
			\hline
			\textit{Andrea Dorigo} & x & x & x & x & x & x & x\\
			\hline
			\textit{Margherita Mitillo} & x & x & x & x & x & x & x\\
			\hline
			\textit{Igli Mezini} & x & x & x & x & x & x & x\\
			\hline
			\textit{Andrea Cecchin} & x & x & x & x & x & x & x\\
			\hline
			\textit{Emma Roveroni} & x & x & x & x & x & x & x\\
			\hline
			\textit{Alfredo Graziano} & x & x & x & x & x & x & x\\
			\hline
			\textit{Mattia Cocco} & x & x & x & x & x & x & x\\
			\hline
		\end{tabularx}
	\captionof{table}{\textbf{Distribuzione delle ore nel periodo di Progettazione di Validazione e collaudo}}
	\end{center}

Il seguente istogramma riassume i dati ottenuti:

\subsection{Prospetto economico}\label{5.5.2}
\quad
\def\tabularxcolumn#1{m{#1}}
{\rowcolors{2}{RawSienna!90!RawSienna!20}{RawSienna!70!RawSienna!40}	
	\begin{center}
		\renewcommand{\arraystretch}{1.4}
		\begin{tabularx}{7cm}{|X|c|c|}
			\hline
			\rowcolor{airforceblue}
			\textbf{Ruolo} & \textbf{Ore} & \textbf{Costo}\\
			\hline
			Responsabile & x & x\\
			\hline
			Amministratore & x & x\\
			\hline
			Analista & x & x\\
			\hline
			Progettista & x & x\\
			\hline
			Programmatore & x & x\\
			\hline
			Verificatore & x & x\\
			\hline
			Totale & x & x\\
			\hline
		\end{tabularx}
	\captionof{table}{\textbf{Prospetto dei costi per ruolo nel periodo di Progettazione di Validazione e collaudo}}
	\end{center}

Il seguente grafico a torta riassume i dati ottenuti:

\section{Riepilogo}\label{5.6}

\subsection{Ore totali}\label{5.6.1}

\subsubsection{Suddivisione lavoro}\label{5.6.1.1}
In questa fase la distribuzione oraria è la seguente:
\quad
\def\tabularxcolumn#1{m{#1}}
{\rowcolors{2}{RawSienna!90!RawSienna!20}{RawSienna!70!RawSienna!40}
	
	\begin{center}
		\renewcommand{\arraystretch}{1.4}
		\begin{tabularx}{\textwidth}{|X|c|c|c|c|c|c|c|}
			\hline
			\rowcolor{airforceblue}
			\textbf{Nominativo} & \textbf{Re} & \textbf{Am} & \textbf{An} & \textbf{Pt} & \textbf{Pr} & \textbf{Ve} & \textbf{Totale ore}\\
			\hline
			\textit{Andrea Dorigo} & x & x & x & x & x & x & x\\
			\hline
			\textit{Margherita Mitillo} & x & x & x & x & x & x & x\\
			\hline
			\textit{Igli Mezini} & x & x & x & x & x & x & x\\
			\hline
			\textit{Andrea Cecchin} & x & x & x & x & x & x & x\\
			\hline
			\textit{Emma Roveroni} & x & x & x & x & x & x & x\\
			\hline
			\textit{Alfredo Graziano} & x & x & x & x & x & x & x\\
			\hline
			\textit{Mattia Cocco} & x & x & x & x & x & x & x\\
			\hline
		\end{tabularx}
	\captionof{table}{\textbf{Distribuzione delle ore totali di investimento e rendicontate}}
	\end{center}

\subsubsection{Prospetto economico}\label{5.6.1.2}
\quad
\def\tabularxcolumn#1{m{#1}}
{\rowcolors{2}{RawSienna!90!RawSienna!20}{RawSienna!70!RawSienna!40}	
	\begin{center}
		\renewcommand{\arraystretch}{1.4}
		\begin{tabularx}{7cm}{|X|c|c|}
			\hline
			\rowcolor{airforceblue}
			\textbf{Ruolo} & \textbf{Ore} & \textbf{Costo}\\
			\hline
			Responsabile & x & x\\
			\hline
			Amministratore & x & x\\
			\hline
			Analista & x & x\\
			\hline
			Progettista & x & x\\
			\hline
			Programmatore & x & x\\
			\hline
			Verificatore & x & x\\
			\hline
			Totale & x & x\\
			\hline
		\end{tabularx}
	\captionof{table}{\textbf{Prospetto dei costi totali delle ore totali di investimento e rendicontate}}
	\end{center}

\subsection{Ore rendicontate}\label{5.6.2}

\subsubsection{Suddivisione lavoro}\label{5.6.2.1}
In questa fase la distribuzione oraria è la seguente:
\quad
\def\tabularxcolumn#1{m{#1}}
{\rowcolors{2}{RawSienna!90!RawSienna!20}{RawSienna!70!RawSienna!40}
	
	\begin{center}
		\renewcommand{\arraystretch}{1.4}
		\begin{tabularx}{\textwidth}{|X|c|c|c|c|c|c|c|}
			\hline
			\rowcolor{airforceblue}
			\textbf{Nominativo} & \textbf{Re} & \textbf{Am} & \textbf{An} & \textbf{Pt} & \textbf{Pr} & \textbf{Ve} & \textbf{Totale ore}\\
			\hline
			\textit{Andrea Dorigo} & x & x & x & x & x & x & x\\
			\hline
			\textit{Margherita Mitillo} & x & x & x & x & x & x & x\\
			\hline
			\textit{Igli Mezini} & x & x & x & x & x & x & x\\
			\hline
			\textit{Andrea Cecchin} & x & x & x & x & x & x & x\\
			\hline
			\textit{Emma Roveroni} & x & x & x & x & x & x & x\\
			\hline
			\textit{Alfredo Graziano} & x & x & x & x & x & x & x\\
			\hline
			\textit{Mattia Cocco} & x & x & x & x & x & x & x\\
			\hline
		\end{tabularx}
	\captionof{table}{\textbf{Distribuzione delle ore rendicontate}}
	\end{center}

\subsubsection{Prospetto economico}\label{5.6.2.2}
\quad
\def\tabularxcolumn#1{m{#1}}
{\rowcolors{2}{RawSienna!90!RawSienna!20}{RawSienna!70!RawSienna!40}	
	\begin{center}
		\renewcommand{\arraystretch}{1.4}
		\begin{tabularx}{7cm}{|X|c|c|}
			\hline
			\rowcolor{airforceblue}
			\textbf{Ruolo} & \textbf{Ore} & \textbf{Costo}\\
			\hline
			Responsabile & x & x\\
			\hline
			Amministratore & x & x\\
			\hline
			Analista & x & x\\
			\hline
			Progettista & x & x\\
			\hline
			Programmatore & x & x\\
			\hline
			Verificatore & x & x\\
			\hline
			Totale & x & x\\
			\hline
		\end{tabularx}
	\captionof{table}{\textbf{Prospetto dei costi totali delle ore rendicontate}}
	\end{center}

\subsection{Conclusioni}\label{5.6.3}
Il costo totale del progetto considerando solamente le ore rendicontate è: xxxx\euro.
\chapter{Consuntivo}\label{Consuntivo}
Di seguito vengono indicate le spese sostenute dal gruppo confrontandole con quanto preventivato. Il bilancio potrà essere:
\begin{itemize}
	\item positivo: la spesa effettiva è minore di quanto preventivato;
	\item pari: la spesa effettiva è uguale a quanto preventivato;
	\item negativo: la spesa effettiva è maggiore di quanto preventivato.
\end{itemize}
\section{Periodo di analisi}\label{ConsuntivoPeriodoDiAnalisi}
Le ore di lavoro che sono state sostenute durante la fase di analisi sono considerate come ore di investimento per questo motivo esse non vengono rendicontate.
\quad
\def\tabularxcolumn#1{m{#1}}
{\rowcolors{2}{RawSienna!90!RawSienna!20}{RawSienna!70!RawSienna!40}
	\begin{center}
		\renewcommand{\arraystretch}{1.4}
		\begin{tabularx}{10cm}{|X|c|c|}
			\hline
			\rowcolor{airforceblue}
			\textbf{Ruolo} & \textbf{Ore} & \textbf{Costo}\\
			\hline
			Responsabile & 26(+0) & 780\euro(+0\euro)\\
			\hline
			Amministratore & 42(+0) & 840\euro(+0\euro)\\
			\hline
			Analista & 49(+15) & 1225\euro(+375\euro)\\
			\hline
			Progettista & 0(+0) & 0\euro(+0\euro)\\
			\hline
			Programmatore & 0(+0) & 0\euro(+0\euro)\\
			\hline
			Verificatore & 33(+10) & 495\euro(+150\euro)\\
			\hline
			\textbf{Totale Preventivo} & \textbf{150} & \textbf{3340\euro}\\
			\hline
			\textbf{Totale Consultivo} & \textbf{175} & \textbf{3865\euro}\\
			\hline
			\textbf{Differenza} & \textbf{25} & \textbf{525\euro}
		\end{tabularx}
		\captionof{table}{\textbf{Consuntivo della fase di Analisi}}
	\end{center}

\subsection{Conclusioni}\label{ConsuntivoPeriodoDiAnalisiConclusioni}
Come emerso dalla tabella precedente il bilancio risulta negativo in quanto il gruppo ha ritenuto necessario impiegare più tempo del previsto nei ruoli di \textit{Analista} e \textit{Verificatore}. Il motivo di tale ritardo è:
\begin{itemize}
	\item l'individuazione dei requisiti é risultata più complessa del previsto;
	\item la grande quantità di lavoro nel revisionare i documenti in quanto essendo un processo nuovo ed ogni componente ha dovuto imparare come farlo in maniera corretta, efficacie ed efficiente.
\end{itemize}

\subsection{Preventivo a finire}\label{ConsuntivoPeriodoDiAnalisiPreventivoAFinire}
Il preventivo a finire, nonostante in questa fase siano state necessarie più ore del previsto, è in linea con quanto descritto nella sezione precedente. Il gruppo non ritiene il surplus di 500\euro{} un problema in quanto le ore lavorative e i costi sostenuti in questa fase non verranno rendicontati. Per questo motivo il gruppo ha deciso di non prendere alcuna contromisura nella pianificazione futura.

\section{Periodo di consolidamento dei requisiti}\label{ConsuntivoPeriodoDiConsolidamentoDeiRequisiti}
Le ore di lavoro calcolate per questo periodo sono considerate come ore di investimento e per tale motivo non vengono rendicontate.

\quad
\def\tabularxcolumn#1{m{#1}}
{\rowcolors{2}{RawSienna!90!RawSienna!20}{RawSienna!70!RawSienna!40}
	\begin{center}
		\renewcommand{\arraystretch}{1.4}
		\begin{tabularx}{10cm}{|X|c|c|}
			\hline
			\rowcolor{airforceblue}
			\textbf{Ruolo} & \textbf{Ore} & \textbf{Costo}\\
			\hline
			Responsabile & 4(+0) & 120\euro(+0\euro)\\
			\hline
			Amministratore & 8(+0) & 160\euro(+0\euro)\\
			\hline
			Analista & 4(0) & 100\euro(+0\euro)\\
			\hline
			Progettista & 0(+0) & 0\euro(+0\euro)\\
			\hline
			Programmatore & 0(+0) & 0\euro(+0\euro)\\
			\hline
			Verificatore & 8(0) & 120\euro(+0\euro)\\
			\hline
			\textbf{Totale Preventivo} & \textbf{24} & \textbf{500\euro}\\
			\hline
			\textbf{Totale Consultivo} & \textbf{24} & \textbf{500\euro}\\
			\hline
			\textbf{Differenza} & \textbf{0} & \textbf{0\euro}
		\end{tabularx}
		\captionof{table}{\textbf{Consuntivo della fase di Consolidamento dei requisiti}}
	\end{center}

\subsection{Conclusioni}\label{ConsuntivoPeriodoDiConsolidamentoDeiRequisitiConclusioni}
Grazie al minor carico di lavoro le ore preventivate sono state rispettate quindi non è presente alcuna differenza rispetto alle ore effettivo di lavoro. Inoltre il gruppo è riuscito a procedere senza alcun problema con lo studio personale per lo svolgimento della fase successiva del lavoro.

\subsection{Preventivo a finire}\label{ConsuntivoPeriodoDiConsolidamentoDeiRequisitiPreventivoAFinire}
In quanto le ore di lavoro previste sono state rispettate il preventivo a finire risulta coerente con quello previsto.

\section{Periodo di progettazione architetturale}\label{ConsuntivoPeriodoDiProgettazioneArchitetturale}

Il gruppo ha suddiviso questo periodo in diverse parti, di conseguenza il consuntivo viene analizzato in funzione di ogni fase(??).

\subsection{Periodo di progettazione architetturale - Incremento e Verifica}\label{ConsuntivoPeriodoDiProgettazioneArchitetturaleIncrementoEVerifica}

Le ore dedicate in questo periodo sono atte al completamento della fase di Incremento e Verifica descritta nella sezione \S~\ref{PianificazioneProgettazioneArchitetturale}.

\quad
\def\tabularxcolumn#1{m{#1}}
{\rowcolors{2}{RawSienna!90!RawSienna!20}{RawSienna!70!RawSienna!40}
	\begin{center}
		\renewcommand{\arraystretch}{1.4}
		\begin{tabularx}{10cm}{|X|c|c|}
			\hline
			\rowcolor{airforceblue}
			\textbf{Ruolo} & \textbf{Ore} & \textbf{Costo}\\
			\hline
			Responsabile & 24(+0) & 720\euro(+0\euro)\\
			\hline
			Amministratore & 26(+0) & 520\euro(+0\euro)\\
			\hline
			Analista & 29(0) & 725\euro(+0\euro)\\
			\hline
			Progettista & 86(+20) & 1892\euro(+440\euro)\\
			\hline
			Programmatore & 12(+0) & 180\euro(+0\euro)\\
			\hline
			Verificatore & 54(+15) & 810\euro(+225\euro)\\
			\hline
			\textbf{Totale Preventivo} & \textbf{231} & \textbf{4847\euro}\\
			\hline
			\textbf{Totale Consultivo} & \textbf{266} & \textbf{5512\euro}\\
			\hline
			\textbf{Differenza} & \textbf{35} & \textbf{665\euro}
		\end{tabularx}
		\captionof{table}{\textbf{Consuntivo della fase di Incremento e Verifica}}
	\end{center}

\subsubsection{Conclusioni}

\subsubsection{Preventivo a finire}

\subsection{Periodo di progettazione architetturale - Technology Baseline (Primo Incremento)}\label{ConsuntivoPeriodoDiProgettazioneArchitetturaleTechnologyBaselinePrimoIncremento}

Le ore dedicate in questo periodo sono atte al completamento del primo incremento della fase di Technology Baseline$_{\scaleto{G}{3pt}}$ descritta nella sezione \S~\ref{PianificazioneProgettazioneArchitetturale}.

\quad
\def\tabularxcolumn#1{m{#1}}
{\rowcolors{2}{RawSienna!90!RawSienna!20}{RawSienna!70!RawSienna!40}
	\begin{center}
		\renewcommand{\arraystretch}{1.4}
		\begin{tabularx}{10cm}{|X|c|c|}
			\hline
			\rowcolor{airforceblue}
			\textbf{Ruolo} & \textbf{Ore} & \textbf{Costo}\\
			\hline
			Responsabile & 24(+0) & 720\euro(+0\euro)\\
			\hline
			Amministratore & 26(+0) & 520\euro(+0\euro)\\
			\hline
			Analista & 29(0) & 725\euro(+0\euro)\\
			\hline
			Progettista & 86(+20) & 1892\euro(+440\euro)\\
			\hline
			Programmatore & 12(+0) & 180\euro(+0\euro)\\
			\hline
			Verificatore & 54(+15) & 810\euro(+225\euro)\\
			\hline
			\textbf{Totale Preventivo} & \textbf{231} & \textbf{4847\euro}\\
			\hline
			\textbf{Totale Consultivo} & \textbf{266} & \textbf{5512\euro}\\
			\hline
			\textbf{Differenza} & \textbf{35} & \textbf{665\euro}
		\end{tabularx}
		\captionof{table}{\textbf{Consuntivo della fase di Technology Baseline (Primo Incremento)}}
	\end{center}

\subsubsection{Conclusioni}

\subsubsection{Preventivo a finire}

\subsection{Periodo di progettazione architetturale - Technology Baseline (Secondo Incremento)}\label{ConsuntivoPeriodoDiProgettazioneArchitetturaleTechnologyBaselineSecondoIncremento}

Le ore dedicate in questo periodo sono atte al completamento del secondo incremento della fase di Technology Baseline$_{\scaleto{G}{3pt}}$ descritta nella sezione \S~\ref{PianificazioneProgettazioneArchitetturale}.

\quad
\def\tabularxcolumn#1{m{#1}}
{\rowcolors{2}{RawSienna!90!RawSienna!20}{RawSienna!70!RawSienna!40}
	\begin{center}
		\renewcommand{\arraystretch}{1.4}
		\begin{tabularx}{10cm}{|X|c|c|}
			\hline
			\rowcolor{airforceblue}
			\textbf{Ruolo} & \textbf{Ore} & \textbf{Costo}\\
			\hline
			Responsabile & 24(+0) & 720\euro(+0\euro)\\
			\hline
			Amministratore & 26(+0) & 520\euro(+0\euro)\\
			\hline
			Analista & 29(0) & 725\euro(+0\euro)\\
			\hline
			Progettista & 86(+20) & 1892\euro(+440\euro)\\
			\hline
			Programmatore & 12(+0) & 180\euro(+0\euro)\\
			\hline
			Verificatore & 54(+15) & 810\euro(+225\euro)\\
			\hline
			\textbf{Totale Preventivo} & \textbf{231} & \textbf{4847\euro}\\
			\hline
			\textbf{Totale Consultivo} & \textbf{266} & \textbf{5512\euro}\\
			\hline
			\textbf{Differenza} & \textbf{35} & \textbf{665\euro}
		\end{tabularx}
		\captionof{table}{\textbf{Consuntivo della fase di Technology Baseline (Secondo Incremento)}}
	\end{center}

\subsubsection{Conclusioni}

\subsubsection{Preventivo a finire}

\subsection{Periodo di progettazione architetturale - Technology Baseline (Terzo Incremento)}\label{ConsuntivoPeriodoDiProgettazioneArchitetturaleTechnologyBaselineTerzoIncremento}

Le ore dedicate in questo periodo sono atte al completamento del terzo incremento della fase di Technology Baseline$_{\scaleto{G}{3pt}}$ descritta nella sezione \S~\ref{PianificazioneProgettazioneArchitetturale}.

\quad
\def\tabularxcolumn#1{m{#1}}
{\rowcolors{2}{RawSienna!90!RawSienna!20}{RawSienna!70!RawSienna!40}
	\begin{center}
		\renewcommand{\arraystretch}{1.4}
		\begin{tabularx}{10cm}{|X|c|c|}
			\hline
			\rowcolor{airforceblue}
			\textbf{Ruolo} & \textbf{Ore} & \textbf{Costo}\\
			\hline
			Responsabile & 24(+0) & 720\euro(+0\euro)\\
			\hline
			Amministratore & 26(+0) & 520\euro(+0\euro)\\
			\hline
			Analista & 29(0) & 725\euro(+0\euro)\\
			\hline
			Progettista & 86(+20) & 1892\euro(+440\euro)\\
			\hline
			Programmatore & 12(+0) & 180\euro(+0\euro)\\
			\hline
			Verificatore & 54(+15) & 810\euro(+225\euro)\\
			\hline
			\textbf{Totale Preventivo} & \textbf{231} & \textbf{4847\euro}\\
			\hline
			\textbf{Totale Consultivo} & \textbf{266} & \textbf{5512\euro}\\
			\hline
			\textbf{Differenza} & \textbf{35} & \textbf{665\euro}
		\end{tabularx}
		\captionof{table}{\textbf{Consuntivo della fase di Technology Baseline (Terzo Incremento)}}
	\end{center}

\subsubsection{Conclusioni}

\subsubsection{Preventivo a finire}

\subsection{Consutivo complessivo delle fasi}\label{ConsuntivoPeriodoDiProgettazioneArchitetturaleConsuntivoComplessivoDelleFasi}

Nella tabella successiva viene descritto il calcolo delle ore totali di tutte le fasi precedentemente descritte.

\quad
\def\tabularxcolumn#1{m{#1}}
{\rowcolors{2}{RawSienna!90!RawSienna!20}{RawSienna!70!RawSienna!40}
	\begin{center}
		\renewcommand{\arraystretch}{1.4}
		\begin{tabularx}{10cm}{|X|c|c|}
			\hline
			\rowcolor{airforceblue}
			\textbf{Ruolo} & \textbf{Ore} & \textbf{Costo}\\
			\hline
			Responsabile & 24(+0) & 720\euro(+0\euro)\\
			\hline
			Amministratore & 26(+0) & 520\euro(+0\euro)\\
			\hline
			Analista & 29(0) & 725\euro(+0\euro)\\
			\hline
			Progettista & 86(+20) & 1892\euro(+440\euro)\\
			\hline
			Programmatore & 12(+0) & 180\euro(+0\euro)\\
			\hline
			Verificatore & 54(+15) & 810\euro(+225\euro)\\
			\hline
			\textbf{Totale Preventivo} & \textbf{231} & \textbf{4847\euro}\\
			\hline
			\textbf{Totale Consultivo} & \textbf{266} & \textbf{5512\euro}\\
			\hline
			\textbf{Differenza} & \textbf{35} & \textbf{665\euro}
		\end{tabularx}
		\captionof{table}{\textbf{Consuntivo della fase di Technology Baseline (Terzo Incremento)}}
	\end{center}

\subsection{Conclusioni}\label{ConsuntivoPeriodoDiProgettazioneArchitetturaleConclusioni}
Il bilancio, come emerge dalla tabella precedente, risulta negativo poiché il gruppo ha ritenuto necessario impiegare più ore nei ruoli di \textit{Progettista} e \textit{Verificatore}. I motivi di tale ritardo sono:
\begin{itemize}
	\item il tempo impiegato per la correzione e l'aggiornamento dei documenti si è rivelato essere più di quello preventivato;
	\item essendo un progetto complesso con tecnologie nuove ad ogni componente del gruppo la parte di progettazione si è rivelata molto più complessa del previsto.
\end{itemize}

\subsection{Preventivo a finire}\label{ConsuntivoPeriodoDiProgettazioneArchitetturalePreventivoAFinire}
Il preventivo a finire risulta quindi con un surplus di 665\euro. AGGIUNGI PARTE DOVE PENSIAMO AD UNA SOLUZIONE
\chapter{Organigramma}\label{Organigramma}

\section{Redazione}\label{7.1}
\quad
\def\tabularxcolumn#1{m{#1}}
{\rowcolors{2}{RawSienna!90!RawSienna!20}{RawSienna!70!RawSienna!40}	
	\begin{center}
		\renewcommand{\arraystretch}{1.4}
		\begin{tabularx}{\textwidth}{|X|c|c|}
			\hline
			\rowcolor{airforceblue}
			\textbf{Nominativo} & \textbf{Data di Redazione} & \textbf{Firma}\\
			\hline
			Andrea Dorigo & gg-mm-aaaa & firma\\
			\hline
		\end{tabularx}
		\captionof{table}{\textbf{Consuntivo della fase di Analisi}}
	\end{center}

\section{Approvazione}\label{7.2}
\quad
\def\tabularxcolumn#1{m{#1}}
{\rowcolors{2}{RawSienna!90!RawSienna!20}{RawSienna!70!RawSienna!40}	
	\begin{center}
		\renewcommand{\arraystretch}{1.4}
		\begin{tabularx}{\textwidth}{|X|c|c|}
			\hline
			\rowcolor{airforceblue}
			\textbf{Nominativo} & \textbf{Data di Approvazione} & \textbf{Firma}\\
			\hline
			Andrea Dorigo & gg-mm-aaaa & firma\\
			\hline
			Tullio Vardanega & &\\
			Riccardo Cardin & &\\
			\hline
		\end{tabularx}
		\captionof{table}{\textbf{Consuntivo della fase di Analisi}}
	\end{center}

\section{Accettazione dei componenti}\label{7.3}
\quad
\def\tabularxcolumn#1{m{#1}}
{\rowcolors{2}{RawSienna!90!RawSienna!20}{RawSienna!70!RawSienna!40}	
	\begin{center}
		\renewcommand{\arraystretch}{1.4}
		\begin{tabularx}{\textwidth}{|X|c|c|}
			\hline
			\rowcolor{airforceblue}
			\textbf{Nominativo} & \textbf{Data di Accettazione} & \textbf{Firma}\\
			\hline
			Andrea Dorigo & 10-01-2021 & firma\\
			\hline
			Margherita Mitillo & 10-01-2021 & \includegraphics[width=0.2\linewidth]{../immagini/firma_margherita.png}\\
			\hline
			Igli Mezini & 10-01-2021 &\\
			\hline
			Emma Roveroni & 10-01-2021 &\\
			\hline
			Mattia Cocco & 10-01-2021 &\\
			\hline
			Alfredo Graziano & 10-01-2021 &\\
			\hline
			Andrea Cecchin & 10-01-2021 &\\
			\hline
		\end{tabularx}
		\captionof{table}{\textbf{Consuntivo della fase di Analisi}}
	\end{center}

\section{Componenti}\label{7.4}
\quad
\def\tabularxcolumn#1{m{#1}}
{\rowcolors{2}{RawSienna!90!RawSienna!20}{RawSienna!70!RawSienna!40}	
	\begin{center}
		\renewcommand{\arraystretch}{1.4}
		\begin{tabularx}{\textwidth}{|X|c|c|}
			\hline
			\rowcolor{airforceblue}
			\textbf{Nominativo} & \textbf{Matricola} & \textbf{Indirizzo di posta elettronica}\\
			\hline
			Andrea Dorigo & 1170610 & andrea.dorigo.3@studenti.unipd.it\\
			\hline
			Margherita Mitillo & 1098971 & margherita.mitillo@studenti.unipd\\
			\hline
			Igli Mezini & 1149009 & igli.mezini@studenti.unipd.it\\
			\hline
			Emma Roveroni & 1187275 & emma.roveroni@studenti.unipd.it\\
			\hline
			Mattia Cocco & 1096738 & mattia.cocco@studenti.unipd.it\\
			\hline
			Alfredo Graziano & 1144530 & alfredo.graziano@studenti.unipd.it\\
			\hline
			Andrea Cecchin & 1171050  & andrea.cecchin.3@studenti.unipd.it\\
			\hline
		\end{tabularx}
		\captionof{table}{\textbf{Consuntivo della fase di Analisi}}
	\end{center}

%PER RENDERE PIÙ CHIARA LA STESURA DEI DOCUMENTI È MEGLIO LASCIARE SEPARATI IN FILE DIVERSI OGNI CAPITOLO

% \input{esempio} -- esempio di codice per inserire un nuovo capitolo

\end{document}

% \documentclass[a4paper,12pt]{report}
% \usepackage[utf8]{inputenc}
% \usepackage{graphicx}
% \usepackage{float}
% \usepackage{tabularx}
% \usepackage{makecell}
% \usepackage{titlesec}
% \usepackage{fancyhdr}
% \usepackage{lastpage}
% \usepackage{xurl}
% \usepackage{hyperref}
% \usepackage{geometry}
% \usepackage{color}
% \usepackage{microtype}
% \usepackage{enumerate}
% \usepackage{listings}
% \usepackage{tabularx}
% \usepackage[table,dvipsnames]{xcolor}
% \usepackage{caption}
% \captionsetup{labelformat=empty}
% \setcounter{tocdepth}{5}
% \setcounter{secnumdepth}{5} %SONO I DUE COMANDI PER AVERE LE SUBSUBSECTION NUMERATE E PRESENTI NELL'INDICE
% \renewcommand{\contentsname}{Indice} %QUESTO SERVE PER AVERE L'INDICE CON IL NOME CHE VOGLIO IO
% \renewcommand{\listfigurename}{Lista delle figure}
% \renewcommand{\listtablename}{Lista delle tabelle}
% \titleformat{\chapter}[display]
% {\normalfont\bfseries}{}{0pt}{\LARGE} %QUESTO SERVE PER AVERE SOLO IL NOME DEL CAPITOLO CHE VOGLIO IO
% \titlespacing*{\chapter}{0cm}{0cm}{0.2cm}
% \geometry{
% 	left=20mm,
% 	right=20mm,
% }
% \fancypagestyle{plain}{
% 	\fancyhf{}
% 	\lhead{\includegraphics[width=3cm]{../immagini/minilogo.jpg}}
% 	\chead{}
% 	\rhead{\fontsize{12}{10}Piano di progetto}
% 	\lfoot{}
% 	\cfoot{\thepage\ di \pageref*{LastPage}}
% 	\rfoot{}
% }
%
% \definecolor{atomlightorange}{rgb}{0.88,0.76,0.55}
% \definecolor{atomdarkgrey}{RGB}{59,62,75}
%
% \usepackage{tikz}
%
% % set listings
% \lstset{%
%     basicstyle=\footnotesize\ttfamily\color{atomlightorange},
%     framesep=20pt,
% 		belowskip=10pt,
% 		aboveskip=10pt
% }
%
% % add frame environment
% \usepackage[%
%     framemethod=tikz,
%     skipbelow=8pt,
%     skipabove=13pt
% ]{mdframed}
% \mdfsetup{%
%     leftmargin=0pt,
%     rightmargin=0pt,
%     backgroundcolor=atomdarkgrey,
%     middlelinecolor=atomdarkgrey,
%     roundcorner=6
% }
%
% \usepackage{etoolbox}% >= v2.1 2011-01-03
% \BeforeBeginEnvironment{lstlisting}{\begin{mdframed}\vspace{-0.7em}}
% \AfterEndEnvironment{lstlisting}{\vspace{-0.5em}\end{mdframed}}
%
% % needed for \lstcapt
% \def\ifempty#1{\def\temparg{#1}\ifx\temparg\empty}
%
% % make new caption command for listings
% \usepackage{caption}
% \newcommand{\lstcapt}[2][]{%
%     \ifempty{#1}%
%         \captionof{lstlisting}{#2}%
%     \else%
%         \captionof{lstlisting}[#1]{#2}%
%     \fi%
%     \vspace{0.75\baselineskip}%
% }
%
% \hypersetup{
%     colorlinks=true,
%     linkcolor=black,
%     filecolor=black,
%     urlcolor=blue,
% 		citecolor=black,
% }
%
% \pagestyle{plain}
%
% \makeatletter
% 	 \def\thebibliography#1{\chapter*{Bibliografia\@mkboth
% 		 {Bibliografia}{Bibliografia}}\list
% 		 {[\arabic{enumi}]}{\settowidth\labelwidth{[#1]}\leftmargin\labelwidth
%  \advance\leftmargin\labelsep
%  \usecounter{enumi}}
%  \def\newblock{\hskip .11em plus .33em minus .07em}
%  \sloppy\clubpenalty4000\widowpenalty4000
%  \sfcode`\.=1000\relax}
% 	 \makeatother
%
% \newcolumntype{Y}{>{\centering\arraybackslash}X}
%
% \begin{document}
%
% \makeatletter
% \begin{titlepage}
%     \begin{center}
%     \vspace*{-5,0cm}
%     \author{Jawa Druids}
%     \title{Piano di progetto}
%     \date{} %LASCIARE QUESTO CAMPO VUOTO, SE LO TOLGO STAMPA LA DATA CORRENTE
%     \includegraphics[width=0.7\linewidth]{../immagini/DRUIDSLOGO.jpg}\\[4ex]
%     {\huge \bfseries  \@title }\\[2ex]
%     {\LARGE  \@author}\\[50ex]
%     \vspace*{-8,0cm}
%     \begin{table}[H]
%         \centering
%         \begin{tabular}{c|c}
%             \textbf{Versione} & x.x.x \\
%             \textbf{Data approvazione} & xx-xx-xxxx\\
%             \textbf{Responsabile} & Nome Cognome\\
%             \textbf{Redattori} & Andrea Dorigo \\
%             \textbf{Verificatori} & Nome Cognome \\
%         %MAKECELL SERVE PER POI ANDARE A CAPO ALL'INTERNO DELLA CELLA
%             \textbf{Stato} & Redazione in corso\\
%             \textbf{Lista distribuzione} & \makecell{Jawa Druids \\ prof. Tullio Vardanega \\ prof. Riccardo Cardin \\ Sync Lab}\\
%             \textbf{Uso} & Esterno
%         \end{tabular}
%     \end{table}
% 					\vspace{.4cm}
% 	\hfill \break
%     \fontsize{17}{10}\textbf{Sommario}\\
% 		\vspace{.3cm}
%     Il presente documento contiene la pianificazione delle attività del gruppo Jawa Druids atte al soddisfacimento del capitolato \normalsize\textit{GDP: Gathering Detection Platform} di Sync Lab.
%     \end{center}
% \end{titlepage}
% \makeatother
%
% \quad
\begin{center}
	\LARGE\textbf{Registro delle modifiche}
\end{center}

\def\tabularxcolumn#1{m{#1}}
{\rowcolors{2}{RawSienna!90!RawSienna!20}{RawSienna!70!RawSienna!40}


\begin{center}
	\renewcommand{\arraystretch}{1.4}
	\begin{longtable}[c]{|p{2cm-1\tabcolsep}|p{2cm}|p{3cm-2\tabcolsep}|p{2,5cm-2\tabcolsep}|p{4cm-2\tabcolsep}|p{2,5cm}|}
		\hline
		\rowcolor{airforceblue}
		\makecell[c]{\textbf{Versione}} & \makecell[c]{\textbf{Data}} & \makecell[c]{\textbf{Autore}} & \makecell[c]{\textbf{Ruolo}} & \makecell[c]{\textbf{Modifica}} & \makecell[c]{\textbf{Verificatore}} \\
		\hline
		\centering v1.0.0 & 2021-04-18 & Andrea Dorigo & \centering \textit{Responsabile} & \textit{Approvazione del documento} & \makecell[c]{-} \\
		\hline
		\centering v0.1.0 & 2021-04-17 &  \centering - & \centering - &  \textit{Revisione complessiva del documento} & Mattia Cocco \\
		\hline
		\centering v0.0.1 & 2021-04-15 & Andrea Cecchin & \centering \textit{Redattore} &\textit{Stesura del documento}  & Mattia Cocco \\
		\hline
	\end{longtable}
\end{center}

% \tableofcontents
% \listoffigures
% \listoftables
% \chapter{Introduzione}

Lo scopo di questo documento è la stesura di un elenco di linee guida ed esempi che i componenti sono incoraggiati a seguire per migliorare l'efficacia della collaborazione.
A differenza delle Norme di Progetto, questo documento è redatto in un linguaggio più informale per facilitarne la comprensione, con spezzati di codice a scopo esemplificativo e riferimenti esterni sulle best practices da seguire.\\
Il documento può essere soggetto a modifiche ed aggiunte per tutta la durata del progetto.

% \chapter{Analisi dei rischi}\label{AnalisiDeiRischi}

\section{Piano per la gestione dei rischi}\label{AnalisiDeiRischiPianoPerLaGestioneDeiRischi}
Con l'intento di prevenire il naturale insorgere di problemi durante lo svolgimento del progetto è stato elaborato un approfondito piano per la gestione dei rischi. Quest'ultimo è suddiviso in quattro attività$_{\scaleto{G}{3pt}}$:
\begin{itemize}
  \item \textbf{Individuazione dei rischi:} attività$_{\scaleto{G}{3pt}}$ di identificazione e documentazione di possibili elementi problematici che possano ostacolare il naturale percorso del progetto;
  \item \textbf{Analisi dei rischi:} attività$_{\scaleto{G}{3pt}}$ di analisi dei fattori di rischio, che si articola in probabilità di occorrenza, indice di gravità e conseguente impatto sul progetto;
  \item \textbf{Pianificazione di controllo:} attività$_{\scaleto{G}{3pt}}$ di pianificazione delle misure da adottare per la prevenzione e contenimento del problema;
  \item \textbf{Monitoraggio dei rischi:} attività$_{\scaleto{G}{3pt}}$ di controllo dei rischi che accompagna tutto lo svolgimento del progetto, al fine di evitarli o agire tempestivamente alla loro occorrenza per contenerne i danni.
\end{itemize}
Le principali tipologie di rischio sono state quindi codificate e categorizzate come segue:
\begin{itemize}
  \item \textbf{RT:} Rischi legati alle tecnologie;
  \item \textbf{RO:} Rischi legati all'organizzazione;
  \item \textbf{RI:} Rischi interpersonali, ovvero legati alle relazioni personali interne ed esterne o alla disponibilità e alle risorse dei componenti.
\end{itemize}

\clearpage
\quad
\begin{center}
	\LARGE\textbf{Rischi legati alle tecnologie}
\end{center}

\def\tabularxcolumn#1{m{#1}}
{\rowcolors{2}{RawSienna!90!RawSienna!20}{RawSienna!70!RawSienna!40}

	\begin{center}
		\renewcommand{\arraystretch}{1.4}
		\begin{longtable}{|p{5cm}|p{12cm}|}
			\hline
			\rowcolor{airforceblue}
			\multicolumn{2}{|c|}{\textit{Inesperienza tecnologica}}\\
			\hline
			\textit{Codice} & RT1 \\
			\hline
			\textit{Descrizione} & Alcune tecnologie utilizzate in questo progetto sono nuove per tutti i membri del gruppo di lavoro. \\
			\hline
			\textit{Conseguenza} & Lo studio e l'apprendimento di tali tecnologie potrebbero richiedere un intervallo di tempo difficile da quantificare, maggiore del previsto e variabile da membro a membro con conseguenti difficoltà operative. \\
			\hline
			\textit{Possibilità di occorrenza} & Alta. \\
			\hline
			\textit{Pericolosità} & Alta. \\
			\hline
			\textit{Precauzioni} & Il \textit{Responsabile di Progetto} dovrà suddividere i compiti$_G$ nel modo più congruo possibile, considerando le conoscenze preliminari di ciascun componente; prevederà inoltre un tempo di Slack$_G$ maggiore per i compiti$_{\scaleto{G}{3pt}}$ assegnati ad un componente senza particolare familiarità con la relativa tecnologia. Il \textit{Responsabile di Progetto} assegnerà i task$_G$ di maggiore complessità a più membri ove necessario.  \\
			\hline
			\textit{Piano di contingenza} & Ciascun membro comunicherà il prima possibile al \textit{Responsabile di progetto} la previsione di un eventuale ritardo o mancanza; egli provvederà a ridistribuire i compiti$_{\scaleto{G}{3pt}}$ se necessario in modo da sanare eventuali lacune o sottostime. \\
			\hline
		\end{longtable}
	\captionof{table}{\textbf{Analisi dei rischi delle tecnologie utilizzate}}
	\end{center}


\def\tabularxcolumn#1{m{#1}}
{\rowcolors{2}{RawSienna!90!RawSienna!20}{RawSienna!70!RawSienna!40}

	\begin{center}
		\renewcommand{\arraystretch}{1.4}
		\begin{longtable}{|p{5cm}|p{12cm}|}
			\hline
			\rowcolor{airforceblue}
			\multicolumn{2}{|c|}{\textit{Software terze parti}}\\
			\hline
			\textit{Codice} & RT2 \\
			\hline
			\textit{Descrizione} & Eventuali problematiche con software di terze parti, quali la mancanza di documentazione o problemi tecnici, sono indipendenti dai membri del gruppo. \\
			\hline
			\textit{Conseguenza} & Ciò causerebbe ritardi pesanti sul proseguo del lavoro e anche possibili ritardi sulla consegna.
			La necessità di cambiare tecnologia potrebbe richiedere molto tempo e risorse per la ricerca di una sostituzione. \\
			\hline
			\textit{Possibilità di occorrenza} & Bassa. \\
			\hline
			\textit{Pericolosità} & Alta. \\
			\hline
			\textit{Precauzioni} & Il gruppo sceglierà i software più stabili e documentati per evitare questi tipi di problemi.  \\
			\hline
			\textit{Piano di contingenza} & Assieme al \textit{Responsabile di progetto} il gruppo di lavoro si attiverà al fine di tentare di risolvere il problema. Se ciò non è possibile sarà necessario un cambio di tecnologia, anche tramite l'aiuto del proponente$_{\scaleto{G}{3pt}}$.  \\
			\hline
		\end{longtable}
		\captionof{table}{\textbf{Analisi dei rischi dei software di terze parti}}
	\end{center}


\def\tabularxcolumn#1{m{#1}}
{\rowcolors{2}{RawSienna!90!RawSienna!20}{RawSienna!70!RawSienna!40}

	\begin{center}
		\renewcommand{\arraystretch}{1.4}
		\begin{longtable}{|p{5cm}|p{12cm}|}
			\hline
			\rowcolor{airforceblue}
			\multicolumn{2}{|c|}{\textit{Validità dei dati}}\\
			\hline
			\textit{Codice} & RT3 \\
			\hline
			\textit{Descrizione} & Problemi legati alla validità e all'elaborazione dei dati. \\
			\hline
			\textit{Conseguenza} & Arresto obbligato del lavoro in corso, con possibilità di invalidazione del lavoro svolto fino a quel momento. \\
			\hline
			\textit{Possibilità di occorrenza} & Medio/Alta. \\
			\hline
			\textit{Pericolosità} & Molto alta. \\
			\hline
			\textit{Precauzioni} & Prima dell'inizio della raccolta dati il gruppo si assicurerà che la fonte sia affidabile e coerente.
			Questa operazione sarà svolta per prima in quanto critica per l'intero sviluppo.  \\
			\hline
			\textit{Piano di contingenza} & Il gruppo, insieme al proponente$_{\scaleto{G}{3pt}}$, valuterà se sarà necessario cambiare solo la fonte di provenienza dei dati oppure simularli in maniera consona. \\
			\hline
		\end{longtable}
		\captionof{table}{\textbf{Analisi dei rischi della validità dei dati}}
	\end{center}

\def\tabularxcolumn#1{m{#1}}
{\rowcolors{2}{RawSienna!90!RawSienna!20}{RawSienna!70!RawSienna!40}
	
	\begin{center}
		\renewcommand{\arraystretch}{1.4}
		\begin{longtable}{|p{5cm}|p{12cm}|}
			\hline
			\rowcolor{airforceblue}
			\multicolumn{2}{|c|}{\textit{Malfunzionamenti hardware o software dei pc dei membri del gruppo}}\\
			\hline
			\textit{Codice} & RT4 \\
			\hline
			\textit{Descrizione} & Può accadere che qualcuno abbia il PC non funzionante o in assistenza e non possa contribuire attivamente alla realizzazione del prodotto. \\
			\hline
			\textit{Conseguenza} & Ciò causerebbe una mole di lavoro maggiore per gli altri componenti del gruppo ed eventuali ritardi.\\
			\hline
			\textit{Possibilità di occorrenza} & Bassa. \\
			\hline
			\textit{Pericolosità} & Media. \\
			\hline
			\textit{Precauzioni} & Ogni membro del team deve monitorare il corretto funzionamento del proprio PC.  \\
			\hline
			\textit{Piano di contingenza} & A seconda della gravità del problema si provvederà alla reinstallazione del software, del sistema operativo o della sostituzione della propria macchina. \\
			\hline
		\end{longtable}
		\captionof{table}{\textbf{Analisi dei rischi del malfunzionamento del PC}}
	\end{center}
\clearpage
\quad
\begin{center}
	\LARGE\textbf{Rischi legati all'organizzazione}
\end{center}

\def\tabularxcolumn#1{m{#1}}
{\rowcolors{2}{RawSienna!90!RawSienna!20}{RawSienna!70!RawSienna!40}

	\begin{center}
		\renewcommand{\arraystretch}{1.4}
		\begin{longtable}{|p{5cm}|p{12cm}|}
			\hline
			\rowcolor{airforceblue}
			\multicolumn{2}{|c|}{\textit{Contrasti tra i componenti}}\\
			\hline
			\textit{Codice} & RO1 \\
			\hline
			\textit{Descrizione} & Può presentarsi la possibilità che durante le varie attività possano emergere contrasti e tensioni tra i componenti. \\
			\hline
			\textit{Conseguenza} & Tensioni o contrasti tra i componenti sfavoriscono il corretto proseguimento del progetto. \\
			\hline
			\textit{Possibilità di occorrenza} & Bassa. \\
			\hline
			\textit{Pericolosità} & Alta. \\
			\hline
			\textit{Precauzioni} & Ogni membro del gruppo di lavoro cercherà di essere più comprensibile e limitare così eventuali tensioni a favore del collettivo. \\
			\hline
			\textit{Piano di contingenza} & Il \textit{Responsabile di Progetto} avrà l'incarico di mediatore in tali controversie.
			Eventualmente, insieme al resto del gruppo si cercherà di sanare le discordie e solamente in casi estremi verrà chiamato in causa il Prof. Tullio Vardanega.  \\
			\hline
		\end{longtable}
	\captionof{table}{\textbf{Analisi dei rischi per i contrasti tra i componenti}}
	\end{center}


\def\tabularxcolumn#1{m{#1}}
{\rowcolors{2}{RawSienna!90!RawSienna!20}{RawSienna!70!RawSienna!40}

	\begin{center}
		\renewcommand{\arraystretch}{1.4}
		\begin{longtable}{|p{5cm}|p{12cm}|}
			\hline
			\rowcolor{airforceblue}
			\multicolumn{2}{|c|}{\textit{Impegni personali}}\\
			\hline
			\textit{Codice} & RO2 \\
			\hline
			\textit{Descrizione} & Può presentarsi la possibilità che in alcuni momenti uno o più componenti del gruppo abbiano degli impegni accademici o personali. \\
			\hline
			\textit{Conseguenza} & Rallentamento del lavoro.\\
			\hline
			\textit{Possibilità di occorrenza} & Media. \\
			\hline
			\textit{Pericolosità} & Media. \\
			\hline
			\textit{Precauzioni} & Essenziale sarà il comunicare gli impegni al \textit{Responsabile di Progetto} appena il componente ne viene a conoscenza. \\
			\hline
			\textit{Piano di contingenza} & Il \textit{Responsabile di Progetto} provvederà ad approvare delle modifiche organizzative per evitare o limitare rallentamenti ai lavori. \\
			\hline
		\end{longtable}
		\captionof{table}{\textbf{Analisi dei rischi sugli impegni personali}}
	\end{center}

\def\tabularxcolumn#1{m{#1}}
{\rowcolors{2}{RawSienna!90!RawSienna!20}{RawSienna!70!RawSienna!40}
	
	\begin{center}
		\renewcommand{\arraystretch}{1.4}
		\begin{longtable}{|p{5cm}|p{12cm}|}
			\hline
			\rowcolor{airforceblue}
			\multicolumn{2}{|c|}{\textit{Calcolo dei tempi e dei costi}}\\
			\hline
			\textit{Codice} & RO3 \\
			\hline
			\textit{Descrizione} & E' possibile che i tempi e i costi preventivati si rivelino imprecisi con l'avanzamento del progetto. \\
			\hline
			\textit{Conseguenza} & Costi preventivati sbagliati. \\
			\hline
			\textit{Possibilità di occorrenza} & Alta. \\
			\hline
			\textit{Pericolosità} & Alta. \\
			\hline
			\textit{Precauzioni} & Nel caso in cui un componente riscontri una differenza dalle ore di lavoro preventivate, dovrà farlo presente al \textit{Responsabile di Progetto}.  \\
			\hline
			\textit{Piano di contingenza} & Nel caso in cui una stima oraria risulti non sufficiente per portare a termine la consegna, il \textit{Responsabile di Progetto} provvederà ad assegnare più risorse in modo da limitare rallentamenti. Eventualmente se ci dovessero essere lo stesso variazioni al preventivo, allora il \textit{Responsabile di Progetto} provvederà a comunicarlo al Committente$_{\scaleto{G}{3pt}}$ \\
			\hline
		\end{longtable}
		\captionof{table}{\textbf{Analisi dei rischi del calcolo dei tempi e dei costi}}
	\end{center}

\def\tabularxcolumn#1{m{#1}}
{\rowcolors{2}{RawSienna!90!RawSienna!20}{RawSienna!70!RawSienna!40}
	
	\begin{center}
		\renewcommand{\arraystretch}{1.4}
		\begin{longtable}{|p{5cm}|p{12cm}|}
			\hline
			\rowcolor{airforceblue}
			\multicolumn{2}{|c|}{\textit{Inesperienza nel coordinamento}}\\
			\hline
			\textit{Codice} & RO4 \\
			\hline
			\textit{Descrizione} & I membri non hanno esperienza di lavoro che richieda il coordinamento di sette persone.\\
			\hline
			\textit{Conseguenza} & Problematiche o ritardi a causa di una mancata o scarsa organizzazione del team con tempi di latenza e compiti svolti più volte da membri differenti. \\
			\hline
			\textit{Possibilità di occorrenza} & Alta. \\
			\hline
			\textit{Pericolosità} & Alta. \\
			\hline
			\textit{Precauzioni} & Il \textit{Responsabile di Progetto} deve, insieme al resto del gruppo, pianificare le mansioni. Si cercherà di avere una rotazione dei ruoli in modo da far collaborare tutti in modo che ciascuna attività venga svolta dai componenti con più esperienza, insieme a quelli che ancora non ne hanno.  \\
			\hline
			\textit{Piano di contingenza} & Qualunque difficoltà sarà notificata al \textit{Responsabile di Progetto}, che dopo essersi consultato con il gruppo, provvederà eventualmente ad assegnare un compito più semplice all'interessato. \\
			\hline
		\end{longtable}
		\captionof{table}{\textbf{Analisi dei rischi per inesperienza nel coordinamento}}
	\end{center}

\def\tabularxcolumn#1{m{#1}}
{\rowcolors{2}{RawSienna!90!RawSienna!20}{RawSienna!70!RawSienna!40}
	
	\begin{center}
		\renewcommand{\arraystretch}{1.4}
		\begin{longtable}{|p{5cm}|p{12cm}|}
			\hline
			\rowcolor{airforceblue}
			\multicolumn{2}{|c|}{\textit{Scarsa comunicazione}}\\
			\hline
			\textit{Codice} & RO5 \\
			\hline
			\textit{Descrizione} & Per avanzare nelle attività pianificate e rispettare le scadenze, è necessaria una comunicazione costante tra tutti i membri del gruppo.\\
			\hline
			\textit{Conseguenza} & Problematiche o ritardi a causa di una scarsa comunicazione del team. \\
			\hline
			\textit{Possibilità di occorrenza} & Medio. \\
			\hline
			\textit{Pericolosità} & Alta. \\
			\hline
			\textit{Precauzioni} & Il \textit{Responsabile di Progetto} provvederà a promuovere un adeguato livello di comunicazione tra i vari componenti del gruppo.  \\
			\hline
			\textit{Piano di contingenza} & Nel caso si rilevi una scarsa comunicazione sarà compito del \textit{Responsabile di Progetto} provvedere a risolverlo, aiutandosi attraverso una riunione interna al gruppo per discuterne la situazione. \\
			\hline
		\end{longtable}
		\captionof{table}{\textbf{Analisi dei rischi per scarsa comunicazione}}
	\end{center}

\clearpage
\def\tabularxcolumn#1{m{#1}}
{\rowcolors{2}{RawSienna!90!RawSienna!20}{RawSienna!70!RawSienna!40}
	
	\begin{center}
		\renewcommand{\arraystretch}{1.4}
		\begin{longtable}{|p{5cm}|p{12cm}|}
			\hline
			\rowcolor{airforceblue}
			\multicolumn{2}{|c|}{\textit{Approvazione errata dei documenti}}\\
			\hline
			\textit{Codice} & RO6 \\
			\hline
			\textit{Descrizione} & E' possibile che il \textit{Responsabile di Progetto} durante l'approvazione non si accorga o commetta alcuni errori.\\
			\hline
			\textit{Conseguenza} & Approvazione e possibile consegna di documenti errati. \\
			\hline
			\textit{Possibilità di occorrenza} & Bassa. \\
			\hline
			\textit{Pericolosità} & Alta. \\
			\hline
			\textit{Precauzioni} & Per ogni documento devono essere eseguiti controlli costanti, in modo che sia possibile identificare in maniera tempestiva gli eventuali errori.  \\
			\hline
			\textit{Piano di contingenza} & Il \textit{Responsabile di Progetto} si dovrà occupare di controllare che i documenti da approvare siano effettivamente validi. \\
			\hline
		\end{longtable}
		\captionof{table}{\textbf{Analisi dei rischi per approvazione errata dei documenti}}
	\end{center}

\def\tabularxcolumn#1{m{#1}}
{\rowcolors{2}{RawSienna!90!RawSienna!20}{RawSienna!70!RawSienna!40}
	
	\begin{center}
		\renewcommand{\arraystretch}{1.4}
		\begin{longtable}{|p{5cm}|p{12cm}|}
			\hline
			\rowcolor{airforceblue}
			\multicolumn{2}{|c|}{\textit{Analisi dei requisiti imperfetta}}\\
			\hline
			\textit{Codice} & RO7 \\
			\hline
			\textit{Descrizione} & E' possibile che a causa dell'inesperienza del gruppo venga prodotta un'Analisi dei Requisiti insoddisfacente.\\
			\hline
			\textit{Conseguenza} & La proposta potrebbe risultare inadeguata alle aspettative. \\
			\hline
			\textit{Possibilità di occorrenza} & Media. \\
			\hline
			\textit{Pericolosità} & Alta. \\
			\hline
			\textit{Precauzioni} & Ogni dubbio verrà discusso con il proponente$_{\scaleto{G}{3pt}}$.  \\
			\hline
			\textit{Piano di contingenza} & Qualsiasi errore verrà corretto con la massima priorità. \\
			\hline
		\end{longtable}
		\captionof{table}{\textbf{Analisi dei rischi per l'analisi dei requisiti imperfetta}}
	\end{center}
\clearpage
\quad
\begin{center}
	\LARGE\textbf{Rischi interpersonali}
\end{center}

\def\tabularxcolumn#1{m{#1}}
{\rowcolors{2}{RawSienna!90!RawSienna!20}{RawSienna!70!RawSienna!40}

	\begin{center}
		\renewcommand{\arraystretch}{1.4}

		\begin{longtable}{|p{5cm}|p{12cm}|}
			\hline
			\rowcolor{airforceblue}
			\multicolumn{2}{|c|}{\textit{Comunicazione interna}}\\
			\hline
			\textit{Codice} & RI1 \\
			\hline
			\textit{Descrizione} & Potrebbero esserci momenti nei quali uno o più componenti non sono reperibili. \\
			\hline
			\textit{Conseguenza} & Rallentamenti del lavoro qualora non si riuscisse a comunicare con la persona interessata. \\
			\hline
			\textit{Possibilità di occorrenza} & Bassa. \\
			\hline
			\textit{Pericolosità} & Alta. \\
			\hline
			\textit{Precauzioni} & E' necessario che ciascun componente riferisca tempestivamente al \textit{Responsabile di Progetto} eventuali momenti nei quali potrebbe non essere reperibile. \\
			\hline
			\textit{Piano di contingenza} & E' stato concordato con il gruppo di svolgere riunioni frequenti per comunicare l'avanzamento del lavoro.\\
			\hline
		\end{longtable}
	\captionof{table}{\textbf{Analisi dei rischi della comunicazione interna}}
	\end{center}

\def\tabularxcolumn#1{m{#1}}
{\rowcolors{2}{RawSienna!90!RawSienna!20}{RawSienna!70!RawSienna!40}
	
	\begin{center}
		\renewcommand{\arraystretch}{1.4}
		
		\begin{longtable}{|p{5cm}|p{12cm}|}
			\hline
			\rowcolor{airforceblue}
			\multicolumn{2}{|c|}{\textit{Comunicazione esterna}}\\
			\hline
			\textit{Codice} & RI2 \\
			\hline
			\textit{Descrizione} & Potrebbero esserci momenti nel quale l'azienda proponente$_{\scaleto{G}{3pt}}$ non è reperibile qualora avessimo necessita di contattarla. \\
			\hline
			\textit{Conseguenza} & Rallentamenti del lavoro qualora non si riuscisse a comunicare. \\
			\hline
			\textit{Possibilità di occorrenza} & Bassa. \\
			\hline
			\textit{Pericolosità} & Media. \\
			\hline
			\textit{Precauzioni} & E' stato creato un canale sulla piattaforma Discord$_G$ per poter comunicare con il proponente$_{\scaleto{G}{3pt}}$ in maniera facile e rapida. \\
			\hline
			\textit{Piano di contingenza} & Qualora si presentasse la necessità di organizzare un incontro con il proponente$_{\scaleto{G}{3pt}}$ è sufficiente richiederlo ed accordarsi con la disponibilità.\\
			\hline
		\end{longtable}
		\captionof{table}{\textbf{Analisi dei rischi della comunicazione esterna}}
	\end{center}

\def\tabularxcolumn#1{m{#1}}
{\rowcolors{2}{RawSienna!90!RawSienna!20}{RawSienna!70!RawSienna!40}
	
	\begin{center}
		\renewcommand{\arraystretch}{1.4}
		
		\begin{longtable}{|p{5cm}|p{12cm}|}
			\hline
			\rowcolor{airforceblue}
			\multicolumn{2}{|c|}{\textit{Stato di malattia}}\\
			\hline
			\textit{Codice} & RI3 \\
			\hline
			\textit{Descrizione} & Uno o più membri del gruppo possono ammalarsi. \\
			\hline
			\textit{Conseguenza} & Può influire sui task$_G$ assegnati. \\
			\hline
			\textit{Possibilità di occorrenza} & Medio/Alta. \\
			\hline
			\textit{Pericolosità} & Media. \\
			\hline
			\textit{Precauzioni} & Il membro del gruppo comunica il proprio stato di poca salute. \\
			\hline
			\textit{Piano di contingenza} & Se la malattia impedisce di lavorare, il componente del gruppo è tenuto a riprendere uno stato di salute ottimale, ed il suo lavoro è ridistribuito tra gli altri componenti del gruppo.\\
			\hline
		\end{longtable}
		\captionof{table}{\textbf{Analisi dei rischi dello stato di malattia}}
	\end{center}

\def\tabularxcolumn#1{m{#1}}
{\rowcolors{2}{RawSienna!90!RawSienna!20}{RawSienna!70!RawSienna!40}
	
	\begin{center}
		\renewcommand{\arraystretch}{1.4}
		
		\begin{longtable}{|p{5cm}|p{12cm}|}
			\hline
			\rowcolor{airforceblue}
			\multicolumn{2}{|c|}{\textit{Stress mentale}}\\
			\hline
			\textit{Codice} & RI4 \\
			\hline
			\textit{Descrizione} & Gli orari di lavoro eccessivi e la situazione di pandemia possono portare a frustazione e disagio. \\
			\hline
			\textit{Conseguenza} & Sintomi fisici di stress ed emotività instabile. \\
			\hline
			\textit{Possibilità di occorrenza} & Media. \\
			\hline
			\textit{Pericolosità} & Media. \\
			\hline
			\textit{Precauzioni} & Il gruppo collabora per creare un clima e delle relazioni all'interno per aiutare a gestire eventuali problematiche. L'appoggio anche se virtuale aiuta a sentirsi parte di un gruppo. \\
			\hline
			\textit{Piano di contingenza} & Per rilasciare la tensione si cerca di fare attività fisica costante.\\
			\hline
		\end{longtable}
		\captionof{table}{\textbf{Analisi dei rischi dello stress mentale}}
	\end{center}


% % \begin{thebibliography}{}
    \bibitem{GitFlowCheatseet}
    Daniel Kummer.
    \textit{Git flow cheatsheet - efficient branching using git-flow by Vincent Driessen}.\\
		\url{https://danielkummer.github.io/git-flow-cheatsheet/}.
		\bibitem{GitFlowWorkflowTutorial}
		Atlassian Bitbucket.
		\textit{Git flow workflow tutorial}.\\
		\url{https://www.atlassian.com/git/tutorials/comparing-workflows/gitflow-workflow}.
		\bibitem{GitCommitGuidelines}
		Cameron McKenzie, TechTarget.
		\textit{How to write a Git commit message properly with examples}.\\
		\url{https://www.theserverside.com/video/Follow-these-git-commit-message-guidelines}.
		\bibitem{TabellaDelleOreLavorative}
		Jawa Druids, 2020.
		\textit{Tabella delle Ore Lavorative}.\\
		\url{https://docs.google.com/spreadsheets/d/12esX1ISWQOKM-fjuHTLmAzRN0cWltksn7eGiPsBHBI0/edit?usp=sharing}.
		\bibitem{TrelloJawaDruids}
		Jawa Druids, 2020.
		\textit{Trello - dashboard Jawa Druids}.\\
		\url{https://trello.com/b/hIEOGbE9/jawadruids}.
\end{thebibliography}

%
% \end{document}
