\documentclass[a4paper,12pt]{report}
\usepackage[utf8]{inputenc}
\usepackage{graphicx}
\usepackage{float}
\usepackage{tabularx}
\usepackage{makecell}
\setcounter{tocdepth}{5}
\setcounter{secnumdepth}{5} %SONO I DUE COMANDI PER AVERE LE SUBSUBSECTION NUMERATE E PRESENTI NELL'INDICE

\begin{document}
    
\makeatletter
\begin{titlepage}
    \begin{center}
    \vspace*{-4,0cm}
    \author{Jawa Druids} 
    \title{Linee Guida}
    \date{} %LASCIARE QUESTO CAMPO VUOTO, SE LO TOLGO STAMPA LA DATA CORRENTE
    \includegraphics[width=0.7\linewidth]{../immagini/DRUIDSLOGO.jpg}\\[4ex]
    {\huge \bfseries  \@title }\\[2ex] 
    {\LARGE  \@author}\\[50ex]
    \vspace*{-8,0cm}
    \begin{table}[H]
        \centering
        \begin{tabular}{c|c}
            \textbf{Versione} & x.x.x \\
            \textbf{Data approvazione} & xx-xx-xxxx\\
            \textbf{Responsabile} & Nome Cognome\\
            \textbf{Redattori} & \makecell{Andrea Dorigo \\ Margherita Mitillo} \\
            \textbf{Verificatori} & \makecell{Nome Cognome \\ Nome Cognome} \\
        %MAKECELL SERVE PER POI ANDARE A CAPO ALL'INTERNO DELLA CELLA
            \textbf{Stato} & stato\\
            \textbf{Lista distribuzione} & \makecell{Jawa Druids \\ Tullio \\ NomeAziendaProponente}\\
            \textbf{Uso} & Interno            
        \end{tabular}
    \end{table}
    \fontsize{16}{10}\textbf{Sommario} \\
    Linee guide da seguire per una sessione lavorativa standard con git flow, Trello e google sheet
    \end{center}
\end{titlepage}
\makeatother
\end{document}