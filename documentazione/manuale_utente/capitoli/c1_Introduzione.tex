\chapter{Introduzione}\label{Introduzione}

\section{Scopo del documento}\label{IntroduzioneScopoDelDocumento}

Lo scopo di questo documento è fornire all'utente tutte le indicazioni per il corretto uso del software da noi prodotti.

\section{Scopo del prodotto}\label{IntroduzioneScopoDelProdotto}

In seguito alla pandemia del virus COVID-19 è nata l'esigenza di limitare il più possibile i
contatti fra le persone, specialmente evitando la formazione di assembramenti. 
Il progetto \textit{GDP: Gathering Detection Platform} di \textit{Sync Lab} ha pertanto l'obiettivo di \textbf{creare una piattaforma in grado di rappresentare graficamente le zone potenzialmente a rischio di assembramento, al fine di prevenirlo.}Il prodotto finale è rivolto specificatamente agli
organi amministrativi delle singole città, cosicché possano gestire al meglio i punti sensibili di
affollamento, come piazze o siti turistici. Lo scopo che il software intende raggiungere non è
solo quello della rappresentazione grafica real-time ma anche di poter riuscire a prevedere
assembramenti in intervalli futuri di tempo.
\\
A tal fine il gruppo \textit{Jawa Druids} si prefigge di sviluppare un prototipo software in grado di acquisire, monitorare ed analizzare i molteplici dati provenienti dai diversi sistemi e dispositivi, a scopo di identificare i possibili eventi che concorrono all'insorgere di variazioni di flussi di utenti. Il gruppo prevede inoltre lo sviluppo di un'applicazione web da interporre fra i dati elaborati e l'utente, per favorirne la consultazione.

\section{Glossario}\label{IntroduzioneGlossario}

All'interno della documentazione viene fornito un \textit{Glossario}, con l'obiettivo di assistere il lettore specificando il significato e contesto d'utilizzo di alcuni termini strettamente tecnici o ambigui, segnalati con una \textit{G} a pedice.

%\section{Riferimenti}\label{IntroduzioneRiferimenti}

%\subsection{Riferimenti informativi}\label{IntroudioneRiferimentiRiferimentiInformativi}