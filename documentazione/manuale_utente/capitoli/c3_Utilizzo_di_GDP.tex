
\chapter{Utilizzo di GDP - Gathering Detection Platform}\label{UtilizzoDiGDPGatheringDetecionPlatform}

\section{Pagina iniziale}\label{UtilizzoDiGDPGatheringDetecionPlatformPaginaIniziale}
La pagina iniziale che si presenta all'avvio è mostrata nella seguente figura.
Al suo interno troviamo il nome della web application$_{\scaleto{G}{3pt}}$ seguito dalle componenti qui elencate e successivamente spiegate:
\begin{enumerate}
	\item La barra di navigazione;
	\item Contenuto centrale;
	\item Il footer$_G$.
\end{enumerate}

\begin{center}
	\begin{figure}[H]
		\centering\includegraphics[width=0.9\linewidth]{../immagini/manualeUtente/mainpage.png}
		\caption{\textbf{\textbf{\textbf{Pagina iniziale}}}}
	\end{figure}
\end{center} 

\section{Barra di navigazione}\label{UtilizzoDiGDPGatheringDetecionPlatformBarraDiNavigazione}

La barra di navigazione dell'applicazione web$_{\scaleto{G}{3pt}}$ è quella rappresentata in figura 3.2. Tramite questa l'utente potrà navigare all'interno della piattaforma. Nella barra di navigazione sono presenti:
\begin{enumerate}
	\item \textit{Home}: link alla pagina iniziale;
	\item \textit{Chi siamo}: link alla pagina "Chi siamo";
	\item \textit{Barra di ricerca}: attraverso la barra di ricerca è possibile cercare, e quindi selezionare tra quelle disponibili, la città di cui si è interessati a visualizzare i dati sulla heat-map$_G$. 
\end{enumerate}
\begin{center}
	\begin{figure}[H]
		\includegraphics[width=1\linewidth]{../immagini/manualeUtente/BarraDiNavigazioe.png}
		\caption{\textbf{Barra di Navigazione}}
	\end{figure}
\end{center}

\section{Contenuto centrale}\label{UtilizzoDiGDPGatheringDetecionPlatformContenutoCentrale}

\subsection{Pagina Iniziale - Home} \label{UtilizzoDiGDPGatheringDetecionPlatformContenutoCentralePaginaInizialeHome}
La pagina iniziale che visualizza l'utente è quella mostrata in figura 3.1.

\subsubsection{Heatmap}\label{UtilizzoDiGDPGatheringDetecionPlatformContenutoCentralePaginaInizialeHomeHeatmap}
Al centro della web app$_{\scaleto{G}{3pt}}$ è presente una heat map$_{\scaleto{G}{3pt}}$ che raffigura, inizialmente, il flusso di persone presenti nella città di Roma nell'orario attuale. Successivamente l'utente potrà modificare la città, attraverso l'elenco delle città (\S~\ref{UtilizzoDiGDPGatheringDetecionPlatformContenutoCentralePaginaInizialeHomeMenùATendina}) oppure tramite la barra di ricerca, l'orario e la data, secondo quanto spiegato in \S~\ref{UtilizzoDiGDPGatheringDetecionPlatformContenutoCentralePaginaInizialeHomeCalendarioESlider}, e visualizzare tramite la heat map$_{\scaleto{G}{3pt}}$ i dati relativi ai campi selezionati.\\
Per facilitare la lettura della mappa, dopo che l'utente ha effettuato lo zoom-in$_G$, viene mostrato un pop-up$_G$, accompagnato da un marker$_G$, che evidenzia sia il nome del luogo che si sta osservando sia il numero di persone effettivamente presenti in quel momento. L'utente ha la possibilità di chiudere il pop-up$_{\scaleto{G}{3pt}}$. Questa funzionalità viene illustrata nella figura successiva.

\begin{center}
	\begin{figure}[H]
		\centering\includegraphics[width=0.9\linewidth]{../immagini/manualeUtente/popup.png}
		\caption{\textbf{\textbf{\textbf{Heatmap con popup visibile}}}}
	\end{figure}
\end{center} 

\subsubsection{Elenco delle città} \label{UtilizzoDiGDPGatheringDetecionPlatformContenutoCentralePaginaInizialeHomeMenùATendina}
L'elenco delle città, posizionato a destra della mappa, viene utilizzato per la selezione della città. Infatti, l'utente, quando apre l'applicazione web$_{\scaleto{G}{3pt}}$, visualizza la mappa centrata sulla città di Roma, la città di default, ma successivamente può scegliere di osservare il flusso di persone relativo ad un'altra città presente tra quelle messe a disposizione nella lista. 
\begin{center}
	\begin{figure}[h]
		\centering\includegraphics[width=0.2\linewidth]{../immagini/manualeUtente/ElencoCittà.png}
		\caption{\textbf{Elenco Città}}
	\end{figure}
\end{center}

\subsubsection{Calendario e slider}\label{UtilizzoDiGDPGatheringDetecionPlatformContenutoCentralePaginaInizialeHomeCalendarioESlider}
L'utente ha a disposizione, a sinistra della mappa, un calendario che gli permette di scegliere l'anno, il mese ed il giorno di cui desidera visualizzare i dati. Per selezionare il mese bisogna spostarsi usando le frecce poste ai lati (1), mentre per l'anno si seleziona sull'anno corrente (2).
\begin{center}
	\begin{figure}[H]
		\centering\includegraphics[width=0.3\linewidth]{../immagini/manualeUtente/Calendario.jpg}
		\caption{\textbf{Calendario}}
	\end{figure}
\end{center}
Al di sopra della mappa, invece, è presente uno slider$_G$ con il quale l'utente può scegliere un orario diverso da quello attuale di cui desidera visualizzare i dati attraverso la heat map$_{\scaleto{G}{3pt}}$. La selezione dell'orario è effettuata su intervalli di tempo di ora in ora. Per selezionare l'ora attravero lo slider$_{\scaleto{G}{3pt}}$, si può sia cliccare sulla scritta dell'orario che si vuole selezionare sia spostare l'etichetta rappresentate l'ora selezionata(1).
\begin{center}
	\begin{figure}[H]
		\centering\includegraphics[width=1\linewidth]{../immagini/manualeUtente/Slider.png}
		\caption{\textbf{Slider}}
	\end{figure}
\end{center}

\subsubsection{Bottone Reload Map} \label{UtilizzoDiGDPGatheringDetecionPlatformContenutoCentralePaginaInizialeHomeBottoneReloadMap}
Nel caso in cui l'utente avesse selezionato una data diversa da quella odierna e/o un'ora differente da quella attuale, cliccando sul pulsante "Reload Map", la mappa si aggiornerà e tornerà a mostrare i dati in tempo reale, quindi relativi a data e ora corrente, rimanendo sulla città che si stava osservando. 
\begin{center}
	\begin{figure}[H]
		\centering\includegraphics[width=0.3\linewidth]{../immagini/manualeUtente/ReloadMap.png}
		\caption{\textbf{Reload Map}}
	\end{figure}
\end{center}

\subsubsection{Messaggio d'errore} \label{UtilizzoDiGDPGatheringDetecionPlatformContenutoCentralePaginaInizialeHomeMessaggioDiErrore}
Nel caso in cui non ci siano dati disponibili nel database per il luogo, il giorno e l'ora in questione, l'utente visualizzerà un messaggio di errore che lo informerà del disguido e la mappa non mostrerà nessun dato. L'utente potrà chiudere il messaggio d'errore premendo il pulsante "OK". 
\begin{center}
	\begin{figure}[H]
		\centering\includegraphics[width=0.5\linewidth]{../immagini/manualeUtente/MessaggioErrore.png}
		\caption{\textbf{Messaggio d'errore}}
	\end{figure}
\end{center}

\subsection{Chi siamo} \label{UtilizzoDiGDPGatheringDetecionPlatformContenutoCentraleChiSiamo}
In questa pagina è possibile visualizzare le informazioni che riguardano il team di sviluppo \textit{Jawa Druids}, l'azienda proponente \textit{Sync Lab} e il progetto \textit{GDP: Gathering Detection Platform}. 

\begin{center}
	\begin{figure}[H]
		\centering\includegraphics[width=0.5\linewidth]{../immagini/manualeUtente/AboutUs.png}
		\caption{\textbf{Chi siamo}}
	\end{figure}
\end{center} 

\section{Footer}\label{UtilizzoDiGDPGatheringDetecionPlatformFooter}
Il footer$_{\scaleto{G}{3pt}}$ è presente in tutte le pagine dell'applicazione web$_{\scaleto{G}{3pt}}$ e riporta alcuni link utili, come quello del sito web dell'azienda \textit{Sync Lab} e la mail del team di sviluppo, da contattare in caso si riscontrino problemi con l'uso del prodotto software$_{\scaleto{G}{3pt}}$. 
%da sistemare
\begin{center}
	\begin{figure}[H]
		\centering\includegraphics[width=1\linewidth]{../immagini/manualeUtente/Footer.png}
		\caption{\textbf{Footer}}
	\end{figure}
\end{center}