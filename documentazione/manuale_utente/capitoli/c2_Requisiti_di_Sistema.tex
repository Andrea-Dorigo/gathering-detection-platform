\chapter{Requisiti di sistema ed installazione}\label{RequisitiDiSistemaEdInstallazione}

\section{Requisiti}\label{RequisitiDiSistemaEdInstallazioneRequisiti}
GDP - Gathering Detection Platform è un'applicazione web$_G$ per dispositivi desktop. 

\subsection{Requisiti di sistema}\label{RequisitiDiSistemaEdInstallazioneRequisitiRequisitiDiSistema}

Il funzionamento della web application$_{\scaleto{G}{3pt}}$ richiede che nella macchina siano installate:
\begin{itemize}
	\item la piattaforma \textit{Node.js}$_G$;
	\item la versione 11 o superiore di \textit{Java}$_G$;
	\item versione 2.6.1 di Vue.js $_G$.
	%\item mongo??
\end{itemize}
Infine, deve essere disponibile un browser attraverso il quale poter visualizzare l'interfaccia delll'applicazione.

\subsection{Browser supportati}\label{RequisitiDiSistemaEdInstallazioneRequisitiBrowserSupportati}
Perché l'applicazione funzioni in modo corretto è necessario che Javascript sia abilitato sul browser. 
Qui di seguito sono riportati i browser per i quali si garantisce la compatibilità. 
\begin{itemize}
	\item \textit{Google Chrome} versione 88 o superiori;
	\item \textit{Microsoft Edge} versione 87;
	\item \textit{Mozzilla Firefox Developer} versione 89;
\end{itemize}
L'applicazione web$_G$ potrebbe funzionare correttamente anche su versioni precedenti e/o successive o su altri browser, ma non si garantisce il supporto. 
 

\section{Installazione}\label{RequisitiDiSistemaEdInstallazioneInstallazione}

 GDP-Gathering Detection Platform è un progetto universitario, per tale motivo la web application$_{\scaleto{G}{3pt}}$ non è raggiungibile online tramite un link internet. Di conseguenza l'utente, per poter visualizzare il prodotto software finale, deve svolgere gli stessi passaggi di installazione di uno sviluppatore.\\
Questa sezione terrà conto delle procedure di installazione all'interno del sistema operativo Linux$_G$, più precisamente Ubuntu$_G$ 20.04 LTS, in quanto utilizzato anche per lo sviluppo del software stesso.

\subsection{Download della  repository}\label{RequisitiDiSistemaEdInstallazioneInstallazioneDownloadRepo}
Come prima fase bisogna scaricare la repository$_G$ relativa al progetto \textit{Gathering-Detection-Platform}, presente al seguente link:
\begin{center} \url{https://github.com/Andrea-Dorigo/gathering-detection-platform}
\end{center}

Per scaricare correttamente i contenuti della repository$_{\scaleto{G}{3pt}}$ è necessario installare \texttt{git} e \texttt{git-lfs}(\textit{Git Large File Storage}).
Su Ubuntu$_{\scaleto{G}{3pt}}$ 20.04, questo è possibile eseguendo il comando:
\begin{lstlisting}
	sudo apt install git git-lfs
\end{lstlisting}
assumendo che le principali repository$_{\scaleto{G}{3pt}}$ per i pacchetti di Ubuntu$_{\scaleto{G}{3pt}}$ siano attive (Universe, Multiverse).

Questo passaggio è richiesto poiché GitHub$_G$ (il sito che ospita la repository del progetto) consente l'upload di file con dimensioni massime fino a 100MB.
L'utilizzo di \textit{Git Large File Storage} permette l'upload e il download di file che superano questo limite, ed in particolare permette l'upload e download dei pesi necessari all'algoritmo YOLOv3 per il rilevamento di oggetti (più precisamente per il rilevamento delle persone in un'immagine), il quale ha una dimensione maggiore di 200MB. Maggiori informazioni riguardo \textit{Git Large File Storage} sono reperibili all'indirizzo:
\begin{center}
	 \url{https://git-lfs.github.com} .
\end{center}

È dunque possibile scaricare correttamente la repository$_{\scaleto{G}{3pt}}$ relativa al progetto \textit{Gathering-Detection-Platform} con il seguente comando:
\begin{lstlisting}
git clone https://github.com/Andrea-Dorigo/gathering-detection- platform.git
\end{lstlisting}

\subsection{Installazione dipendenze}\label{RequisitiDiSistemaEdInstallazioneInstallazioneInstallazioneDipendenze}
Dopo eseguito il passo sopra descritto, è obbligatorio installare le dipendenze necessarie per eseguire il prodotto software$_{\scaleto{G}{3pt}}$ adeguatamente.
Per fare ciò sarà necessario aprire il terminale, all'interno della cartella \textit{gathering-detection-platform}, e copiare il seguente comando:
\begin{lstlisting}
sudo apt install python3-opencv python3-pip mongo maven npm Git Git-LFS 
&& pip3 install mongoengine
\end{lstlisting}

Una volta conclusa questa operazione il programma potrà essere eseguito senza problemi.

\subsection{Inizializzazione modulo Acquisition}\label{RequisitiDiSistemaEdInstallazioneInstallazioneInizializzazioneModuloAcquisition}
Per eseguire il modulo di acquisizione, in modo da iniziare a raccogliere i dati dalle webcam salvate, basterà posizionarsi all'interno della cartella "acquisition", dopodiché nella cartella "main" e da terminale eseguire il comando:
\begin{lstlisting}
python3 detect.py
\end{lstlisting}
Se i passi precedenti sono stati eseguiti correttamente allora si vedranno comparire sul terminale i vari passaggi che svolge il modulo.

%\subsubsection{Aggiunta di una webcam}
%Per aggiungere una webcam al modulo Acquisition è molto semplice.
%Per una questione di codifica, il link della webcam dev'essere conforme a quelle già presenti, ovvero provenire da \url{https://www.whatsupcams.com/}.
%Dopo aver aggiunto il link tratta di accedere al file \textit{webcams.json} ed inserire la webcam prendendo spunto dallo schema delle altre già inserite.


\subsection{Inizializzazione modulo Prediction}\label{RequisitiDiSistemaEdInstallazioneInstallazioneInizializzazioneModuloPrediction}
Per eseguire il modulo di predizione, che tramite il machina-learning si occupa di calcolare, appunto, le predizioni per lassi di tempo futuri, bisognerà posizionarsi all'interno della cartella "prediction" ed eseguire il seguente comando da terminale:

\begin{lstlisting}
python3 DataPrediction.py
\end{lstlisting}
In tal modo verrà attivato il modulo per le predizioni sui dati.


\subsection{Inizializzazione modulo Web-App}\label{RequisitiDiSistemaEdInstallazioneInstallazioneInizializzazioneModuloWebApp}
Per avviare la web-app$_{\scaleto{G}{3pt}}$, e le sue funzioni, si devono eseguire alcuni comandi, sempre da terminale, a partire dalla cartella webapp.
\begin{enumerate}
	\item posizionarsi all'interno della cartella "webapp" ed eseguire:
	\begin{lstlisting}
	mvn spring-boot:run
	\end{lstlisting}
	\item posizionarsi all'interno della cartella "vue-js-client-crud" ed eseguire:
	\begin{lstlisting}
	npm install
	\end{lstlisting}
	\item successivamente, all'interno della stessa cartella, bisogna eseguire:
	\begin{lstlisting}
	npm run serve
	\end{lstlisting}
	\item infine, la piattaforma \textit{Gathering-Detection-Platform} sarà disponibile all'indirizzo di default \url{http://localhost:8081} 
\end{enumerate}
