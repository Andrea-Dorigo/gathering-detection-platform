\chapter{Glossario} \label{Glossario}
\textbf{F} \\
\\
\textbf{Footer} \\
Il footer indica la parte inferiore di un sito e normalmente ha lo stesso aspetto per tutte le pagine del dominio.Tipicamente contiene i link alle note legali, email  e ad altri elementi di navigazione del sito.
\\
\\
\textbf{H} \\
\\
\textbf{Heat map} \\
 Rappresentazione grafica dei dati dove i singoli valori contenuti in una matrice sono rappresentati da colori. \\
\\
\textbf{J} \\
\\
\textbf{Java} \\
Linguaggio di programmazione ad alto livello, orientato agli oggetti e a tipizzazione statica, che si appoggia sull'omonima piattaforma software di esecuzione, specificatamente progettato per essere il più possibile indipendente dalla piattaforma hardware di esecuzione. Le principali caratteristiche di Java sono la portabilità, cioè il codice sorgente
è compilato in bytecode e può essere eseguito su ogni PC che ha JVM (Java Virtual Machine), e la robustezza. \\
\\
\\
\textbf{L} \\
\\
\textbf{Linux} \\
Linux è una famiglia di sistemi operativi open source che hanno come caratteristica comune quella di utilizzare come nucleo il kernel Linux. Linux è il primo rappresentante del software "libero", ovvero quel software che viene distribuito con una licenza che ne permette l'utilizzo da parte di chiunque. \\
\\
\textbf{N}\\
\\
\textbf{Node.js} \\
Node.js è un’applicazione, per la precisione un framework, che viene usata per scrivere applicazioni in Javascript lato server. \\
\\
\textbf{R} \\
\\
\textbf{Repository} \\
Ambiente di un sistema informativo in cui vengono conservati e gestiti file, documenti e metadati relativi ad un’attività$_G$ di progetto. \\
\\
\textbf{S} \\
\\
\textbf{Slider} \\
Lo slider è un componente grafico usato dall'utente come cursore per regolare una proprietà. Infatti, attraverso tale componente, un utente può impostare un valore muovendo un indicatore, solitamente con uno spostamento orizzontale. In alcuni casi, l'utente può anche cliccare in un punto dello slider per cambiare le impostazioni.  \\
\\
\textbf{U} \\
\\
\textbf{Ubuntu} \\
\\
Ubuntu è un sistema operativo di distribuzione Linux, basata su Debian e che ha in Unity il suo ambiente desktop. Si basa esclusivamente su software libero distribuito liberamente con licenza GNU GPL ma supporta anche software proprietario. \\
\\
\textbf{V}\\
\\
\textbf{Vue.js}\\
 Framework open-source per lo sviluppo di applicazioni web, interfacce utente e applicazioni. \\
\\
\textbf{W} \\
\\
\textbf{Web application o web app}\\
Con web application (o web app) si intende un'applicazione accessibile via web attraverso un network. 
\\
 
 
 Mancano 
 -prodotto software/ software ??
 - gathering detection platform ?? 
 - dublicare ?