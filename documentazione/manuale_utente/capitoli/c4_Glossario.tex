\chapter{Glossario} \label{Glossario}
\textbf{A} \\
\\
\textbf{Applicazione web} \\
Con applicazione web si intende un'applicazione accessibile via web attraverso un network. \\
\\
\textbf{Framework} \\
In informatica e specificamente nello sviluppo software$_{\scaleto{G}{3pt}}$, è un'architettura logica di supporto sulla quale un software$_{\scaleto{G}{3pt}}$ può essere progettato e realizzato.\\
\\
\textbf{G} \\
\\
\textbf{Git}\\
Sistema di controllo gratuito a versione distribuita progettato per tenere traccia del lavoro svolto durante l'intero periodo di sviluppo del software$_{\scaleto{G}{3pt}}$.
Utilizzato anche per tenere traccia di tutte le modifiche fatte nei file.
I suoi punti di forza sono l'integrità dei dati e il supporto per flussi di lavoro distribuiti e non lineari.\\
\\
\textbf{GitHub} \\
GitHub è un servizio di hosting per progetti software$_{\scaleto{G}{3pt}}$. Il nome deriva dal fatto che esso è una implementazione dello strumento di controllo versione distribuito Git$_G$. \\
\\
\textbf{H} \\
\\
\textbf{Heat map} \\
 Rappresentazione grafica dei dati dove i singoli valori contenuti in una matrice sono rappresentati da colori. \\
\\
\textbf{J} \\
\\
\textbf{Java} \\
Linguaggio di programmazione ad alto livello, orientato agli oggetti e a tipizzazione statica, che si appoggia sull'omonima piattaforma software$_{\scaleto{G}{3pt}}$ di esecuzione, specificatamente progettato per essere il più possibile indipendente dalla piattaforma hardware di esecuzione. Le principali caratteristiche di Java sono la portabilità, cioè il codice sorgente
è compilato in bytecode e può essere eseguito su ogni PC che ha JVM (Java Virtual Machine), e la robustezza. \\
\\
\textbf{Javascript} \\
Linguaggio di programmazione orientato ad oggetti ed eventi. Utilizzato specficatamente
per la programmazione Web lato client. Aggiunge al sito effetti dinamici tramite funzioni
invocate da un'azione eseguita sulla pagine Web(es. click del mouse, movimento del
mouse, caricamento pagina). \\
\\
\textbf{L} \\
\\
\textbf{Linux} \\
Linux è una famiglia di sistemi operativi open source$_G$ che hanno come caratteristica comune quella di utilizzare come nucleo il kernel Linux. Linux è il primo rappresentante del software$_{\scaleto{G}{3pt}}$ "libero", ovvero quel software$_{\scaleto{G}{3pt}}$ che viene distribuito con una licenza che ne permette l'utilizzo da parte di chiunque. \\
\\
\textbf{O} \\
\\
\textbf{Opern-Source} \\
Un software$_{\scaleto{G}{3pt}}$ open-source è reso tale per mezzo di una licenza attraverso cui i detentori dei
diritti favoriscono la modifica, lo studio, l'utilizzo e la redistribuzione del codice sorgente.\\
\\
\textbf{R} \\
\\
\textbf{Repository} \\
Ambiente di un sistema informativo in cui vengono conservati e gestiti file, documenti e metadati relativi ad un’attività di progetto. \\
\\
\textbf{S} \\
\\
\textbf{Slider} \\
Lo slider è un componente grafico usato dall'utente come cursore per regolare una proprietà. Infatti, attraverso tale componente, un utente può impostare un valore muovendo un indicatore, solitamente con uno spostamento orizzontale. In alcuni casi, l'utente può anche cliccare in un punto dello slider per cambiare le impostazioni.  \\
\\
\textbf{Software} \\
Un software èl'insieme delle procedure e delle istruzioni in un sistema di elaborazione dati. In contrapposizione all'hardware, si identifica con un insieme di programmi. \\
\\
\textbf{Spring} \\
In informatica Spring è un framework$_{\scaleto{G}{3pt}}$ open-source$_{\scaleto{G}{3pt}}$ per lo sviluppo di applicazioni su piattaforma Java$_{\scaleto{G}{3pt}}$.\\
\\
\textbf{T} \\
\\
\textbf{Topic}\\ 
Nell'ambito di Apache Kafka, si intende una categoria per utilizzata per raggruppare i messaggi.\\ 
\\
\textbf{U} \\
\\
\textbf{Ubuntu} \\
Ubuntu è un sistema operativo di distribuzione Linux$_{\scaleto{G}{3pt}}$, basata su Debian e che ha in Unity il suo ambiente desktop. Si basa esclusivamente su software$_{\scaleto{G}{3pt}}$ libero distribuito liberamente con licenza GNU GPL ma supporta anche software$_{\scaleto{G}{3pt}}$ proprietario. \\
\\
\textbf{W} \\
\\
\textbf{Web application o web app}\\
\textit{Si rimanda alla voce "Applicazione web".}