
\input{packages}
\input{config}

\begin{document}
	
	\makeatletter
	\begin{titlepage}
		\begin{center}
			\vspace*{-4cm}
			\author{Jawa Druids} 
			\title{Piano di Qualifica}
			\date{} %LASCIARE QUESTO CAMPO VUOTO, SE LO TOLGO STAMPA LA DATA CORRENTE
			\includegraphics[width=0.5\linewidth]{../immagini/DRUIDSLOGO.jpg}\\[4ex]
			{\huge \bfseries  \@title }\\[2ex] 
			{\LARGE  \@author}\\[50ex]
			\vspace*{-9cm}
			\begin{table}[H]
				\renewcommand{\arraystretch}{1.4}
				\centering
				\begin{tabular}{r | l}
					\textbf{Versione} & 4.0.0 \\%RIGA PER INSERIRE LA VERSIONE ULTIMA DEL DOCUMENTO
					\textbf{Data approvazione} & 2021-05-23\\
					\textbf{Responsabile} & Andrea Cecchin\\
					\textbf{Redattori} & \makecell[tl]{ Emma Roveroni \\ Margherita Mitillo} \\
					\textbf{Verificatori} & \makecell[tl]{Andrea Dorigo \\ Andrea Cecchin \\ Emma Roveroni} \\
					%MAKECELL SERVE PER POI ANDARE A CAPO ALL'INTERNO DELLA CELLA
					\textbf{Stato} & Approvato\\
					\textbf{Lista distribuzione} & \makecell[tl]{Jawa Druids \\ Prof. Tullio Vardanega \\ Prof. Riccardo Cardin \\ Sync Lab}\\
					\textbf{Uso} & Esterno  
				\end{tabular}
			\end{table}
			\vspace{0.1cm}
			\hfill \break
			\fontsize{17}{10}\textbf{Sommario} \\
			\vspace{0.1cm}
			Il documento contiene le strategie di verifica e validazione seguite dal gruppo \emph{\normalsize\textit{Jawa Druids}} durante lo svolgimento del progetto \emph{\normalsize\textit{Gathering Detection Platform}}.
		\end{center}
	\end{titlepage}
	\makeatother
	
	\quad
\begin{center}
	\LARGE\textbf{Registro delle modifiche}
\end{center}

\def\tabularxcolumn#1{m{#1}}
{\rowcolors{2}{RawSienna!90!RawSienna!20}{RawSienna!70!RawSienna!40}


\begin{center}
	\renewcommand{\arraystretch}{1.4}
	\begin{longtable}[c]{|p{2cm-1\tabcolsep}|p{2cm}|p{3cm-2\tabcolsep}|p{2,5cm-2\tabcolsep}|p{4cm-2\tabcolsep}|p{2,5cm}|}
		\hline
		\rowcolor{airforceblue}
		\makecell[c]{\textbf{Versione}} & \makecell[c]{\textbf{Data}} & \makecell[c]{\textbf{Autore}} & \makecell[c]{\textbf{Ruolo}} & \makecell[c]{\textbf{Modifica}} & \makecell[c]{\textbf{Verificatore}} \\
		\hline
		\centering v1.0.0 & 2021-04-18 & Andrea Dorigo & \centering \textit{Responsabile} & \textit{Approvazione del documento} & \makecell[c]{-} \\
		\hline
		\centering v0.1.0 & 2021-04-17 &  \centering - & \centering - &  \textit{Revisione complessiva del documento} & Mattia Cocco \\
		\hline
		\centering v0.0.1 & 2021-04-15 & Andrea Cecchin & \centering \textit{Redattore} &\textit{Stesura del documento}  & Mattia Cocco \\
		\hline
	\end{longtable}
\end{center}

	\tableofcontents{}
	\listoffigures
	\listoftables
	\chapter{Introduzione}\label{Introduzione}

\section{Scopo del documento}\label{IntroduzioneScopoDelDocumento}

Lo scopo di questo documento è fornire all'utente tutte le indicazioni per il corretto uso del software$_G$ da noi prodotto.

\section{Scopo del prodotto}\label{IntroduzioneScopoDelProdotto}

In seguito alla pandemia del virus COVID-19 è nata l'esigenza di limitare il più possibile i
contatti fra le persone, specialmente evitando la formazione di assembramenti. 
Il progetto \textit{GDP: Gathering Detection Platform} di \textit{Sync Lab} ha pertanto l'obiettivo di \textbf{creare una piattaforma in grado di rappresentare graficamente le zone potenzialmente a rischio di assembramento, al fine di prevenirlo.}Il prodotto finale è rivolto specificatamente agli
organi amministrativi delle singole città, cosicché possano gestire al meglio i punti sensibili di
affollamento, come piazze o siti turistici. Lo scopo che il software$_{\scaleto{G}{3pt}}$ intende raggiungere non è
solo quello della rappresentazione grafica real-time ma anche di poter riuscire a prevedere
assembramenti in intervalli futuri di tempo.
\\
A tal fine il gruppo \textit{Jawa Druids} si prefigge di sviluppare un prototipo software$_{\scaleto{G}{3pt}}$ in grado di acquisire, monitorare ed analizzare i molteplici dati provenienti dai diversi sistemi e dispositivi, a scopo di identificare i possibili eventi che concorrono all'insorgere di variazioni di flussi di utenti. Il gruppo prevede inoltre lo sviluppo di un'applicazione web$_G$ da interporre fra i dati elaborati e l'utente, per favorirne la consultazione.

\section{Glossario}\label{IntroduzioneGlossario}

All'interno della documentazione viene fornito un \textit{Glossario}, con l'obiettivo di assistere il lettore specificando il significato e contesto d'utilizzo di alcuni termini strettamente tecnici o ambigui, segnalati con una \textit{G} a pedice.

%\section{Riferimenti}\label{IntroduzioneRiferimenti}

%\subsection{Riferimenti informativi}\label{IntroudioneRiferimentiRiferimentiInformativi}
	\chapter{Qualità di processo}\label{2}
Per garantire un prodotto di qualità, che rispetti i costi ed i tempi stabiliti dal \textit{Piano di Progetto 1.0.0}, il nostro gruppo ha deciso di aderire allo standard ISO/IEC 15504, anche noto come SPICE (\textit{Software Process Improvement and Capability Determination}) e di utilizzare il Ciclo di Deming.
Lo standard ISO/IEC 15504 garantisce la qualità di tutti i processi che compongono il prodotto attraverso una definizione chiara degli obiettivi di ognuno di essi e delle soglie prestabilite, mentre  il Ciclo di Deming ha come fine  il principio del miglioramento continuo di tale qualità.
Per una descrizione più dettagliata dello standard ISO/IEC 15504 e del ciclo di Deming,
riferirsi rispettivamente ai capitoli §6 e §7 in \textit{Norme di Progetto 1.0.0}.
\section{Processo di pianificazione}\label{2.1}
La pianificazione è un’attività significativa della gestione di progetto. 
Consiste nel governare le risorse a disposizione, ovvero tempi, costi e ruoli, monitorarle nel tempo e reagire efficacemente ai cambiamenti. 

\subsection{Obiettivi}\label{2.1.1}
Gli obiettivi relativi a questo processo sono:
\begin{itemize}
	\item avere a disposizioni piani ben definiti;
	\item aver definito ruoli, responsabilità, obblighi e autorità a cui rispondere;
	\item aver allocato le risorse ed i beni necessari;
	\item attivare il piano per sostenere il progetto.
\end{itemize}

\subsection{Strategia}\label{2.1.2}
La strategia messa in atto per il raggiungimento di tali obiettivi consiste nei seguenti punti:
\begin{itemize}
	\item produrre la pianificazione delle attività
	\item mantenerla aggiornata mentre esse vengono svolte;
	\item usarla come riferimento e supporto.
\end{itemize}

\subsection{Metriche}\label{2.1.3}
Di seguito si presentano le metriche relative alla qualità dei processi, come stabilito nel
documento \textit{Norme di Progetto 1.0.0}, indispensabili per ottenere gli obiettivi di qualità.

\subsubsection{MQPS01 Budget at Completion}\label{2.1.3.1}
Quantità di budget totale allocato per il progetto.
\begin{itemize}
	\item \textbf{misurazione:} numero intero;
	\item \textbf{valore preferibile:} corrispondente al preventivo;
	\item \textbf{valore accettabile:} il valore del preventivo con un errore massimo del 5\%, ossia:
	\begin{center}
		\textit{preventivo -5\% $\leq$ BAC $\leq$ preventivo + 5\%}
	\end{center}
\end{itemize}

\subsubsection{MQPS02 Planned value}\label{2.1.3.2}
Si tratta del valore del lavoro pianificato al momento del calcolo.
\begin{itemize}
	\item \textbf{misurazione:} BAC *\% di lavoro pianificato
	\item \textbf{valore preferibile:} $>$ 0;
	\item \textbf{valore accettabile:} $\geq$ 0.
\end{itemize}

\subsubsection{MQPS03 Actual cost}\label{2.1.3.3}
Il denaro speso fino al momento del calcolo per lo svolgimento del progetto.
E’ necessario un monitoramento continuo per avere un Actual Cost al di sotto della soglia del Planned Value.
\begin{itemize}
	\item \textbf{misurazione:} numero intero;
	\item \textbf{valore preferibile:} 0 $\leq$ AC $<$ PV
	\item \textbf{valore accettabile:}  0 $\leq$ AC $\leq$  budget totale.
\end{itemize}

\subsubsection{MQPS04 Earned value}\label{2.1.3.4}
\begin{itemize}
	\item \textbf{misurazione:}  BAC * \% di lavoro completato ;
	\item \textbf{valore preferibile:} $\geq$ PV;
	\item \textbf{valore accettabile:} $\geq$ 0.
\end{itemize}

\subsubsection{MQPS05 Schedule Variance}\label{2.1.3.5}
Indica lo stato di avanzamento nello svolgimento del progetto rispetto a quanto pianificato.
\begin{itemize}
	\item \textbf{misurazione:} SV = EV - PV;
	\item \textbf{valore preferibile:} $>$ 0;
	\item \textbf{valore accettabile:} $\geq$ 0.
\end{itemize}
In base al risultato ottenuto:
\begin{itemize}
	\item SV $>$ 0  indica che il gruppo è in anticipo rispetto alla pianificazione; in futuro,
	le previsioni dovranno essere eseguite con più precisione e tenendo conto di questo
	risultato;
	\item SV = 0  indica che il gruppo è in linea con la pianificazione; i criteri adottati per
	fare la pianificazione sono quindi efficaci, e dovranno essere usati anche per previsioni
	future;
	\item SV $<$ 0 indica che il gruppo è in ritardo rispetto alla pianificazione: è necessaria
	una revisione delle pianificazioni da quel momento in poi, in modo da ridistribuire le
	risorse ed evitare di accumulare ulteriori ritardi.
\end{itemize}

\subsubsection{MQPS06 Cost variance}\label{2.1.3.6}
Indica la differenza tra il costo di lavoro effettivamente completato ed il costo attualmente sostenuto.
\begin{itemize}
	\item \textbf{misurazione:} CV = EV - AC;
	\item \textbf{valore preferibile:} $>$ 0;
	\item \textbf{valore accettabile:} $\geq$ 0.
\end{itemize}
In base al risultato ottenuto:
\begin{itemize}
	\item CV $>$ 0 indica che lo svolgimento del progetto si mantiene al di sotto del budget;
	\item CV = 0 indica che il progetto è al pari con il budget;
	\item CV $<$ 0 indica che il progetto è al di sopra del budget a disposizione, si devono correggere i metodi di lavoro.
\end{itemize}
	\chapter{Qualità del prodotto} \label{QualitàDelProdotto}

Per valutare la qualità del prodotto il gruppo \textit{JawaDruids} ha stabilito di usare come riferimento lo standard ISO/IEC 9126$_G$, che definisce le caratteristiche, descritte attraverso dei parametri,  da considerare affinché il prodotto finale sia di buona qualità. 
Per un approfondimento sullo standard si rimanda alla lettura del paragrafo §5 delle \textit{Norme di Progetto v1.0.0}. 
Si riportano di seguito i parametri dello standard ritenuti più interessanti dal gruppo, nel contesto del  progetto.
Le metriche qui riportate si limitano a quelle individuate fino alla stesura di tale documento, dunque l’elenco di queste sarà opportunamente ampliato in futuro, se necessario per l’aumento della completezza della valutazione della qualità.

\section{Funzionalità} \label{QualitàDelProdottoFunzionalità}
Si tratta della capacità del prodotto software di fornire le funzioni appropriate e necessarie per soddisfare i bisogni emersi nell’\textit{Analisi dei requisiti} e per operare in un determinato contesto. 
\subsection{Metriche} \label{QualitàDelProdottoFunzionalitàMetriche}
\subsubsection{MQPD01 Totalita dell’implementazione} \label{QualitàDelProdottoFunzionalitàMetricheMQPD01}
Indice riportante l’interezza del prodotto software, rispetto ai requisiti$_G$ posti, mediante un valore in percentuale:
\begin{center}
	\textbf{T=($1-\frac{RnI}{RI}$)*100}
\end{center}
Dove:
\begin{itemize}
	\item \textbf{T} sta per \textit{Totalità}, riferito ai requisiti$_{\scaleto{G}{3pt}}$ da implementare;
	\item \textbf{RnI} sta per \textit{Requisito non Implementato};
	\item \textbf{RI} sta per \textit{Requisito Implementato}.
\end{itemize}
I range accettabili per il risultato di \textbf{T} sono così suddivisi:
\begin{itemize}
	\item 90\% $<$ \textbf{T} $\leq$ 100\% indica che la copertura dei requisiti$_{\scaleto{G}{3pt}}$ proposti è quasi totale;
	\item 80\% $<$ \textbf{T} $\leq$ 90\% indica che la copertura dei requisiti$_{\scaleto{G}{3pt}}$ proposti è sufficiente, buona;
	\item \textbf{T} $\leq$ 80\% indica che la copertura dei requisiti$_{\scaleto{G}{3pt}}$ proposti è insufficiente;
	\item \textbf{Valore Preferibile:} 100\%;
	\item \textbf{Valore  Accettabile:} $\geq$ 90\%. 
\end{itemize}

\section{Affidabilità} \label{QualitàDelProdottoAffidabilità}
Si tratta della capacità del prodotto software di mantenere il livello di prestazione elevato anche se usato in condizioni specifiche, che possono essere anomale o critiche. 
\subsection{Metriche} \label{QualitàDelProdottoAffidabilitàMetriche}
\subsubsection{MQPD03 Rilevamento Errori} \label{QualitàDelProdottoAffidabilitàMetricheMQPD03}
Indice che mostra qual’è la percentuale di errore basata sui test fatti. Come formula viene usata la seguente:
\begin{center}
	\textbf{RE = ($1-\frac{TE}{TT}$)*100}
\end{center}
\begin{itemize}
	\item \textbf{RE} sta per \textit{Rilevamento Errori};
	\item \textbf{TE} sta per \textit{Test con Errori};
	\item \textbf{TT} sta per \textit{Test Totali}.
\end{itemize}
\textit{Il gruppo JawaDruids ha valutato precoce la scelta di stabilire in questa prima fase dei valori soglia per tale metrica, di conseguenza il gruppo si riserva di integrarli successivamente in modo opportuno.}

\section{Usabilità}\label{QualitàDelProdottoUsabilità}
Si tratta della capacità del prodotto software di essere di facile comprensione e utilizzo da parte dell’utente, sotto determinate condizioni. 
\subsection{Metriche} \label{QualitàDelProdottoUsabilitàMetriche}
\subsubsection{MQPD04 Validità dei dati in input} \label{QualitàDelProdottoUsabilitàMetricheMQPD04}
Questo indice misura la veridicità dei dati che arrivano in input al software. Ovviamente più i dati si avvicinano alla realtà più elevato sarà il valore dell’indice.
Viene usata la seguente formula:
\begin{center}
	\textbf{VD = $\frac{DIV}{DP}$ *100}
\end{center}
\begin{itemize}
	\item \textbf{VD} sta per \textit{Validità Dati};
	\item \textbf{DIV} sta per \textit{Dati Input Validati};
	\item \textbf{DP} sta per \textit{Dati Previsti}.
\end{itemize}

\textit{I range di valori accettabili non si possono ancora esprimere in quanto, concordi con l'azienda, si stabiliranno in futuro.}
\subsubsection{Indice di Gulpease} \label{QualitàDelProdottoUsabilitàMetricheIndiceDiGulpease}
L’indice di Gulpease$_G$ riporta il grado di leggibilità di un testo redatto in lingua italiana.
La formula adottata è:
\begin{center}
	\textbf{GULP= 89+ $\frac{300*(numero frasi)-10*(numero parole)}{numero lettere}$}
\end{center}
L'indice così calcolato può pertanto assumere valori compresi tra 0 e 100, in cui:
\begin{itemize}
	\item \textbf{GULP$<$ 80:} indica una leggibilità difficile per un utente con licenza elementare;
	\item \textbf{GULP$<$ 60:} indica una leggibilità difficile per un utente con licenza media;
	\item \textbf{GULP$<$ 40:} indica una leggibilità difficile per un utente con licenza superiore;
	\item \textbf{Valore Preferibile:} $>$ 80;
	\item \textbf{Valore  Accettabile:} $>$ 60.
\end{itemize}
\subsubsection{Errori Ortografici} \label{QualitàDelProdottoUsabilitàMetricheErroriOrtografici}
La correttezza ortografica della lingua italiana è verificata attraverso l’apposito strumento integrato in di TexStudio, il quale sottolinea in tempo reale le parole ove ritiene sia presente un errore, consentendone la correzione.
\begin{itemize}
	\item \textbf{Valore Preferibile:} 0 errori;
	\item \textbf{Valore  Accettabile:} 0 errori.
\end{itemize}

\section{Efficienza} \label{QualitàDelProdottoEfficienza}
Si tratta della capacità di un prodotto software di realizzare le funzioni richieste nel minor tempo possibile e sfruttando al meglio le risorse necessarie, quando opera in determinate condizioni. 
\subsection{Valutazione sulla Caratteristica} \label{QualitàDelProdottoEfficienzaValutazioneSullaCaratteristica}
\textit{I membri del gruppo non hanno valutato opportuno stabilire già delle metriche di qualità riguardo questa sezione in quanto il proponente$_G$ non ha ancora espresso requisiti$_{\scaleto{G}{3pt}}$ in termini di efficienza. Se ritenuto necessario, successivamente, dopo una conoscenza più approfondita della gestione delle risorse e dell’ambiente di rilascio del prodotto software, il gruppo si preoccuperà di integrare efficacemente la suddetta sezione.}

\section{Portabilità} \label{QualitàDelProdottoPortabilità}
La portabilità è definita come la capacità di un software nell’essere “trasportato” da un ambiente di lavoro, inteso sia come organizzativo che tecnologico,  ad un altro.

\subsection{Valutazione sulla Caratteristica} \label{QualitàDelProdottoPortabilitàValutazioneSullaCaratteristica}
\textit{Il gruppo, dopo aver preso visione degli obbietti riguardanti questa caratteristica, si è soffermato sulla }\textbf{Adattabilità} \textit{.
Questo perché il prodotto software, essendo una web-app, dovrà essere capace di funzionare su qualsiasi piattaforma internet senza problemi.
Se sarà necessario il paragrafo verrà adeguatamente integrato con gli altri obbiettivi presi in considerazione.}

\section{Manutenibilità} \label{QualitàDelProdottoManutenibilità}
E' la capacità di un prodotto software di essere modificato. Le modifiche possono includere correzioni, adattamenti o miglioramenti del software.
\subsection{Metriche}  \label{QualitàDelProdottoManutenibilitàMetriche}
\subsubsection{MQPD05 Comprensione del codice}\label{QualitàDelProdottoManutenibilitàMetricheMQPD05}
Con questa metriche si intende calcolare un indice, in percentuale, riferito alla facilità della comprensione del codice da parte dell'utente.
La formula utilizzata è la seguente:
\begin{center}
	\textbf{F=$\frac{N_{Lc}}{N_{Lcod}}$*100}
\end{center}
Dove:
\begin{itemize}
	\item \textbf{F}: è l'indice di facilità di comprensione;
	\item \textbf{N\_{Lc}}: indica il numero di linee di commento presenti all'interno del codice;
	\item \textbf{N\_{Lcod}}: indica il numero di linee di codice presente.
\end{itemize}

\textit{Non avendo ancora iniziato l'attività$_{\scaleto{G}{3pt}}$ di codifica, il gruppo si riserva di porre range di valori ottimali ed accettabili in un futuro momento.}
	\chapter{Specifica dei test} \label{SpecificaDeiTest}

Per assicurare un’ottima qualità del software prodotto, il gruppo \textit{Jawa Druids}, dopo essersi confrontato, ha deciso di utilizzare come modello di sviluppo software il \textit{V-Model}, o Modello a V, il quale è un’estensione del modello a cascata.
Questo modello prevede un lavoro parallelo tra lo sviluppo dei test e le attività$_{\scaleto{G}{3pt}}$ di analisi e progettazione.
Grazie a ciò, i test permettono di verificare sia il corretto funzionamento delle parti di software programmate, sia la corretta implementazione di tutti i requisiti$_{\scaleto{G}{3pt}}$ del progetto.
Vengono utilizzate delle sigle, all’interno di tabelle, per fornire una comprensione più agevolata riguardo gli output prodotti tramite i test, specificando se il risultato è quello atteso, errato o non coerente con quanto aspettato.
Le sigle per lo stato dei test sono:
\begin{itemize}
	\item \textbf{NI}: non implementato;
	\item \textbf{I}: implementato.
\end{itemize}
Per quanto riguarda la qualità dei test si usa:
\begin{itemize}
	\item \textbf{NS}: il test non ha soddisfatto la richiesta;
	\item \textbf{S}: il test ha soddisfatto la richiesta.
\end{itemize}

I test di \textit{Sistema} e \textit{Accettazione} hanno la seguente nomenclatura:

\begin{center}
	\textbf{[TipoTest]RS[classificazione][tipo\_di\_requisito][codice\_requisito]}
\end{center}
dove:
\begin{itemize}
	\item \textbf{TipoTest}: specifica il tipo di test applicato;
	\item \textbf{classificazione}:
	\begin{itemize}
		\item[-] \textbf{F}: indica se il requisito è funzionale;
		\item[-] \textbf{Q}: indica se il requisito è qualitativo;
		\item[-] \textbf{V}: indica se il requisito è vincolante;
		\item[-] \textbf{P}: indica se il requisito è prestazionale.
	\end{itemize}
	\item \textbf{tipo\_di\_requisito}: assume i seguenti valori:
		\begin{itemize}
			\item[-] \textbf{O} per i requisiti$_{\scaleto{G}{3pt}}$ obbligatori;
			\item[-] \textbf{D} per i requisiti$_{\scaleto{G}{3pt}}$ desiderabili;
			\item[-] \textbf{F} per i requisiti$_{\scaleto{G}{3pt}}$ facoltativi.
		\end{itemize}
	\item \textbf{codice\_requisito}: un numero incrementale per rendere univoco il requisito.
\end{itemize}

Invece i test di \textit{Unità}, \textit{Integrazione} e \textit{Regressione} sono denominati nel seguente modo:
\begin{center}
	\textbf{[TipoTest][Id]}
\end{center}
dove:
\begin{itemize}
	\item \textbf{Id} rappresenta un numero incrementale che inizia da 1.
\end{itemize}

\section{Tipi di test} \label{SpecificaDeiTestTipiDiTest}
I test che verranno effettuati sul prodotto software sono così divisi:
\begin{itemize}
	\item \textbf{Test di unità}: i test di unità servono per verificare le singole unità del software, ovvero le componenti con funzionamento autonomo.
	Il superamento di tali test non implica il corretto funzionamento del software.
	Viene contrassegnata da \textbf{[TU]}.

	\item \textbf{Test di integrazione}: questa tipologia di test verifica i singoli moduli del software come fossero un gruppo unico.
	Vengono svolti successivamente ai \textit{TU} e prima dei \textit{TS}.
	Sono contrassegnati da \textbf{[TI]};

	\item \textbf{Test di sistema}: i test di sistema vengono eseguiti per verificare che i requisiti$_{\scaleto{G}{3pt}}$, scritti nel documento \textit{Analisi dei Requisiti}, siano stati implementati e funzionanti.
	Viene rappresentato mediante la sigla \textbf{[TS]};

	\item \textbf{Test di accettazione}: i test di accettazione hanno come scopo la verifica che il software sviluppato soddisfi i requisiti$_{\scaleto{G}{3pt}}$ presenti nel capitolato d’appalto$_G$ e concordati col proponente$_{\scaleto{G}{3pt}}$.
	Questi saranno eseguiti durante il collaudo finale del prodotto software sotto l'osservazione sia dell'azienda proponente$_{\scaleto{G}{3pt}}$ sia del gruppo di lavoro.
	Rappresentati mediante la sigla \textbf{[TA]};

	\item \textbf{Test di regressione}: Servono a garantire il corretto funzionamento del prodotto a seguito di modifiche del codice o di inserimento di nuove funzionalità.
	Vengono etichettati nel seguente modo \textbf{[TR]};

\end{itemize}

\clearpage
\section{Test di unità}\label{SpecificaDeiTestTestDiUnita}
Sono stati individuati i seguenti test di unità per garantire il funzionamento di ogni minimo componente autonomo del sistema.

\begin{center}
	\renewcommand{\arraystretch}{1.4}
	\begin{longtable}{|p{3cm}|p{9cm}|p{2cm}|p{2cm}|}
		\hline
		\rowcolor{airforceblue}
		\multicolumn{4}{|c|}{\textbf{Test di unità Python}} \\
		\hline
		\rowcolor{airforceblue}
		\makecell[c]{\textbf{Id Test}} & \makecell[c]{\textbf{Descrizione}} & \makecell[c]{\textbf{Esito}} & \makecell[c]{\textbf{Qualità}} \\
		\hline
		\centering \textit{TU01} & Verifica che il metodo get{\_}onecall{\_}api{\_}response in weather{\_}forecast.py ottenga i dati delle 48 ore successive & \makecell[tc]{\textit{I}} & \makecell[tc]{\textit{S}} \\
		\hline
		\centering \textit{TU02} & Verifica che il metodo fetch{\_}read{\_}m3u8 in detect.py scarichi e legga correttamente un file m3u8& \makecell[tc]{\textit{I}} & \makecell[tc]{\textit{S}}\\
		\hline
		\centering \textit{TU03} &  Verifica che il metodo extract{\_}frame{\_}from{\_}video{\_}url in get{\_}frames.py estragga e legga correttamente un frame da un video di cui viene fornito il link &\makecell[tc]{\textit{I}} & \makecell[tc]{\textit{S}}\\
		\hline
		\rowcolor{white}
		\caption{\textbf{Elenco test di unità - Python}}
	\end{longtable}
\end{center}

\begin{center}
	\renewcommand{\arraystretch}{1.4}
	\begin{longtable}{|p{3cm}|p{9cm}|p{2cm}|p{2cm}|}
		\hline
		\rowcolor{airforceblue}
		\multicolumn{4}{|c|}{\textbf{Test di unità Spring}} \\
		\hline
		\rowcolor{airforceblue}
		\makecell[c]{\textbf{Id Test}} & \makecell[c]{\textbf{Descrizione}} & \makecell[c]{\textbf{Esito}} & \makecell[c]{\textbf{Qualità}} \\
		\hline
		\centering \textit{TU04} & Verifica che il metodo in cui viene eseguita la query per il recupero della lista delle città presenti nel database ritorni effettivamente il tipo di ritorno corretto, ovvero List$<$String$>$ & \makecell[tc]{\textit{I}} & \makecell[tc]{\textit{S}} \\
		\hline
		\centering \textit{TU05} & Verifica che il metodo in cui viene eseguita la query per il recupero del'ultimo dato presente nel database ritorni effettivamente il tipo di ritorno corretto, ovvero un Detection & \makecell[tc]{\textit{I}} & \makecell[tc]{\textit{S}}\\
		\hline
		\centering \textit{TU06} &  Verifica che il metodo in cui viene eseguita la query per il recupero dei dati presente nel database in base alla città, alla data e all'orario, ritorni effettivamente il tipo di ritorno corretto, ovvero un List$<$Detection$>$ &\makecell[tc]{\textit{I}} & \makecell[tc]{\textit{S}}\\
		\hline
		\centering \textit{TU07} &  Verifica che il metodo in cui viene eseguita la query per il recupero della latitudine e/o longitudine relativa a una città dal database ritorni effettivamente il tipo di ritorno corretto, ovvero List$<$String$>$.
		&\makecell[tc]{\textit{I}} & \makecell[tc]{\textit{S}}\\
		\hline
		\centering \textit{TU08} &  Verifica che il metodo in cui viene eseguita la query per il recupero della città in base all'id{\_}webcam dal database ritorni effettivamente il tipo di ritorno corretto, ovvero List$<$String$>$.
		&\makecell[tc]{\textit{I}} & \makecell[tc]{\textit{S}}\\
		\hline
		\centering \textit{TU09} &  Verifica che il metodo in cui viene eseguita la query per il recupero di tutti i dati storici di una specifica città dal database ritorni effettivamente il tipo di ritorno corretto, ovvero List$<$Detection$>$.
		&\makecell[tc]{\textit{I}} & \makecell[tc]{\textit{S}}\\
		\hline
		\rowcolor{white}
		\caption{\textbf{Elenco test di unità - Spring}}
	\end{longtable}
\end{center}

\begin{center}
	\renewcommand{\arraystretch}{1.4}
	\begin{longtable}{|p{3cm}|p{9cm}|p{2cm}|p{2cm}|}
		\hline
		\rowcolor{airforceblue}
		\multicolumn{4}{|c|}{\textbf{Test di unità Vue.js}} \\
		\hline
		\rowcolor{airforceblue}
		\makecell[c]{\textbf{Id Test}} & \makecell[c]{\textbf{Descrizione}} & \makecell[c]{\textbf{Esito}} & \makecell[c]{\textbf{Qualità}} \\
		\hline
		\centering \textit{TU10} & Verifica che la componente "aboutUs.vue" esista & \makecell[tc]{\textit{I}} & \makecell[tc]{\textit{S}} \\
		\hline
		\centering \textit{TU11} & Verifica che il file "httpcommon.js" esista & \makecell[tc]{\textit{I}} & \makecell[tc]{\textit{S}} \\
		\hline
		\centering \textit{TU12} & Verifica che il metodo "getCities" presente nel file "httprequest.js" esista & \makecell[tc]{\textit{I}} & \makecell[tc]{\textit{S}} \\
		\hline
		\centering \textit{TU13} & Verifica che il metodo "getCoo" presente nel file "httprequest.js" esista & \makecell[tc]{\textit{I}} & \makecell[tc]{\textit{S}} \\
		\hline
		\centering \textit{TU14} & Verifica che il metodo "getDataRT" presente nel file "httprequest.js" esista & \makecell[tc]{\textit{I}} & \makecell[tc]{\textit{S}} \\
		\hline
		\centering \textit{TU15} & Verifica che il metodo "getLastValue" presente nel file "httprequest.js" esista & \makecell[tc]{\textit{I}} & \makecell[tc]{\textit{S}} \\
		\hline
		\centering \textit{TU16} & Verifica che la componente "autocompleteSearch.vue" esista & \makecell[tc]{\textit{I}} & \makecell[tc]{\textit{S}} \\
		\hline
		\centering \textit{TU17} & Verifica che il tag di input della componente "autocompleteSearch.vue" sia vuoto & \makecell[tc]{\textit{I}} & \makecell[tc]{\textit{S}} \\
		\hline
		\centering \textit{TU18} & Verifica che il tag di input della componente "autocompleteSearch.vue" non sia vuoto dopo che l'utente ha interagito scrivendo & \makecell[tc]{\textit{I}} & \makecell[tc]{\textit{S}} \\
		\hline
		\centering \textit{TU19} & Verifica che il metodo "getNameCity" presente nel file "autocompleteSearch.vue" esista & \makecell[tc]{\textit{I}} & \makecell[tc]{\textit{S}} \\
		\hline
		\centering \textit{TU20} & Verifica che il metodo "getNameCity" presente nel file "autocompleteSearch.vue" venga chiamato almeno una volta & \makecell[tc]{\textit{I}} & \makecell[tc]{\textit{S}} \\
		\hline
		\centering \textit{TU21} & Verifica che il metodo "itemSelected" presente nel file "autocompleteSearch.vue" esista & \makecell[tc]{\textit{I}} & \makecell[tc]{\textit{S}} \\
		\hline
		\centering \textit{TU22} & Verifica che la componente "datePicker.vue" esista & \makecell[tc]{\textit{I}} & \makecell[tc]{\textit{S}} \\
		\hline
		\centering \textit{TU23} & Verifica che la componente "listCity.vue" esista & \makecell[tc]{\textit{I}} & \makecell[tc]{\textit{S}} \\
		\hline
		\centering \textit{TU24} & Verifica che la componente "mainPage.vue" esista & \makecell[tc]{\textit{I}} & \makecell[tc]{\textit{S}} \\
		\hline
		\centering \textit{TU25} & Verifica che la componente "heatmap" sia presente nel file "mainPage.vue" & \makecell[tc]{\textit{I}} & \makecell[tc]{\textit{S}} \\
		\hline
		\centering \textit{TU26} & Verifica che la componente "confrontoCittà.vue" esista & \makecell[tc]{\textit{I}} & \makecell[tc]{\textit{S}} \\
		\hline
		\rowcolor{white}
		\caption{\textbf{Elenco test di unità - Vue.js}}
	\end{longtable}
\end{center}


\subsection{Tracciamento dei test}\label{SpecificaDeiTestTestDiUnitaTracciamentoDeiTest}

\begin{center}
	\renewcommand{\arraystretch}{1.4}
	\begin{longtable}{|p{1.5cm}|p{9cm}|p{6cm}|}
		\hline
		\rowcolor{airforceblue}
		\multicolumn{3}{|c|}{\textbf{Tracciamento test di unità Python}} \\
		\hline
		\rowcolor{airforceblue}
		\makecell[c]{\textbf{Id Test}} & \makecell[c]{\textbf{Percorso file}} & \makecell[c]{\textbf{Metodo}} \\
		\hline
		\centering TU01	& \makecell[c]{acquisition/main/test/test{\_}weather{\_}forecast.py} & \makecell[c]{test{\_}response}\\
		\hline
		\centering TU02 & \makecell[c]{acquisition/main/test/test{\_}detect.py} & \makecell[c]{test{\_}fetch{\_}read{\_}m3u8}\\
		\hline
		\centering TU03 & \makecell[c]{acquisition/main/test/test{\_}detect.py} & \makecell[c]{test{\_}extract{\_}frame{\_}from{\_}video{\_}url}\\
		\hline
		\rowcolor{white}
		\caption{\textbf{Tracciamento dei test di unità - Python}}
	\end{longtable}
\end{center}

\begin{center}
	\renewcommand{\arraystretch}{1.4}
	\begin{longtable}{|p{1.5cm}|p{11.5cm}|p{3.5cm}|}
		\hline
		\rowcolor{airforceblue}
		\multicolumn{3}{|c|}{\textbf{Tracciamento test di unità Spring}} \\
		\hline
		\rowcolor{airforceblue}
		\makecell[c]{\textbf{Id Test}} & \makecell[c]{\textbf{Percorso file}} & \makecell[c]{\textbf{Metodo}} \\
		\hline
		\centering TU04	& \makecell[c]{proof{\_}of{\_}concept/webapp/webapp/src/test/java/com/webapp/\\data/mongodb/DetectionCustomRepositoryImplTest/\\DetectionCustomRepositoryImplTest.java} & \makecell[c]{getCitiesTest()}\\
		\hline
		\centering TU05 & \makecell[c]{proof{\_}of{\_}concept/webapp/webapp/src/test/java/com/webapp/\\data/mongodb/DetectionCustomRepositoryImplTest/\\DetectionCustomRepositoryImplTest.java} & \makecell[c]{getLastValueTest()}\\
		\hline
		\centering TU06 & \makecell[c]{proof{\_}of{\_}concept/webapp/webapp/src/test/java/com/webapp/\\data/mongodb/DetectionCustomRepositoryImplTest/\\DetectionCustomRepositoryImplTest.java} & \makecell[c]{getDataRTTest()}\\
		\hline
		\centering TU07 & \makecell[c]{proof{\_}of{\_}concept/webapp/webapp/src/test/java/com/webapp/\\data/mongodb/DetectionCustomRepositoryImplTest/\\DetectionCustomRepositoryImplTest.java} & \makecell[c]{getLatLngsTest()}\\
		\hline
		\centering TU08 & \makecell[c]{proof{\_}of{\_}concept/webapp/webapp/src/test/java/com/webapp/\\data/mongodb/DetectionCustomRepositoryImplTest/\\DetectionCustomRepositoryImplTest.java} & \makecell[c]{getCityByIdTest()}\\
		\hline
		\centering TU09 & \makecell[c]{proof{\_}of{\_}concept/webapp/webapp/src/test/java/com/webapp/\\data/mongodb/DetectionCustomRepositoryImplTest/\\DetectionCustomRepositoryImplTest.java} & \makecell[c]{getAllValueTest()}\\
		\hline
		\rowcolor{white}
		\caption{\textbf{Tracciamento dei test di unità - Spring}}
	\end{longtable}
\end{center}

\begin{center}
	\renewcommand{\arraystretch}{1.4}
	\begin{longtable}{|p{1.5cm}|p{11.5cm}|p{3.5cm}|}
		\hline
		\rowcolor{airforceblue}
		\multicolumn{3}{|c|}{\textbf{Tracciamento test di unità Vue.js}} \\
		\hline
		\rowcolor{airforceblue}
		\makecell[c]{\textbf{Id Test}} & \makecell[c]{\textbf{Percorso file}} & \makecell[c]{\textbf{Metodo}} \\
		\hline
		\centering TU10 & \makecell[c]{proof{\_}of{\_}concept/webapp/vue-js-client-crud/src/tests/unit/\\appTest.spec.js} &
		aboutUs:check if exist\\
		\hline
		\centering TU11 & \makecell[c]{proof{\_}of{\_}concept/webapp/vue-js-client-crud/src/tests/unit/\\appTest.spec.js} & httpcommon:check if exist\\
		\hline
		\centering TU12 & \makecell[c]{proof{\_}of{\_}concept/webapp/vue-js-client-crud/src/tests/unit/\\appTest.spec.js} & httpRequest:check if getCities work\\
		\hline
		\centering TU13 & \makecell[c]{proof{\_}of{\_}concept/webapp/vue-js-client-crud/src/tests/unit/\\appTest.spec.js} & httpRequest:check if getCoo work\\
		\hline
		\centering TU14 & \makecell[c]{proof{\_}of{\_}concept/webapp/vue-js-client-crud/src/tests/unit/\\appTest.spec.js} & httpRequest:check if getDataRT work\\
		\hline
		\centering TU15 & \makecell[c]{proof{\_}of{\_}concept/webapp/vue-js-client-crud/src/tests/unit/\\appTest.spec.js} & httpRequest:check if getLastValue work\\
		\hline
		\centering TU16 & \makecell[c]{proof{\_}of{\_}concept/webapp/vue-js-client-crud/src/tests/unit/\\appTest.spec.js} & {autocompleteSearch: check if exist}\\
		\hline
		\centering TU17 & \makecell[c]{proof{\_}of{\_}concept/webapp/vue-js-client-crud/src/tests/unit/\\appTest.spec.js} & {autocompleteSearch: check if search bar is empty}\\
		\hline
		\centering TU18 & \makecell[c]{proof{\_}of{\_}concept/webapp/vue-js-client-crud/src/tests/unit/\\appTest.spec.js} & {autocompleteSearch: user should have written something}\\
		\hline
		\centering TU19 & \makecell[c]{proof{\_}of{\_}concept/webapp/vue-js-client-crud/src/tests/unit/\\appTest.spec.js} & {autocompleteSearch: check if the get work}\\
		\hline
		\centering TU20 & \makecell[c]{proof{\_}of{\_}concept/webapp/vue-js-client-crud/src/tests/unit/\\appTest.spec.js} & {autocompleteSearch: check if the get is called}\\
		\hline
		\centering TU21 & \makecell[c]{proof{\_}of{\_}concept/webapp/vue-js-client-crud/src/tests/unit/\\appTest.spec.js} & {autocompleteSearch: check if the itemSelected work}\\
		\hline
		\centering TU22 & \makecell[c]{proof{\_}of{\_}concept/webapp/vue-js-client-crud/src/tests/unit/\\appTest.spec.js} & {datePicker:check if exist}\\
		\hline
		\centering TU23 & \makecell[c]{proof{\_}of{\_}concept/webapp/vue-js-client-crud/src/tests/unit/\\appTest.spec.js} & {listCity:check if exist}\\		
		\hline
		\centering TU24 & \makecell[c]{proof{\_}of{\_}concept/webapp/vue-js-client-crud/src/tests/unit/\\appTest.spec.js} & {mainPage:check if exist}\\
		\hline
		\centering TU25 & \makecell[c]{proof{\_}of{\_}concept/webapp/vue-js-client-crud/src/tests/unit/\\appTest.spec.js} & {mainPage:check if component heatmap exixst}\\
		\hline
		\centering TU26 & \makecell[c]{proof{\_}of{\_}concept/webapp/vue-js-client-crud/src/tests/unit/\\appTest.spec.js} & {confrontoCittà: check if exixst }\\
		\hline
		\rowcolor{white}
		\caption{\textbf{Tracciamento dei test di unità - Vue.js}}
	\end{longtable}
\end{center}

\section{Test di integrazione}\label{SpecificaDeiTestTestDiIntegrazione}
Sono stati individuati i seguenti test di integrazione per garantire il funzionamento dei diversi moduli nel momento in cui vengono messi in relazione tra di loro.
\def\tabularxcolumn#1{m{#1}}
{\rowcolors{2}{RawSienna!90!RawSienna!20}{RawSienna!70!RawSienna!40}

	\begin{center}
		\renewcommand{\arraystretch}{1.4}
		\begin{longtable}{|p{3cm}|p{8cm}|p{2cm}|p{2cm}|}
			\hline
			\rowcolor{airforceblue}
			\makecell[c]{\textbf{Id Test}} & \makecell[c]{\textbf{Descrizione}} & \makecell[c]{\textbf{Esito}} & \makecell[c]{\textbf{Qualità}} \\
			\hline
			\centering \textit{TI01} & Verifica che il sistema acquisica i dati attraverso il modulo contapersone che utlizza l'algoritmo di Yolo v3 & \makecell[tc]{\textit{I}} & \makecell[tc]{\textit{S}} \\
			\hline
			\centering \textit{TI02} & Verifica che il sistema invii i dati ottenuti dal modulo contapersone al database in modo corretto & \makecell[tc]{\textit{I}} & \makecell[tc]{\textit{S}}\\
			\hline
			\centering \textit{TI03} &  Verifica che il sistema invii i dati ottenuti dal modulo di machine learning$_{\scaleto{G}{3pt}}$ al database in modo corretto &\makecell[tc]{\textit{I}} & \makecell[tc]{\textit{S}}\\
			\hline
			\centering \textit{TI04} &  Verifica che il sistema invii i dati presenti nel  database alla web application in modo corretto &\makecell[tc]{\textit{I}} & \makecell[tc]{\textit{S}}\\
			\hline
			\rowcolor{white}
			\caption{\textbf{Elenco test di integrazione}}
		\end{longtable}
	\end{center}

\section{Test di sistema}\label{SpecificaDeiTestTestDiSistema}
Sono stati individuati i seguenti test di sistema per garantire il funzionamento del prodotto sviluppato. I test di sistema sono stati identificati attraverso i requisiti indicati nel documento \textit{Analisi dei Requisiti 3.0.0}.
\def\tabularxcolumn#1{m{#1}}
{\rowcolors{2}{RawSienna!90!RawSienna!20}{RawSienna!70!RawSienna!40}

	\begin{center}
		\renewcommand{\arraystretch}{1.4}
		\begin{longtable}{|p{3cm}|p{8cm}|p{2cm}|p{2cm}|}
			\hline
			\rowcolor{airforceblue}
			\makecell[c]{\textbf{Id Test}} & \makecell[c]{\textbf{Descrizione}} & \makecell[c]{\textbf{Esito}} & \makecell[c]{\textbf{Qualità}} \\
			\hline
			\textit{TSRSFO1} & Verifica che il sistema utilizzi motori software 'contapersone' & \makecell[tc]{\textit{I}} & \makecell[tc]{\textit{S}} \\
			\hline
			\textit{TSRSFF2} & Verifica che il sistema utilizzi simulatori di dati storici & \makecell[tc]{\textit{NI}} & \makecell[tc]{\textit{-}}\\
			\hline
			\textit{TSRSFO3} & Verifica della visualizzazione di un messaggio d'errore in caso mancanza dati nella generazione della heat-map$_G$ &\makecell[tc]{\textit{I}} & \makecell[tc]{\textit{S}}\\
			\hline
			\textit{TSRSFO4} & Verifica che il sistema archivi tutti i dati nel database & \makecell[tc]{\textit{I}} & \makecell[tc]{\textit{S}}\\
			\hline
			\textit{TSRSFO5} & Verifica che il sistema elabori i dati dalle sorgenti esterne in tempo reale & \makecell[tc]{\textit{I}} & \makecell[tc]{\textit{S}}\\
			\hline
			\textit{TSRSFD6} & Verifica che il sistema effettui una previsione dell’insorgenza futura di variazioni significative di flussi di persone & \makecell[tc]{\textit{NI}} & \makecell[tc]{\textit{-}}\\
			\hline
			\textit{TSRSFO7} & Verifica che l’utente possa poter visualizzare i dati tramite heat map$_{\scaleto{G}{3pt}}$ & \makecell[tc]{\textit{I}} & \makecell[tc]{\textit{S}}\\
			\hline
			\textit{TSRSFO8} & Verifica che Apache kafka crei una comunicazione tra il programma con il software contapersone e il database & \makecell[tc]{\textit{I}} & \makecell[tc]{\textit{S}}\\
			\hline
			\textit{TSRSFO9} & Verifica che l’utente possa poter visualizzare i dati in tempo reale tramite heat map$_{\scaleto{G}{3pt}}$ & \makecell[tc]{\textit{I}} & \makecell[tc]{\textit{S}}\\
			\hline
			\textit{TSRSFO10} & Verifica che l’utente possa poter visualizzare i dati storicizzati tramite heat map$_{\scaleto{G}{3pt}}$ & \makecell[tc]{\textit{I}} & \makecell[tc]{\textit{S}}\\
			\hline
			\textit{TSRSFO11} & Verifica che l’utente possa poter visualizzare una previsione tramite heat map$_{\scaleto{G}{3pt}}$ & \makecell[tc]{\textit{I}} & \makecell[tc]{\textit{S}}\\
			\hline
			\textit{TSRSFF12} & Verifica che l'utente possa poter distinguere tra i dati simulati e quelli reali & \makecell[tc]{\textit{I}} & \makecell[tc]{\textit{S}}\\
			\hline
			\textit{TSRSFD13} & Verfica che l’utente possa poter visualizzare un indice di affidabilità della previsione nella mappa & \makecell[tc]{\textit{NI}} & \makecell[tc]{\textit{-}}\\
			\hline
			\textit{TSRSFD14} & Verifica che l’utente possa poter visualizzare un indice di affidabilità dei dati in tempo reale nella mappa & \makecell[tc]{\textit{NI}} & \makecell[tc]{\textit{-}}\\
			\hline
			\textit{TSRSFF15} & Verifica che l’utente possa poter applicare dei filtri ai dati (reali, simulati) & \makecell[tc]{\textit{NI}} & \makecell[tc]{\textit{-}}\\
			\hline
			\textit{TSRSFF16} & Verifica che l’utente abbia la possibilità di scegliere le sorgenti dati da cui prelevare dati tempo reale & \makecell[tc]{\textit{NI}} & \makecell[tc]{\textit{-}}\\
			\hline
			\textit{TSRSFO17} & Verifica che il sistema aggiorni la mappa automaticamente ogni 10 minuti & \makecell[tc]{\textit{I}} & \makecell[tc]{\textit{S}}\\
			\hline
			\textit{TSRSFO18} & Verifica che il modello di machine learning$_{\scaleto{G}{3pt}}$ salvi i pesi e le predizioni in un file & \makecell[tc]{\textit{I}} & \makecell[tc]{\textit{S}}\\
			\hline
			\textit{TSRSFO19} & Verifica che venga inviato un messaggio di errore al front end , dal backend, se non ci sono i dati richiesti & \makecell[tc]{\textit{I}} & \makecell[tc]{\textit{S}}\\
			\hline
			\textit{TSRSFO20} & Verifica che l’utente possa selezionare una città tra quelle disponibili & \makecell[tc]{\textit{I}} & \makecell[tc]{\textit{S}}\\
			\hline
			\textit{TSRSFO21} & Verifica che l'utente visualizzi le zone delle città rispettivamente alle zone utilizzate & \makecell[tc]{\textit{I}} & \makecell[tc]{\textit{S}}\\
			\hline
			\textit{TSRSFO22} & Verifica che il sistema archivi i dati in tempo reale con la data e orario di riferimento associata & \makecell[tc]{\textit{I}} & \makecell[tc]{\textit{S}}\\
			\hline
			\textit{TSRSFF23} & Verifica che il sistema utilizzi i dati delle predizioni in caso di mancanza di dati in tempo reale & \makecell[tc]{\textit{I}} & \makecell[tc]{\textit{S}}\\
			\hline
			\textit{TSRSFO24} & Verifica che l'utente possa selezionare l'intervallo orario in fasce orarie & \makecell[tc]{\textit{I}} & \makecell[tc]{\textit{S}}\\
			\hline
			\textit{TSRSFO25} & Verifica che il sistema utilizzi in modo prioritario i dati reali se presenti anche quelli determinati per le predizioni & \makecell[tc]{\textit{I}} & \makecell[tc]{\textit{S}}\\
			\hline
			\textit{TSRSFO26} & Verifica che il sistema aggiorni automaticamente la mappa alla selezione di un diverso orario & \makecell[tc]{\textit{I}} & \makecell[tc]{\textit{S}}\\
			\hline
			\textit{TSRSFO27} & Verifica che l’utente possa poter selezionare la data del giorno di cui vuole visualizzare i dati & \makecell[tc]{\textit{I}} & \makecell[tc]{\textit{S}}\\
			\hline
			\textit{TSRSFO28} & Verifica che l’utente possa poter ripristinare la visione in tempo reale tramite un pulsante di ripristino & \makecell[tc]{\textit{I}} & \makecell[tc]{\textit{S}}\\
			\hline
			\textit{TSRSFD29} & Verifica che il sistema prelevi dati da diverse fonti e le formatti nel tipo di default & \makecell[tc]{\textit{I}} & \makecell[tc]{\textit{S}}\\
			\hline
			\textit{TSRSFO30} & Verifica che il sistema utilizzi un software contapersone già allenato & \makecell[tc]{\textit{I}} & \makecell[tc]{\textit{S}}\\
			\hline
			\textit{TSRSFF31} & Verifica che l’utente possa poter reperire il manuale d'uso & \makecell[tc]{\textit{I}} & \makecell[tc]{\textit{S}}\\
			\hline
			\textit{TSRSFO32} & Verifica che l’utente possa poter variare il livello di zoom della heat map$_{\scaleto{G}{3pt}}$ & \makecell[tc]{\textit{I}} & \makecell[tc]{\textit{S}}\\
			\hline
			\textit{TSRSFO32.1} & Verifica che l’utente possa poter aumentare il livello di zoom della heat map$_{\scaleto{G}{3pt}}$ & \makecell[tc]{\textit{I}} & \makecell[tc]{\textit{S}}\\
			\hline
			\textit{TSRSFO32.1.1} & Verifica che l’utente possa poter attuare il drag$_{\scaleto{G}{3pt}}$ della heat map$_{\scaleto{G}{3pt}}$ & \makecell[tc]{\textit{I}} & \makecell[tc]{\textit{S}}\\
			\hline
			\textit{TSRSFO32.1.2} & Verifica che l’utente possa poter visualizzare il pop-up$_{\scaleto{G}{3pt}}$ legato ad un punto di interesse & \makecell[tc]{\textit{I}} & \makecell[tc]{\textit{S}}\\
			\hline
			\textit{TSRSFO32.1.3} & Verifica che l’utente possa poter chiudere il pop-up$_{\scaleto{G}{3pt}}$ legato ad un punto di interesse & \makecell[tc]{\textit{I}} & \makecell[tc]{\textit{S}}\\
			\hline
			\textit{TSRSFO32.2} & Verifica che l’utente possa poter diminuire il livello di zoom della heat map$_{\scaleto{G}{3pt}}$ & \makecell[tc]{\textit{I}} & \makecell[tc]{\textit{S}}\\
			\hline
			\textit{TSRSFD33} & Verifica che l’utente possa poter ricercare tramite una barra di ricerca le città presenti nel database & \makecell[tc]{\textit{I}} & \makecell[tc]{\textit{S}}\\
			\hline
			\hline
			\textit{TSRSFD33.1} & Verifica che l’utente possa poter ricercare tramite una barra di ricerca per nome le città presenti nel database & \makecell[tc]{\textit{I}} & \makecell[tc]{\textit{S}}\\
			\hline
			\textit{TSRSFD33.2} & Verifica che l’utente possa poter ricercare tramite una barra di ricerca tramite codice identificativo le città presenti nel database & \makecell[tc]{\textit{I}} & \makecell[tc]{\textit{S}}\\
			\textit{TSRSFD34} & Verifica che l’utente possa poter visualizzare il messaggio d'errore relativo alla mancanza dei dati ricercati attraverso la barra di ricerca nel database & \makecell[tc]{\textit{I}} & \makecell[tc]{\textit{S}}\\
			\hline
			\textit{TSRSFD35} & Verifica che l’utente possa poter visualizzare il confronto dei dati di due città selezionate dall'utente & \makecell[tc]{\textit{I}} & \makecell[tc]{\textit{S}}\\
			\hline
			\textit{TSRSFD36} & Verifica che l’utente possa poter salvare in un file locale i dati della città della mappa che sta visualizzando & \makecell[tc]{\textit{I}} & \makecell[tc]{\textit{S}}\\
			\hline
			\textit{TSRSFD36.1} & Verifica che l’utente possa poter salvare in un file formato PDF locale i dati della città della mappa che sta visualizzando & \makecell[tc]{\textit{I}} & \makecell[tc]{\textit{S}}\\
			\hline
			\textit{TSRSFD36.2} & Verifica che l’utente possa poter salvare in un file formato cvs locale i dati della città della mappa che sta visualizzando & \makecell[tc]{\textit{I}} & \makecell[tc]{\textit{S}}\\
			\hline
			\textit{TSRSFD37} & Verifica che l’utente possa poter inserire l'email per il ricevimento delle informazioni delle informazioni della città selezionata & \makecell[tc]{\textit{NI}} & \makecell[tc]{\textit{-}}\\
			\hline
			\textit{TSRSFD37.1} & Verifica che l’utente possa poter visualizzare un messaggio di errore nel caso l'email inserita sia scritta in modo errato & \makecell[tc]{\textit{NI}} & \makecell[tc]{\textit{-}}\\
			\hline
			\textit{TSRSFD38} & Verifica che il sistema abbia inserito correttamente l'email e la città correlata nel database & \makecell[tc]{\textit{NI}} & \makecell[tc]{\textit{-}}\\
			\hline
			\textit{TSRSFD39} & Verifica che il sistema invii correttamente l'email all'utente & \makecell[tc]{\textit{NI}} & \makecell[tc]{\textit{-}}\\
			\hline
			\textit{TSRSFD40} & Verifica che l’utente possa poter visualizzare la lista delle città più ricercate & \makecell[tc]{\textit{NI}} & \makecell[tc]{\textit{-}}\\
			\hline
			\textit{TSRSFD41} & Verifica che l’utente possa poter visualizzare la lista di tutte le città presenti nel database & \makecell[tc]{\textit{I}} & \makecell[tc]{\textit{S}}\\
			\hline
			\rowcolor{white}
			\caption{\textbf{Elenco test di sistema}}\\
		\end{longtable}
	\end{center}
	\def\tabularxcolumn#1{m{#1}}
	{\rowcolors{2}{RawSienna!90!RawSienna!20}{RawSienna!70!RawSienna!40}

		\subsection{Tracciamento dei test}\label{SpecificaDeiTestTestDiSistemaTracciamentoDeiTest}

		\begin{center}
			\renewcommand{\arraystretch}{1.4}
			\begin{longtable}{|p{3cm}|p{3cm}|}
				\hline
				\rowcolor{airforceblue}
				\makecell[c]{\textbf{Id Test}} & \makecell[c]{\textbf{Id Requisito}} \\
				\hline
				\hline
				TSRSFO1	& \makecell{RSFO1} \\
				\hline
				TSRSFF2 & \makecell{RSFF2} \\
				\hline
				TSRSFO3 & \makecell{RSFO3}  \\
				\hline
				TSRSFO4 & \makecell{RSFO4 \\ RSFO4.1 \\ RSFO4.2} \\
				\hline
				TSRSFO5 & \makecell{RSFO5} \\
				\hline
				TSRSFD6 & \makecell{RSFD6} \\
				\hline
				TSRSFO7 & \makecell{RSFO7} \\
				\hline
				TSRSFO8 & \makecell{RSFO8} \\
				\hline
				TSRSFO9 & \makecell{RSFO9} \\
				\hline
				TSRSFO10 & \makecell{RSFO10} \\
				\hline
				TSRSFO11 & \makecell{RSFO11} \\
				\hline
				TSRSFF12 & \makecell{RSFF12} \\
				\hline
				TSRSFD13 & \makecell{RSFD13} \\
				\hline
				TSRSFD14 & \makecell{RSFD14} \\
				\hline
				TSRSFF15 & \makecell{RSFF15} \\
				\hline
				TSRSFF16 & \makecell{RSFF16}\\
				\hline
				TSRSFO17 & \makecell{RSFO17} \\
				\hline
				TSRSFO18 & \makecell{RSFO18 \\ RSFO18.1} \\
				\hline
				TSRSFO19 & \makecell{RSFO19} \\
				\hline
				TSRSFO20 & \makecell{RSFO20} \\
				\hline
				TSRSFO21 & \makecell{RSFO21} \\
				\hline
				TSRSFO22 & \makecell{RSFO22 \\ RSFO22.1 \\ RSFO22.2} \\
				\hline
				TSRSFF23 & \makecell{RSFF23} \\
				\hline
				TSRSFO24 & \makecell{RSFO24} \\
				\hline
				TSRSFO25 & \makecell{RSFO25} \\
				\hline
				TSRSFO26 & \makecell{RSFO26} \\
				\hline
				TSRSFO27 & \makecell{RSFO27} \\
				\hline
				TSRSFO28 & \makecell{RSFO28} \\
				\hline
				TSRSFD29 & \makecell{RSFD29}\\
				\hline
				TSRSFO30 & \makecell{RSFO30} \\
				\hline
				TSRSFF31 & \makecell{RSFF31} \\
				\hline
				TSRSFO32 & \makecell{RSFO32}\\
				\hline
				TSRSFO32.1 & \makecell{RSFO32.1} \\
				\hline
				TSRSFO32.1.1 & \makecell{RSFO32.1.1} \\
				\hline
				TSRSFO32.1.2 & \makecell{RSFO32.1.2} \\
				\hline
				TSRSFO32.1.3 & \makecell{RSFO32.1.3} \\
				\hline
				TSRSFO32.2 & \makecell{RSFO32.2}\\
				\hline
				TSRSFD33 & \makecell{RSFD33}\\
				\hline
				TSRSFD33.1 & \makecell{RSFD33.1}\\
				\hline
				TSRSFD33.2 & \makecell{RSFD33.2}\\
				\hline
				TSRSFD34 & \makecell{RSFD34}\\
				\hline
				TSRSFD35 & \makecell{RSFD35}\\
				\hline
				TSRSFD36 & \makecell{RSFD36} \\
				\hline
				TSRSFD36.1 & \makecell{RSFD36.1} \\
				\hline
				TSRSFD36.2 & \makecell{RSFD36.2} \\
				\hline
				TSRSFD37 & \makecell{RSFD37} \\
				\hline
				TSRSFD37.1 & \makecell{RSFD37.1} \\
				\hline
				TSRSFD38 & \makecell{RSFD38} \\
				\hline
				TSRSFD39 & \makecell{RSFD39} \\
				\hline
				TSRSFD40 & \makecell{RSFD40} \\
				\hline
				TSRSFD41 & \makecell{RSFD41} \\
				\hline
				\rowcolor{white}
				\caption{\textbf{Tracciamento dei test di sistema con i requisiti}}
			\end{longtable}

		\end{center}

\section{Test di accettazione}\label{SpecificaDeiTestTestDiAccettazione}
I test di accettazione hanno lo scopo di dimostrare che il prodotto software realizzato soddifisfi i requisiti minimi concordati con il proponente. \\
Questi test si compongono dei test di sistema e sono eseguiti durante il il collaudio finale sotto l'osservazione sia del gruppo \textit{JawaDruids} che dell'azienda proponente \textit{SyncLab}.

\def\tabularxcolumn#1{m{#1}}
{\rowcolors{2}{RawSienna!90!RawSienna!20}{RawSienna!70!RawSienna!40}

	\begin{center}
		\renewcommand{\arraystretch}{1.4}
		\begin{longtable}{|p{3cm}|p{8cm}|p{2cm}|p{2cm}|}
			\hline
			\rowcolor{airforceblue}
			\makecell[c]{\textbf{Id Test}} & \makecell[c]{\textbf{Descrizione}} & \makecell[c]{\textbf{Esito}} & \makecell[c]{\textbf{Qualità}} \\
			\hline
			\textit{TARSFO1} & Verifica che il sistema utilizzi motori software 'contapersone' & \makecell[tc]{\textit{I}} & \makecell[tc]{\textit{S}} \\
			\hline
			\textit{TARSFF2} & Verifica che il sistema utilizzi simulatori di dati storici & \makecell[tc]{\textit{NI}} & \makecell[tc]{\textit{-}}\\
			\hline
			\textit{TARSFO3} & Verifica della visualizzazione di un messaggio d'errore in caso mancanza dati nella generazione della heat-map$_G$ &\makecell[tc]{\textit{I}} & \makecell[tc]{\textit{S}}\\
			\hline
			\textit{TARSFO4} & Verifica che il sistema archivi tutti i dati nel database & \makecell[tc]{\textit{I}} & \makecell[tc]{\textit{S}}\\
			\hline
			\textit{TARSFO5} & Verifica che il sistema elabori i dati dalle sorgenti esterne in tempo reale & \makecell[tc]{\textit{I}} & \makecell[tc]{\textit{S}}\\
			\hline
			\textit{TARSFD6} & Verifica che il sistema effettui una previsione dell’insorgenza futura di variazioni significative di flussi di persone & \makecell[tc]{\textit{NI}} & \makecell[tc]{\textit{-}}\\
			\hline
			\textit{TARSFO7} & Verifica che l’utente possa poter visualizzare i dati tramite heat map$_{\scaleto{G}{3pt}}$ & \makecell[tc]{\textit{I}} & \makecell[tc]{\textit{S}}\\
			\hline
			\textit{TARSFO8} & Verifica che Apache kafka crei una comunicazione tra il programma con il software contapersone e il database & \makecell[tc]{\textit{I}} & \makecell[tc]{\textit{S}}\\
			\hline
			\textit{TARSFF9} & Verifica che l'utente possa poter distinguere tra i dati simulati e quelli reali & \makecell[tc]{\textit{I}} & \makecell[tc]{\textit{S}}\\
			\hline
			\textit{TARSFD1O} & Verfica che l’utente possa poter visualizzare un indice di affidabilità dei dati nella mappa & \makecell[tc]{\textit{NI}} & \makecell[tc]{\textit{-}}\\
			\hline
			\textit{TARSFF11} & Verifica che l’utente possa poter applicare dei filtri ai dati (reali, simulati) & \makecell[tc]{\textit{NI}} & \makecell[tc]{\textit{-}}\\
			\hline
			\textit{TARSFF12} & Verifica che l’utente abbia la possibilità di scegliere le sorgenti dati da cui prelevare dati tempo reale & \makecell[tc]{\textit{NI}} & \makecell[tc]{\textit{-}}\\
			\hline
			\textit{TARSFO13} & Verifica che il sistema aggiorni la mappa automaticamente ogni 10 minuti & \makecell[tc]{\textit{I}} & \makecell[tc]{\textit{S}}\\
			\hline
			\textit{TARSFO14} & Verifica che il modello di machine learning$_{\scaleto{G}{3pt}}$ salvi i pesi e le predizioni in un file & \makecell[tc]{\textit{I}} & \makecell[tc]{\textit{S}}\\
			\hline
			\textit{TARSFO15} & Verifica che venga inviato un messaggio di errore al front end , dal backend, se non ci sono i dati richiesti & \makecell[tc]{\textit{I}} & \makecell[tc]{\textit{S}}\\
			\hline
			\textit{TARSFO16} & Verifica che l’utente possa selezionare una città tra quelle disponibili & \makecell[tc]{\textit{I}} & \makecell[tc]{\textit{S}}\\
			\hline
			\textit{TARSFO17} & Verifica che l'utente visualizzi le zone delle città rispettivamente alle zone utilizzate & \makecell[tc]{\textit{I}} & \makecell[tc]{\textit{S}}\\
			\hline
			\textit{TARSFO18} & Verifica che il sistema archivi i dati in tempo reale con la data e orario di riferimento associata & \makecell[tc]{\textit{I}} & \makecell[tc]{\textit{S}}\\
			\hline
			\textit{TARSFF19} & Verifica che il sistema utilizzi i dati delle predizioni in caso di mancanza di dati in tempo reale & \makecell[tc]{\textit{I}} & \makecell[tc]{\textit{S}}\\
			\hline
			\textit{TARSFO20} & Verifica che l'utente possa selezionare l'intervallo orario in fasce orarie & \makecell[tc]{\textit{I}} & \makecell[tc]{\textit{S}}\\
			\hline
			\textit{TARSFO21} & Verifica che il sistema utilizzi in modo prioritario i dati reali se presenti anche quelli determinati per le predizioni & \makecell[tc]{\textit{I}} & \makecell[tc]{\textit{S}}\\
			\hline
			\textit{TARSFO22} & Verifica che il sistema aggiorni automaticamente la mappa alla selezione di un diverso orario & \makecell[tc]{\textit{I}} & \makecell[tc]{\textit{S}}\\
			\hline
			\textit{TARSFO23} & Verifica che l’utente possa poter selezionare la data del giorno di cui vuole visualizzare i dati & \makecell[tc]{\textit{I}} & \makecell[tc]{\textit{S}}\\
			\hline
			\textit{TARSFO24} & Verifica che l’utente possa poter ripristinare la visione in tempo reale tramite un pulsante di ripristino & \makecell[tc]{\textit{I}} & \makecell[tc]{\textit{S}}\\
			\hline
			\textit{TARSFD25} & Verifica che il sistema prelevi dati da diverse fonti e le formatti nel tipo di default & \makecell[tc]{\textit{I}} & \makecell[tc]{\textit{S}}\\
			\hline
			\textit{TARSFO26} & Verifica che il sistema utilizzi un software contapersone già allenato & \makecell[tc]{\textit{I}} & \makecell[tc]{\textit{S}}\\
			\hline
			\textit{TARSFF27} & Verifica che l’utente possa poter reperire il manuale d'uso & \makecell[tc]{\textit{I}} & \makecell[tc]{\textit{S}}\\
			\hline
			\textit{TARSFO28} & Verifica che l’utente possa poter variare il livello di zoom della heat map$_{\scaleto{G}{3pt}}$ & \makecell[tc]{\textit{I}} & \makecell[tc]{\textit{S}}\\
			\hline
			\textit{TARSFD29} & Verifica che l’utente possa poter ricercare tramite una barra di ricerca le città presenti nel database & \makecell[tc]{\textit{I}} & \makecell[tc]{\textit{S}}\\
			\hline
			\textit{TARSFD30} & Verifica che l’utente possa poter visualizzare il messaggio d'errore relativo alla mancanza dei dati ricercati attraverso la barra di ricerca nel database & \makecell[tc]{\textit{I}} & \makecell[tc]{\textit{S}}\\
			\hline
			\textit{TARSFD31} & Verifica che l’utente possa poter visualizzare il confronto dei dati di due città selezionate dall'utente & \makecell[tc]{\textit{I}} & \makecell[tc]{\textit{S}}\\
			\hline
			\textit{TARSFD32} & Verifica che l’utente possa poter salvare in un file locale i dati della città della mappa che sta visualizzando & \makecell[tc]{\textit{I}} & \makecell[tc]{\textit{S}}\\
			\hline
			\textit{TARSFD33} & Verifica che l’utente possa poter inserire l'email per il ricevimento delle informazioni delle informazioni della città selezionata & \makecell[tc]{\textit{NI}} & \makecell[tc]{\textit{-}}\\
			\hline
			\textit{TARSFD34} & Verifica che il sistema abbia inserito correttamente l'email e la città correlata nel database & \makecell[tc]{\textit{NI}} & \makecell[tc]{\textit{-}}\\
			\hline
			\textit{TARSFD35} & Verifica che il sistema invii correttamente l'email all'utente & \makecell[tc]{\textit{NI}} & \makecell[tc]{\textit{-}}\\
			\hline
			\textit{TARSFD36} & Verifica che l’utente possa poter visualizzare la lista delle città più ricercate & \makecell[tc]{\textit{NI}} & \makecell[tc]{\textit{-}}\\
			\hline
			\textit{TARSFD37} & Verifica che l’utente possa poter visualizzare la lista di tutte le città presenti nel database & \makecell[tc]{\textit{I}} & \makecell[tc]{\textit{S}}\\
			\hline
			\rowcolor{white}
			\caption{\textbf{Elenco test di accettazione}}\\
		\end{longtable}
	\end{center}
	\def\tabularxcolumn#1{m{#1}}
	{\rowcolors{2}{RawSienna!90!RawSienna!20}{RawSienna!70!RawSienna!40}

		\subsection{Tracciamento dei test}\label{SpecificaDeiTestTestDiAccettazioneTracciamentoDeiTest}

		\begin{center}
			\renewcommand{\arraystretch}{1.4}
			\begin{longtable}{|p{3cm}|p{3cm}|}
				\hline
				\rowcolor{airforceblue}
				\makecell[c]{\textbf{Id Test}} & \makecell[c]{\textbf{Id Requisito}} \\
				\hline
				\hline
				TARSFO1	& \makecell{RSFO1} \\
				\hline
				TARSFF2 & \makecell{RSFF2} \\
				\hline
				TARSFO3 & \makecell{RSFO3}  \\
				\hline
				TSRSFO4 & \makecell{RSFO4 \\ RAFO4.1 \\ RSFO4.2} \\
				\hline
				TARSFO5 & \makecell{RSFO5} \\
				\hline
				TARSFD6 & \makecell{RSFD6} \\
				\hline
				TARSFO7 & \makecell{RSFO7 \\ RSFO9 \\ RSFO10 \\ RSFO11 } \\
				\hline
				TARSFO8 & \makecell{RSFO8} \\
				\hline
				TARSFF9 & \makecell{RSFF12} \\
				\hline
				TARSFD10 & \makecell{RSFD13 \\ RSFD14} \\
				\hline
				TARSFF11 & \makecell{RSFF15} \\
				\hline
				TARSFF12 & \makecell{RSFF16}\\
				\hline
				TARSFO13 & \makecell{RSFO17} \\
				\hline
				TARSFO14 & \makecell{RSFO18 \\ RSFO18.1} \\
				\hline
				TARSFO15 & \makecell{RSFO19} \\
				\hline
				TARSFO16 & \makecell{RSFO20} \\
				\hline
				TARSFO17 & \makecell{RSFO21} \\
				\hline
				TARSFO18 & \makecell{RSFO22 \\ RSFO22.1 \\ RSFO22.2} \\
				\hline
				TARSFF19 & \makecell{RSFF23} \\
				\hline
				TARSFO20 & \makecell{RSFO24} \\
				\hline
				TARSFO21 & \makecell{RSFO25} \\
				\hline
				TARSFO22 & \makecell{RSFO26} \\
				\hline
				TARSFO23 & \makecell{RSFO27} \\
				\hline
				TARSFO24 & \makecell{RSFO28} \\
				\hline
				TARSFD25 & \makecell{RSFD29}\\
				\hline
				TARSFO26 & \makecell{RSFO30} \\
				\hline
				TARSFF27 & \makecell{RSFF31} \\
				\hline
				TARSFO28 & \makecell{RSFO32 \\ RSFO32.1 \\ RSFO32.1.1 \\ RSFO32.1.2 \\ RSFO32.1.3 \\ RSFO32.2 }\\
				\hline
				TARSFD29 & \makecell{RSFD33 \\ RSFD33.1 \\ RSFD33.2}\\
				\hline
				TARSFD30 & \makecell{RSFD34}\\
				\hline
				TARSFD31 & \makecell{RSFD35}\\
				\hline
				TARSFD32 & \makecell{RSFD36 \\ RSFD36.1 \\ RSFD36.2} \\
				\hline
				TARSFD33 & \makecell{RSFD37 \\ RSFD37.1} \\
				\hline
				TARSFD34 & \makecell{RSFD38} \\
				\hline
				TARSFD35 & \makecell{RSFD39} \\
				\hline
				TARSFD36 & \makecell{RSFD40} \\
				\hline
				TARSFD37 & \makecell{RSFD41} \\
				\hline
				\rowcolor{white}
				\caption{\textbf{Tracciamento dei test di accettazione con i requisiti}}
			\end{longtable}

		\end{center}

	\chapter{Resoconto attività di verifica } \label{ResocontoAttivitàDiVerifica}
In questa sezione si riportano gli esiti, descritti ed analizzati, di tutte le attività di verifica svolte.
\section{Revisione dei Requisiti}  \label{ResocontoAttivitàDiVerificaRevisioneDeiRequisiti}
Tutta la documentazione sviluppata nella prima fase da consegnare per la Revisione dei Requisiti ha subito una meticolosa ed attenta revisione da parte dei Verificatori. Questi ultimi hanno seguito, in questa attività, per ogni documento i metodi di \textit{Walkthrough$_G$} ed \textit{Inspection$_G$} relative all’analisi statica, stabilite nelle \textit{Norme di Progetto}. %% ?????? da rivedere quando scriviamo 3.4 sulle norme.
\subsection{Strategia adoperata per l’analisi statica dei documenti} \label{ResocontoAttivitàDiVerificaRevisioneDeiRequisitiStrategiaPerAnalisiStatica}
Il \textit{Verificatore} si è occupato di valutare la correttezza del documento, concentrandosi nell’individuare gli errori presenti in questo. Una volta individuati gli errori la strategia adottata è la seguente: 
\begin{itemize}
	\item Correzione degli errori sia ortografici che sintattici, non fedeli alle norme tipografiche fissate nelle \textit{Norme di Progetto v1.0.0};
\end{itemize}

\subsection{Esiti verifica} \label{ResocontoAttivitàDiVerificaRevisioneDeiRequisitiEsitiVerifica}
Per ciascun documento stilato si è calcolato l’indice di Gulpease$_G$. I risultati sono mostrati qui di seguito.
Per evitare risultati errati nel calcolo di tale indice, non si sono tenuti in considerazione:
\begin{itemize}
	\item il frontespizio di ogni documento;
	\item le eventuali tabelle presenti nel documenti;
	\item i diari delle modifiche di ogni documento.
\end{itemize}

\quad
\def\tabularxcolumn#1{m{#1}}
{\rowcolors{2}{RawSienna!90!RawSienna!20}{RawSienna!70!RawSienna!40}	
	\begin{center}
		\renewcommand{\arraystretch}{1.4}
		\begin{tabularx}{11.65cm}{|c|c|c|}
			\hline
			\rowcolor{airforceblue}
			\textbf{Documento} & \textbf{Indice di Gulpease} & \textbf{Esito}\\
			\hline
			\textit{Analisi dei Requisiti v1.0.0} & 96  & \textit{Superato}\\
			\hline
			\textit{Norme di Progetto v1.0.0} & 75 & \textit{Superato}\\
			\hline
			\textit{Studio di Fattibilità v1.0.0} & 70 & \textit{Superato}\\
			\hline
			\textit{Piano di Progetto v1.0.0} & 77 & \textit{Superato}\\
			\hline
			\textit{Piano di Qualifica v1.0.0} & 80 & \textit{Superato}\\
			\hline
		\end{tabularx}
		\captionof{table}{\textbf{Elenco Indici di Gulpease$_{\scaleto{G}{3pt}}$ dei documenti versione v1.0.0}}
	\end{center}


\begin{figure}[!h]
	\begin{center}
		\includegraphics[width=1\linewidth]{../immagini/IndexGulpeaseAdR.png}
		\caption{\textbf{Andamento Indice di Gulpease$_{\scaleto{G}{3pt}}$ Analisi dei Requisiti}}
	\end{center}
\end{figure}

\begin{figure}[!h]
	\begin{center}
		\includegraphics[width=1\linewidth]{../immagini/IndexGulpeaseNdP.png}
		\caption{\textbf{Andamento Indice di Gulpease$_{\scaleto{G}{3pt}}$ Norme di Progetto}}
	\end{center}
\end{figure}

\begin{figure}[!h]
	\begin{center}
		\includegraphics[width=1\linewidth]{../immagini/IndexGulpeasePdQ.png}
		\caption{\textbf{Andamento Indice di Gulpease$_{\scaleto{G}{3pt}}$ Piano di Qualifica}}
	\end{center}
\end{figure}

\begin{figure}[!h]	
	\begin{center}
		\includegraphics[width=1\linewidth]{../immagini/IndexGulpeasePdP.png}
		\caption{\textbf{Andamento Indice di Gulpease$_{\scaleto{G}{3pt}}$ Piano di Progetto}}
	\end{center}
\end{figure}

\begin{figure}[!h]
	\begin{center}
		\includegraphics[width=1\linewidth]{../immagini/IndexGulpeaseSdF.png}
		\caption{\textbf{Andamento Indice di Gulpease$_{\scaleto{G}{3pt}}$ Studio di Fattibilità}}
	\end{center}
\end{figure}
\clearpage

Per quanto riguarda gli Indici di Gulpease$_{\scaleto{G}{3pt}}$ dei verbali si è deciso di rappresentare i risultati in forma tabellare.
Questo in quanto il verbale viene scritto tutta in una volta, quindi utilizzare un grafico temporale risulta non idoneo.
\quad
\def\tabularxcolumn#1{m{#1}}
{\rowcolors{2}{RawSienna!90!RawSienna!20}{RawSienna!70!RawSienna!40}	
	\begin{center}
		\renewcommand{\arraystretch}{1.4}
		\begin{tabularx}{11.65cm}{|c|c|c|}
			\hline
			\rowcolor{airforceblue}
			\textbf{Documento} & \textbf{Indice di Gulpease} & \textbf{Esito}\\
			\hline
			\textit{verbale\_interno\_28-10-2020} & 100  & \textit{Superato}\\
			\hline
			\textit{verbale\_interno\_19-11-2020} & 100 & \textit{Superato}\\
			\hline
			\textit{verbale\_interno\_24-11-2020} & 99 & \textit{Superato}\\
			\hline
			\textit{verbale\_esterno\_04-12-2020} & 98 & \textit{Superato}\\
			\hline
			\textit{verbale\_interno\_29-12-2020} & 100 & \textit{Superato}\\
			\hline
			\textit{verbale\_interno\_03-01-2021} & 100 & \textit{Superato}\\
			\hline
			\textit{verbale\_interno\_06-01-2021} & 100 & \textit{Superato}\\
			\hline
		\end{tabularx}
		\captionof{table}{\textbf{Elenco Indici di Gulpease$_{\scaleto{G}{3pt}}$ dei verbali versione v1.0.0}}
	\end{center}

	\chapter{Valutazioni per il miglioramento} \label{ValutazionePerIlMiglioramento}
Questa sezione riporta una valutazione complessiva sul lavoro svolto fino ad ora, con l’obiettivo di far emergere e, quindi, risolvere in maniera efficace tutte le problematiche portate all'attenzione durante agli incontri con il gruppo e segnalate di conseguenza nei verbali interni. In questo modo il gruppo si impegna ad evitare che queste si ripresentino in futuro.\\
A causa dell’assenza di una figura esterna che possa effettivamente fornire una valutazione oggettiva del lavoro svolto, questa si basa su un'autovalutazione di ciascun membro del gruppo.
Nel caso in cui si presentassero nuove problematiche con l’avanzamento del lavoro, il gruppo provvederà ad integrare opportunamente la seguente sezione. Qui di seguito si trovano in forma tabellare le difficoltà incontrate durante ciascuna revisione per ogni tipologia di problema:
\begin{itemize}
	\item la prima è legata all'organizzazione (\S~\ref{ValutazionePerIlMiglioramentoValutazioneSuOrganizzazione});
	\item la seconda è legata ai ruoli (\S~\ref{ValutazionePerIlMiglioramentoValutazioneSuiRuoli});
	\item la terza è legata agli strumenti (\S~\ref{ValutazionePerIlMiglioramentoValutazioneSuStrumentiDiLavoro}).
\end{itemize}
Nella tabella di ogni problematiche è presentata la descrizione del problema con la rispettiva soluzione, ed inoltre, ad ognuna è attribuito un livello di gravità che varia da 1, che indica il livello minimo di difficoltà nella corretta risoluzione, a 3, che indica il livello massimo.
La strutturazione di queste tabelle è stata decisa ed uniformata all'interno delle \textit{Norme di Progetto 3.0.0}

\textit{Questa sezione verrà continuamente aggiornata allo scadere delle revisioni in modo da tenere traccia dei problemi riscontrati durante lo sviluppo del progetto e delle relative soluzioni.}
\clearpage

\section{Valutazione su organizzazione}  \label{ValutazionePerIlMiglioramentoValutazioneSuOrganizzazione}

\subsection{Revisione dei requisiti}\label{ValutazionePerIlMiglioramentoValutazioneSuOrganizzazioneRevisioneDeiRequisiti}

\quad
\def\tabularxcolumn#1{m{#1}}
{\rowcolors{2}{RawSienna!90!RawSienna!20}{RawSienna!70!RawSienna!40}

	\begin{center}
		\renewcommand{\arraystretch}{1.4}
		\begin{tabularx}{\textwidth}[c]{|p{3,5cm}|p{5cm}|p{1,9cm}|p{4,85cm}|}
			\hline
			\rowcolor{airforceblue}
			\textbf{Problema} & \textbf{Descrizione} & \textbf{Gravità} & \textbf{Soluzione}\\
			\hline
			Incontro con il gruppo & Si è riscontrata una difficoltà nel riuscire ad organizzare tutti gli incontri in modo che ogni membro del gruppo fosse presente. & \centering2 & Si è fatto un Poll sul canale Discord del gruppo, in cui ciascun membro ha votato la propria preferenza. Alla fine si è raggiunti ad una decisione unanime. \\
			 \hline
		\end{tabularx}
		\captionof{table}{\textbf{Tabella dei problemi relativi all'organizzazione - Revisione dei requisiti}}
	\end{center}

\subsection{Revisione di progettazione}\label{ValutazionePerIlMiglioramentoValutazioneSuOrganizzazioneRevisioneDiProgettazione}

\quad
\def\tabularxcolumn#1{m{#1}}
{\rowcolors{2}{RawSienna!90!RawSienna!20}{RawSienna!70!RawSienna!40}

	\begin{center}
		\renewcommand{\arraystretch}{1.4}
		\begin{tabularx}{\textwidth}[c]{|p{3,5cm}|p{5cm}|p{1,9cm}|p{4,85cm}|}
			\hline
			\rowcolor{airforceblue}
			\textbf{Problema} & \textbf{Descrizione} & \textbf{Gravità} & \textbf{Soluzione}\\
			\hline
			Comunicazione interna del gruppo & Si è riscontrata una difficoltà nel riuscire a comunicare internamente al sorgere di problemi con il rischio di accumularli e doverli risolvere troppo sommariamente & \centering2 & Si è deciso di aumentare l'impegno da parte di ogni componente di comunicare più spesso in modo da poter risolvere i possibili problemi che possono sorgere durante lo sviluppo \\
			\hline
			Organizzazione interna del gruppo & Si è riscontrata una difficoltà nello scandire il tempo ed i compiti portando ad uno sviluppo poco organizzato sia dei documenti che del prodotto software & \centering2 & Si è deciso di suddividere a monte ogni obiettivo da raggiungere per poter avere una migliore organizzazione e tracciabilità di cosa manca per ottenere un prodotto di qualità \\
			\hline
			Utilizzo di Trello & Si è riscontrata una difficoltà nell'utilizzo di Trello & \centering2 & Per risolvere il problema il gruppo ha deciso di impegnarsi nell'utilizzo della piattaforma\\
		\end{tabularx}
		\captionof{table}{\textbf{Tabella dei problemi relativi all'organizzazione - Revisione di progettazione}}
	\end{center}

\subsection{Revisione di qualifica}\label{ValutazionePerIlMiglioramentoValutazioneSuOrganizzazioneRevisioneDiQualifica}
	\begin{center}
	\renewcommand{\arraystretch}{1.4}
	\begin{tabularx}{\textwidth}[c]{|p{3,5cm}|p{5cm}|p{1,9cm}|p{4,85cm}|}
		\hline
		\rowcolor{airforceblue}
		\textbf{Problema} & \textbf{Descrizione} & \textbf{Gravità} & \textbf{Soluzione}\\
		Lavoro interno al gruppo & Alcuni componenti del gruppo non hanno lavorato in maniera efficiente in modo da pesare sul lavoro generale in prossimità delle scadenze & \centering2 & Per arginare questa difficoltà abbiamo deciso di organizzare ulteriormente il lavoro e aumentare i resoconti del lavoro svolto in modo da poter tenere traccia degli obiettivi da svolgere \\
	\end{tabularx}
	\captionof{table}{\textbf{Tabella dei problemi relativi all'organizzazione - Revisione di qualifica}}
\end{center}

\subsection{Revisione di accettazione}\label{ValutazionePerIlMiglioramentoValutazioneSuOrganizzazioneRevisioneDiAccettazione}
\begin{center}
	\renewcommand{\arraystretch}{1.4}
	\begin{tabularx}{\textwidth}[c]{|p{3,5cm}|p{5cm}|p{1,9cm}|p{4,85cm}|}
		\hline
		\rowcolor{airforceblue}
		\textbf{Problema} & \textbf{Descrizione} & \textbf{Gravità} & \textbf{Soluzione}\\
		Organizzazione interna al gruppo dopo l'inizio stage & Alcuni membri del gruppo hanno iniziato l'attività di stage nelle prime settimane di maggio, portando ad una pianificazione proporzionata del lavoro da svolegere. & \centering2 & Il lavoro da svolgere è stato ripartito in base al tempo che i membri in tirocinio potevano deidcare al progetto. In questo modo i membri più disponibili hanno potuto svolgere i compiti in autonomia, tenendo sempre aggiornati gli altri membri attraverso gli appositi canali di comunicazione, in particolare Trello e Discord.  \\
	\end{tabularx}
	\captionof{table}{\textbf{Tabella dei problemi relativi all'organizzazione - Revisione di accettazione}}
\end{center}


\section{Valutazione sui ruoli}\label{ValutazionePerIlMiglioramentoValutazioneSuiRuoli}

\subsection{Revisione dei requisiti}\label{ValutazionePerIlMiglioramentoValutazioneSuiRuoliRevisioneDeiRequisiti}

\quad
\def\tabularxcolumn#1{m{#1}}
{\rowcolors{2}{RawSienna!90!RawSienna!20}{RawSienna!70!RawSienna!40}

	\begin{center}
		\renewcommand{\arraystretch}{1.4}
		\begin{tabularx}{\textwidth}[c]{|p{3,5cm}|p{5cm}|p{1,9cm}|p{4,85cm}|}
			\hline
			\rowcolor{airforceblue}
			\textbf{Problema} & \textbf{Descrizione} & \textbf{Gravità} & \textbf{Soluzione}\\
			\hline
			Rivestimento del ruolo di \textit{Responsabile} &A causa dell'inesperienza, la maggiore difficoltà riscontrata nel rivestire il ruolo di \textit{Responsabile} è stata la stima delle risorse necessarie ed un'assegnazione adeguata delle stesse & \centering2 & Per arginare tale difficoltà,  in questa fase iniziale del progetto, il gruppo si aggiorna con maggior frequenza per avere un riscontro sulle stime e per poterle correggere \\
			\hline
			Rivestimento del ruolo di \textit{Analista}& Nessuno del gruppo ha redatto tale documentazione prima, per questo motivo è risultato difficile comprendere la struttura e le "competenze" di ogni documento & \centering2 & Abbiamo cercato di capire più a fondo le indicazioni del committente$_G$ e ci siamo confrontati tra di noi per cercare di trovare la soluzione migliore. \\
			\hline
			Rivestimento del ruolo di \textit{Amministratore} & Il ruolo di \textit{Amministratore} inizialmente ha creato delle problematiche relative all'approfondimento degli standard ISO per capire come adattarli al nostro progetto, mantenendo la qualità. & \centering2 & Tutti i membri del gruppo hanno contribuito alla ricerca di materiale informativo e condiviso le informazioni con gli altri membri, per velocizzare l'apprendimento iniziale. \\
			\hline
		\end{tabularx}
		\captionof{table}{\textbf{Tabella dei problemi relativi ai ruoli - Revisione dei requisiti}}
	\end{center}

\subsection{Revisione di progettazione}\label{ValutazionePerIlMiglioramentoValutazioneSuiRuoliRevisioneDiProgettazione}

\begin{center}
	\renewcommand{\arraystretch}{1.4}
	\begin{tabularx}{\textwidth}[c]{|p{3,5cm}|p{5cm}|p{1,9cm}|p{4,85cm}|}
		\hline
		\rowcolor{airforceblue}
		\textbf{Problema} & \textbf{Descrizione} & \textbf{Gravità} & \textbf{Soluzione}\\
		\hline
		Rivestimento del ruolo di \textit{Progettista} & A causa dell'inesperienza tecnologica il gruppo ha ritrovato difficoltà ad organizzare lo sviluppo dell'architeuttra del prodotto software & \centering3 & Come soluzione il gruppo ha deciso di impiegare più ore nello studio personale per conoscere meglio le tecnologie con cui dovrà lavorare \\
		\hline
		Rivestimento del ruolo di \textit{Analista} & Il gruppo, dato lo studio poco approfondito del progetto e dei suoi requisiti, ha riscontrato difficoltà nello svolgere un'analisi corretta che ha portato ad un'Analisi dei Requisiti non sufficiente & \centering2 & Il gruppo ha organizzato incontri con l'azienda proponente ed il professor Riccardo Cardin per poter sanare le lacune e produrre un'Analisi dei Requisiti soddisfacente\\
		\hline
	\end{tabularx}
	\captionof{table}{\textbf{Tabella dei problemi relativi ai ruoli - Revisione di progettazione}}
\end{center}


\subsection{Revisione di qualifica}\label{ValutazionePerIlMiglioramentoValutazioneSuiRuoliRevisioneDiQualifica}

\begin{center}
	\renewcommand{\arraystretch}{1.4}
	\begin{tabularx}{\textwidth}[c]{|p{3,5cm}|p{5cm}|p{1,9cm}|p{4,85cm}|}
		\hline
		\rowcolor{airforceblue}
		\textbf{Problema} & \textbf{Descrizione} & \textbf{Gravità} & \textbf{Soluzione}\\
		Rivestimento del ruolo di \textit{Programmatore} & A causa dell'inesperienza tecnologica, il gruppo ha riscontrato difficoltà a padroneggiare correttamente  i linguggi e framework da usare. Ciò ha portato anche a produrre del codice poco chiaro ed efficiente & \centering3 & Per arginare il problema i componenti del gruppo hanno deciso di dedicare molto tempo allo lo studio individuale per poter colmare le lacune e produrre del codice soddisfacente\\
		Rivestimento del ruolo di \textit{Analista} & Il gruppo, in seguito alle correzioni ricevute alla revisione di progettazione, ha dovuto impiegare più ore di quelle preventivate per questo ruolo per produrre una versione migliore del documento \textit{Analisi dei Requisiti} & \centering3 & Per risolvere il problema il gruppo ha organizzato incontri con l'azienda proponente e ha seguito le indicazioni del professor Riccardo Cardin tramite corrispondenza via email \\
	\end{tabularx}
	\captionof{table}{\textbf{Tabella dei problemi relativi ai ruoli - Revisione di qualifica}}
\end{center}

\subsection{Revisione di accettazione}\label{ValutazionePerIlMiglioramentoValutazioneSuiRuoliRevisioneDiAccettazione}
\begin{center}
	\renewcommand{\arraystretch}{1.4}
	\begin{tabularx}{\textwidth}[c]{|p{3,5cm}|p{5cm}|p{1,9cm}|p{4,85cm}|}
		\hline
		\rowcolor{airforceblue}
		\textbf{Problema} & \textbf{Descrizione} & \textbf{Gravità} & \textbf{Soluzione}\\
		Rivestimento del ruolo di \textit{Programmatore} & I \textit{Programmatori} hanno dovuto riorganizzare il lavoro a causa delle poche settimane a disposizione per terminare il progetto & \centering2 & Per sfruttare al meglio il tempo a disposizione, i \textit{Proggrammatori} hanno iniziato a lavorare subito, senza quindi aspettare l'esito della \textit{Revisione di qualifica}, dove possibile, con aggiunte, modifiche e correzzioni al prodotto sowftware. Inoltre per rientrare nei costi e nei tempi stabiliti, sono stati scelti, in accordo con il \textit{Proponente}, alcuni requisiti facoltativi/desiderabili più semplici da realizzare in fase di \textit{Revisione di Accettazione}.  \\
	\end{tabularx}
		\captionof{table}{\textbf{Tabella dei problemi relativi ai ruoli - Revisione di accettazione}}
\end{center}

\section{Valutazione su strumenti di lavoro}\label{ValutazionePerIlMiglioramentoValutazioneSuStrumentiDiLavoro}

\subsection{Revisione dei requisiti}\label{ValutazionePerIlMiglioramentoValutazioneSuStrumentiDiLavoroRevisioneDeiRequisiti}

\quad
\def\tabularxcolumn#1{m{#1}}
{\rowcolors{2}{RawSienna!90!RawSienna!20}{RawSienna!70!RawSienna!40}

	\begin{center}
		\renewcommand{\arraystretch}{1.4}
		\begin{tabularx}{\textwidth}[c]{|p{3,5cm}|p{5cm}|p{1,9cm}|p{4,85cm}|}
			\hline
			\rowcolor{airforceblue}
			\textbf{Problema} & \textbf{Descrizione} & \textbf{Gravità} & \textbf{Soluzione}\\
			\hline
			\textit{GitHub$_G$} & Alcuni membri del gruppo avevano meno esperienza con l’uso di questo strumento, quindi ci sono state alcune difficoltà iniziali. & \centering2 & Per risolvere tale problema, i membri meno pratici si sono impegnati nel sanare le loro lacune e quelli più ferrati, invece, si resi disponibili nell’aiutare chi in difficoltà. \\
			\hline
			\LaTeX & Per via dell’inesperienza della maggior parte dei membri del gruppo riguardo l’uso di tale strumento, si sono riscontrate diverse difficoltà, specie con la costruzione di tabelle ed il frontespizio. & \centering2 & Per cercare di risolvere in breve tempo il problema, si è dedicato del tempo nelle prime settimane all’apprendimento di questo strumento. \\
			\hline
		\end{tabularx}
		\captionof{table}{\textbf{Tabella dei problemi relativi agli strumenti di lavoro - Revisione dei requisiti}}
	\end{center}

\subsection{Revisione dei progettazione}\label{ValutazionePerIlMiglioramentoValutazioneSuStrumentiDiLavoroRevisioneDiProgettazione}

	\begin{center}
	\renewcommand{\arraystretch}{1.4}
	\begin{tabularx}{\textwidth}[c]{|p{3,5cm}|p{5cm}|p{1,9cm}|p{4,85cm}|}
		\hline
		\rowcolor{airforceblue}
		\textbf{Problema} & \textbf{Descrizione} & \textbf{Gravità} & \textbf{Soluzione}\\
		Software di riconoscimento oggetti & Il gruppo ha riscontrato difficoltà nel trovare ed implementare un software di riconoscimento oggetti & \centering3 & Per risolvere questa problematica il gruppo ha effettuato più ricerche contemporaneamente testando ogni risultato ottenuto per poi decidere quale software potesse essere il migliore per lo sviluppo del nostro progetto.\\
	\end{tabularx}
	\captionof{table}{\textbf{Tabella dei problemi relativi agli strumenti di lavoro - Revisione di progettazione}}
\end{center}


\subsection{Revisione di qualifica}\label{ValutazionePerIlMiglioramentoValutazioneSuStrumentiDiLavoroRevisioneDiQualifica}

\begin{center}
	\renewcommand{\arraystretch}{1.4}
	\begin{longtable}[c]{|p{3,5cm}|p{5cm}|p{1,9cm}|p{4,85cm}|}
		\hline
		\rowcolor{airforceblue}
		\textbf{Problema} & \textbf{Descrizione} & \textbf{Gravità} & \textbf{Soluzione}\\
		Difficoltà nell'implementare il framework leaflet & Per generare correttamente la mappa il gruppo ha riscontrato difficoltà nell'implementare il framework leaflet come componente di Vue.js & \centering3 & Più membri del gruppo hanno dedicato diverse ore a documentarsi per risolvere questa problematica il prima possibile, così da riuscire ad avanzare con lo sviluppo della web application\\
		Difficoltà nel gestire le dipendenze in Vue.js & Il gruppo ha riscontrato difficoltà nel gestire le dipendenze tra le differenti componenti & \centering2 & I componenti del gruppo che dovevano sviluppare questa parte hanno impiegato molto tempo a leggere la documentazione e a fare esercizio pratico per poter comprendere al meglio il funzionamento del framework. \\
		Difficoltà nella scrittura delle query lato backend & Il gruppo ha riscontrato una difficoltà iniziale a comprendere come strutturare il codice per eseguire correttamente le query al database & \centering2 & Per risolvere questa problematica si è effettuato uno studio della documentazione trovando la soluzione più adatta al nostro problema \\
		Sviluppo del modulo legato al machine learning$_{\scaleto{G}{3pt}}$ & Il gruppo ha riscontrato diverse difficoltà nello sviluppare un modello funzionante di machine learning & \centering3 & Per risolvere questa problematica Dario Stagnitto, dipendente dell'azeinda proponente, durante un incontro, ha proposto di non utilizzare più keras, libreria troppo complessa per le conoscenze pregresse del gruppo legate al machine learning, e di utilizzare invece scikit learn che ha una curva di apprendimento molto più bassa. Inoltre il gruppo ha mantenuto una corrispondenza con Dario Stagnitto il quale ha aiutato allo sviluppo dell'ambiente in modo da fornire delle buone basi da cui partire per procedere con lo sviluppo.\\
\end{longtable}
	\captionof{table}{\textbf{Tabella dei problemi relativi agli strumenti di lavoro - Revisione di qualifica}}

\end{center}

	\chapter{Esiti delle revisioni}\label{EsitiDelleRevisioni}

\section{Revisione dei requisiti}\label{EsitiDelleRevisioniRevisioneDeiRequisiti}

Successivamente alla prima revisione il gruppo, basandosi sulla prima valutazione, ha apportato diverse modifiche. Di seguito vengono elencate le modifiche effettuate:
\begin{itemize}
	\item aggiunta il numero del capitolo in ogni documento;
	\item modifica della denominazione dei verbali in modo da poter ordinarli tramite una codifica alfanumerica; 
	\item in tutti i documenti il gruppo ha rielaborato la tabella del registro delle modifiche in modo tale che sia coerente con lo scatto di versione legato al modello incrementale;
	\item ristrutturazione dell'\textit{Analisi dei requisiti} attraverso l'aggiunta: 
	\begin{itemize}
		\item dei casi d'uso concordati con l'azienda e i requisiti collegati ad essi;
		\item della tabella riassuntiva rappresentate il numero dei requisiti ed il loro tipo;
	\end{itemize}
	\item ristrutturazione del \textit{Piano di Progetto} attraverso la modifica del capitolo \ref{QualitàDelProdotto}, relativo al modello di sviluppo;
	\item ristrutturazione delle \textit{Norme di Progetto} attraverso la modifica dei capitoli \ref{QualitàDiProcesso} e \ref{QualitàDelProdotto};
	\item ristrutturazione del \textit{Piano di Qualifica} attraverso:
	\begin{itemize}
		\item l'aggiunta del capitolo \ref{EsitiDelleRevisioni} e della \S~\ref{ResocontoAttivitàDiVerificaRevisioneDiProgettazione};%label da aggiungere 5.2
		\item riorganizzazione dei capitoli \ref{QualitàDiProcesso},\ref{SpecificaDeiTest},\ref{ResocontoAttivitàDiVerifica}.
	\end{itemize}
\end{itemize}

\section{Revisione di progettazione}\label{EsitiDelleRevisioniRevisioneDiProgettazione}

In seguito alla Revisione di progettazione, basandosi sulla valutazione data, il gruppo ha apportato diverse modifiche sia ai documenti che al codice prodotto. Di seguito vengono elencate le modifiche effettuate:
\begin{itemize}
	\item riscrittura del codice già presente per renderlo più leggibile ed ordinato;
	\item stesura dei verbali interni ed esterni di tutti gli incontri interni e delle comunicazioni esterne;
	\item ristrutturazione della tabella del registro delle modifiche in tutti i documenti in modo da rendere più chiaro chi effettua la modifica e chi la verifica;
	\item modifica del formato della data in tutti i documenti in modo tale da seguire il formato AAAA-MM-GG;
	\item stesura del documento \textit{Manuale Utente};
	\item stesura del documento \textit{Manuale Sviluppatore};
	\item stesura del documento \textit{Product Baseline};
	\item ristrutturazione del \textit{Piano di Progetto} attraverso:
	\begin{itemize}
		\item rivisitazione dei capitoli 5 e 6;
		\item aggiunta del capitolo 7;
	\end{itemize}
	\item ristrutturazione dell'\textit{Analisi dei requisiti} attraverso:
	\begin{itemize}
		\item studio ulteriore e modifica dei casi d'uso;
		\item aggiornamento dei requisiti in funzione della nuova organizzazione dei casi d'uso;
	\end{itemize}
	\item ristrutturazione delle \textit{Norme di Progetto} attraverso:
	\begin{itemize}
		\item l'aggiunta della sezione legata ai processi di miglioramento del capitolo \ref{SpecificaDeiTest}
	\end{itemize}
	\item ristrutturazione del \textit{Piano di Qualifica} attraverso:
	\begin{itemize}
		\item l'aggiornamento dei capitoli \ref{SpecificaDeiTest}, \ref{ValutazionePerIlMiglioramento} e \ref{EsitiDelleRevisioni};
		\item l'aggiunta di grafici legati alle metriche del capitolo \ref{ResocontoAttivitàDiVerifica}.
	\end{itemize}
\end{itemize}
	
	
	
	
\end{document}
