\chapter{Qualità di processo}\label{QualitaDiProcesso}
Per garantire un prodotto di qualità, che rispetti i costi ed i tempi stabiliti dal \textit{Piano di Progetto 2.0.0}, il nostro gruppo ha deciso di aderire allo standard ISO/IEC 15504$_G$, anche noto come SPICE (\textit{Software Process Improvement and Capability Determination}).
Lo standard ISO/IEC 15504$_{\scaleto{G}{3pt}}$ garantisce la qualità di tutti i processi che compongono il prodotto attraverso una definizione chiara degli obiettivi di ognuno di essi e delle soglie prestabilite.
Per una descrizione più dettagliata dello standard ISO/IEC 15504$_{\scaleto{G}{3pt}}$
riferirsi al capitolo §6 nel documento \textit{Norme di Progetto 2.0.0}.

\section{Processi di supporto}\label{QualitaDiProcessoProcessiDiSupporto}
Di seguito si presentano le metriche relative alla qualità dei processi, come stabilito nel
documento \textit{Norme di Progetto 2.0.0}, indispensabili per ottenere gli obiettivi di qualità.
\subsection{Pianificazione}\label{QualitaDiProcessoProcessiDiSupportoPianificazione}

\subsubsection{Metriche}\label{QualitaDiProcessoProcessiDiSupportoMetriche}

\begin{itemize}
	\item[] \textbf{MQPS01 Budget at Completion} \\
	Quantità di budget totale allocato per il progetto.
	La misurazione viene effettuata tramite un numero intero.
	\begin{itemize}
		\item \textbf{valore preferibile:} corrispondente al preventivo;
		\item \textbf{valore accettabile:} il valore del preventivo con un errore massimo del 5\%, ossia:
		\begin{center}
			\textit{preventivo -5\% $\leq$ BAC $\leq$ preventivo + 5\%}
		\end{center}
	\end{itemize}

	\item[] \textbf{MQPS02 Planned value} \\
	Si tratta del valore del lavoro pianificato al momento del calcolo.
	La formula adottata è:
	\begin{center}
		\textbf{PV = BAC *\% di lavoro pianificato}
	\end{center}
	\begin{itemize}
		\item \textbf{valore preferibile:} $>$ 0;
		\item \textbf{valore accettabile:} $\geq$ 0.
	\end{itemize}

	\item[] \textbf{MQPS03 Actual cost} \\
	Il denaro speso fino al momento del calcolo per lo svolgimento del progetto.
	E’ necessario un monitoramento continuo per avere un Actual Cost al di sotto della soglia del Planned Value.
	Il valore è dato da un numero intero.
	\begin{itemize}
		\item \textbf{valore preferibile:} 0 $\leq$ AC $<$ PV;
		\item \textbf{valore accettabile:}  0 $\leq$ AC $\leq$  budget totale.
	\end{itemize}

	\item[] \textbf{MQPS04 Earned value} \\
	Si tratta del valore del lavoro fatto fino al momento del calcolo.
	La formula corrispondente è:
	\begin{center}
		\textbf{BAC * \% di lavoro completato}
	\end{center}
	\begin{itemize}
		\item \textbf{valore preferibile:} $\geq$ PV;
		\item \textbf{valore accettabile:} $\geq$ 0.
	\end{itemize}

	\item[] \textbf{MQPS05 Schedule Variance} \\
	Indica lo stato di avanzamento nello svolgimento del progetto rispetto a quanto pianificato.
	La formula adottata è:
	\begin{center}
		\textbf{SV = EV - PV}
	\end{center}
	\begin{itemize}
		\item \textbf{valore preferibile:} $>$ 0;
		\item \textbf{valore accettabile:} $\geq$ 0.
	\end{itemize}
	In base al risultato ottenuto:
	\begin{itemize}
		\item SV $>$ 0  indica che il gruppo è in anticipo rispetto alla pianificazione; in futuro,
		le previsioni dovranno essere eseguite con più precisione e tenendo conto di questo
		risultato;
		\item SV = 0  indica che il gruppo è in linea con la pianificazione; i criteri adottati per
		fare la pianificazione sono quindi efficaci, e dovranno essere usati anche per previsioni
		future;
		\item SV $<$ 0 indica che il gruppo è in ritardo rispetto alla pianificazione: è necessaria
		una revisione delle pianificazioni da quel momento in poi, in modo da ridistribuire le
		risorse ed evitare di accumulare ulteriori ritardi.
	\end{itemize}

	\item[] \textbf{MQPS06 Cost variance} \\
	Indica la differenza tra il costo di lavoro effettivamente completato ed il costo attualmente sostenuto.
	La formula adottata è:
	\begin{center}
		\textbf{CV = EV - AC}
	\end{center}
	\begin{itemize}
		\item \textbf{valore preferibile:} $>$ 0;
		\item \textbf{valore accettabile:} $\geq$ 0.
	\end{itemize}
	In base al risultato ottenuto:
	\begin{itemize}
		\item CV $>$ 0 indica che lo svolgimento del progetto si mantiene al di sotto del budget;
		\item CV = 0 indica che il progetto è al pari con il budget;
		\item CV $<$ 0 indica che il progetto è al di sopra del budget a disposizione, si devono correggere i metodi di lavoro.
	\end{itemize}

\end{itemize}

\section{Processi di sviluppo}\label{QualitaDiProcessoProcessiDiSviluppo}
La pianificazione è un’attività significativa della gestione di progetto.
Consiste nel governare le risorse a disposizione, ovvero tempi, costi e ruoli, monitorarle nel tempo e reagire efficacemente ai cambiamenti.

\subsection{Analisi dei requisiti}\label{QualitaDiProcessoProcessiDiSviluppoAnalisiDeiRequisiti}
\subsubsection{Metriche}\label{MetricheProcessiDiSviluppo}
\begin{itemize}
	\item[] \textbf{PROI: Percentuale Requisiti Obbligatori Implementati}
	Indicatore, mediante percentuale, dei requisiti$_{\scaleto{G}{3pt}}$ obbligatori che sono stati implementati nel software.
	\begin{center}
		\textbf{PROI = $\frac{ROI}{ROT}$*100}
	\end{center}
	Dove:
	\begin{itemize}
		\item \textbf{ROI:} requisiti obbligatori implementati;
		\item \textbf{ROT:} requisiti obbligatori totali.
	\end{itemize}
	I range accettabili per il risultato di \textbf{PROI} sono così suddivisi:
	\item \textbf{valore preferibile:} 100\%;
	\item \textbf{valore accettabile:} 100\%.
\end{itemize}
