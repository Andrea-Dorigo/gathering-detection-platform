\chapter{Valutazioni per il miglioramento} \label{ValutazionePerIlMiglioramento}
Questa sezione riporta una valutazione complessiva sul lavoro svolto fino ad ora, con l’obiettivo di far emergere e, quindi, risolvere in maniera efficace tutte le problematiche sorte, così da evitare che queste si ripresentino in futuro. 
I problemi affrontati riguardano: 
\begin{itemize}
\item \textbf{Organizzazione:} valutazione rispetto all'organizzazione e alla comunicazione interna fra i membri del gruppo;
\item \textbf{Ruoli:} valutazione rispetto alla copertura di un ruolo in maniera corretta ed efficiente;
\item \textbf{Strumenti di lavoro:} valutazione rispetto all’uso degli strumenti di lavoro scelti.
\end{itemize}
A causa dell’assenza di una figura esterna che possa effettivamente fornire una valutazione oggettiva del lavoro svolto, questa si basa su un'autovalutazione di ciascun membro del gruppo. 
Nel caso in cui si presentassero nuove problematiche con l’avanzamento del lavoro, il gruppo provvederà ad integrare opportunamente la seguente sezione.
Qui di seguito si trovano, in forma tabellare, le difficoltà incontrate per ogni tipologia di problema. Nella tabella di ogni problematiche è presentata la descrizione del problema con la rispettiva soluzione, ed inoltre, ad ognuna è attribuito un livello di gravità: con gravità 1 si intende che un livello di difficoltà minimo per la corretta risoluzione del problema. 

\section{Valutazione su organizzazione}  \label{ValutazionePerIlMiglioramentoValutazioneSuOrganizzazione}

\quad
\def\tabularxcolumn#1{m{#1}}
{\rowcolors{2}{RawSienna!90!RawSienna!20}{RawSienna!70!RawSienna!40}
	
	\begin{center}
		\renewcommand{\arraystretch}{1.4}
		\begin{tabularx}{\textwidth}{|c|X|c|X|}
			\hline
			\rowcolor{airforceblue}
			\textbf{Problema} & \textbf{Descrizione} & \textbf{Gravità} & \textbf{Soluzione}\\
			\hline
			Incontro con il gruppo & Si è riscontrata una difficoltà nel riuscire ad organizzare tutti gli incontri in modo che ogni membro del gruppo fosse presente. & 2 & Si è fatto un Poll sul canale Discord del gruppo, in cui ciascun membro ha votato la propria preferenza. Alla fine si è raggiunti ad una decisione unanime. \\
			\hline
		\end{tabularx}
		\captionof{table}{\textbf{Tabella dei problemi relativi all'organizzazione}}
	\end{center}
\clearpage

\section{Valutazione sui ruoli}  \label{ValutazionePerIlMiglioramentoValutazioneSuiRuoli}

\quad
\def\tabularxcolumn#1{m{#1}}
{\rowcolors{2}{RawSienna!90!RawSienna!20}{RawSienna!70!RawSienna!40}
	
	\begin{center}
		\renewcommand{\arraystretch}{1.4}
		\begin{tabularx}{\textwidth}{|X|X|c|X|}
			\hline
			\rowcolor{airforceblue}
			\textbf{Problema} & \textbf{Descrizione} & \textbf{Gravità} & \textbf{Soluzione}\\
			\hline
			Rivestimento del ruolo di \textit{Responsabile} &A causa dell'inesperienza, la maggiore difficoltà riscontrata nel rivestire il ruolo di \textit{Responsabile} è stata la stima delle risorse necessarie ed un'assegnazione adeguata delle stesse & 2 & Per arginare tale difficoltà,  in questa fase iniziale del progetto, il gruppo si aggiorna con maggior frequenza per avere un riscontro sulle stime e per poterle correggere \\
			\hline
			Rivestimento del ruolo di \textit{Analista}& Nessuno del gruppo ha redatto tale documentazione prima, per questo motivo è risultato difficile comprendere la struttura e le "competenze" di ogni documento & 2 & Abbiamo cercato di capire più a fondo le indicazioni del committente e ci siamo confrontati tra di noi per cercare di trovare la soluzione migliore. \\
			\hline
			Rivestimento del ruolo di \textit{Amministratore} & Il ruolo di \textit{Amministratore} inizialmente ha creato delle problematiche relative all'approfondimento degli standard ISO per capire come adattarli al nostro progetto, mantenendo la qualità. & 2 & Tutti i membri del gruppo hanno contribuito alla ricerca di materiale informativo e condiviso le informazioni con gli altri membri, per velocizzare l'apprendimento iniziale. \\
			\hline
		\end{tabularx}
		\captionof{table}{\textbf{Tabella dei problemi relativi ai ruoli}}
	\end{center}
\clearpage

\section{Valutazione su strumenti di lavoro}  \label{ValutazionePerIlMiglioramentoValutazioneSuStrumentiDiLavoro}

\quad
\def\tabularxcolumn#1{m{#1}}
{\rowcolors{2}{RawSienna!90!RawSienna!20}{RawSienna!70!RawSienna!40}
	
	\begin{center}
		\renewcommand{\arraystretch}{1.4}
		\begin{tabularx}{\textwidth}{|X|X|c|X|}
			\hline
			\rowcolor{airforceblue}
			\textbf{Problema} & \textbf{Descrizione} & \textbf{Gravità} & \textbf{Soluzione}\\
			\hline
			\textit{GitHub} & Alcuni membri del gruppo avevano meno esperienza con l’uso di questo strumento, quindi ci sono state alcune difficoltà iniziali. & 2 & Per risolvere tale problema, i membri meno pratici si sono impegnati nel sanare le loro lacune e quelli più ferrati, invece, si resi disponibili nell’aiutare chi in difficoltà. \\
			\hline
			\LaTeX & Per via dell’inesperienza della maggior parte dei membri del gruppo riguardo l’uso di tale strumento, si sono riscontrate diverse difficoltà, specie con la costruzione di tabelle ed il frontespizio. & 2 & Per cercare di risolvere in breve tempo il problema, si è dedicato del tempo nelle prime settimane all’apprendimento di questo strumento. \\
			\hline
		\end{tabularx}
		\captionof{table}{\textbf{Tabella dei problemi relativi agli strumenti di lavoro}}
	\end{center}