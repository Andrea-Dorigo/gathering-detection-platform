\chapter{Valutazioni per il miglioramento} \label{ValutazionePerIlMiglioramento}
Questa sezione riporta una valutazione complessiva sul lavoro svolto fino ad ora, con l’obiettivo di far emergere e, quindi, risolvere in maniera efficace tutte le problematiche portate all'attenzione durante agli incontri con il gruppo e segnalate di conseguenza nei verbali interni. In questo modo il gruppo si impegna ad evitare che queste si ripresentino in futuro.\\
A causa dell’assenza di una figura esterna che possa effettivamente fornire una valutazione oggettiva del lavoro svolto, questa si basa su un'autovalutazione di ciascun membro del gruppo.
Nel caso in cui si presentassero nuove problematiche con l’avanzamento del lavoro, il gruppo provvederà ad integrare opportunamente la seguente sezione. Qui di seguito si trovano in forma tabellare le difficoltà incontrate durante ciascuna revisione per ogni tipologia di problema:
\begin{itemize}
	\item la prima è legata all'organizzazione (\S~\ref{ValutazionePerIlMiglioramentoValutazioneSuOrganizzazione});
	\item la seconda è legata ai ruoli (\S~\ref{ValutazionePerIlMiglioramentoValutazioneSuiRuoli});
	\item la terza è legata agli strumenti (\S~\ref{ValutazionePerIlMiglioramentoValutazioneSuStrumentiDiLavoro}).
\end{itemize}
Nella tabella di ogni problematiche è presentata la descrizione del problema con la rispettiva soluzione, ed inoltre, ad ognuna è attribuito un livello di gravità che varia da 1, che indica il livello minimo di difficoltà nella corretta risoluzione, a 3, che indica il livello massimo.
La strutturazione di queste tabelle è stata decisa ed uniformata all'interno delle \textit{Norme di Progetto 3.0.0}

\textit{Questa sezione verrà continuamente aggiornata allo scadere delle revisioni in modo da tenere traccia dei problemi riscontrati durante lo sviluppo del progetto e delle relative soluzioni.}
\clearpage

\section{Valutazione su organizzazione}  \label{ValutazionePerIlMiglioramentoValutazioneSuOrganizzazione}

\subsection{Revisione dei requisiti}\label{ValutazionePerIlMiglioramentoValutazioneSuOrganizzazioneRevisioneDeiRequisiti}

\quad
\def\tabularxcolumn#1{m{#1}}
{\rowcolors{2}{RawSienna!90!RawSienna!20}{RawSienna!70!RawSienna!40}

	\begin{center}
		\renewcommand{\arraystretch}{1.4}
		\begin{tabularx}{\textwidth}[c]{|p{3,5cm}|p{5cm}|p{1,9cm}|p{4,85cm}|}
			\hline
			\rowcolor{airforceblue}
			\textbf{Problema} & \textbf{Descrizione} & \textbf{Gravità} & \textbf{Soluzione}\\
			\hline
			Incontro con il gruppo & Si è riscontrata una difficoltà nel riuscire ad organizzare tutti gli incontri in modo che ogni membro del gruppo fosse presente. & \centering2 & Si è fatto un Poll sul canale Discord del gruppo, in cui ciascun membro ha votato la propria preferenza. Alla fine si è raggiunti ad una decisione unanime. \\
			 \hline
		\end{tabularx}
		\captionof{table}{\textbf{Tabella dei problemi relativi all'organizzazione - Revisione dei requisiti}}
	\end{center}

\subsection{Revisione di progettazione}\label{ValutazionePerIlMiglioramentoValutazioneSuOrganizzazioneRevisioneDiProgettazione}

\quad
\def\tabularxcolumn#1{m{#1}}
{\rowcolors{2}{RawSienna!90!RawSienna!20}{RawSienna!70!RawSienna!40}

	\begin{center}
		\renewcommand{\arraystretch}{1.4}
		\begin{tabularx}{\textwidth}[c]{|p{3,5cm}|p{5cm}|p{1,9cm}|p{4,85cm}|}
			\hline
			\rowcolor{airforceblue}
			\textbf{Problema} & \textbf{Descrizione} & \textbf{Gravità} & \textbf{Soluzione}\\
			\hline
			Comunicazione interna del gruppo & Si è riscontrata una difficoltà nel riuscire a comunicare internamente al sorgere di problemi con il rischio di accumularli e doverli risolvere troppo sommariamente & \centering2 & Si è deciso di aumentare l'impegno da parte di ogni componente di comunicare più spesso in modo da poter risolvere i possibili problemi che possono sorgere durante lo sviluppo \\
			\hline
			Organizzazione interna del gruppo & Si è riscontrata una difficoltà nello scandire il tempo ed i compiti portando ad uno sviluppo poco organizzato sia dei documenti che del prodotto software & \centering2 & Si è deciso di suddividere a monte ogni obiettivo da raggiungere per poter avere una migliore organizzazione e tracciabilità di cosa manca per ottenere un prodotto di qualità \\
			\hline
		\end{tabularx}
		\captionof{table}{\textbf{Tabella dei problemi relativi all'organizzazione - Revisione di progettazione}}
	\end{center}

\subsection{Revisione di qualifica}\label{ValutazionePerIlMiglioramentoValutazioneSuOrganizzazioneRevisioneDiQualifica}


\section{Valutazione sui ruoli}\label{ValutazionePerIlMiglioramentoValutazioneSuiRuoli}

\subsection{Revisione dei requisiti}\label{ValutazionePerIlMiglioramentoValutazioneSuiRuoliRevisioneDeiRequisiti}

\quad
\def\tabularxcolumn#1{m{#1}}
{\rowcolors{2}{RawSienna!90!RawSienna!20}{RawSienna!70!RawSienna!40}

	\begin{center}
		\renewcommand{\arraystretch}{1.4}
		\begin{tabularx}{\textwidth}[c]{|p{3,5cm}|p{5cm}|p{1,9cm}|p{4,85cm}|}
			\hline
			\rowcolor{airforceblue}
			\textbf{Problema} & \textbf{Descrizione} & \textbf{Gravità} & \textbf{Soluzione}\\
			\hline
			Rivestimento del ruolo di \textit{Responsabile} &A causa dell'inesperienza, la maggiore difficoltà riscontrata nel rivestire il ruolo di \textit{Responsabile} è stata la stima delle risorse necessarie ed un'assegnazione adeguata delle stesse & \centering2 & Per arginare tale difficoltà,  in questa fase iniziale del progetto, il gruppo si aggiorna con maggior frequenza per avere un riscontro sulle stime e per poterle correggere \\
			\hline
			Rivestimento del ruolo di \textit{Analista}& Nessuno del gruppo ha redatto tale documentazione prima, per questo motivo è risultato difficile comprendere la struttura e le "competenze" di ogni documento & \centering2 & Abbiamo cercato di capire più a fondo le indicazioni del committente$_G$ e ci siamo confrontati tra di noi per cercare di trovare la soluzione migliore. \\
			\hline
			Rivestimento del ruolo di \textit{Amministratore} & Il ruolo di \textit{Amministratore} inizialmente ha creato delle problematiche relative all'approfondimento degli standard ISO per capire come adattarli al nostro progetto, mantenendo la qualità. & \centering2 & Tutti i membri del gruppo hanno contribuito alla ricerca di materiale informativo e condiviso le informazioni con gli altri membri, per velocizzare l'apprendimento iniziale. \\
			\hline
		\end{tabularx}
		\captionof{table}{\textbf{Tabella dei problemi relativi ai ruoli - Revisione dei requisiti}}
	\end{center}

\subsection{Revisione di progettazione}\label{ValutazionePerIlMiglioramentoValutazioneSuiRuoliRevisioneDiProgettazione}

\begin{center}
	\renewcommand{\arraystretch}{1.4}
	\begin{tabularx}{\textwidth}[c]{|p{3,5cm}|p{5cm}|p{1,9cm}|p{4,85cm}|}
		\hline
		\rowcolor{airforceblue}
		\textbf{Problema} & \textbf{Descrizione} & \textbf{Gravità} & \textbf{Soluzione}\\
		\hline
		Rivestimento del ruolo di \textit{Progettista} & A causa dell'inesperienza tecnologica il gruppo ha ritrovato difficoltà ad organizzare lo sviluppo dell'architeuttra del prodotto software & \centering3 & Come soluzione il gruppo ha deciso di impiegare più ore nello studio personale per conoscere meglio le tecnologie con cui dovrà lavorare \\
		\hline
		Rivestimento del ruolo di \textit{Analista} & Il gruppo, dato lo studio poco approfondito del progetto e dei suoi requisiti, ha riscontrato difficoltà nello svolgere un'analisi corretta che ha portato ad un'Analisi dei Requisiti non sufficiente & \centering2 & Il gruppo ha organizzato incontri con l'azienda proponente ed il professor Riccardo Cardin per poter sanare le lacune e produrre un'Analisi dei Requisiti soddisfacente\\
		\hline
	\end{tabularx}
	\captionof{table}{\textbf{Tabella dei problemi relativi ai ruoli}}
\end{center}


\subsection{Revisione di qualifica}\label{ValutazionePerIlMiglioramentoValutazioneSuiRuoliRevisioneDiQualifica}

\section{Valutazione su strumenti di lavoro}\label{ValutazionePerIlMiglioramentoValutazioneSuStrumentiDiLavoro}

\subsubsection{Revisione dei requisiti}\label{ValutazionePerIlMiglioramentoValutazioneSuStrumentiDiLavoroRevisioneDeiRequisiti}

\quad
\def\tabularxcolumn#1{m{#1}}
{\rowcolors{2}{RawSienna!90!RawSienna!20}{RawSienna!70!RawSienna!40}

	\begin{center}
		\renewcommand{\arraystretch}{1.4}
		\begin{tabularx}{\textwidth}[c]{|p{3,5cm}|p{5cm}|p{1,9cm}|p{4,85cm}|}
			\hline
			\rowcolor{airforceblue}
			\textbf{Problema} & \textbf{Descrizione} & \textbf{Gravità} & \textbf{Soluzione}\\
			\hline
			\textit{GitHub$_G$} & Alcuni membri del gruppo avevano meno esperienza con l’uso di questo strumento, quindi ci sono state alcune difficoltà iniziali. & \centering2 & Per risolvere tale problema, i membri meno pratici si sono impegnati nel sanare le loro lacune e quelli più ferrati, invece, si resi disponibili nell’aiutare chi in difficoltà. \\
			\hline
			\LaTeX & Per via dell’inesperienza della maggior parte dei membri del gruppo riguardo l’uso di tale strumento, si sono riscontrate diverse difficoltà, specie con la costruzione di tabelle ed il frontespizio. & \centering2 & Per cercare di risolvere in breve tempo il problema, si è dedicato del tempo nelle prime settimane all’apprendimento di questo strumento. \\
			\hline
		\end{tabularx}
		\captionof{table}{\textbf{Tabella dei problemi relativi agli strumenti di lavoro}}
	\end{center}

\subsubsection{Revisione dei progettazione}\label{ValutazionePerIlMiglioramentoValutazioneSuStrumentiDiLavoroRevisioneDiProgettazione}

	\begin{center}
	\renewcommand{\arraystretch}{1.4}
	\begin{tabularx}{\textwidth}[c]{|p{3,5cm}|p{5cm}|p{1,9cm}|p{4,85cm}|}
		\hline
		\rowcolor{airforceblue}
		\textbf{Problema} & \textbf{Descrizione} & \textbf{Gravità} & \textbf{Soluzione}\\
		da chiedere agli andrea e cocco & ciao & 2 & soluzione \\
	\end{tabularx}
	\captionof{table}{\textbf{Tabella dei problemi relativi agli strumenti di lavoro}}
\end{center}


\subsubsection{Revisione di qualifica}\label{ValutazionePerIlMiglioramentoValutazioneSuStrumentiDiLavoroRevisioneDiQualifica}