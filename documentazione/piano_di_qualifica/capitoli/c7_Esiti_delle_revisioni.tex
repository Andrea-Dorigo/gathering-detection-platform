\chapter{Esiti delle revisioni}\label{EsitiDelleRevisioni}

\section{Revisione dei requisiti}\label{EsitiDelleRevisioniRevisioneDeiRequisiti}

Successivamente alla prima revisione il gruppo, basandosi sulla prima valutazione, ha apportato diverse modifiche. Di seguito vengono elencate le modifiche effettuate:
\begin{itemize}
	\item aggiunta il numero del capitolo in ogni documento;
	\item modifica della denominazione dei verbali in modo da poter ordinarli tramite una codifica alfanumerica; 
	\item in tutti i documenti il gruppo ha rielaborato la tabella del registro delle modifiche in modo tale che sia coerente con lo scatto di versione legato al modello incrementale;
	\item ristrutturazione dell'\textit{Analisi dei requisiti} attraverso l'aggiunta: 
	\begin{itemize}
		\item dei casi d'uso concordati con l'azienda e i requisiti collegati ad essi;
		\item della tabella riassuntiva rappresentate il numero dei requisiti ed il loro tipo;
	\end{itemize}
	\item ristrutturazione del \textit{Piano di Progetto} attraverso la modifica del capitolo \ref{QualitàDelProdotto}, relativo al modello di sviluppo;
	\item ristrutturazione delle \textit{Norme di Progetto} attraverso la modifica dei capitoli \ref{QualitàDiProcesso} e \ref{QualitàDelProdotto};
	\item ristrutturazione del \textit{Piano di Qualifica} attraverso:
	\begin{itemize}
		\item l'aggiunta del capitolo \ref{EsitiDelleRevisioni} e della \S~\ref{ResocontoAttivitàDiVerificaRevisioneDiProgettazione};%label da aggiungere 5.2
		\item riorganizzazione dei capitoli \ref{QualitàDiProcesso},\ref{SpecificaDeiTest},\ref{ResocontoAttivitàDiVerifica}.
	\end{itemize}
\end{itemize}

\section{Revisione di progettazione}\label{EsitiDelleRevisioniRevisioneDiProgettazione}

In seguito alla Revisione di progettazione, basandosi sulla valutazione data, il gruppo ha apportato diverse modifiche sia ai documenti che al codice prodotto. Di seguito vengono elencate le modifiche effettuate:
\begin{itemize}
	\item riscrittura del codice già presente per renderlo più leggibile ed ordinato;
	\item stesura del documento \textit{Manuale Utente};
	\item stesura del documento \textit{Manuale Sviluppatore};
	\item stesura del documento \textit{Product Baseline};
	\item ristrutturazione del \textit{Piano di Progetto};
	\item ristrutturazione dell'\textit{Analisi dei requisiti} attraverso:
	\begin{itemize}
		\item studio ulteriore e modifica dei casi d'uso;
		\item aggiornamento dei requisiti in funzione della nuova organizzazione dei casi d'uso;
	\end{itemize}
	\item ristrutturazione delle \textit{Norme di Progetto} attraverso:
	\begin{itemize}
		\item l'aggiunta della sezione legata ai processi di miglioramento del capitolo \ref{SpecificaDeiTest}
	\end{itemize}
	\item ristrutturazione del \textit{Piano di Qualifica} attraverso:
	\begin{itemize}
		\item l'aggiornamento dei capitoli \ref{SpecificaDeiTest}, \ref{ValutazionePerIlMiglioramento} e \ref{EsitiDelleRevisioni};
		\item l'aggiunta di grafici legati alle metriche del capitolo \ref{ResocontoAttivitàDiVerifica};
	\end{itemize}
	\item stesura dei verbali interni ed esterni di tutti gli incontri interni e delle comunicazioni esterne.
\end{itemize}