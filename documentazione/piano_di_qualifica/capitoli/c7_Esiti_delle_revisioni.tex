\chapter{Esiti delle revisioni}\label{EsitiDelleRevisioni}

\section{Revisione dei requisiti}\label{EsitiDelleRevisioniRevisioneDeiRequisiti}

Successivamente alla prima revisione il gruppo, basandosi sulla prima valutazione, ha apportato diverse modifiche. Di seguito vengono elencate le modifiche effettuate:
\begin{itemize}
	\item aggiunta il numero del capitolo in ogni documento;
	\item modifica della denominazione dei verbali in modo da poter ordinarli tramite una codifica alfanumerica; 
	\item in tutti i documenti il gruppo ha rielaborato la tabella del registro delle modifiche in modo tale che sia coerente con lo scatto di versione legato al modello incrementale;
	\item ristrutturazione dell'\textit{Analisi dei requisiti} attraverso l'aggiunta: 
	\begin{itemize}
		\item dei casi d'uso concordati con l'azienda e i requisiti collegati ad essi;
		\item della tabella riassuntiva rappresentate il numero dei requisiti ed il loro tipo;
	\end{itemize}
	\item ristrutturazione del \textit{Piano di Progetto} attraverso la modifica del capitolo \S~3, relativo al modello di sviluppo;
	\item ristrutturazione del \textit{Piano di Qualifica} attraverso:
	\begin{itemize}
		\item l'aggiunta del capitolo \S~\ref{EsitiDelleRevisioni} e della sezione \S~\ref{ResocontoAttivitàDiVerificaRevisioneDiProgettazione};%label da aggiungere 5.2
		\item riorganizzazione dei capitoli \S~\ref{QualitàDelProcesso},\S~\ref{SpecificaDeiTest},\S~\ref{ResocontoAttivitàDiVerifica}.
	\end{itemize}
\end{itemize}