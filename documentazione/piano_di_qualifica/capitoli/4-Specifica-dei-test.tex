\chapter{Specifica dei test} \label{SpecificaDeiTest}
Per assicurare un’ottima qualità del software prodotto, il gruppo \textit{JawaDruids}, dopo essersi confrontato, ha deciso di utilizzare come modello di sviluppo software il \textit{V-Model}, o Modello a V, il quale è un’estensione del modello a cascata.
Questo modello prevede un lavoro parallelo tra lo sviluppo dei test e le attività di analisi e progettazione.
Grazie a ciò, i test permettono di verificare sia il corretto funzionamento delle parti di software programmate, sia la corretta implementazione di tutti i requisiti del progetto.
Vengono utilizzate delle sigle, all’interno di tabelle, per fornire una comprensione più agevolata riguardo gli output prodotti tramite i test, specificando se il risultato è quello atteso, errato o non coerente con quanto aspettato.
Le sigle per lo stato dei test sono:
\begin{itemize}
	\item \textbf{NI}: non implementato;
	\item \textbf{I}: implementato;
\end{itemize}
Per quando riguarda la qualità dei test si usa:
\begin{itemize}
	\item \textbf{NS}: il test non ha soddisfatto la richiesta;
	\item \textbf{S}: il test ha soddisfatto la richiesta; 
\end{itemize}
Inoltre i test hanno la seguente nomenclatura
\begin{center}
	\textbf{[TipoTest][TipoRequisito][Importanza][Codice]}
\end{center}
dove:
\begin{itemize}
	\item \textbf{TipoTest}: specifica il tipo di test applicato;
	\item \textbf{TipoRequisito}: assume i seguenti valori
		\begin{itemize}
			\item[-] \textbf{O} per i requisiti obbligatori;
			\item[-] \textbf{D} per i requisiti desiderabili;
			\item[-] \textbf{F} per i requisiti facoltativi;
		\end{itemize}
	\item \textbf{Importanza}: come si evince, questa label indica l'importanza del requisito;
	\item 
\end{itemize}

\section{Test di convalida}
I test di convalida hanno come scopo la verifica che il software sviluppato soddisfi i requisiti presenti nel capitolato(g) d’appalto e concordati col proponente.
Questi saranno eseguiti durante il collaudo finale del prodotto software sotto l'osservazione sia dell'azienda proponente sia del gruppo di lavoro.

\section{Test di sistema}
I test di sistema vengono eseguiti per verificare che i requisiti, scritti nel documento \textit{Analisi dei Requisiti}, siano stati implementati e funzionanti.

