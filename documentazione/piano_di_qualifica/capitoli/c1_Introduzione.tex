\chapter{Introduzione}\label{Introduzione}

\section{Scopo del documento}\label{IntroduzioneScopoDelDocumento}
Il Piano di Qualifica è un documento su cui si prevede di operare per l’intera durata del progetto 
e il cui scopo è presentare e descrivere le strategie di verifica e validazione adottate 
dal gruppo \textit{Jawa Druids} al fine di garantire la qualità di prodotto e di processo.  
Per raggiungere questo obbiettivo viene applicato un sistema di verifica continua sui processi in corso e 
sulle attività$_{\scaleto{G}{3pt}}$ svolte, in modo da rilevare e correggere subito eventuali anomalie, riducendo lo spreco di risorse 
ed il rischio di reiterare gli stessi errori.
\section{Scopo del prodotto}\label{IntroduzioneScopodelProdotto}
In seguito alla pandemia del virus COVID-19 è nata l'esigenza di limitare il più possibile i
contatti fra le persone, specialmente evitando la formazione di assembramenti. 
Il progetto \textit{GDP: Gathering Detection Platform} di \textit{Sync Lab} ha pertanto l'obiettivo di \textbf{creare una piattaforma in grado di rappresentare graficamente le zone potenzialmente a rischio di assembramento, al fine di prevenirlo.}Il prodotto finale è rivolto specificatamente agli
organi amministrativi delle singole città, cosicché possano gestire al meglio i punti sensibili di
affollamento, come piazze o siti turistici. Lo scopo che il software intende raggiungere non è
solo quello della rappresentazione grafica real-time ma anche di poter riuscire a prevedere
assembramenti in intervalli futuri di tempo.
\\
A tal fine il gruppo \textit{Jawa Druids} si prefigge di sviluppare un prototipo software in grado di acquisire, monitorare ed analizzare i molteplici dati provenienti dai diversi sistemi e dispositivi, a scopo di identificare i possibili eventi che concorrono all'insorgere di variazioni di flussi di utenti. Il gruppo prevede inoltre lo sviluppo di un'applicazione web da interporre fra i dati elaborati e l'utente, per favorirne la consultazione.
\section{Glossario}\label{IntroduzioneGlossario}
All'interno della documentazione viene fornito un \textit{Glossario}, con l'obiettivo di assistere il lettore specificando il significato e contesto d'utilizzo di alcuni termini strettamente tecnici o ambigui, segnalati con una \textit{G} a pedice.
\section{Riferimenti}\label{IntroduzioneRiferimenti}
\subsection{Riferimenti normativi}\label{IntroduzioneRiferimentiRiferimentiNormativi}
\begin{itemize}
	\item \textit{Norme di Progetto 1.0.0}.
\end{itemize}
\subsection{Riferimenti informativi}\label{IntroduzioneRiferimentiRiferimentiInformativi}
\begin{itemize}
	\item \textit{Qualità di processo:}\\
		\url{https://www.math.unipd.it/~tullio/IS-1/2020/Dispense/L13.pdf}
	\item \textit{Qualità di prodotto:}\\
		\url{https://www.math.unipd.it/~tullio/IS-1/2020/Dispense/L12.pdf}
	\item \textit{Verifica e validazione: introduzione:}\\
		\url{https://www.math.unipd.it/~tullio/IS-1/2020/Dispense/L14.pdf}
	\item \textit{Indice di Gulpease:}\\
		\url{https://it.wikipedia.org/wiki/Indice_Gulpease}
	\item \textit{IEEE Recommended Practice for Software Requirements Specifications:}\\
		\url{https://ieeexplore.ieee.org/document/720574}
	\item \textit{Validating the ISO/IEC 15504 measure of software requirements analysis process capability:}\\
		\url{https://ieeexplore.ieee.org/document/852742}
	\item \textit{Seminario per approfondimenti tecnici del capitolato C3:}\\
		\url{https://www.math.unipd.it/~tullio/IS-1/2020/Progetto/ST1.pdf}	
	\item \textit{Standard ISO/IEC 9126:} \\
		\url{http://www.colonese.it/00-Manuali_Pubblicatii/07-ISO-IEC9126_v2.pdf}
	\item \textit{Metrica Comprensione del Codice:} \\
		\url{https://www.aivosto.com/project/help/pm-loc.html}
	\item \textit{Metriche per valutazione della Qualità di Prodotto:} \\
		\url{https://www.tricentis.com/blog/64-essential-testing-metrics-for-measuring-quality-assurance-success/?utm_source=qasymphony&utm_medium=redirect&utm_campaign=qas-redirects&utm_content=%2F%2Fwww.qasymphony.com%2Fblog%2F64-test-metrics%2F}
	\item \textit{Metriche per la valutazione della Qualità di Processo:} \\
		\url{https://it.wikipedia.org/wiki/Metriche_di_progetto}
\end{itemize}