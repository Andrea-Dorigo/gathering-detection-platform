\chapter{Introduzione}\label{Introduzione}

\section{Scopo del documento}
Il Piano di Qualifica è un documento su cui si prevede di operare per l’intera durata del progetto 
e il cui scopo è presentare e descrivere le strategie di verifica e validazione adottate 
dal gruppo Jawa Druids al fine di garantire la qualità di prodotto e di processo.  
Per raggiungere questo obbiettivo viene applicato un sistema di verifica continua sui processi in corso e 
sulle attività svolte, in modo da rilevare e correggere subito eventuali anomalie, riducendo lo spreco di risorse 
ed il rischio di reiterare gli stessi errori.
\section{Scopo del prodotto}
L'obiettivo del prodotto \textit{GDP - Gathering Detection Platform} di \textit{Sync Lab} è la creazione di un sistema software in 
grado di acquisire, monitorare, utilizzare e correlare tra loro tutti i dati acquisiti tramite sensoristica (come telecamere di videosorveglianza)
e altre sorgenti dati (es. orari con il maggior numero di visite degli esercizi commerciali estrapolati da Google) con lo
scopo di avere indicazioni di potenziali assembramenti e quindi di fornire supporto per trovare soluzioni che limitino tali condizioni.
\section{Glossario}
Al fine di favorire la comprensione totale della documentazione relativa al progetto, viene fornito un documento denominato \textit{Glossario -- fornire versione}
che riporta la spiegazione della terminologia specifica e tecnica utilizzata per la stesura di questo ed altri documenti.
Tali termini, quando utilizzati, vengono contrassegnati con un apice $_G$ alla fine della parola.
\section{Standard di progetto}
Decidere se scrivere di aver utilizzato un determinato standard ISO per il software, se si, definirlo nelle norme di progetto!
\section{Riferimenti}
\subsection{Riferimenti normativi}
\begin{itemize}
	\item \textit{Norme di Progetto 1.0.0}
\end{itemize}
\subsection{Riferimenti informativi}
\begin{itemize}
	\item \textit{IEEE Recommended Practice for Software Requirements Specifications:}\\
	\url{https://ieeexplore.ieee.org/document/720574}
	\item \textit{Seminario per approfondimenti tecnici del capitolato C3:}\\
	\url{https://www.math.unipd.it/~tullio/IS-1/2020/Progetto/ST1.pdf}		
\end{itemize}