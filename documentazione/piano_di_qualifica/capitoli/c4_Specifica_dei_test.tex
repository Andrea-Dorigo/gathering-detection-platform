\chapter{Specifica dei test} \label{SpecificaDeiTest}
Per assicurare un’ottima qualità del software prodotto, il gruppo \textit{JawaDruids}, dopo essersi confrontato, ha deciso di utilizzare come modello di sviluppo software il \textit{V-Model}, o Modello a V, il quale è un’estensione del modello a cascata.
Questo modello prevede un lavoro parallelo tra lo sviluppo dei test e le attività di analisi e progettazione.
Grazie a ciò, i test permettono di verificare sia il corretto funzionamento delle parti di software programmate, sia la corretta implementazione di tutti i requisiti del progetto.
Vengono utilizzate delle sigle, all’interno di tabelle, per fornire una comprensione più agevolata riguardo gli output prodotti tramite i test, specificando se il risultato è quello atteso, errato o non coerente con quanto aspettato.
Le sigle per lo stato dei test sono:
\begin{itemize}
	\item \textbf{NI}: non implementato;
	\item \textbf{I}: implementato;
\end{itemize}
Per quando riguarda la qualità dei test si usa:
\begin{itemize}
	\item \textbf{NS}: il test non ha soddisfatto la richiesta;
	\item \textbf{S}: il test ha soddisfatto la richiesta; 
\end{itemize}
Inoltre i test hanno la seguente nomenclatura
\begin{center}
	\textbf{[TipoTest]RS[tipo\_di\_requisito][codice\_requisito]}
\end{center}
dove:
\begin{itemize}
	\item \textbf{TipoTest}: specifica il tipo di test applicato;
	\item \textbf{tipo\_di\_requisito}: assume i seguenti valori
		\begin{itemize}
			\item[-] \textbf{O} per i requisiti obbligatori;
			\item[-] \textbf{D} per i requisiti desiderabili;
			\item[-] \textbf{F} per i requisiti facoltativi;
		\end{itemize}
	\item \textbf{codice\_requisito}: un numero incrementale per rendere univoco il requisito.
\end{itemize}

\section{Tipi di test} \label{4.1}
I test sono che verranno effettuati sul prodotto software sono così divisi:
\begin{itemize}
	\item \textbf{Test di convalida}: i test di convalida hanno come scopo la verifica che il software sviluppato soddisfi i requisiti presenti nel capitolato(g) d’appalto e concordati col proponente.
	Questi saranno eseguiti durante il collaudo finale del prodotto software sotto l'osservazione sia dell'azienda proponente sia del gruppo di lavoro.
	Rappresentati mediante la sigla \textbf{[TC]};
	
	\item \textbf{Test di sistema}: i test di sistema vengono eseguiti per verificare che i requisiti, scritti nel documento \textit{Analisi dei Requisiti}, siano stati implementati e funzionanti.
	Viene rappresentato mediante la sigla \textbf{[TS]};
		
	\item \textbf{Test di integrazione}: questa tipologia di test verifica i singoli moduli del software come fossero un gruppo unico.
	Vengono svolti successivamente ai \textit{TU} e prima dei \textit{TS}.
	Sono contrassegnati da \textbf{[TI]}.
	
	\item \textbf{Test di unità}: i test di unità servono per verificare le singole unità del software, ovvero le componenti con funzionamento autonomo.
	Il superamento di tali test non implica il corretto funzionamento del software.
	Viene contassegnata da \textbf{[TU]}.
\end{itemize}



\quad
\def\tabularxcolumn#1{m{#1}}
{\rowcolors{2}{RawSienna!90!RawSienna!20}{RawSienna!70!RawSienna!40}
	
	\begin{center}
		\renewcommand{\arraystretch}{1.4}
		\begin{tabularx}{\textwidth}{|c|X|c|}
			\hline
			\rowcolor{airforceblue}
			\textbf{Requisito} & \textbf{Descrizione} & \textbf{Esito}\\
			\hline
			\textit{TSRSO1} & L'utente deve poter vedere la heat-map in tempo reale & \textit{NI}\\
			\hline
			\textit{TSRSO2} & L'utente deve poter vedere le heat-map in un intervallo di tempo futuro & \textit{NI}\\
			\hline
			\textit{TSRSO3} & L'utente deve aver accesso ai dati raccolti e storicizzati nel tempo & \textit{NI}\\
			\hline
		\end{tabularx}
	\captionof{table}{\textbf{Elenco test di sistema}}
	\end{center}