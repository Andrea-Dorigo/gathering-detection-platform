\chapter{Specifica dei test} \label{SpecificaDeiTest}

Per assicurare un’ottima qualità del software prodotto, il gruppo \textit{Jawa Druids}, dopo essersi confrontato, ha deciso di utilizzare come modello di sviluppo software il \textit{V-Model}, o Modello a V, il quale è un’estensione del modello a cascata.
Questo modello prevede un lavoro parallelo tra lo sviluppo dei test e le attività$_{\scaleto{G}{3pt}}$ di analisi e progettazione.
Grazie a ciò, i test permettono di verificare sia il corretto funzionamento delle parti di software programmate, sia la corretta implementazione di tutti i requisiti$_{\scaleto{G}{3pt}}$ del progetto.
Vengono utilizzate delle sigle, all’interno di tabelle, per fornire una comprensione più agevolata riguardo gli output prodotti tramite i test, specificando se il risultato è quello atteso, errato o non coerente con quanto aspettato.
Le sigle per lo stato dei test sono:
\begin{itemize}
	\item \textbf{NI}: non implementato;
	\item \textbf{I}: implementato.
\end{itemize}
Per quanto riguarda la qualità dei test si usa:
\begin{itemize}
	\item \textbf{NS}: il test non ha soddisfatto la richiesta;
	\item \textbf{S}: il test ha soddisfatto la richiesta.
\end{itemize}

I test di \textit{Sistema} e \textit{Accettazione} hanno la seguente nomenclatura:

\begin{center}
	\textbf{[TipoTest]RS[classificazione][tipo\_di\_requisito][codice\_requisito]}
\end{center}
dove:
\begin{itemize}
	\item \textbf{TipoTest}: specifica il tipo di test applicato;
	\item \textbf{classificazione}:
	\begin{itemize}
		\item[-] \textbf{F}: indica se il requisito è funzionale;
		\item[-] \textbf{Q}: indica se il requisito è qualitativo;
		\item[-] \textbf{V}: indica se il requisito è vincolante;
		\item[-] \textbf{P}: indica se il requisito è prestazionale.
	\end{itemize}
	\item \textbf{tipo\_di\_requisito}: assume i seguenti valori:
		\begin{itemize}
			\item[-] \textbf{O} per i requisiti$_{\scaleto{G}{3pt}}$ obbligatori;
			\item[-] \textbf{D} per i requisiti$_{\scaleto{G}{3pt}}$ desiderabili;
			\item[-] \textbf{F} per i requisiti$_{\scaleto{G}{3pt}}$ facoltativi.
		\end{itemize}
	\item \textbf{codice\_requisito}: un numero incrementale per rendere univoco il requisito.
\end{itemize}

Invece i test di \textit{Unità}, \textit{Integrazione} e \textit{Regressione} sono denominati nel seguente modo:
\begin{center}
	\textbf{[TipoTest][Id]}
\end{center}
dove:
\begin{itemize}
	\item \textbf{Id} rappresenta un numero incrementale che inizia da 1.
\end{itemize}

\section{Tipi di test} \label{SpecificaDeiTestTipiDiTest}
I test che verranno effettuati sul prodotto software sono così divisi:
\begin{itemize}
	\item \textbf{Test di unità}: i test di unità servono per verificare le singole unità del software, ovvero le componenti con funzionamento autonomo.
	Il superamento di tali test non implica il corretto funzionamento del software.
	Viene contrassegnata da \textbf{[TU]}.
	
	\item \textbf{Test di integrazione}: questa tipologia di test verifica i singoli moduli del software come fossero un gruppo unico.
	Vengono svolti successivamente ai \textit{TU} e prima dei \textit{TS}.
	Sono contrassegnati da \textbf{[TI]};
	
	\item \textbf{Test di sistema}: i test di sistema vengono eseguiti per verificare che i requisiti$_{\scaleto{G}{3pt}}$, scritti nel documento \textit{Analisi dei Requisiti}, siano stati implementati e funzionanti.
	Viene rappresentato mediante la sigla \textbf{[TS]};
	
	\item \textbf{Test di accettazione}: i test di accettazione hanno come scopo la verifica che il software sviluppato soddisfi i requisiti$_{\scaleto{G}{3pt}}$ presenti nel capitolato d’appalto$_G$ e concordati col proponente$_{\scaleto{G}{3pt}}$.
	Questi saranno eseguiti durante il collaudo finale del prodotto software sotto l'osservazione sia dell'azienda proponente$_{\scaleto{G}{3pt}}$ sia del gruppo di lavoro.
	Rappresentati mediante la sigla \textbf{[TA]};

	\item \textbf{Test di regressione}: Servono a garantire il corretto funzionamento del prodotto a seguito di modifiche del codice o di inserimento di nuove funzionalità.
	Vengono etichettati nel seguente modo \textbf{[TR]};

\end{itemize}

\clearpage
\section{Test di unità}\label{SpecificaDeiTestTestDiUnita}
Sono stati individuati i seguenti test di unità per garantire il funzionamento di ogni minimo componente autonomo del sistema.

\begin{center}
	\renewcommand{\arraystretch}{1.4}
	\begin{longtable}{|p{3cm}|p{9cm}|p{2cm}|p{2cm}|}
		\hline
		\rowcolor{airforceblue}
		\multicolumn{4}{|c|}{\textbf{Test di unità Python}} \\
		\hline
		\rowcolor{airforceblue}
		\makecell[c]{\textbf{Id Test}} & \makecell[c]{\textbf{Descrizione}} & \makecell[c]{\textbf{Esito}} & \makecell[c]{\textbf{Qualità}} \\
		\hline
		\centering \textit{TU01} & Verifica che il metodo get{\_}onecall{\_}api{\_}response in weather{\_}forecast.py ottenga i dati delle 48 ore successive & \makecell[tc]{\textit{I}} & \makecell[tc]{\textit{S}} \\
		\hline
		\centering \textit{TU02} & Verifica che il metodo fetch{\_}read{\_}m3u8 in detect.py scarichi e legga correttamente un file m3u8& \makecell[tc]{\textit{I}} & \makecell[tc]{\textit{S}}\\
		\hline
		\centering \textit{TU03} &  Verifica che il metodo extract{\_}frame{\_}from{\_}video{\_}url in get{\_}frames.py estragga e legga correttamente un frame da un video di cui viene fornito il link &\makecell[tc]{\textit{I}} & \makecell[tc]{\textit{S}}\\
		\hline
		\rowcolor{white}
		\caption{\textbf{Elenco test di unità - Python}}
	\end{longtable}
\end{center}

\begin{center}
	\renewcommand{\arraystretch}{1.4}
	\begin{longtable}{|p{3cm}|p{9cm}|p{2cm}|p{2cm}|}
		\hline
		\rowcolor{airforceblue}
		\multicolumn{4}{|c|}{\textbf{Test di unità Spring}} \\
		\hline
		\rowcolor{airforceblue}
		\makecell[c]{\textbf{Id Test}} & \makecell[c]{\textbf{Descrizione}} & \makecell[c]{\textbf{Esito}} & \makecell[c]{\textbf{Qualità}} \\
		\hline
		\centering \textit{TU04} & Verifica che il metodo in cui viene eseguita la query per il recupero della lista delle città presenti nel database ritorni effettivamente il tipo di ritorno corretto, ovvero List$<$String$>$ & \makecell[tc]{\textit{I}} & \makecell[tc]{\textit{S}} \\
		\hline
		\centering \textit{TU05} & Verifica che il metodo in cui viene eseguita la query per il recupero del'ultimo dato presente nel database ritorni effettivamente il tipo di ritorno corretto, ovvero un Detection & \makecell[tc]{\textit{I}} & \makecell[tc]{\textit{S}}\\
		\hline
		\centering \textit{TU06} &  Verifica che il metodo in cui viene eseguita la query per il recupero dei dati presente nel database in base alla città, alla data e all'orario, ritorni effettivamente il tipo di ritorno corretto, ovvero un List$<$Detection$>$ &\makecell[tc]{\textit{I}} & \makecell[tc]{\textit{S}}\\
		\hline
		\centering \textit{TU07} &  Verifica che il metodo in cui viene eseguita la query per il recupero della latitudine e/o longitudine relativa a una città dal database ritorni effettivamente il tipo di ritorno corretto, ovvero List$<$String$>$.
		&\makecell[tc]{\textit{I}} & \makecell[tc]{\textit{S}}\\
		\hline
		\rowcolor{white}
		\caption{\textbf{Elenco test di unità - Spring}}
	\end{longtable}
\end{center}

\begin{center}
	\renewcommand{\arraystretch}{1.4}
	\begin{longtable}{|p{3cm}|p{9cm}|p{2cm}|p{2cm}|}
		\hline
		\rowcolor{airforceblue}
		\multicolumn{4}{|c|}{\textbf{Test di unità Vue.js}} \\
		\hline
		\rowcolor{airforceblue}
		\makecell[c]{\textbf{Id Test}} & \makecell[c]{\textbf{Descrizione}} & \makecell[c]{\textbf{Esito}} & \makecell[c]{\textbf{Qualità}} \\
		\hline
		\centering \textit{TU08} & Verifica che la componente "aboutUs.vue" esista & \makecell[tc]{\textit{I}} & \makecell[tc]{\textit{S}} \\
		\hline
		\centering \textit{TU09} & Verifica che il file "httpcommon.js" esista & \makecell[tc]{\textit{I}} & \makecell[tc]{\textit{S}} \\
		\hline
		\centering \textit{TU10} & Verifica che il metodo "getCities" presente nel file "httprequest.js" esista & \makecell[tc]{\textit{I}} & \makecell[tc]{\textit{S}} \\
		\hline
		\centering \textit{TU11} & Verifica che il metodo "getCoo" presente nel file "httprequest.js" esista & \makecell[tc]{\textit{I}} & \makecell[tc]{\textit{S}} \\
		\hline
		\centering \textit{TU12} & Verifica che il metodo "getDataRT" presente nel file "httprequest.js" esista & \makecell[tc]{\textit{I}} & \makecell[tc]{\textit{S}} \\
		\hline
		\centering \textit{TU13} & Verifica che il metodo "getLastValue" presente nel file "httprequest.js" esista & \makecell[tc]{\textit{I}} & \makecell[tc]{\textit{S}} \\
		\hline
		\centering \textit{TU14} & Verifica che la componente "autocompleteSearch.vue" esista & \makecell[tc]{\textit{I}} & \makecell[tc]{\textit{S}} \\
		\hline
		\centering \textit{TU15} & Verifica che il tag di input della componente "autocompleteSearch.vue" sia vuoto & \makecell[tc]{\textit{I}} & \makecell[tc]{\textit{S}} \\
		\hline
		\centering \textit{TU16} & Verifica che il tag di input della componente "autocompleteSearch.vue" non sia vuoto dopo che l'utente interagisce scrivendo & \makecell[tc]{\textit{I}} & \makecell[tc]{\textit{S}} \\
		\hline
		\centering \textit{TU17} & Verifica che il metodo "getNameCity" presente nel file "autocompleteSearch.vue" esista & \makecell[tc]{\textit{I}} & \makecell[tc]{\textit{S}} \\
		\hline
		\centering \textit{TU18} & Verifica che il metodo "getNameCity" presente nel file "autocompleteSearch.vue" venga chiamato almeno una volta & \makecell[tc]{\textit{I}} & \makecell[tc]{\textit{S}} \\
		\hline
		\centering \textit{TU19} & Verifica che il metodo "itemSelected" presente nel file "autocompleteSearch.vue" esista & \makecell[tc]{\textit{I}} & \makecell[tc]{\textit{S}} \\
		\hline
		\centering \textit{TU20} & Verifica che la componente "datePicker.vue" esista & \makecell[tc]{\textit{I}} & \makecell[tc]{\textit{NS}} \\
		\hline
		\centering \textit{TU21} & Verifica che la componente "slider.vue" esista & \makecell[tc]{\textit{I}} & \makecell[tc]{\textit{NS}} \\
		\hline
		\centering \textit{TU22} & Verifica che il metodo "getTime" presente nel file "slider.vue" esista & \makecell[tc]{\textit{I}} & \makecell[tc]{\textit{NS}} \\
		\hline
		\centering \textit{TU23} & Verifica che la componente "app.vue" esista & \makecell[tc]{\textit{I}} & \makecell[tc]{\textit{S}} \\
		\hline
		\centering \textit{TU24} & Verifica che la componente "listCity.vue" esista & \makecell[tc]{\textit{I}} & \makecell[tc]{\textit{NS}} \\
		\hline
		\centering \textit{TU25} & Verifica che la componente "heatmap.vue" esista & \makecell[tc]{\textit{I}} & \makecell[tc]{\textit{NS}} \\
		\hline
		\centering \textit{TU26} & Verifica che la componente "vue2LeafletHeatmap.vue" esista & \makecell[tc]{\textit{I}} & \makecell[tc]{\textit{NS}} \\
		\hline
		\centering \textit{TU27} & Verifica che la componente "mainPage.vue" esista & \makecell[tc]{\textit{I}} & \makecell[tc]{\textit{S}} \\
		\hline
		\centering \textit{TU28} & Verifica che il metodo "getRetrieveCoordinate" presente nel file "mainPage.vue" venga chiamato dopo il click dell'utente sul bottone & \makecell[tc]{\textit{I}} & \makecell[tc]{\textit{S}} \\
		\hline
		\centering \textit{TU29} & Verifica che la componente "heatmap.vue" presente nel file "mainPage.vue" esista & \makecell[tc]{\textit{I}} & \makecell[tc]{\textit{S}} \\
		\hline
		\rowcolor{white}
		\caption{\textbf{Elenco test di unità - Vue.js}}
	\end{longtable}
\end{center}


\subsection{Tracciamento dei test}\label{SpecificaDeiTestTestDiUnitaTracciamentoDeiTest}

\begin{center}
	\renewcommand{\arraystretch}{1.4}
	\begin{longtable}{|p{1.5cm}|p{9cm}|p{6cm}|}
		\hline
		\rowcolor{airforceblue}
		\multicolumn{3}{|c|}{\textbf{Tracciamento test di unità Python}} \\
		\hline
		\rowcolor{airforceblue}
		\makecell[c]{\textbf{Id Test}} & \makecell[c]{\textbf{Percorso file}} & \makecell[c]{\textbf{Metodo}} \\
		\hline
		\centering TU01	& \makecell[c]{acquisition/main/test/test{\_}weather{\_}forecast.py} & \makecell[c]{test{\_}response}\\
		\hline
		\centering TU02 & \makecell[c]{acquisition/main/test/test{\_}detect.py} & \makecell[c]{test{\_}fetch{\_}read{\_}m3u8}\\
		\hline
		\centering TU03 & \makecell[c]{acquisition/main/test/test{\_}detect.py} & \makecell[c]{test{\_}extract{\_}frame{\_}from{\_}video{\_}url}\\
		\hline
		\rowcolor{white}
		\caption{\textbf{Tracciamento dei test di unità - Python}}
	\end{longtable}
\end{center}

\begin{center}
	\renewcommand{\arraystretch}{1.4}
	\begin{longtable}{|p{1.5cm}|p{11.5cm}|p{3.5cm}|}
		\hline
		\rowcolor{airforceblue}
		\multicolumn{3}{|c|}{\textbf{Tracciamento test di unità Spring}} \\
		\hline
		\rowcolor{airforceblue}
		\makecell[c]{\textbf{Id Test}} & \makecell[c]{\textbf{Percorso file}} & \makecell[c]{\textbf{Metodo}} \\
		\hline
		\centering TU04	& \makecell[c]{proof{\_}of{\_}concept/webapp/webapp/src/test/java/com/webapp/\\data/mongodb/DetectionCustomRepositoryImplTest/\\DetectionCustomRepositoryImplTest.java} & \makecell[c]{getCitiesTest()}\\
		\hline
		\centering TU05 & \makecell[c]{proof{\_}of{\_}concept/webapp/webapp/src/test/java/com/webapp/\\data/mongodb/DetectionCustomRepositoryImplTest/\\DetectionCustomRepositoryImplTest.java} & \makecell[c]{getLastValueTest()}\\
		\hline
		\centering TU06 & \makecell[c]{proof{\_}of{\_}concept/webapp/webapp/src/test/java/com/webapp/\\data/mongodb/DetectionCustomRepositoryImplTest/\\DetectionCustomRepositoryImplTest.java} & \makecell[c]{getDataRTTest()}\\
		\hline
		\centering TU07 & \makecell[c]{proof{\_}of{\_}concept/webapp/webapp/src/test/java/com/webapp/\\data/mongodb/DetectionCustomRepositoryImplTest/\\DetectionCustomRepositoryImplTest.java} & \makecell[c]{getLatLngsTest()}\\
		\hline
		\rowcolor{white}
		\caption{\textbf{Tracciamento dei test di unità - Spring}}
	\end{longtable}	
\end{center}

\begin{center}
	\renewcommand{\arraystretch}{1.4}
	\begin{longtable}{|p{1.5cm}|p{11.5cm}|p{3.5cm}|}
		\hline
		\rowcolor{airforceblue}
		\multicolumn{3}{|c|}{\textbf{Tracciamento test di unità Vue.js}} \\
		\hline
		\rowcolor{airforceblue}
		\makecell[c]{\textbf{Id Test}} & \makecell[c]{\textbf{Percorso file}} & \makecell[c]{\textbf{Metodo}} \\
		\hline
		\centering TU08 & \makecell[c]{proof{\_}of{\_}concept/webapp/vue-js-client-crud/src/tests/unit/\\appTest.spec.js} & \makecell[c]{}\\
		\hline
		\rowcolor{white}
		\caption{\textbf{Tracciamento dei test di unità - Vue.js}}
	\end{longtable}	
\end{center}

\section{Test di integrazione}\label{SpecificaDeiTestTestDiIntegrazione}
Sono stati individuati i seguenti test di integrazione per garantire il funzionamento dei diversi moduli nel momento in cui vengono messi in relazione tra di loro.
\def\tabularxcolumn#1{m{#1}}
{\rowcolors{2}{RawSienna!90!RawSienna!20}{RawSienna!70!RawSienna!40}
	
	\begin{center}
		\renewcommand{\arraystretch}{1.4}
		\begin{longtable}{|p{3cm}|p{8cm}|p{2cm}|p{2cm}|}
			\hline
			\rowcolor{airforceblue}
			\makecell[c]{\textbf{Id Test}} & \makecell[c]{\textbf{Descrizione}} & \makecell[c]{\textbf{Esito}} & \makecell[c]{\textbf{Qualità}} \\
			\hline
			\centering \textit{TI01} & Verifica che il sistema acquisica i dati attraverso il modulo contapersone che utlizza l'algoritmo di Yolo v3 & \makecell[tc]{\textit{I}} & \makecell[tc]{\textit{S}} \\
			\hline
			\centering \textit{TI02} & Verifica che il sistema invii i dati ottenuti dal modulo contapersone al database in modo corretto & \makecell[tc]{\textit{I}} & \makecell[tc]{\textit{S}}\\
			\hline
			\centering \textit{TI03} &  Verifica che il sistema invii i dati ottenuti dal modulo di machine learning$_{\scaleto{G}{3pt}}$ al database in modo corretto &\makecell[tc]{\textit{I}} & \makecell[tc]{\textit{S}}\\
			\hline
			\centering \textit{TI04} &  Verifica che il sistema invii i dati presenti nel  database alla web application in modo corretto &\makecell[tc]{\textit{I}} & \makecell[tc]{\textit{S}}\\
			\hline
			\rowcolor{white}
			\caption{\textbf{Elenco test di accettazione}}
		\end{longtable}
	\end{center}

\section{Test di sistema}\label{SpecificaDeiTestTestDiSistema}
Sono stati individuati i seguenti test di sistema per garantire il funzionamento del prodotto sviluppato. I test di sistema sono stati identificati attraverso i requisiti indicati nel documento \textit{Analisi dei Requisiti 3.0.0}. 
\def\tabularxcolumn#1{m{#1}}
{\rowcolors{2}{RawSienna!90!RawSienna!20}{RawSienna!70!RawSienna!40}
	
	\begin{center}
		\renewcommand{\arraystretch}{1.4}
		\begin{longtable}{|p{3cm}|p{8cm}|p{2cm}|p{2cm}|}
			\hline
			\rowcolor{airforceblue}
			\makecell[c]{\textbf{Id Test}} & \makecell[c]{\textbf{Descrizione}} & \makecell[c]{\textbf{Esito}} & \makecell[c]{\textbf{Qualità}} \\
			\hline
			\textit{TSRSFO1} & Verifica che il sistema utilizzi motori software 'contapersone' & \makecell[tc]{\textit{I}} & \makecell[tc]{\textit{S}} \\
			\hline
			\textit{TSRSFF2} & Verifica che il sistema utilizzi simulatori di dati storici & \makecell[tc]{\textit{NI}} & \makecell[tc]{\textit{-}}\\
			\hline
			\textit{TSRSFO3} & Verifica della visualizzazione di un messaggio d'errore in caso mancanza dati nella generazione della heat-map$_G$ &\makecell[tc]{\textit{I}} & \makecell[tc]{\textit{S}}\\
			\hline
			\textit{TSRSFO4} & Verifica che il sistema archivi tutti i dati nel database & \makecell[tc]{\textit{I}} & \makecell[tc]{\textit{S}}\\
			\hline
			\textit{TSRSFO5} & Verifica che il sistema elabori i dati dalle sorgenti esterne in tempo reale & \makecell[tc]{\textit{I}} & \makecell[tc]{\textit{S}}\\
			\hline
			\textit{TSRSFD6} & Verifica che il sistema effettui una previsione dell’insorgenza futura di variazioni significative di flussi di persone & \makecell[tc]{\textit{NI}} & \makecell[tc]{\textit{-}}\\
			\hline
			\textit{TSRSFO7} & Verifica che l’utente possa poter visualizzare i dati in tempo reale tramite heat map$_{\scaleto{G}{3pt}}$ & \makecell[tc]{\textit{I}} & \makecell[tc]{\textit{S}}\\
			\hline
			\textit{TSRSFO8} & Verifica che l’utente possa poter visualizzare i dati storicizzati tramite heat map$_{\scaleto{G}{3pt}}$ & \makecell[tc]{\textit{I}} & \makecell[tc]{\textit{S}}\\
			\hline
			\textit{TSRSFO9} & Verifica che l’utente possa poter visualizzare una previsione tramite heat map$_{\scaleto{G}{3pt}}$ & \makecell[tc]{\textit{I}} & \makecell[tc]{\textit{S}}\\
			\hline
			\textit{TSRSFF10} & Verifica che l'utente possa poter distinguere tra i dati simulati e quelli reali & \makecell[tc]{\textit{NI}} & \makecell[tc]{\textit{-}}\\
			\hline
			\textit{TSRSFD11} & Verfica che l’utente possa poter visualizzare un indice di affidabilità della previsione nella mappa & \makecell[tc]{\textit{NI}} & \makecell[tc]{\textit{-}}\\
			\hline
			\textit{TSRSFD12} & Verifica che l’utente possa poter visualizzare un indice di affidabilità dei dati in tempo reale nella mappa & \makecell[tc]{\textit{NI}} & \makecell[tc]{\textit{-}}\\
			\hline
			\textit{TSRSFF13} & Verifica che l’utente possa poter applicare dei filtri ai dati (reali, simulati) & \makecell[tc]{\textit{NI}} & \makecell[tc]{\textit{-}}\\
			\hline
			\textit{TSRSFF14} & Verifica che l’utente abbia la possibilità di scegliere le sorgenti dati da cui prelevare dati tempo reale & \makecell[tc]{\textit{NI}} & \makecell[tc]{\textit{-}}\\
			\hline
			\textit{TSRSFO15} & Verifica che il sistema aggiorni la mappa automaticamente ogni 10 minuti & \makecell[tc]{\textit{I}} & \makecell[tc]{\textit{S}}\\
			\hline
			\textit{TSRSFO16} & Verifica che il modello di machine learning$_{\scaleto{G}{3pt}}$ salvi i pesi e le predizioni in un file & \makecell[tc]{\textit{I}} & \makecell[tc]{\textit{S}}\\
			\hline
			\textit{TSRSFO17} & Verifica che venga inviato un messaggio di errore al front end , dal backend, se non ci sono i dati richiesti & \makecell[tc]{\textit{I}} & \makecell[tc]{\textit{S}}\\
			\hline
			\textit{TSRSFO18} & Verifica che l’utente possa selezionare una città tra quelle disponibili & \makecell[tc]{\textit{I}} & \makecell[tc]{\textit{S}}\\
			\hline
			\textit{TSRSFO19} & Verifica che l'utente visualizzi le zone delle città rispettivamente alle zone utilizzate & \makecell[tc]{\textit{I}} & \makecell[tc]{\textit{S}}\\
			\hline
			\textit{TSRSFO20} & Verifica che il sistema archivi i dati in tempo reale con la data e orario di riferimento associata & \makecell[tc]{\textit{I}} & \makecell[tc]{\textit{S}}\\
			\hline
			\textit{TSRSFF21} & Verifica che il sistema utilizzi i dati delle predizioni in caso di mancanza di dati in tempo reale & \makecell[tc]{\textit{I}} & \makecell[tc]{\textit{S}}\\
			\hline
			\textit{TSRSFO22} & Verifica che l'utente possa selezionare l'intervallo orario in fasce orarie & \makecell[tc]{\textit{I}} & \makecell[tc]{\textit{S}}\\
			\hline
			\textit{TSRSFO23} & Verifica che il sistema utilizzi in modo prioritario i dati reali se presenti anche quelli determinati per le predizioni & \makecell[tc]{\textit{I}} & \makecell[tc]{\textit{S}}\\
			\hline
			\textit{TSRSFO24} & Verifica che il sistema aggiorni automaticamente la mappa alla selezione di un diverso orario & \makecell[tc]{\textit{I}} & \makecell[tc]{\textit{S}}\\
			\hline
			\textit{TSRSFO25} & Verifica che l’utente possa poter selezionare la data del giorno di cui vuole visualizzare i dati & \makecell[tc]{\textit{I}} & \makecell[tc]{\textit{S}}\\
			\hline
			\textit{TSRSFO26} & Verifica che l’utente possa poter ripristinare la visione in tempo reale tramite un pulsante di ripristino & \makecell[tc]{\textit{I}} & \makecell[tc]{\textit{S}}\\
			\hline
			\textit{TSRSFF27} & Verifica che l’utente possa poter reperire il manuale d'uso & \makecell[tc]{\textit{NI}} & \makecell[tc]{\textit{-}}\\
			\hline
			\textit{TSRSFO28} & Verifica che l’utente possa poter variare il livello di zoom della heat map$_{\scaleto{G}{3pt}}$ & \makecell[tc]{\textit{I}} & \makecell[tc]{\textit{S}}\\
			\hline
			\textit{TSRSFO28.1} & Verifica che l’utente possa poter aumentare il livello di zoom della heat map$_{\scaleto{G}{3pt}}$ & \makecell[tc]{\textit{I}} & \makecell[tc]{\textit{S}}\\
			\hline
			\textit{TSRSFO28.1.1} & Verifica che l’utente possa poter attuare il drag$_{\scaleto{G}{3pt}}$ della heat map$_{\scaleto{G}{3pt}}$ & \makecell[tc]{\textit{I}} & \makecell[tc]{\textit{S}}\\
			\hline
			\textit{TSRSFO28.1.2} & Verifica che l’utente possa poter visualizzare il pop-up$_{\scaleto{G}{3pt}}$ legato ad un punto di interesse & \makecell[tc]{\textit{I}} & \makecell[tc]{\textit{S}}\\
			\hline
			\textit{TSRSFO28.1.3} & Verifica che l’utente possa poter chiudere il pop-up$_{\scaleto{G}{3pt}}$ legato ad un punto di interesse & \makecell[tc]{\textit{I}} & \makecell[tc]{\textit{S}}\\
			\hline
			\textit{TSRSFO28.2} & Verifica che l’utente possa poter diminuire il livello di zoom della heat map$_{\scaleto{G}{3pt}}$ & \makecell[tc]{\textit{I}} & \makecell[tc]{\textit{S}}\\
			\hline
			\textit{TSRSFD29} & Verifica che l’utente possa poter ricercare tramite una barra di ricerca le città presenti nel database & \makecell[tc]{\textit{I}} & \makecell[tc]{\textit{S}}\\
			\hline
			\textit{TSRSFD30} & Verifica che l’utente possa poter visualizzare il messaggio d'errore relativo alla mancanza dei dati ricercati attraverso la barra di ricerca nel database & \makecell[tc]{\textit{NI}} & \makecell[tc]{\textit{-}}\\
			\hline
			\textit{TSRSFD31} & Verifica che l’utente possa poter visualizzare il confronto dei dati di due città selezionate dall'utente & \makecell[tc]{\textit{NI}} & \makecell[tc]{\textit{-}}\\
			\hline
			\textit{TSRSFD32} & Verifica che l’utente possa poter salvare in un file locale i dati della città della mappa che sta visualizzando & \makecell[tc]{\textit{NI}} & \makecell[tc]{\textit{-}}\\
			\hline
			\textit{TSRSFD33} & Verifica che l’utente possa poter inserire l'email per il ricevimento delle informazioni delle informazioni della città selezionata & \makecell[tc]{\textit{NI}} & \makecell[tc]{\textit{-}}\\
			\hline
			\textit{TSRSFD33.1} & Verifica che l’utente possa poter visualizzare un messaggio di errore nel caso l'email inserita sia scritta in modo errato & \makecell[tc]{\textit{NI}} & \makecell[tc]{\textit{-}}\\
			\hline
			\textit{TSRSFD34} & Verifica che il sistema abbia inserito correttamente l'email e la città correlata nel database & \makecell[tc]{\textit{NI}} & \makecell[tc]{\textit{-}}\\
			\hline
			\textit{TSRSFD35} & Verifica che il sistema invii correttamente l'email all'utente & \makecell[tc]{\textit{NI}} & \makecell[tc]{\textit{-}}\\
			\hline
			\textit{TSRSFD36} & Verifica che l’utente possa poter visualizzare la lista delle città più ricercate & \makecell[tc]{\textit{NI}} & \makecell[tc]{\textit{-}}\\
			\hline
			\textit{TSRSFD37} & Verifica che l’utente possa poter visualizzare la lista di tutte le città presenti nel database & \makecell[tc]{\textit{NI}} & \makecell[tc]{\textit{-}}\\
			\hline
			\rowcolor{white}
			\caption{\textbf{Elenco test di sistema}}\\
		\end{longtable}
		
	\end{center}
	\def\tabularxcolumn#1{m{#1}}
	{\rowcolors{2}{RawSienna!90!RawSienna!20}{RawSienna!70!RawSienna!40}
		
		\subsection{Tracciamento dei test}\label{SpecificaDeiTestTestDiSistemaTracciamentoDeiTest}
		
		\begin{center}
			\renewcommand{\arraystretch}{1.4}
			\begin{longtable}{|p{3cm}|p{3cm}|}
				\hline
				\rowcolor{airforceblue}
				\makecell[c]{\textbf{Id Test}} & \makecell[c]{\textbf{Id Requisito}} \\
				\hline
				\hline
				TSRSFO1	& \makecell{RSFO1} \\
				\hline
				TSRSFF2 & \makecell{RSFF2} \\
				\hline
				TSRSFO3 & \makecell{RSFO3}  \\
				\hline
				TSRSFO4 & \makecell{RSFO4 \\ RSFO4.1 \\ RSFO4.2} \\
				\hline
				TSRSFO5 & \makecell{RSFO5} \\
				\hline
				TSRSFD6 & \makecell{RSFD6} \\
				\hline
				TSRSFO7 & \makecell{RSFO9} \\
				\hline
				TSRSFO8 & \makecell{RSFO10} \\
				\hline
				TSRSFO9 & \makecell{RSFO7 \\ RSFO11} \\
				\hline
				TSRSFF10 & \makecell{RSFF12} \\
				\hline
				TSRSFD11 & \makecell{RSFD13} \\
				\hline
				TSRSFD12 & \makecell{RSFD14} \\
				\hline
				TSRSFF13 & \makecell{RSFF15} \\
				\hline
				TSRSFF14 & \makecell{RSFF16} \\
				\hline
				TSRSFO15 & \makecell{RSFO17} \\
				\hline
				TSRSFO16 & \makecell{RSFO18 \\ RSFO18.1 }\\
				\hline
				TSRSFO17 & \makecell{RSFO19} \\
				\hline
				TSRSFO18 & \makecell{RSFO20} \\
				\hline
				TSRSFO19 & \makecell{RSFO21} \\
				\hline
				TSRSFO20 & \makecell{RSFO22 \\ RSFO22.1} \\
				\hline
				TSRSFF21 & \makecell{RSFF23} \\
				\hline
				TSRSFO22 & \makecell{RSFO24} \\
				\hline
				TSRSFO23 & \makecell{RSFO25} \\
				\hline
				TSRSFO24 & \makecell{RSFO26} \\
				\hline
				TSRSFO25 & \makecell{RSFO27} \\
				\hline
				TSRSFO26 & \makecell{RSFO28} \\
				\hline
				TSRSFF27 & \makecell{RSFF31} \\
				\hline
				TSRSFO28 & \makecell{RSFO32} \\
				\hline
				TSRSFO28.1 & \makecell{RSFO32.1} \\
				\hline
				TSRSFO28.1.1 & \makecell{RSFO32.1.1} \\
				\hline
				TSRSFO28.1.2 & \makecell{RSFO32.1.2} \\
				\hline
				TSRSFO28.1.3 & \makecell{RSFO32.1.3} \\
				\hline
				TSRSFO28.2 & \makecell{RSFO32.2}\\
				\hline
				TSRSFD29 & \makecell{RSFD33 \\RSFD33.1 \\ RSFD33.2}\\
				\hline
				TSRSFD30 & \makecell{RSFD34} \\
				\hline
				TSRSFD31 & \makecell{RSFD35} \\
				\hline
				TSRSFD32 & \makecell{RSFD36 \\ RSFD36.1 \\ RSFD36.2}\\
				\hline
				TSRSFD33 & \makecell{RSFD37}\\
				\hline
				TSRSFD33.1 & \makecell{RSFD37.1}\\
				\hline
				TSRSFD34 & \makecell{RSFD38}\\
				\hline
				TSRSFD35 & \makecell{RSFD39}\\
				\hline
				TSRSFD36 & \makecell{RSFD40} \\
				\hline
				TSRSFD37 & \makecell{RSFD41} \\
				\hline
				\rowcolor{white}
				\caption{\textbf{Tracciamento dei test di sistema con i requisiti}}
			\end{longtable}
			
		\end{center}

\section{Test di accettazione}\label{SpecificaDeiTestTestDiAccettazione}
\textit{Essendo una tipologia di test che verrà eseguita in periodo di accettazione il gruppo si impegna a completare questa sezione in previsione della Revsione di Accettazione.}
