\chapter{Specifica dei test} \label{SpecificaDeiTest}

Per assicurare un’ottima qualità del software prodotto, il gruppo \textit{Jawa Druids}, dopo essersi confrontato, ha deciso di utilizzare come modello di sviluppo software il \textit{V-Model}, o Modello a V, il quale è un’estensione del modello a cascata.
Questo modello prevede un lavoro parallelo tra lo sviluppo dei test e le attività$_{\scaleto{G}{3pt}}$ di analisi e progettazione.
Grazie a ciò, i test permettono di verificare sia il corretto funzionamento delle parti di software programmate, sia la corretta implementazione di tutti i requisiti$_{\scaleto{G}{3pt}}$ del progetto.
Vengono utilizzate delle sigle, all’interno di tabelle, per fornire una comprensione più agevolata riguardo gli output prodotti tramite i test, specificando se il risultato è quello atteso, errato o non coerente con quanto aspettato.
Le sigle per lo stato dei test sono:
\begin{itemize}
	\item \textbf{NI}: non implementato;
	\item \textbf{I}: implementato.
\end{itemize}
Per quando riguarda la qualità dei test si usa:
\begin{itemize}
	\item \textbf{NS}: il test non ha soddisfatto la richiesta;
	\item \textbf{S}: il test ha soddisfatto la richiesta.
\end{itemize}

I test di \textit{Sistema} e \textit{Accettazione} hanno la seguente nomenclatura:
\begin{center}
	\textbf{[TipoTest]RS[classificazione][tipo\_di\_requisito][codice\_requisito]}
\end{center}
dove:
\begin{itemize}
	\item \textbf{TipoTest}: specifica il tipo di test applicato;
	\item \textbf{classificazione}:
	\begin{itemize}
		\item[-] \textbf{F}: indica se il requisito è funzionale;
		\item[-] \textbf{Q}: indica se il requisito è qualitativo;
		\item[-] \textbf{V}: indica se il requisito è vincolante;
		\item[-] \textbf{P}: indica se il requisito è prestazionale.
	\end{itemize}
	\item \textbf{tipo\_di\_requisito}: assume i seguenti valori:
		\begin{itemize}
			\item[-] \textbf{O} per i requisiti$_{\scaleto{G}{3pt}}$ obbligatori;
			\item[-] \textbf{D} per i requisiti$_{\scaleto{G}{3pt}}$ desiderabili;
			\item[-] \textbf{F} per i requisiti$_{\scaleto{G}{3pt}}$ facoltativi.
		\end{itemize}
	\item \textbf{codice\_requisito}: un numero incrementale per rendere univoco il requisito.
\end{itemize}

Invece i test di \textit{Unità}, \textit{Integrazione} e \textit{Regressione} sono denominati nel seguente modo:
\begin{center}
	\textbf{[TipoTest][Id]}
\end{center}
dove:
\begin{itemize}
	\item \textbf{Id} rappresenta un numero incrementale che inizia da 1.
\end{itemize}

\section{Tipi di test} \label{SpecificaDeiTestTipiDiTest}
I test che verranno effettuati sul prodotto software sono così divisi:
\begin{itemize}
	\item \textbf{Test di accettazione}: i test di accettazione hanno come scopo la verifica che il software sviluppato soddisfi i requisiti$_{\scaleto{G}{3pt}}$ presenti nel capitolato d’appalto$_G$ e concordati col proponente$_{\scaleto{G}{3pt}}$.
	Questi saranno eseguiti durante il collaudo finale del prodotto software sotto l'osservazione sia dell'azienda proponente$_{\scaleto{G}{3pt}}$ sia del gruppo di lavoro.
	Rappresentati mediante la sigla \textbf{[TA]};

	\item \textbf{Test di sistema}: i test di sistema vengono eseguiti per verificare che i requisiti$_{\scaleto{G}{3pt}}$, scritti nel documento \textit{Analisi dei Requisiti}, siano stati implementati e funzionanti.
	Viene rappresentato mediante la sigla \textbf{[TS]};

	\item \textbf{Test di integrazione}: questa tipologia di test verifica i singoli moduli del software come fossero un gruppo unico.
	Vengono svolti successivamente ai \textit{TU} e prima dei \textit{TS}.
	Sono contrassegnati da \textbf{[TI]};

	\item \textbf{Test di regressione}: Servono a garantire il corretto funzionamento del prodotto a seguito di modifiche del codice o di inserimento di nuove funzionalità.
	Vengono etichettati nel seguente modo \textbf{[TR]};

	\item \textbf{Test di unità}: i test di unità servono per verificare le singole unità del software, ovvero le componenti con funzionamento autonomo.
	Il superamento di tali test non implica il corretto funzionamento del software.
	Viene contrassegnata da \textbf{[TU]}.
\end{itemize}
\section{Test di sistema}\label{TestDiSistema}
Sono stati individuati i seguenti test di sistema per garantire il funzionamento del prodotto sviluppato. I test di sistema sono stati identificati attraverso i requisiti indicati nel documento \textit{Analisi dei Requisiti 3.0.0}. 
\def\tabularxcolumn#1{m{#1}}
{\rowcolors{2}{RawSienna!90!RawSienna!20}{RawSienna!70!RawSienna!40}
	
	\begin{center}
		\renewcommand{\arraystretch}{1.4}
		\begin{longtable}{|p{3cm}|p{8cm}|p{3cm}|}
			\hline
			\rowcolor{airforceblue}
			\makecell[c]{\textbf{Id Test}} & \makecell[c]{\textbf{Descrizione}} & \makecell[c]{\textbf{Esito}} \\
			\hline
			\textit{TSRSFO1} & Verifica che il sistema utilizzi motori software 'contapersone' & \makecell[tc]{\textit{NI}} \\
			\hline
			\textit{TSRSFF2} & Verifica che il sistema utilizzi simulatori di dati storici & \makecell[tc]{\textit{NI}}\\
			\hline
			\textit{TSRSFO3} & Verifica della visualizzazione di un messaggio d'errore in caso mancanza dati nella generazione della heat-map$_G$ &\makecell[tc]{\textit{NI}}\\
			\hline
			\textit{TSRSFO4} & Verifica che il sistema archivi tutti i dati nel database & \makecell[tc]{\textit{NI}}\\
			\hline
			\textit{TSRSFO5} & Verifica che il sistema elabori i dati dalle sorgenti esterne in tempo reale & \makecell[tc]{\textit{NI}}\\
			\hline
			\textit{TSRSFO7} & Verifica della visualizzazione dei dati elaborati attraverso heat map$_{\scaleto{G}{3pt}}$ & \makecell[tc]{\textit{NI}}\\
			\hline
			\textit{TSRSFO9} & Verifica che l’utente possa poter visualizzare i dati in tempo reale tramite heat map$_{\scaleto{G}{3pt}}$ & \makecell[tc]{\textit{NI}}\\
			\hline
			\textit{TSRSFO10} & Verifica che l’utente possa poter visualizzare i dati storicizzati tramite heat map$_{\scaleto{G}{3pt}}$ & \makecell[tc]{\textit{NI}}\\
			\hline
			\textit{TSRSFO11} & Verifica che l’utente possa poter visualizzare una previsione tramite heat map$_{\scaleto{G}{3pt}}$ & \makecell[tc]{\textit{NI}}\\
			\hline
			\textit{TSRSFF12} & Verifica che l'utente possa poter distinguere tra i dati simulati e quelli reali & \makecell[tc]{\textit{NI}}\\
			\hline
			\textit{TSRSFD13} & Verfica che l’utente possa poter visualizzare un indice di affidabilità della previsione nella mappa & \makecell[tc]{\textit{NI}}\\
			\hline
			\textit{TSRSFD14} & Verifica che l’utente possa poter visualizzare un indice di affidabilità dei dati in tempo reale nella mappa & \makecell[tc]{\textit{NI}}\\
			\hline
			\textit{TSRSFF15} & Verifica che l’utente possa poter applicare dei filtri ai dati (reali, simulati) & \makecell[tc]{\textit{NI}}\\
			\hline
			\textit{TSRSFF16} & Verifica che l’utente abbia la possibilità di scegliere le sorgenti dati da cui prelevare dati tempo reale & \makecell[tc]{\textit{NI}}\\
			\hline
			\textit{TSRSFO17} & Verifica che il sistema aggiorni la mappa automaticamente ogni 10 minuti & \makecell[tc]{\textit{NI}}\\
			\hline
			\textit{TSRSFO18} & Verifica che il modello di machine learning salvi i pesi e le predizioni in un file & \makecell[tc]{\textit{NI}}\\
			\hline
			\textit{TSRSFO19} & Verifica che venga inviato un messaggio di errore al front end , dal backend, se non ci sono i dati richiesti & \makecell[tc]{\textit{NI}}\\
			\hline
			\textit{TSRSFO20} & Verifica che l’utente possa selezionare una città tra quelle disponibili & \makecell[tc]{\textit{NI}}\\
			\hline
			\textit{TSRSFO21} & Verifica che l'utente visualizzi le zone delle città rispettivamente alle zone utilizzate & \makecell[tc]{\textit{NI}}\\
			\hline
			\textit{TSRSFO22} & Verifica che il sistema archivi i dati in tempo reale con la data e orario di riferimento associata & \makecell[tc]{\textit{NI}}\\
			\hline
			\textit{TSRSFF23} & Verifica che il sistema utilizzi i dati delle predizioni in caso di mancanza di dati in tempo reale & \makecell[tc]{\textit{NI}}\\
			\hline
			\textit{TSRSFO24} & Verifica che l'utente possa selezionare l'intervallo orario in fasce orarie & \makecell[tc]{\textit{NI}}\\
			\hline
			\textit{TSRSFO25} & Verifica che il sistema utilizzi in modo prioritario i dati reali se presenti anche quelli determinati per le predizioni & \makecell[tc]{\textit{NI}}\\
			\hline
			\textit{TSRSFO26} & Verifica che il sistema aggiorni automaticamente la mappa alla selezione di un diverso orario & \makecell[tc]{\textit{NI}}\\
			\hline
			\textit{TSRSFO27} & Verifica che l’utente possa poter selezionare la data del giorno di cui vuole visualizzare i dati & \makecell[tc]{\textit{NI}}\\
			\hline
			\textit{TSRSFO28} & Verifica che l’utente possa poter ripristinare la visione in tempo reale tramite un pulsante di ripristino & \makecell[tc]{\textit{NI}}\\
			\hline
			\textit{TSRSFF31} & Verifica che l’utente possa poter reperire il manuale d'uso & \makecell[tc]{\textit{NI}}\\
			\hline
			\textit{TSRSFO32} & Verifica che l’utente possa poter variare il livello di zoom della heat map$_{\scaleto{G}{3pt}}$ & \makecell[tc]{\textit{NI}}\\
			\hline
			\textit{TSRSFO32.1} & Verifica che l’utente possa poter aumentare il livello di zoom della heat map$_{\scaleto{G}{3pt}}$ & \makecell[tc]{\textit{NI}}\\
			\hline
			\textit{TSRSFO32.1.1} & Verifica che l’utente possa poter attuare il drag$_{\scaleto{G}{3pt}}$ della heat map$_{\scaleto{G}{3pt}}$ & \makecell[tc]{\textit{NI}}\\
			\hline
			\textit{TSRSFO32.1.2} & Verifica che l’utente possa poter visualizzare il pop-up$_{\scaleto{G}{3pt}}$ legato ad un punto di interesse & \makecell[tc]{\textit{NI}}\\
			\hline
			\textit{TSRSFO32.1.3} & Verifica che l’utente possa poter chiudere il pop-up$_{\scaleto{G}{3pt}}$ legato ad un punto di interesse & \makecell[tc]{\textit{NI}}\\
			\hline
			\textit{TSRSFO32.2} & Verifica che l’utente possa poter diminuire il livello di zoom della heat map$_{\scaleto{G}{3pt}}$ & \makecell[tc]{\textit{NI}}\\
			\hline
			\textit{TSRSFD33} & Verifica che l’utente possa poter ricercare tramite una barra di ricerca le città presenti nel database & \makecell[tc]{\textit{NI}}\\
			\hline
			\textit{TSRSFD34} & Verifica che l’utente possa poter visualizzare il messaggio d'errore relativo alla mancanza dei dati ricercati attraverso la barra di ricerca nel database & \makecell[tc]{\textit{NI}}\\
			\hline
			\textit{TSRSFD35} & Verifica che l’utente possa poter visualizzare il confronto dei dati di due città selezionate dall'utente & \makecell[tc]{\textit{NI}}\\
			\hline
			\textit{TSRSFD36} & Verifica che l’utente possa poter salvare in un file locale i dati della città della mappa che sta visualizzando & \makecell[tc]{\textit{NI}}\\
			\hline
			\textit{TSRSFD37} & Verifica che l’utente possa poter inserire l'email per il ricevimento delle informazioni delle informazioni della città selezionata & \makecell[tc]{\textit{NI}}\\
			\hline
			\textit{TSRSFD37.1} & Verifica che l’utente possa poter visualizzare un messaggio di errore nel caso l'email inserita sia scritta in modo errato & \makecell[tc]{\textit{NI}}\\
			\hline
			\textit{TSRSFD38} & Verifica che il sistema abbia inserito correttamente l'email e la città correlata nel database & \makecell[tc]{\textit{NI}}\\
			\hline
			\textit{TSRSFD39} & Verifica che il sistema invii correttamente l'email all'utente & \makecell[tc]{\textit{NI}}\\
			\hline
			\textit{TSRSFD40} & Verifica che l’utente possa poter visualizzare la lista delle città più ricercate & \makecell[tc]{\textit{NI}}\\
			\hline
			\textit{TSRSFD41} & Verifica che l’utente possa poter visualizzare la lista di tutte le città presenti nel database & \makecell[tc]{\textit{NI}}\\
			\hline
			\rowcolor{white}
			\caption{\textbf{Elenco test di sistema}}\\
		\end{longtable}
	
	\end{center}
\def\tabularxcolumn#1{m{#1}}
{\rowcolors{2}{RawSienna!90!RawSienna!20}{RawSienna!70!RawSienna!40}
	
	\begin{center}
		\renewcommand{\arraystretch}{1.4}
		\begin{longtable}{|p{3cm}|p{3cm}|}
			\hline
			\rowcolor{airforceblue}
			\makecell[c]{\textbf{Id Test}} & \makecell[c]{\textbf{Id Requisito}} \\
			\hline
			\hline
			TSRSFO1	& \makecell{RSFO1} \\
			\hline
			TSRSFF2 & \makecell{RSFF2} \\
			\hline
			TSRSFO3 & \makecell{RSFO3}  \\
			\hline
			TSRSFO4 & \makecell{RSFO4 \\ RSFO4.1 \\ RSFO4.2} \\
			\hline
			TSRSFO5 & \makecell{RSFO5} \\
			\hline
			TSRSFO7 & \makecell{RSFO7} \\
			\hline
			TSRSFO9 & \makecell{RSFO9} \\
			\hline
			TSRSFO10 & \makecell{RSFO10} \\
			\hline
			TSRSFO11 & \makecell{RSFO11} \\
			\hline
			TSRSFF12 & \makecell{RSFF12} \\
			\hline
			TSRSFD13 & \makecell{RSFD13} \\
			\hline
			TSRSFD14 & \makecell{RSFD14} \\
			\hline
			TSRSFF15 & \makecell{RSFF15} \\
			\hline
			TSRSFF16 & \makecell{RSFF16} \\
			\hline
			TSRSFO17 & \makecell{RSFO17} \\
			\hline
			TSRSFO18 & \makecell{RSFO18 \\ RSFO18.1 }\\
			\hline
			TSRSFO19 & \makecell{RSFO19} \\
			\hline
			TSRSFO20 & \makecell{RSFO20} \\
			\hline
			TSRSFO21 & \makecell{RSFO21} \\
			\hline
			TSRSFO22 & \makecell{RSFO22} \\
			\hline
			TSRSFF23 & \makecell{RSFF23} \\
			\hline
			TSRSFO24 & \makecell{RSFO24} \\
			\hline
			TSRSFO25 & \makecell{RSFO25} \\
			\hline
			TSRSFO26 & \makecell{RSFO26} \\
			\hline
			TSRSFO27 & \makecell{RSFO27} \\
			\hline
			TSRSFO28 & \makecell{RSFO28} \\
			\hline
			TSRSFF31 & \makecell{RSFF31} \\
			\hline
			TSRSFO32 & \makecell{RSFO32} \\
			\hline
			TSRSFO32.1 & \makecell{RSFO32.1} \\
			\hline
			TSRSFO32.1.1 & \makecell{RSFO32.1.1} \\
			\hline
			TSRSFO32.1.2 & \makecell{RSFO32.1.2} \\
			\hline
			TSRSFO32.1.3 & \makecell{RSFO32.1.3} \\
			\hline
			TSRSFO32.2 & \makecell{RSFO32.2}\\
			\hline
			TSRSFD33 & \makecell{RSFD33 \\RSFD33.1 \\ RSFD33.2}\\
			\hline
			TSRSFD34 & \makecell{RSFD34} \\
			\hline
			TSRSFD35 & \makecell{RSFD35} \\
			\hline
			TSRSFD36 & \makecell{RSFD36 \\ RSFD36.1 \\ RSFD36.2}\\
			\hline
			TSRSFD37 & \makecell{RSFD37}\\
			\hline
			TSRSFD37.1 & \makecell{RSFD37.1}\\
			\hline
			TSRSFD38 & \makecell{RSFD38}\\
			\hline
			TSRSFD39 & \makecell{RSFD39}\\
			\hline
			TSRSFD40 & \makecell{RSFD40} \\
			\hline
			TSRSFD41 & \makecell{RSFD41} \\
			\hline
			\rowcolor{white}
			\caption{\textbf{Tracciamento dei test di sistema con i requisiti}}
	\end{longtable}

\end{center}