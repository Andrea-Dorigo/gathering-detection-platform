\chapter{Processi Primari}\label{ProcessiPrimari}
\section{Fornitura}\label{ProcessiPrimariFornitura}
\subsection{Scopo}\label{ProcessiPrimariFornituraScopo}
La fornitura secondo lo standard ISO/IEC 12207:1995 descrive le attività e i compiti svolti dal fornitore al fine di sviluppare un prodotto soddisfacente e che rispetti appieno le richieste del committente.
Durante questa fase si prevede la compilazione di diversi documenti, i quali verranno inviati al committente per guadagnare la possibilità di lavorare al progetto offerto dall'azienda \textit{Sync Lab}.
Il fornitore esegue un'attività di analisi e stesura dello \textit{Studio di Fattibilità}, documento che rileva i rischi e le criticità riscontrate nella richiesta di appalto.
Si definisce inoltre un accordo contrattuale con il proponente mediante il quale si regolano i rapporti con l'azienda, la consegna e la manutenzione del prodotto sviluppato. 

\subsection{Studio di Fattibilità}\label{ProcessiPrimariFornituraStudioDiFattibilità}
Lo \textit{Studio di Fattibilità} consiste nell'analisi e nella valutazione sistematica delle caratteristiche, dei costi, e dei possibili risultati di un progetto sulla base di una preliminare idea di massima.
A seguito della presentazione dei capitolati d'appalto da parte di ogni proponente avvenuta il 05-11-2020, il \textit{Responsabile di Progetto} si è impegnato a programmare incontri con tutti i componenti del gruppo \textit{Jawa Druids} per valutare le scelte di ogni membro e attuare così un primo scambio di idee. Una volta individuato il capitolato d'interesse gli \textit{Analisti} hanno provveduto alla stesura dello \textit{Studio di Fattibilità}, i quali hanno fornito un'analisi accurata dei capitolati presentati.
Nella stesura dello \textit{Studio di Fattibilità} per ogni capitolato si riporterà:
\begin{itemize}
	\item informazioni generali: informazioni riguardanti il proponente;
	\item descrizione del capitolato: sintesi del progetto da sviluppare; 
	\item finalità del progetto: finalità richieste dal capitolato d'appalto;
	\item tecnologie interessate: tecnologie che verranno utilizzate nello svolgimento del capitolato
	\item aspetti positivi: aspetti favorevoli alla scelta del capitolato;
	\item criticità e fattori di rischio: problematiche che potrebbero sorgere durante lo svolgimento del capitolato;
	\item conclusioni: accettazione o rifiuto del capitolato in base alle informazioni illustrate precedentemente e anche all'interesse dimostrato da ogni membro nel gruppo.
\end{itemize}
\subsection{Altra documentazione da fornire}\label{ProcessiPrimariFornituraAltraDocumentazioneDaFornire}
Oltre allo \textit{Studio di Fattibilità} vengono consegnati altri documenti all'azienda \textit{Sync Lab} ed ai committenti \textit{Prof. Tullio Vardanega} e \textit{Prof. Riccardo Cardin}. Questi documenti sono necessari al fine di tracciare le attività di Analisi, Pianificazione, Verifica, Validazione e Controllo di Qualità per assicurare una completa trasparenza durante tutta la durata del ciclo di vita del progetto.
I documenti in questione sono:
\begin{itemize}
	\item \textit{Analisi dei Requisiti}: identifica e dettaglia in modo completo ed esaustivo i requisiti del sistema descritto nel capitolato che il fornitore si impegna a soddisfare;
	\item \textit{Piano di Qualifica}: illustra la strategia complessiva di verifica e validazione proposta dal fornitore per pervenire al collaudo del sistema con la massima efficienza ed efficacia;
	\item \textit{Piano di Progetto}: presenta l'organigramma dettagliato del fornitore, lo schema proposto per l'assegnazione e la rotazione dei ruoli di progetto, l'impegno complessivo previsto per ogni ruolo e per ogni individuo, l'analisi dei rischi, la pianificazione di massima per la realizzazione del prodotto, e il corrispondente conto economico preventivo.
\end{itemize} 
Alla documentazione appena illustrata il gruppo \textit{Jawa Druids} allegherà inoltre una lettera di presentazione con la quale si formalizza l'impegno nel portare al termine il capitolato prescelto entro i termini definiti nella lettera e rispettandone i requisiti minimi.
\subsection{Strumenti}\label{ProcessiPrimariFornituraStrumenti}
Di seguito sono riportati gli strumenti impiegati dal gruppo durante il progetto per il processo di fornitura.
\subsubsection{Vuoto per ora}\label{2.1.4.1}


\section{Sviluppo}\label{ProcessiPrimariSviluppo}
\subsection{Scopo}\label{ProcessiPrimariScopo}
Il processo di sviluppo contiene tutte le attività che riguardano la produzione del software richiesto dal cliente, in particolare analisi dei requisiti, design, codifica, integrazione, test e installazione.
\subsection{Descrizione}\label{ProcessiPrimariDescrizione}
Di seguito vengono elencate le varie attività che caratterizzano tale processo:
\begin{itemize}
	\item Analisi dei requisiti;
	\item Progettazione architetturale;
	\item Codifica del software.
\end{itemize}
\subsection{Prospettive}\label{ProcessiPrimariProspettive}
Le prospettive alla fine della stesura del processo in questione sono le seguenti:
\begin{itemize}
	\item individuare e stabilire gli obbiettivi di sviluppo;
	\item individuare e stabilire i vincoli tecnologici;
	\item individuare e stabilire i vincoli di design;
	\item produrre un prodotto finale che rispecchi gli obiettivi imposti nello sviluppo e che superi i test e i controlli di qualità stabiliti dal proponente.
\end{itemize}
\subsubsection{Analisi dei requisiti}\label{ProcessiPrimariProspettiveAnalisiDeiRequisiti}
\paragraph{Scopo}\label{ProcessiPrimariProspettiveAnalisiDeiRequisitiScopo}\mbox{}\\
L'analisi dei requisiti viene redatto dagli \textit{Analisti}, lo scopo è quello di definire le funzionalità che il nuovo prodotto deve offrire, ovvero i requisiti che devono essere soddisfatti dal software sviluppato.\\
Gli obiettivi della stesura dell'\textit{Analisi dei Requisiti} sono:
\begin{itemize}
	\item stabilire lo scopo nello sviluppo del prodotto;
	\item definire riferimenti precisi ed affidabili ai \textit{Progettisti};
	\item stabilire i requisiti e le funzionalità concordate con il cliente;
	\item individuare per i \textit{Verificatori} riferimenti per le attività di controllo dei test.
\end{itemize}
\paragraph{Descrizione}\label{ProcessiPrimariProspettiveAnalisiDeiRequisitiDescrizione}\mbox{}\\
I requisiti possono essere individuati in diverse fonti, quali:
\begin{itemize}
	\item \textit{Capitolato d'Appalto}: i requisiti sono stati individuati attraverso la lettura del documento fornito dal proponente \textit{Sync Lab} sul capitolato proposto;
	\item \textit{Verbali Interni}: attraverso le riunioni attuate internamente dagli \textit{Analisti} sono emersi vari requisiti;
	\item \textit{Verbali Esterni}: attraverso contatti e discussioni effettuate con il responsabile aziendale Fabio Pallaro sono emersi requisiti, i quali vi sarà assegnato un codice presente nella tabella dei tracciamenti;
\end{itemize}
\paragraph{Prospettive}\label{ProcessiPrimariProspettiveAnalisiDeiRequisitiProspettive}\mbox{}\\
L'obiettivo dell'\textit{Analisi dei Requisti} è quello di redigere un documento che racchiuda al suo interno tutti i requisiti richiesti dal proponente.
\paragraph{Struttura}\label{ProcessiPrimariProspettiveAnalisiDeiRequisitiStruttura}\mbox{}\\ %wip non è ancora perfettamente definito ???
L'\textit{Analisi dei requisiti} è strutturato nel seguente modo:
\begin{itemize}
	\item \textbf{Introduzione}: in questa sezione si introduce lo scopo del documento e riferimenti a documenti esterni;
	\item \textbf{Descrizione generale}: il prodotto viene descritto, si individuano le fasi generali del progetto e l'utenza a cui è destinato il prodotto;
	\item \textbf{Fasi del prodotto}: si elencano in gruppi le fasi del progetto, le quali vengono suddivise in parti per individuare più dettagliatamente tutti i processi di sviluppo del software;
	\item \textbf{Requisiti}: utilizzando le fasi descritte e tutti i documenti sopra indicati si dettagliano tutti i requisiti obbligatori, facoltativi e desiderabili da implementare.
\end{itemize}
\paragraph{Classificazione dei requisiti}\label{ProcessiPrimariProspettiveAnalisiDeiRequisitiClassificazioneDeiRequisiti}\mbox{}\\% RS1.2 = rs(requisiti specifici) 1(fase generale a cui si riferiscono) 2(numero del requisito incrementale)
I requisiti sono stati individuati utilizzando la seguente codifica:
\begin{center}
	\textbf{RS[classificazione][tipo\_di\_requisito][codice\_requisito]}
\end{center} 
La descrizione della classificazione è la seguente:
\begin{itemize}
	\item \textbf{RS}: è l'acronimo per Requisito Specifico;
	\item \textbf{Classificazione}: individua la classificazione del requisito:
	\begin{itemize}
		\item Funzionale: indicato dalla lettera "F";
		\item Qualità: indicato dalla lettera "Q";
		\item Vincolo: indicato dalla lettera "V";
		\item Prestazionale: indicato dalla lettera "P".
	\end{itemize}
	\item \textbf{Tipo\_di\_requisito}: individua la tipologia di requisito:
	\begin{itemize}
		\item Obbligatorio: indicato con la lettera "O" individua un requisito essenziale allo sviluppo del progetto e necessario al suo completamento;
		\item Desiderabile: indicato con la lettera "D" individua un requisito utile al prodotto e che dà valore aggiunto ad esso, ma non essenziale al suo completamento;
		\item Facoltativo: indicato con la lettera "F" individua un requisito che può essere sviluppato, ma può anche non essere completato.
	\end{itemize};
	\item \textbf{Codice\_requisito}: è rappresentato da un codice identificativo univoco nella forma gerarchica padre/figlio.
\end{itemize}
Ogni requisito è strutturato nella tabella nel seguente modo:
\begin{itemize}
	\item \textbf{Identificativo}: individua univocamente il requisito;
	\item \textbf{Descrizione}: descrizione del requisito richiesto nella determinata fase;
	\item \textbf{Tipo di requisito}: individua la tipologia di requisito obbligatorio, desiderabile e facoltativo;
	\item \textbf{Fonte}: ogni requisito è ricavato da una delle seguenti fonti:
	\begin{itemize}
		\item Capitolato$_{\scaleto{G}{3pt}}$: individua il documento del capitolato$_{\scaleto{G}{3pt}}$;
		\item Interno: il requisito è stato individuato dagli analisti;
		\item Verbale: si tratta del documento in riferimento alla discussione con il proponente;
		\item Fasi del capitolato: il requisito è stato individuato dalla fase del capitolato individuata con il proprio codice.
	\end{itemize}
\end{itemize}
\paragraph{Classificazione delle fasi}\label{ProcessiPrimariProspettiveAnalisiDeiRequisitiClassificazioneDelleFasi}\mbox{}\\% FC1.2 = fc(fasi capitolato) 1 (macro-fase) 2(fase precisa)
La descrizione dettagliata delle fasi è stata incorporata nel documento per individuare la fase in cui il requisito è necessario che venga implementato o sia possibile l'implementazione.\\
La classificazione delle fasi è definita nel seguente modo:
\begin{center}
	\textbf{FC[codice\_macro-fase].[codice\_fase]}
\end{center}
La descrizione della classificazione è la seguente:
\begin{itemize}
	\item \textbf{FC}: è l'acronimo per Fasi Capitolato;
	\item \textbf{Codice\_macro-fase}: individua un numero riferito alla fase dello sviluppo del progetto;
	\item \textbf{Codice\_fase}: individua un numero riferito ad una suddivisione della fase del progetto.
\end{itemize}
Ogni suddivisione delle fasi è strutturato nel seguente modo:
\begin{itemize}
	\item \textbf{Principali}:
	\begin{itemize}
		\item \textbf{Identificativo}: inserito prima del nome della fase per individuarla univocamente;
		\item \textbf{Nome}: identifica il nome della fase;
		\item \textbf{Descrizione}: descrizione dello scopo della fase in analisi.
	\end{itemize}
	\item \textbf{Aggiuntivi}:
\begin{itemize}
	\item \textbf{Input}: dati in ingresso necessari allo sviluppo della fase;
	\item \textbf{Output}: dati in uscita elaborati o modificati durante la fase;
	\item \textbf{Linguaggio di programmazione}: linguaggio di programmazione utilizzato per lo sviluppo della fase;
	\item \textbf{Processo}: sono le operazioni svolte nella fase;
	\item \textbf{Risposta ad errori}: operazione della fase nel caso di errori;
	\item \textbf{Tipi di algoritmi}: individua gli algoritmi utilizzabili;
	\item \textbf{Strumenti}: sono strumenti utilizzati nello svolgimento della fase;
	\item \textbf{Vincolo}: vincoli individuati dal gruppo o dal proponente.
\end{itemize}
\end{itemize}
\paragraph{Qualità dei requisiti}\label{ProcessiPrimariProspettiveAnalisiDeiRequisitiQualitàDeiRequisiti}\mbox{}\\
Ogni requisito deve rispettare le seguenti qualità:
\begin{itemize}
	\item devono essere correttamente descritti;
	\item non devono essere ambigui, ogni requisito fornirà un'unica interpretazione;
	\item devono essere completi, cioè descrivere in modo completo e in tutte le sue parti la funzionalità da implementare;
	\item ogni requisito deve essere consistente, cioè non deve avere conflitti con altri requisiti individuati;
	\item devono essere modificabili, cioè ogni requisito nel corso dello sviluppo del progetto può essere rivalutato e modificato e bisogna mantenere uno storico dei cambiamenti;
	\item ogni requisito deve essere tracciabile, cioè ogni requisito deve essere tracciabile ad ogni suo test o codice di implementazione o alla sua origine;
	\item ogni requisito deve essere classificato per importanza o stabilità;
	\item ogni requisito deve essere verificabile, cioè deve essere possibile una sua verifica attraverso un processo nel quale una persona o una macchina può controllarlo;
\end{itemize}

\paragraph{Metriche}\label{ProcessiPrimariProspettiveAnalisiDeiRequisitiMetriche}\mbox{}\\
Con \textbf{MQPD01} si intende la \textbf{Totalità delle implementazioni}. Indice riportante l'interezza del prodotto software, rispetto ai requisiti posti, mediante un valore percentuale.:

\begin{center}
	\textbf{T=($1-\frac{RnI}{RI}$)*100}
\end{center}
Dove:
\begin{itemize}
	\item \textbf{T} sta per \textit{Totalità}, riferito ai requisiti da implementare.
	\item \textbf{RnI} sta per \textit{Requisito non Implementato};
	\item \textbf{RI} sta per \textit{Requisito Implementato};
\end{itemize}
I range accettabili per il risultato di \textbf{T} sono così suddivisi:
\begin{itemize}
	\item 90\% $<$ \textbf{T} $\leq$ 100\% indica che la copertura dei requisiti proposti è quasi totale;
	\item 80\% $<$ \textbf{T} $\leq$ 90\% indica che la copertura dei requisiti proposti è sufficiente, buona.
	\item \textbf{T} $\leq$ 80\% indica che la copertura dei requisiti proposti è insufficiente.
\end{itemize}


\subsection{Progettazione}\label{ProcessiPrimariProgettazione}
\subsubsection{Scopo}\label{ProcessiPrimariProgettazioneScopo}
La Progettazione è un'attività svolta dai \textit{Progettisti}. In questa fase si individuano, attraverso l'\textit{Analisi dei Requisiti}, le caratteristiche che il prodotto deve avere per soddisfare tutti i requisiti richiesti dal proponente. Lo scopo è quello di determinare la soluzione migliore per risolvere ogni requisito individuato. 
\subsubsection{Aspettative}\label{ProcessiPrimariProgettazioneAspettative}
Alla conclusione della stesura della progettazione si ha come risultato la realizzazione dell'architettura del software da sviluppare. Tale architettura è necessaria ai programmatori per individuare le istruzioni necessarie a sviluppare il prodotto finito.
\subsubsection{Descrizione}\label{ProcessiPrimariProgettazioneDescrizione}
La progettazione è suddivisa principalmente in due parti:
\begin{itemize}
	\item \textbf{Technology baseline}: contiene tutte le specifiche adottate nella progettazione ad alto livello e delle sue componenti, i test di verifica e l'elenco dei diagrammi UML utilizzati per la definizione dell'architettura;
	\item \textbf{Product baseline}: perfeziona ulteriormente la fase di progettazione, integrando ciò che è riportato nella technology baseline, definendo inoltre i test necessari alla verifica.
\end{itemize}

\subsubsection{Technology baseline}\label{ProcessiPrimariProgettazioneTecnologyBaseline}
La technology baseline viene redatta dal \textit{Progettista} ed include:
\begin{itemize}
	\item \textbf{Diagramma UML}: diagrammi sulle classi usate, package, attività e sequenze.
	\item \textbf{Tecnologie}: descrizione delle tecnologie utilizzate nel progetto, i loro vantaggi e svantaggi;
	\item \textbf{Design pattern}: descrizione di ogni design pattern per la realizzazione dell'architettura, ognuno di essi viene accompagnato da un diagramma che ne espone il significato e la struttura; 
	\item \textbf{Tracciamento delle componenti}: ogni componente viene riferito al requisito che soddisfa;
	\item \textbf{Test di integrazione}: corrispondono a tutti i test necessari alla verifica del sistema in tutte le sue unità.
\end{itemize}
\subsubsection{Product baseline}\label{ProcessiPrimariProgettazioneProductBaseline}
La product baseline viene redatta dal \textit{Progettista} ed include:
\begin{itemize}
	\item \textbf{definizioni delle classi e funzioni}: descrizione univoca delle classi e funzioni evitando ridondanze;
	\item \textbf{tracciamento}: ogni requisito viene tracciato attraverso le classi e funzioni, questo è necessario per individuare se è stato soddisfatto il requisito;
	\item \textbf{test di unità}: definiscono test per la verifica delle funzionalità del prodotto.
\end{itemize}
\subsection{Codifica}\label{ProcessiPrimariCodifica}

\subsubsection{Scopo}\label{ProcessiPrimariCodificaScopo}
La fase di codifica è la scrittura del codice per sviluppare la miglior soluzione del prodotto. In questa sezione si introdurranno tutte le norme necessarie allo sviluppo di un codice uniformato tra tutti i \textit{Programmatori} e rispettoso delle regole standard indicate nel documento.
\subsubsection{Aspettative}\label{ProcessiPrimariCodificaAspettative}
Conclusa la fase di codifica ci si attende un codice pulito e facile da leggere, utile nelle successive validazioni, modifiche e per agevolare la sua manutenzione. L'obiettivo è quello di sviluppare un prodotto conforme alle richieste individuate con il proponente. 

\subsubsection{Descrizione}\label{ProcessiPrimariCodificaDescrizione}
In questa fase la programmazione del prodotto dovrà rispettare le norme descritte nel documento. Perseguendo gli obiettivi individuati nel documento \textit{Piano di qualifica v1.0.0} si produrrà un software con un'alta qualità di codice.

\subsubsection{Stile di codifica}\label{ProcessiPrimariCodificaStileDiCodifica}
All'interno di questo paragrafo vengono elencate le norme da rispettare da ogni membro del gruppo per raggiungere uniformità del codice:

\begin{itemize}
	\item \textbf{Indentazione}: ogni blocco di codice scritto per il prodotto da sviluppare deve essere ben indentato e deve rispettare una misura di 4 spazi (equivalenti alla pressione del tasto \textit{tab}). Fanno eccezione a questa regola i commenti che vengono inseriti per spiegazioni di blocchi di codice;
	\item \textbf{Parentesizzazione}: le parentesi devono essere utilizzate in linea col blocco di codice scritto e non in una linea sottostante per separarlo;
	\item \textbf{Univocità delle variabili, metodi e funzioni}: ogni variabile, metodo e funzione utilizzata deve avere un nome significativo, esplicativo ed univoco;
	\item \textbf{Classi}: ogni classe deve avere il proprio nome scritto con l'iniziale maiuscola;
	\item \textbf{Metodi e funzioni}: il nome di metodi e funzioni devono iniziare per lettera minuscola e se composti da più parole le successive devono essere scritte con lettera maiuscola (stile \textit{CamelCase});
	\item \textbf{Lingua}: i commenti devono essere scritti in lingua inglese.
\end{itemize}

\subsection{Strumenti}\label{ProcessiPrimariStrumenti}

\subsubsection{PragmaDB}\label{ProcessiPrimariStrumentiPragmaDB}
Programma utilizzato per il tracciamento dei requisiti.
\subsubsection{Draw.io}\label{ProcessiPrimariStrumentiDrawIo}
Programma utilizzato per la stesura dei diagrammi UML.