\chapter{Processi Primari}\label{ProcessiPrimari}
\section{Fornitura}\label{2.1}
\subsection{Scopo}\label{2.1.1}
La fornitura secondo lo standard ISO/IEC 12207:1995 descrive le attività e i compiti svolti dal fornitore al fine di sviluppare un prodotto soddisfacente e che rispetti appieno le richieste del committente.
Durante questa fase si prevede la compilazione di diversi documenti, i quali verranno inviati al committente per guadagnare la possibilità di lavorare al progetto offerto dall'azienda \textit{Sync Lab}.
Il fornitore esegue un'attività di analisi e stesura dello \textit{Studio di Fattibilità}, documento che rileva i rischi e le criticità riscontrate nella richiesta di appalto.
Si definisce inoltre un accordo contrattuale con il proponente mediante il quale si regolano i rapporti con l'azienda, la consegna e la manutenzione del prodotto sviluppato. 

\subsection{Studio di Fattibilità}\label{2.1.2}
Lo \textit{Studio di Fattibilità} consiste nell'analisi e nella valutazione sistematica delle caratteristiche, dei costi, e dei possibili risultati di un progetto sulla base di una preliminare idea di massima.
A seguito della presentazione dei capitolati d'appalto da parte di ogni proponente avvenuta il 05-11-2020, il \textit{Responsabile di Progetto} si è impegnato a programmare incontri con tutti i componenti del gruppo \textit{Jawa Druids} per valutare le scelte di ogni membro e attuare così un primo scambio di idee. Una volta individuato il capitolato d'interesse gli \textit{Analisti} hanno provveduto alla stesura dello \textit{Studio di Fattibilità}, i quali hanno fornito un'analisi accurata dei capitolati presentati.
Nella stesura dello \textit{Studio di Fattibilità} per ogni capitolato si riporterà:
\begin{itemize}
	\item informazioni generali: informazioni riguardanti il proponente;
	\item descrizione del capitolato: sintesi del progetto da sviluppare; 
	\item finalità del progetto: finalità richieste dal capitolato d'appalto;
	\item tecnologie interessate: tecnologie che verranno utilizzate nello svolgimento del capitolato
	\item aspetti positivi: aspetti favorevoli alla scelta del capitolato;
	\item criticità e fattori di rischio: problematiche che potrebbero sorgere durante lo svolgimento del capitolato;
	\item conclusioni: accettazione o rifiuto del capitolato in base alle informazioni illustrate precedentemente e anche all'interesse dimostrato da ogni membro nel gruppo.
\end{itemize}
\subsection{Altra documentazione da fornire}\label{2.1.3}
Oltre allo \textit{Studio di Fattibilità} vengono consegnati altri documenti all'azienda \textit{Sync Lab} ed ai committenti \textit{Prof. Tullio Vardanega} e \textit{Prof. Riccardo Cardin}. Questi documenti sono necessari al fine di tracciare le attività di Analisi, Pianificazione, Verifica, Validazione e Controllo di Qualità per assicurare una completa trasparenza durante tutta la durata del ciclo di vita del progetto.
I documenti in questione sono:
\begin{itemize}
	\item \textit{Analisi dei Requisiti}: identifica e dettaglia in modo completo ed esaustivo i requisiti del sistema descritto nel capitolato che il fornitore si impegna a soddisfare;
	\item \textit{Piano di Qualifica}: illustra la strategia complessiva di verifica e validazione proposta dal fornitore per pervenire al collaudo del sistema con la massima efficienza ed efficacia;
	\item \textit{Piano di Progetto}: presenta l'organigramma dettagliato del fornitore, lo schema proposto per l'assegnazione e la rotazione dei ruoli di progetto, l'impegno complessivo previsto per ogni ruolo e per ogni individuo, l'analisi dei rischi, la pianificazione di massima per la realizzazione del prodotto, e il corrispondente conto economico preventivo.
\end{itemize} 
Alla documentazione appena illustrata il gruppo \textit{Jawa Druids} allegherà inoltre una lettera di presentazione con la quale si formalizza l'impegno nel portare al termine il capitolato prescelto entro i termini definiti nella lettera e rispettandone i requisiti minimi.
\subsection{Strumenti}\label{2.1.4}
Di seguito sono riportati gli strumenti impiegati dal gruppo durante il progetto per il processo di fornitura.
\subsubsection{Vuoto per ora}\label{2.1.4.1}


\section{Sviluppo}\label{2.2}
\subsection{Scopo}\label{2.2.1}
Il processo di sviluppo contiene tutte le attività che riguardano la produzione del software richiesto dal cliente, in particolare analisi dei requisiti, design, codifica, integrazione, test e installazione.
\subsection{Descrizione}\label{2.2.2}
Di seguito vengono elencate le varie attività che caratterizzano tale processo:
\begin{itemize}
	\item Analisi dei requisiti;
	\item Progettazione architetturale;
	\item Codifica del software.
\end{itemize}
\subsection{Prospettive}\label{2.2.3}
Le prospettive alla fine della stesura del processo in questione sono le seguenti:
\begin{itemize}
	\item individuare e stabilire gli obbiettivi di sviluppo;
	\item individuare e stabilire i vincoli tecnologici;
	\item individuare e stabilire i vincoli di design;
	\item produrre un prodotto finale che rispecchi gli obiettivi imposti nello sviluppo e che superi i test e i controlli di qualità stabiliti dal proponente.
\end{itemize}
\subsubsection{Analisi dei requisiti}\label{2.2.3.1}
\subsubsection{Scopo}\label{2.2.3.2}
L'analisi dei requisiti viene redatto dagli \textit{Analisti}, lo scopo è quello di definire le funzionalità che il nuovo prodotto deve offrire, ovvero i requisiti che devono essere soddisfatti dal software sviluppato.\\
Gli obiettivi della stesura dell'\textit{Analisi dei Requisiti} sono:
\begin{itemize}
	\item stabilire lo scopo nello sviluppo del prodotto;
	\item definire riferimenti precisi ed affidabili ai \textit{Progettisti};
	\item stabilire i requisiti e le funzionalità concordate con il cliente;
	\item individuare per i \textit{Verificatori} riferimenti per le attività di controllo dei test.
\end{itemize}
\subsubsection{Descrizione}\label{2.2.3.3}
I requisiti possono essere individuati in diverse fonti, quali:
\begin{itemize}
	\item \textit{Capitolati d'Appalto}: i requisiti sono stati individuati attraverso la lettura del documento fornito dal proponente \textit{Sync Lab} sul capitolato proposto;
	\item \textit{Verbali Interni}: attraverso le riunioni attuate internamente dagli \textit{Analisti} sono emersi vari requisiti;
	\item \textit{Verbali Esterni}: attraverso contatti e discussioni effettuate con il responsabile aziendale Fabio Pallaro sono emersi requisiti, i quali vi sarà assegnato un codice presente nella tabella dei tracciamenti;
	\item Casi d'uso: attraverso le modalità d'uso del prodotto si sono individuati dei requisiti particolari.
\end{itemize}
\subsubsection{Prospettive}\label{2.2.3.4}
L'obiettivo dell'\textit{Analisi dei Requisti} è quello di redigere un documento che racchiuda al suo interno tutti i requisiti richiesti dal proponente.
\subsubsection{Struttura}\label{2.2.3.5}
L'\textit{Analisi dei requisiti} è strutturato nel seguente modo:
\begin{itemize}
	\item 
\end{itemize}
\subsubsection{Classificazione dei requisiti}\label{2.2.3.6}
\subsubsection{Classificazione dei casi d'uso}\label{2.2.3.7}
\subsubsection{Metriche}\label{2.2.3.8}


\subsection{Progettazione}\label{2.2.4}
\subsubsection{Scopo}\label{2.2.4.1}
La Progettazione è un'attività svolta dai \textit{Progettisti}. In questa fase si individuano, attraverso l'\textit{Analisi dei Requisiti}, le caratteristiche che il prodotto deve avere per soddisfare tutti i requisiti richiesti dal proponente. Lo scopo è quello di determinare la soluzione migliore per ogni requisito.
