\chapter{Processi Organizzativi}\label{Processi Organizzativi}

\section{Processo di coordinamento}\label{4.1}

\subsection{Scopo}\label{4.1.1}
Questa sezione mostra i metodi di coordinamento adottati dal gruppo Jawa Druids in termini di riunioni, comunicazione, ruoli del progetto e assegnazione dei compiti. Verranno inoltre brevemente introdotti lo strumento selezionato e la sua configurazione di base. La struttura del processo di coordinamento secondo lo standard ISO/ IEC 12207 è la seguente:
\begin{itemize}
	\item \textbf{Comunicazione:} interna oppure esterna;
	\item \textbf{Riunioni:} interne oppure esterne.
\end{itemize}
\subsection{Comunicazione}\label{4.1.2}
Durante il progetto, il team di Jawa Druids comunicherà su due diversi livelli: interno ed esterno.
Per quanto riguarda la comunicazione esterna, esse avverranno con i seguenti soggetti:
\begin{itemize}
\item \textbf{Proponente:} l'azienda Sync Lab S.r.l., rappresentata dal signor Fabio Pallaro;
	\item \textbf{Committenti:} nella persona del prof. Tullio Vardanega e del prof. Riccardo Cardin;
		\item \textbf{Competitor:} questo punto verrà chiarito dopo la Revisione dei Requisiti quando i capitolati saranno assegnati ai relativi gruppi;
			\item \textbf{Esperti interni:} da consultare eventualmente previo accordo con il proponente ed i committenti.
\end{itemize}
Il gruppo si rivolgerà a tutti i soggetti mediante comunicazioni scritte e/o meeting.
\subsubsection{Comunicazione interna}\label{4.1.2.1}
Il metodo di comunicazione standard per l'interazione scritta tra i membri del gruppo Jawa Druids è il servizio di messaggistica istantanea Discord.
La strategia per la gestione delle discussioni sarà anche creare un canale specifico per ogni attività non ignorabile che il gruppo deve scegliere (ad esempio, tutte le discussioni sui documenti saranno condotte all'interno di quel canale specifico).
Oltre a questa tipologia di canali, le discussioni saranno suddivise anche nelle seguenti categorie:
\begin{itemize}
\item \textbf{git-github:} per qualsiasi discussione e/o problemi riguardanti le repository;
	\item\textbf{generale:} per qualsiasi discussione generica o riguardante il progetto;
	\item\textbf{latex:} per qualsiasi discussione riguardante Latex;
	\item\textbf{domande per azienda} per tutte le domande/ dubbi da porre al proponente.
\end{itemize}
Discord permette di notificare un particolare messaggio a una o più persone, includendo nel corpo del messaggio le seguenti parole chiave:
\begin{itemize}
\item \textbf{@everyone:} il messaggio verrà notificato a tutti i componenti del gruppo;
	\item \textbf{@username:} inserendo l'username specifico, il messaggio verrà notificato all'utente desiderato.
\end{itemize}
Inoltre, un'altro modo di comunicazione è la video-chiamata di Discord, che è stato preferito ad altri servizi per la sua versatilità, il fatto di essere open-source e perchè multi-piattaforma.	
\subsubsection{Comunicazione esterna}\label{4.1.2.2}
Il Responsabile di Progetto rappresenterà l'intero team e manterrà i contatti esterni tramite un indirizzo email appositamente creato.
L' e-mail utilizzata sarà: 
\begin{center}
	\textbf {JawaDruids@gmail.com}
\end{center}
Il Responsabile di Progetto deve utilizzare regole relative alle comunicazioni interne per notificare a ciascun membro del team eventuali comunicazioni ricevute dal cliente e dal proponente.
Ogni email inviata avrà la seguente struttura:
\begin{itemize}
\item \textbf{Oggetto:} l'oggetto della mail sarà preceduto dalla sigla "[SWE][UNIPD]", in modo che l'ambito di riferimento dell'e-mail sia immediatamente chiaro ed espresso nel modo più conciso possibile per eliminare ambiguità e migliorare la comprensione dell'oggetto;
	\item \textbf{Corpo:} il tono da mantenere è formale, ci si rivolgerà dando del Lei. Il corpo sarà sempre preceduto da una formula di apertura formale, come "Egregio", "Alla cortese attenzione di Sync Lab S.r.l.", "Spettabile", a seconda del destinatario.
	Il contenuto dovrà inoltre essere sintetico ed esaustivo, per esporre al meglio il problema e/o eventuali richieste.
\end{itemize}
\subsection{Riunioni}\label{4.1.3}
Ogni riunione nominerà un segretario il cui compito è tenere traccia di ciò che viene discusso, eseguire l'ordine del giorno e infine utilizzare le informazioni raccolte per redigere i verbali della riunione.
\subsubsection{Riunioni interne}\label{4.1.3.1}
Solo i membri del team possono partecipare alle riunioni interne.
Almeno quattro persone devono partecipare alla riunione per confermarne la validità.
Le decisioni saranno prese a maggioranza semplice ed inoltre, affinché la riunione sia considerata efficace, il responsabile del progetto deve completare le seguenti attività:
\begin{itemize}
\item fissare preventivamente la data della riunione, previa verifica della disponibilità dei membri del gruppo;
\item stabilire un ordine del giorno;
\item nominare un segretario per la riunione
\end{itemize}
I partecipanti della riunione, invece, devono:
\begin{itemize}
	\item avvertire preventivamente in caso di assenze o ritardi;
	\item presentarsi puntuali al meeting nell'ora prestabilita;
	\item partecipare attivamente alla discussione.
\end{itemize}
Ogni membro del gruppo avrà la possibilità di richiedere un incontro interno.
\subsubsection{Riunioni esterne}\label{4.1.3.2}
Le riunioni esterne prevedono la partecipazione di soggetti esterni, quali il proponente e i committenti, oltre ai componenti del gruppo Jawa Druids.
Così come le riunioni interne, anche quelle esterne, prevedono la nomina di un segretario che dovrà poi redigere un verbale di riunione.
Le riunioni esterne con il proponente saranno condotte attraverso il canale Discord creato appositamente.
\subsection{Strumenti utilizzati per il processo di coordinamento}\label{4.1.4}
\begin{itemize}
	\item \textbf{Discord:} \url{https://discordapp.com/company/};
	\item \textbf{Gmail} \url{https://www.google.com/intl/it/gmail/about/}.
\end{itemize}
\section{Processo di pianificazione}\label{4.2}

\subsection{Scopo}\label{4.2.1}

\subsection{Ruoli di progetto}\label{4.2.2}

\subsubsection{Responsabile di progetto}\label{4.2.2.1}

\subsubsection{Amministratore}\label{4.2.2.2}

\subsubsection{Analista}\label{4.2.2.3}

\subsubsection{Programmatore}\label{4.2.2.4}

\subsubsection{Verificatore}\label{4.2.2.5}

\subsection{Assegnazione dei compiti}\label{4.2.3}

\subsection{Ciclo di vita del ticket}\label{4.2.4}

\subsection{Metriche}\label{4.2.5}

\subsubsection{MPR01 Budget at Completion}\label{4.2.5.1}

\subsubsection{MPR02 Schedule Variance}\label{4.2.5.2}

\subsubsection{MPR03 Budget Variance}\label{4.2.5.3}

\subsubsection{MPR04 Actual Cost}\label{4.2.5.4}

\subsection{Strumenti}\label{4.2.6}

\section{Formazione}\label{4.3}