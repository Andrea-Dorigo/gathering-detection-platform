\chapter{Processi Organizzativi}\label{Processi Organizzativi}

\section{Processo di coordinamento}\label{4.1}

\subsection{Scopo}\label{4.1.1}

\subsection{Comunicazione}\label{4.1.2}

\subsubsection{Comunicazione interna}\label{4.1.2.1}

\subsubsection{Comunicazione esterna}\label{4.1.2.2}

\subsection{Riunioni}\label{4.1.3}

\subsubsection{Riunioni interne}\label{4.1.3.1}

\subsubsection{Riunioni esterne}\label{4.1.3.2}

\subsection{Strumenti utilizzati per il processo di coordinamento}\label{4.1.4}

\section{Processo di pianificazione}\label{4.2}

\subsection{Scopo}\label{4.2.1}

Lo scopo di questa sezione è spiegare come il gruppo intende pianificare il lavoro, dall'assegnazione dei ruoli fino alla concreta assegnazione dei compiti di ogni componente di \textit{Java Druids}. In conformità allo standard ISO/IEC 12207, il processo di pianificazione è strutturato in due parti:
\begin{itemize}
	\item ruoli di progetto;
	\item assegnazione dei ruoli;
\end{itemize}

\subsection{Ruoli di progetto}\label{4.2.2}

I vari componenti del gruppo ricopriranno i seguenti ruoli:
\begin{itemize}
	\item \textit{Responsabile di Progetto};
	\item \textit{Amministratore di Progetto};
	\item \textit{Analista};
	\item \textit{Progettista};
	\item \textit{Programmatore};
	\item \textit{Verificatore}.
\end{itemize}

Il gruppo stabilirà un calendario in modo tale che ogni membro riesca a ricoprire almeno una volta ciascuno ruolo in un periodo di tempo omogeneo senza gravare sullo svolgimento delle attività. Come previsto dal \textit{Piano di Progetto} l'assegnazione di un ruolo comporta lo svolgimento di determinati compiti, inoltre il gruppo si impegnerà per eliminare eventuali conflitti: un componente non potrà mai redigere e successivamente verificare ciò che ha prodotto.

\subsubsection{Responsabile di Progetto}\label{4.2.2.1}

Il \textit{Responsabile di Progetto}, ruolo fondamentale e presente per l'intera durata del lavoro, rappresenta il gruppo presso il proponente ed i committenti. Il suo principale compito è inoltre quello di coordinare la struttura e organizzare il lavoro. In particolare questo ruolo comporta:
\begin{itemize}
	\item il coordinamento dei membri del gruppo e dei compiti che devono portare a termine;
	\item la gestione della pianificazione, ossia l'attività da svolgere e le relative scadenze da rispettare;
	\item avere la responsabilità della stima dei costi e dell'analisi dei rischi;
	\item la gestione delle relazioni che il gruppo intrattiene con i soggetti esterni;
	\item l'amministrazione delle risorse umane e dell'assegnazione dei ruoli;
	\item l'approvazione dei documenti.
\end{itemize}

\subsubsection{Amministratore di Progetto}\label{4.2.2.2}

L'\textit{Amministratore del Progetto} ha il compito di gestire, controllare e curare gli strumenti che il gruppo utilizza per lo svolgimento del proprio lavoro; è colui che garantisce l'affidabilità e l'efficacia dei mezzi. Questa figura persegue l'idea che la buona gestione dell'ambiente del lavoro favorisca la produttività, per questo motivo deve:
\begin{itemize}
	\item amministrare le infrastrutture e i servizi necessari ai processi di supporto;
	\item gestire il versionamento e la configurazione dei prodotti;
	\item controllare la documentazione per assicurarsi che venga corretta, verificata ed approvata;
	\item facilitare il reperimento della documentazione;
	\item risolvere eventuali problemi legati alla gestione dei processi;
	\item redigere e manutenere le norme e le procedure che regolano il lavoro;
	\item individuare gli strumenti utili all'automazione dei processi.
\end{itemize}

\subsubsection{Analista}\label{4.2.2.3}

L'\textit{Analista} è presente nelle fasi iniziali del progetto, soprattutto quando avviene la stesura dell'\textit{Analisi dei Requisiti} e il suo compito è quello di evidenziare i punti fondamentali del problema in questione per comprenderne le sue peculiarità. Quindi la sua figura è fondamentale per la buona riuscita del lavoro, in quanto senza la sua analisi potrebbero essere presenti errori o mancanze nell'individuazione dei requisti che possono compromettere la successiva attività di progettazione.\\
L'\textit{Analista}:
\begin{itemize}
	\item studia e definisce il problema in oggetto;
	\item analizza le richieste;
	\item fissa quali sono i requisiti studiando i bisogni impliciti ed espliciti;
	\item analizza gli utenti e i casi d'uso;
	\item redige lo \textit{Studio di Fattibilità} e l'\textit{Analisi dei Requisiti}. 
\end{itemize}

\subsubsection{Progettista}\label{4.2.2.4}

Il \textit{Progettista} ha il compiti di sviluppare una soluzione che soddisfi i bisogni individuati dal lavoro dell'\textit{Analista}. Lo scopo di questo compito, di natura sintetica, è quello di produrre un'architettura che modelli il problema a partire da un insieme di requisiti. Egli deve:
\begin{itemize}
	\item applicare i principi noti e collaudati per produrre un'architettura coerente e consistente;
	\item produrre una soluzione sostenibile e realizzabile che rientri nei costi stabiliti dal preventivo;
	\item costruire una struttura che soddisfi i requisiti e che sia aperta alla comprensione;
	\item limitare il più possibile il grado di accoppiamento tra le varie componenti;
	\item sforzarsi di cercare l'efficienza, la flessibilità e la riusabilità;
	\item elaborare una soluzione capace di interagire in modi diversi con l'ambiente in cui si pone e che sia sicura rispetto ad eventuali anomalie e intrusioni esterne;
	\item ricercare la massima disponibilità e affidabilità per l'architettura proposta.
\end{itemize}

\subsubsection{Programmatore}\label{4.2.2.5}

Il \textit{Programmatore} è incaricato della codifica: il suo compito è quello di implementare l'architettura prodotta dal \textit{Progettista} in modo tale che aderisca alle specifiche. Egli è responsabile della manutenzione del codice creato, infatti i suoi compiti sono:
\begin{itemize}
	\item codificare secondo le specifiche dettate dal \textit{Progettista}, inoltre il codice prodotto deve essere documentato, versionabile e strutturato così da agevolare la futura manutenzione;
	\item creare le componenti che servono per la verifica e la validazione del codice;
	\item scrivere il \textit{Manuale Utente} relativo al prodotto.
\end{itemize}

\subsubsection{Verificatore}\label{4.2.2.6}

Il \textit{Verificatore} deve essere presente per tutta la durata del progetto e si occupa di controllare che le attività svolte rispettino le norme e le attese prefissate. Egli deve:
\begin{itemize}
	\item accertarsi che l'esecuzione delle attività di processo non provochi errori;
	\item redigere la parte retrospettiva del \textit{Piano di Qualifica} che chiarisce le verifiche e le prove effettuate.
\end{itemize}

\subsection{Assegnazione dei compiti}\label{4.2.3}

La progressione del progetto può essere vista come il completamento sequenziale o parallelo di una serie di compiti, ognuno con scadenza temporale, i quali producono risultati utili per la realizzazione degli obiettivi. I compiti possono essere determinati da:
\begin{itemize}
	\item contingenza;
	\item processi in atto;
	\item un insieme dei fattori precedenti.
\end{itemize}

La figura che si occupa della suddivisione ed assegnazione dei compiti è il \textit{Responsabile di Progetto}, il quale:
\begin{itemize}
	\item individua il compito da svolgere;
	\item se ritiene il compito complesso lo suddivide in diversi sotto-compiti;
	\item individua uno o più componenti del gruppo a cui assegnare il compito;
	\item crea le schede su Trello e aggiorna la Timeline di GitKraken.
\end{itemize}

Di conseguenza i membri del gruppo devono impegnarsi per svolgere il compito entro la data prefissata, avvisando nel caso in cui riscontrino problemi a rispettare le scadenze.

\subsection{Trello e Gitkraken}\label{4.2.4}

Dopo aver suddiviso i compiti, il \textit{Responsabile di Progetto} modificherà la pagina di Trello del gruppo. Ogni scheda avrà la descrizione del compito da svolgere, i componenti del gruppo che devono svolgerlo e la data di scadenza. \\
Durante lo sviluppo i membri del gruppo possono aggiungere commenti ed informazioni utili allo svolgimento del compito, se quest'ultimo è suddiviso in più parti sarà presente un elenco puntato che indica ogni suddivisione che l'addetto allo svolgimento potrà spuntare quando conclusa. \\
Per aiutare il gruppo viene fornita anche una rappresentazione grafica delle varie scadenze da rispettare grazie allo strumento Timeline di GitKraken.

\subsection{Metriche}\label{4.2.5}

\subsubsection{MPR01 Budget at Completion}\label{4.2.5.1}

\subsubsection{MPR02 Schedule Variance}\label{4.2.5.2}

\subsubsection{MPR03 Budget Variance}\label{4.2.5.3}

\subsubsection{MPR04 Actual Cost}\label{4.2.5.4}

\subsection{Strumenti}\label{4.2.6}

\section{Formazione}\label{4.3}