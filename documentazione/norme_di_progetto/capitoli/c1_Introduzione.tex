\chapter{Introduzione}

\section{Scopo del documento}
Lo scopo del documento è quello di formalizzare tutte le regole e procedure fondamentali che ciascun membro di JawaDruids si impegna a rispettare per tutta la durata dello sviluppo del progetto. 
Le norme verranno aggiunte passo dopo passo a seguito di un'attenta analisi e concordate all'interno del gruppo preventivamente. L'attuazione di queste regole e norme permette di ottenere un'organizzazione uniforme ed efficiente dei file prodotti.
\section{Scopo del prodotto}
L'idea del capitolato C3 - GDP: Gathering Detection Protocol nasce in seguito alla pandemia del virus COVID-19 che ha forzato i cittadini ad una quarantena di circa 3 mesi. Successivamente è stata permessa una libera circolazione, nella quale è possibile generare assembramenti nello svolgimento della vita quotidiana. Lo scopo del prodotto è quello di creare non solo una piattaforma che rappresenti mediante visualizzazione grafica zone potenzialmente a rischio di assembramento, ma addirittura cercare di prevenirle.  
\section{Glossario}
All'interno del documento sono presenti termini che possono risultare ambigui o incongruenti a seconda del contesto in cui si trovano. Per evitare il sorgere di incomprensioni viene fornito un glossario individuabile nel file \textit{Glossario} contenente i suddetti termini con la loro relativa spiegazione.
\\Nella seguente documentazione per favorire chiarezza ed evitare inutili ridondanze tali termini verranno indicati mettendo la lettera "G" come pedice ad ogni prima ricorrenza che si incontra ad inizio di ogni sezione.

\section{Riferimenti}
\subsection{Riferimenti normativi}
\begin{itemize}
	\item \textit{Norme di Progetto}
\end{itemize}
\subsection{Riferimenti informativi}
\begin{itemize}
	\item \textit{IEEE Recommended Practice for Software Requirements Specifications:}\\
		\url{https://ieeexplore.ieee.org/document/720574}
	\item \textit{Seminario per approfondimenti tecnici del capitolato C3:}\\
		\url{https://www.math.unipd.it/~tullio/IS-1/2020/Progetto/ST1.pdf}		
\end{itemize}