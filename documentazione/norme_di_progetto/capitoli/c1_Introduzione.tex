\chapter{Introduzione}\label{Introduzione}

\section{Scopo del documento}\label{IntroduzioneScopoDelDocumento}
Lo scopo del documento è quello di formalizzare tutte le regole e procedure$_G$ fondamentali che ciascun membro di \textit{Jawa Druids} si impegna a rispettare per tutta la durata dello sviluppo del progetto. 
Le norme verranno aggiunte passo dopo passo a seguito di un'attenta analisi e concordate all'interno del gruppo preventivamente. L'attuazione di queste regole e norme permette di ottenere un'organizzazione uniforme ed efficiente dei file prodotti.
\section{Scopo del prodotto}\label{1.2}
In seguito alla pandemia del virus COVID-19 è nata l'esigenza di limitare il più possibile i contatti fra le persone, specialmente evitando la formazione di assembramenti. Il progetto \textit{GDP: Gathering Detection Platform} di \textit{Sync Lab} ha pertanto l'obiettivo di \textbf{creare una piattaforma in grado di rappresentare graficamente le zone potenzialmente a rischio di assembramento, al fine di prevenirlo.}
Il prodotto finale è rivolto specificatamente agli organi amministrativi delle singole città, cosicché possano gestire al meglio i punti sensibili di affollamento, come piazze o siti turistici.
Lo scopo che il software intende raggiungere non è solo quello della rappresentazione grafica real-time ma anche quella di poter riuscire a prevedere assembramenti in intervalli futuri di tempo.
% il paragrafo sottostante richiede particolare attenzione durante la revisione
% e probabilmente non dovrebbe essere qui sotto gli Scopi del prodotto
% La raffigurazione di queste aree può portare a molteplici benefici. Il più evidente è che un utente cittadino, consapevole delle zone più a rischio, possa evitarle. Un altro vantaggio risiede nella possibilità di fornire ai progettisti urbani una mappa delle aree di maggior affluenza per poter...

Al tal fine il gruppo \textit{Jawa Druids} si prefigge di sviluppare un prototipo software in grado di acquisire, monitorare ed analizzare i molteplici dati provenienti dai diversi sistemi e dispositivi, a scopo di identificare i possibili eventi che concorrono all’insorgere di variazioni di flussi di utenti. Il gruppo prevede inoltre lo sviluppo di un'applicazione web da interporre fra i dati elaborati e l'utente, per favorirne la consultazione.
\section{Glossario}\label{1.3}
All'interno della documentazione viene fornito un \textit{Glossario}, con l'obiettivo di assistere il lettore specificando il significato e contesto d'utilizzo di alcuni termini strettamente tecnici o ambigui, segnalati con una \textit{G} a pedice.

\section{Riferimenti}\label{IntroduzioneRiferimenti}
\subsection{Riferimenti normativi}\label{IntroduzioneRiferimentiNormativi}
\begin{itemize}
	\item \textit{Norme di Progetto};
	\item \textit{Piano di Progetto}.
\end{itemize}
\subsection{Riferimenti informativi}\label{IntroduzioneRiferimentiInformativi}
\begin{itemize}
	\item \textit{IEEE Recommended Practice for Software Requirements Specifications:}\\
		\url{https://ieeexplore.ieee.org/document/720574}
	\item \textit{Standard ISO/IEC 12207:1995:}\\
		\url{https://www.math.unipd.it/~tullio/IS-1/2009/Approfondimenti/ISO_12207-1995.pdf}
	\item \textit{Seminario per approfondimenti tecnici del capitolato$_G$ C3:}\\
		\url{https://www.math.unipd.it/~tullio/IS-1/2020/Progetto/ST1.pdf}		
\end{itemize}