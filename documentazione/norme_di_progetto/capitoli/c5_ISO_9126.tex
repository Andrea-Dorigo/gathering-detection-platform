\chapter{Standard ISO/IEC 9126}\label{StandardISO/IEC9126}
ISO/IEC 9126$_G$ è uno standard internazionale per la valutazione della qualità del software.
Tale standard definisce i seguenti modelli:
\begin{itemize}
	\item \textbf{Qualità interna:} definisce le metriche applicabili al codice sorgente del software non eseguibile durante le fasi di progettazione e codifica. Tramite tale metriche è possibile identificare eventuali problemi che potrebbero minacciare la qualità del software prima che esso vada in esecuzione. Inoltre possono misurare il comportamento del prodotto finale tramite simulazione. Il rilevamento della qualità interna avviene tramite l’analisi statica. Attraverso le metriche interne è possibile, inoltre, prevedere il livello di qualità esterna ed in uso del prodotto software finale, in quanto, idealmente, la qualità interna determina la qualità esterna. Infine, le metriche interne si applicano anche alla documentazione del prodotto;
	\item \textbf{Qualità esterna:} definisce le metriche applicabili al software in esecuzione che ne misurano i comportamenti attraverso i test, l’operatività, l’osservazione durante l’esecuzione, in funzione degli obiettivi stabiliti. Il rilevamento della qualità esterna avviene tramite l’analisi dinamica. Idealmente, la qualità esterna determina la qualità in uso;
	\item \textbf{Qualità in uso:} definisce le metriche applicabili solo quando il prodotto software è finito e usato in condizioni reali. Inoltre misurano il grado con cui il prodotto permette agli utenti di svolgere le proprie attività$_{\scaleto{G}{3pt}}$ con efficacia, sicurezza, produttività e soddisfazione nel contesto operativo previsto. 
\end{itemize}
\section{Modello della qualità esterna ed interna del software}\label{StandardISO/IEC9126ModelloDellaQualitàEsternaEdInternaDelSoftware}
Il modello della qualità del software descritto dallo standard ISO/IEC 9126$_{\scaleto{G}{3pt}}$ definisce le caratteristiche e le sotto-caratteristiche del software, ciascuna misurabile da metriche interne o esterne. le caratteristiche sono quelle riportate qui di seguito.
\begin{enumerate}
	\item \textbf{Funzionalità:} capacità del software di di fornire le funzioni appropriate, necessarie per soddisfare i bisogni espressi nell'\textit{Analisi dei Requisiti 3.0.0} e per operare in un determinato contesto. Più specificatamente, il software deve soddisfare le seguenti caratteristiche:  
 	\begin{itemize}
 		\item \textbf{Adeguatezza:} capacità di fornire un appropriato insieme di funzioni che permettano di raggiungere gli obiettivi prefissati;
 		\item \textbf{Accuratezza:} capacità di fornire i risultati o gli effetti attesi in maniera corretta e con la precisione richiesta;
 		\item \textbf{Interoperabilità:} capacità di interagire con uno o più sistemi specificati;
 		\item \textbf{Sicurezza:} capacità di proteggere le informazioni ed i dati;
 		\item \textbf{Aderenza:} capacità di aderire agli standard.
 	\end{itemize}
 	\item \textbf{Affidabilità:} capacità di un prodotto software di mantenere il livello di prestazione quando viene usato in condizioni specifiche. Più specificatamente, il software deve soddisfare le seguenti caratteristiche: 
 	\begin{itemize}
 		\item \textbf{Maturità:} capacità di evitare che si verifichino errori o risultati non corretti in fase di esecuzione;
 		\item \textbf{Tolleranza ai guasti:} capacità di mantenere il livello di prestazione in caso di errori nel software o usi scorretti del prodotto finale;
 		\item \textbf{Recuperabilità:} capacità di ripristinare il livello di prestazioni e di recuperare i dati rilevanti in caso di errori o malfunzionamenti;
 		\item \textbf{Aderenza:} capacità di aderire agli standard.
 	\end{itemize}
 	\item \textbf{Usabilità:} capacità di un prodotto software di essere comprensibile ed attraente per l’utente, sotto determinate condizioni. Più specificatamente, il software deve soddisfare le seguenti caratteristiche:
 	\begin{itemize}
 		\item \textbf{Comprensibilità:} capacità di permettere all'utente di capire la sua funzionalità e come poterla utilizzare con successo;
 		\item \textbf{Apprendibilità:} capacità di permettere all'utente di imparare ad usare l’applicazione;
 		\item \textbf{Operabilità:} capacità di permettere all'utente di utilizzarlo e di controllarlo;
 		\item \textbf{Attrattività:} capacità di risultare attraente per l’utente;
 		\item \textbf{Aderenza:} capacità di aderire agli standard.
 	\end{itemize}
 	\item \textbf{Efficienza:} capacità di un prodotto software di realizzare le funzioni richieste nel minor tempo possibile e sfruttando al meglio le risorse necessarie, quando opera in determinate condizioni. Più specificatamente, il software deve soddisfare le seguenti caratteristiche:
 	\begin{itemize}
 		\item \textbf{Comportamento rispetto al tempo:} capacità di fornire appropriati tempi di risposta, tempi di elaborazione e quantità di lavoro eseguendo le funzionalità previste sotto determinate condizioni di utilizzo;
 		\item \textbf{Utilizzo delle risorse:} capacità di utilizzare un appropriato numero e tipo di risorse in maniera adeguata;
 		\item \textbf{Aderenza:} capacità di aderire agli standard.
 	\end{itemize}
 	\item \textbf{Manutenibilità:} capacità di un prodotto software di essere modificato. Le modifiche possono includere correzioni, adattamenti o miglioramenti del software. Più specificatamente, il software deve soddisfare le seguenti caratteristiche:
 	\begin{itemize}
 		\item \textbf{Analizzabilità:} capacità di poter effettuare la diagnosi sul software ed individuare le cause di errori o malfunzionamenti;
 		\item \textbf{Modificabilità:} capacità di consentire lo sviluppo di modifiche al software originale. L’implementazione include modifiche al codice, alla progettazione ed alla documentazione;
 		\item \textbf{Stabilità:} capacità di evitare effetti non desiderati a seguito di modifiche al software;
 		\item \textbf{Testabilità:} capacità di consentire la verifica e validazione del software modificato, cioè di eseguire i test;
 		\item \textbf{Aderenza:} capacità di aderire agli standard.
 	\end{itemize}
	\item \textbf{Portabilità:} capacità di un prodotto software di poter essere trasportato da un ambiente hardware/software ad un altro. Più specificatamente, il software deve soddisfare le seguenti caratteristiche:
	\begin{itemize}
		\item \textbf{Adattabilità:} capacità di adattarsi a diversi ambienti senza richiedere azioni specifiche diverse da quelle previste dal software per tali attività$_{\scaleto{G}{3pt}}$;
		\item \textbf{Installabilità:} capacità di essere installato in un determinato ambiente;
		\item \textbf{Coesistenza:} capacità di coesistere con altre applicazioni indipendenti in ambienti comuni e di condividere le risorse;
		\item \textbf{Sostituibilità:} capacità di sostituire un altro software specifico indipendente per lo stesso scopo e nello stesso ambiente;
		\item \textbf{Aderenza:} capacità di aderire agli standard.
	\end{itemize}
\end{enumerate}

\section{Modello della qualità in uso del software}\label{StandardISO/IEC9126ModelloDellaQualitàInUsoDelSoftware}
Questo modello consente agli utenti di ottenere risultati specifici con:
\begin{itemize}
	\item \textbf{Efficacia:} capacità di permettere all'utente di raggiungere determinati obiettivi con accuratezza e completezza in uno specifico contesto di utilizzo;
	\item \textbf{Produttività:} capacità di permettere all'utente di impiegare un numero definito di risorse, in relazione all'efficienza raggiunta in uno specifico contesto di utilizzo;
	\item \textbf{Sicurezza:} capacità di raggiungere un livello accettabile di rischio rispetto a danni nei confronti di persone, apparecchiature tecniche, software ed ambiente d’uso;
	\item \textbf{Soddisfazione:} capacità di soddisfare gli utenti.
\end{itemize}