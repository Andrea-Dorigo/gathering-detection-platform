\chapter{Standard ISO/IEC 15504}\label{Standard ISO/IEC 15504}
Il modello ISO/IEC 15504$_G$, noto anche come SPICE, acronimo di \textit{Software Process Improvement and Capability Determination} (“capability” è una parola chiave, ovvero la \textit{“capacità”} intesa come abilità di un processo nel raggiungere un obiettivo).
SPICE è uno standard di riferimento per valutare la qualità dei processi con il fine di migliorarli.
Questo standard permette di misurare indipendentemente la \textit{capacità} di ogni processo tramite degli attributi, studiando i risultati che si ottengono eseguendolo, tali risultati devono essere ripetibili, oggettivi e comparabili perchè possano contribuire, appunto, al miglioramento dei processi.
Gli attributi associati alle capacità di ogni processo sono:
\begin{itemize}
	\item \textbf{Process definition:} indica in quale modo il processo fa riferimento agli standard;
	\item \textbf{Process performance:} indica il risultato rispetto al raggiungimento degli 
		obiettivi fissati;
	\item \textbf{Process distribution:} indica quanto si discosta il processo standard rispetto ad un processo 
		definito in grado di raggiungere sempre gli stessi risultati; 
	\item \textbf{Process measurement:} indica il grado in cui i risultati delle misurazioni
		sono utilizzati per garantire che il processo raggiunga gli obiettivi fissati;
	\item \textbf{Process control:} indica quanto il processo è stabile, capace e predicibile 
		entro certo limiti definiti;
	\item \textbf{Process change:} indica in quale misura le modifiche da apportare al processo sono
	identificate grazie ad una fase di analisi delle performance e allo studio di approcci innovativi;
	\item \textbf{Process improvement:} indica in che modo i cambiamenti alle performance e alla definizione del processo 
		hanno un impatto effettivo, che porta a raggiungere importanti obiettivi di miglioramento al processo.
	\item \textbf{Performance management:} indica il grado di organizzazione con cui sono raggiunti gli obiettivi fissati;
	\item \textbf{Work product management:} indica in quale misura i prodotti sono gestiti correttamente per quanto riguarda documentazione, controllo e verifica;
\end{itemize}
Ogni attributo viene misurato e classificato con uno dei seguenti livelli:
\begin{itemize}
	\item \textbf{N - not implemented:} il processo non implementa tale attributo o lo fa in modo molto carente;
	\item \textbf{P - partially implemented:} il processo implementa parzialmente tale attributo 
		tramite un approccio sistematico, tuttavia alcuni aspetti non sono ancora prevedibili;
	\item \textbf{L - largely implemented:} il processo possiede significativamente tale attributo 
		tramite un approccio sistematico, ma il modo in cui è attuato varia nelle diverse unità;
	\item \textbf{F - fully implemented:} l'attributo è stato completamente ottenuto grazie ad un approccio sistematico 
		e l'attuazione è uguale in tutte le unità.
\end{itemize}
Perciò, in base alla classificazione degli attributi, ad un processo viene assegnato un livello di capacità pari a:
\begin{itemize}
	\item[-] \textbf{0 - Incomplete:} il processo è incompleto in quanto non è stato implementato o 
		fallisce nel raggiungere il proprio	obiettivo. Questo livello non ha alcun attributo associato.
	\item[-] \textbf{1 - Performed:} il processo è stato implementato e ha successo nel raggiungere il proprio obiettivo.
		L'attributo associato a questo livello è "process performance".
	\item[-] \textbf{2 - Managed:} il processo, che già apparteneva al livello \textit{performed}, è implementato in 
		maniera organizzata tramite pianificazione, controllo e correzioni; i suoi prodotti sono sicuri. 
		Gli attributi associati a questo livello sono "performance management" e "work product management".
	\item[-] \textbf{3 - Established:} il processo, che già apparteneva al livello \textit{managed},
		è stato implementato come processo definito in grado di raggiungere sempre gli stessi risultati.
		Gli attributi associati a questo livello sono "process definition" e "process distribution".
	\item[-] \textbf{4 - Predictable:} il processo, che già apparteneva al livello \textit{established},
		opera entro limiti definiti per raggiungere i propri risultati.
		Gli attributi associati a questo livello sono "process control" e "process measurement".
	\item[-] \textbf{5 - Optimizing:} il processo, che già apparteneva al livello \textit{predictable},
		è oggetto di miglioramento continuo per raggiungere gli obiettivi di progetto/aziendali.
		Gli attributi associati a questo livello sono "process change" e "process improvement".
\end{itemize}
\begin{figure}[!ht]
	\begin{center}
		\includegraphics[width=1\linewidth]{../immagini/Process_Level.jpg}
		\caption{SPICE Capability (Immagine tratta da HM\&S)}
	\end{center}
\end{figure}


	