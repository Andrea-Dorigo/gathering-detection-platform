\input{packages}
\input{config}

\begin{document}
\makeatletter
\begin{titlepage}
	\begin{center}
		\vspace*{-4cm}
		\author{Jawa Druids}
		\title{Manuale Sviluppatore}
		\date{} %LASCIARE QUESTO CAMPO VUOTO, SE LO TOLGO STAMPA LA DATA CORRENTE
		\includegraphics[width=0.5\linewidth]{../immagini/DRUIDSLOGO.jpg}\\[4ex]
		{\huge \bfseries  \@title }\\[2ex]
		{\LARGE  \@author}\\[50ex]
		\vspace*{-9cm}
		\begin{table}[H]
			\renewcommand{\arraystretch}{1.4}
			\centering
			\begin{tabular}{r | l}
				\textbf{Versione} & 1.0.0 \\%RIGA PER INSERIRE LA VERSIONE ULTIMA DEL DOCUMENTO
				\textbf{Data approvazione} & 2021-05-23 \\
				\textbf{Responsabile} & Andrea Cecchin \\
				\textbf{Redattori} & \makecell[tl]{Mattia Cocco} \\
				\textbf{Verificatori} & \makecell[tl]{Andrea Cecchin \\ Alfredo Graziano} \\
				%MAKECELL SERVE PER POI ANDARE A CAPO ALL'INTERNO DELLA CELLA
				\textbf{Stato} & Approvato\\
				\textbf{Lista distribuzione} & \makecell[tl]{Jawa Druids \\ Prof. Tullio Vardanega \\ Prof. Riccardo Cardin}\\
				\textbf{Uso} & Esterno
			\end{tabular}
		\end{table}
		\vspace{0.1cm}
		\hfill \break
		\fontsize{17}{10}\textbf{Sommario} \\
		\vspace{0.1cm}
		Il documento ha lo scopo di presentare le tecnologie e l'architettura del sistema agli sviluppatori interessati al software \emph{\normalsize{\textit{GDP - Gathering Detection Platform}}}.
	\end{center}
\end{titlepage}
\makeatother
	\def\myformat#1{
	\centering\huge#1
}

\hfill \break
\section*{\myformat{Registro delle modifiche}}

{\rowcolors{2}{Apricot!90!Bittersweet!20}{Bittersweet!70!Apricot!40}
\begin{table}[H]
	\centering
	\begin{tabular}{|c|c|c|c|c|}
		\hline
		\rowcolor{Melon} 
		\textbf{Modifica} & \textbf{Autore} & \textbf{Ruolo} & \textbf{Data} & \textbf{Versione}
		\\
		\hline
		\textit{Capitolo 2.1.2, 2.1.3, 2.1.4} & Andrea Cecchin & \textit{Analista} & 26-11-2020 & v0.0.3 
		\\
		\hline
		\textit{Capitolo 2.1.1} & Andrea Cecchin & \textit{Analista} & 25-11-2020 & v0.0.2 
		\\
		\hline
		\textit{Prima stesura del documento} & Andrea Cecchin & \textit{Analista} & 24-11-2020 & v0.0.1
		\\
		\hline
	\end{tabular}
\end{table}

	\tableofcontents{}
	\listoffigures{}
	\chapter{Introduzione}\label{Introduzione}

\section{Scopo del documento}\label{IntroduzioneScopoDelDocumento}

Lo scopo di questo documento è fornire all'utente tutte le indicazioni per il corretto uso del software$_G$ da noi prodotto.

\section{Scopo del prodotto}\label{IntroduzioneScopoDelProdotto}

In seguito alla pandemia del virus COVID-19 è nata l'esigenza di limitare il più possibile i
contatti fra le persone, specialmente evitando la formazione di assembramenti. 
Il progetto \textit{GDP: Gathering Detection Platform} di \textit{Sync Lab} ha pertanto l'obiettivo di \textbf{creare una piattaforma in grado di rappresentare graficamente le zone potenzialmente a rischio di assembramento, al fine di prevenirlo.}Il prodotto finale è rivolto specificatamente agli
organi amministrativi delle singole città, cosicché possano gestire al meglio i punti sensibili di
affollamento, come piazze o siti turistici. Lo scopo che il software$_{\scaleto{G}{3pt}}$ intende raggiungere non è
solo quello della rappresentazione grafica real-time ma anche di poter riuscire a prevedere
assembramenti in intervalli futuri di tempo.
\\
A tal fine il gruppo \textit{Jawa Druids} si prefigge di sviluppare un prototipo software$_{\scaleto{G}{3pt}}$ in grado di acquisire, monitorare ed analizzare i molteplici dati provenienti dai diversi sistemi e dispositivi, a scopo di identificare i possibili eventi che concorrono all'insorgere di variazioni di flussi di utenti. Il gruppo prevede inoltre lo sviluppo di un'applicazione web$_G$ da interporre fra i dati elaborati e l'utente, per favorirne la consultazione.

\section{Glossario}\label{IntroduzioneGlossario}

All'interno della documentazione viene fornito un \textit{Glossario}, con l'obiettivo di assistere il lettore specificando il significato e contesto d'utilizzo di alcuni termini strettamente tecnici o ambigui, segnalati con una \textit{G} a pedice.

%\section{Riferimenti}\label{IntroduzioneRiferimenti}

%\subsection{Riferimenti informativi}\label{IntroudioneRiferimentiRiferimentiInformativi}
	\chapter{Requisiti di sistema}\label{RequisitiDiSistema}
\section{Requisiti minimi}\label{RequisitiDiSistemaRequisitiMinimi}
Il software è stato testato, con esito positivo, su una macchina con i seguenti requisiti minimi:
\begin{itemize}
  \item RAM: 4Gb;
  \item hard disk: 10Gb;
  \item processore: Intel(R) Core(TM) i7-4710HQ CPU @ 2.50GHz CPU.
\end{itemize}

\section{Browser}\label{RequisitiDiSistemaBrowser}
I browser sui quali è stato testato il software sono i seguenti:
\begin{itemize}
  \item Google Chrome v88 o superiore;
  \item Firefox developer v89.0b1 o superiore;
  \item Edge v o superiore;
  \item Safari v o superiore.
\end{itemize}

	\chapter{Procedure di installazione}\label{ProceduraDiInstallazione}
Questa sezione esporrà le procedure di installazione all'interno del sistema operativo Linux$_G$, più precisamente Ubuntu$_G$ 20.04 LTS, in quanto utilizzato anche per lo sviluppo del software stesso.
Rimane comunque possibile installare il software su altri sistemi operativi soddisfando le dipendenze necessarie, ma non verrà qui esplicitato.

\section{Download della repository}\label{ProceduraDiInstallazioneDownloadRepo}
Per scaricare correttamente i contenuti della repository$_G$ è necessario installare \texttt{git} e \texttt{git-lfs} (\textit{Git Large File Storage}).
Su Ubuntu$_{\scaleto{G}{3pt}}$ 20.04, questo è possibile eseguendo il comando:
\begin{lstlisting}
  sudo apt install git git-lfs
\end{lstlisting}
assumendo che le principali repository$_{\scaleto{G}{3pt}}$ per i pacchetti di Ubuntu$_{\scaleto{G}{3pt}}$ siano attive (Universe, Multiverse).

Questo passaggio è richiesto poiché GitHub$_G$ (il sito che ospita la repository del progetto) consente l'upload di file con dimensioni massime fino a 100MB.
L'utilizzo di \textit{Git Large File Storage} permette l'upload e il download di file che superano questo limite, ed in particolare permette l'upload e download dei pesi necessari all'algoritmo YOLOv3 per il rilevamento di oggetti (più precisamente per il rilevamento delle persone in un'immagine), il quale ha una dimensione maggiore di 200MB. Maggiori informazioni riguardo \textit{Git Large File Storage} sono reperibili all'indirizzo:
\begin{center}
  \item \url{https://git-lfs.github.com} .
\end{center}

È dunque possibile scaricare correttamente la repository$_{\scaleto{G}{3pt}}$ relativa al progetto \textit{Gathering-Detection-Platform} con il seguente comando:
\begin{lstlisting}
  git clone https://github.com/Andrea-Dorigo/gathering-detection-platform.git
\end{lstlisting}
% \begin{center}
%   \item \url{https://github.com/Andrea-Dorigo/gathering-detection-platform}
% \end{center}

\section{Installazione delle dipendenze}\label{ProceduraDiInstallazioneInstallazioneDipendenze}
Dopo aver eseguito il passo sopra descritto, è necessario installare le dipendenze necessarie a far eseguire il prodotto software$_{\scaleto{G}{3pt}}$ adeguatamente.
Per fare ciò è sufficiente aprire il terminale all'interno della cartella \textit{gathering-detection-platform$_{\scaleto{G}{3pt}}$}, ed eseguire i seguenti comandi:
\begin{lstlisting}
  sudo apt install python3-opencv python3-pip mongodb maven npm
\end{lstlisting}
\begin{lstlisting}
  pip3 install mongoengine
\end{lstlisting}
\begin{lstlisting}
  pip3 install kafka-python
\end{lstlisting}
\begin{lstlisting}
  pip3 install image_slicer
\end{lstlisting}
Per installare la dipendenza concernente a Kafka$_G$ si sono seguiti i passaggi presenti al seguente link:
\begin{center}
    \url{https://kafka.apache.org/quickstart}
\end{center}
Durante il suo processo di configurazione il nome del topic$_G$ da inserire è \textbf{numtest}.
Una volta concluse queste operazioni con esito positivo, il programma potrà essere eseguito.

\section{Inizializzazione del modulo di acquisizione}\label{ProceduraDiInstallazioneInizializzazioneModuloAcquisition}
Per eseguire il modulo di acquisizione, in modo da iniziare a raccogliere i dati dalle webcam salvate, bisognerà posizionarsi all'interno della cartella \texttt{acquisition/main/} e da terminale eseguire il comando:
\begin{lstlisting}
    python3 detect.py
\end{lstlisting}
Infine, tramite un'altra finestra del terminale, posizionarsi nella cartella \texttt{acquisition/kafka/} ed eseguire il comando:
\begin{lstlisting}
	python3 kafkaConsumer.py
\end{lstlisting}
Se i passi precedenti sono stati eseguiti correttamente allora si visualizzeranno sul terminale i vari passaggi che svolge il modulo.



\section{Inizializzazione modulo Prediction}\label{ProceduraDiInstallazioneInizializzazioneModuloPrediction}
Per eseguire il modulo di predizione, che tramite il machine-learning$_G$ si occupa di calcolare, appunto, le predizioni del periodo di tempo futuro, bisogna posizionarsi all'interno della cartella \texttt{prediction/} ed eseguire il seguente comando da terminale:

\begin{lstlisting}
    python3 DataPrediction.py
\end{lstlisting}
In tal modo verrà attivato il modulo per le predizioni sui dati.
Saranno visibili sul terminale gli step eseguiti dal programma.


\section{Inizializzazione modulo Web-App}\label{ProceduraDiInstallazioneInizializzazioneModuloWebApp}
Per avviare la web-app$_G$, e le sue funzioni, si devono eseguire alcuni comandi, sempre da terminale, a partire dalla cartella \textit{webapp}.
\begin{enumerate}
  \item posizionarsi all'interno della cartella \texttt{webapp/} ed eseguire:
  \begin{lstlisting}
    mvn spring-boot:run
  \end{lstlisting}
  \item posizionarsi all'interno della cartella "vue-js-client-crud" ed eseguire:
  \begin{lstlisting}
    npm install
  \end{lstlisting}
  \item infine, all'interno della stessa cartella bisogna eseguire:
  \begin{lstlisting}
    npm run serve
  \end{lstlisting}
\end{enumerate}
Il primo comando inizializza il server di Spring$_G$ che fornisce i servizi per prelevare le informazioni dal database, mentre i comandi successivi installano i moduli necessari tramite npm$_G$ ed eseguono l'applicazione.

	\chapter{Tecnologie coinvolte}\label{TecnologieCoinvolte}
In questa sezione vengono elencate le tecnologie, e librerie di terze parti, utilizzate per sviluppare il prodotto software \textit{Gathering-Detection-Platform}.

\subsection{Tecnologie}\label{Tecnologie}
\subsubsection{Python}\label{TecnologiePython}
\textit{Python$_{\scaleto{G}{3pt}}$} è un linguaggio di programmazione definito "ad alto livello" rispetto alla maggior parte di essi.
Si tratta di un linguaggio orientato ad oggetti, utile a sviluppare script, computazione numerica e sviluppare software.
Nel progetto \textit{Gathering-Detection-Platform}, Python$_{\scaleto{G}{3pt}}$ è il linguaggio su cui si basa tutto il backend$_{\scaleto{G}{3pt}}$, compreso il modulo del machine-learning$_{\scaleto{G}{3pt}}$.

\begin{itemize}
  \item Versione utilizzata: 3.8.x;
  \item Link download: \url{https://www.python.org/downloads/} .
\end{itemize}

\subsubsection{MongoDB}\label{TecnologieMongoDB}
\textit{MongoDB} è stato scelto come database$_{\scaleto{G}{3pt}}$ nel quale salvare i dati ottenuti dal modulo di acquisizione e dal modulo di machine-learning$_{\scaleto{G}{3pt}}$.
Si tratta di un database$_{\scaleto{G}{3pt}}$ non relazionale e orientato ai documenti.
Classificato come tipo NoSQL$_{\scaleto{G}{3pt}}$, MongoDB$_{\scaleto{G}{3pt}}$ non utilizza la classica struttura basata su tabelle ma invece si basa su tipi di documenti JSON$_{\scaleto{G}{3pt}}$, facilitando così l'integrazioni di alcuni tipi di dati.

\begin{itemize}
    \item Versione utilizzata: 4.4.4;
    \item Link download: \url{https://www.mongodb.com/it }.
\end{itemize}


\subsubsection{HTML 5}\label{TecnologieHTML}

\subsubsection{CSS}\label{TecnologieCSS}

\subsubsection{Bootstrap}\label{TecnologieBootstrap}

	\chapter{Architettura del Prodotto}\label{ArchitetturaDelProdotto}
L'architettura generale del software$_{\scaleto{G}{3pt}}$ \textit{Gathering-Detection-Platform} è un'architettura monolitica distribuita.
Si tratta di due file eseguibili, scritti in Python$_{\scaleto{G}{3pt}}$, che sono rispettivamente il modulo Acquisition e il modulo Prediction.
Il primo viene utilizzato per acquisire le informazione estrapolate dalle live webcams delle città, il secondo invece viene utilizzato per calcolare predizioni su periodi di tempo futuri, utilizzando il machine-learning$_{\scaleto{G}{3pt}}$.
Entrambi sono collegati ad un database, il quale serve per salvare ed esportare i dati in esso.
Il terzo, ed ultimo, modulo riguarda la web-app$_{\scaleto{G}{3pt}}$ vera e propria, che permette all'utente utilizzatore di visualizzare la heat-map$_{\scaleto{G}{3pt}}$ relativa alla città.


\section{Architettura modulo Acquisition}
\subsection{Diagramma dei Package}
\begin{figure}[!h]
  \begin{center}
    \includegraphics[width=1\linewidth]{../immagini/diag_PB/diag_pack_acqui.png}
    \caption{Diagramma dei package del modulo Acquisition}
  \end{center}
\end{figure}

\subsection{Diagrammi di attività}
\begin{figure}[!h]
  \begin{center}
    \includegraphics[width=1\linewidth]{../immagini/diag_PB/detection.png}
    \caption{Diagramma di attività dell'eseguibile Detection}
  \end{center}
\end{figure}
\begin{figure}[!h]
  \begin{center}
    \includegraphics[width=1\linewidth]{../immagini/diag_PB/detection.png}
    \caption{Diagramma di sotto-attività dell'acquisizione delle previsioni meteo}
  \end{center}
\end{figure}
\begin{figure}[!h]
  \begin{center}
    \includegraphics[width=1\linewidth]{../immagini/diag_PB/download_e_cut_frames.png}
    \caption{Diagramma di sotto-attività di download e taglio frame}
  \end{center}
\end{figure}
\begin{figure}[!h]
  \begin{center}
    \includegraphics[width=1\linewidth]{../immagini/diag_PB/conta_persone.png}
    \caption{Diagramma di sotto-attività del conta persone}
  \end{center}
\end{figure}
\begin{figure}[!h]
  \begin{center}
    \includegraphics[width=1\linewidth]{../immagini/diag_PB/kafka.png}
    \caption{Diagramma di attività di Kafka}
  \end{center}
\end{figure}


\section{Architettura modulo Prediction}
\subsection{Diagramma dei package}
\begin{figure}[!h]
  \begin{center}
    \includegraphics[width=1\linewidth]{../immagini/diag_PB/diag_pack_pred.png}
    \caption{Diagramma dei package del modulo Acquisition}
  \end{center}
\end{figure}

\subsection{Diagramma di attività}
\begin{figure}[!h]
  \begin{center}
    \includegraphics[width=1\linewidth]{../immagini/diag_PB/prediction_activity.png}
    \caption{Diagramma di attività dell'eseguibile Detection}
  \end{center}
\end{figure}



\section{Architettura modulo Web-app}
\subsection{Diagrammi dei package}
\begin{figure}[!h]
  \begin{center}
    \includegraphics[width=1\linewidth]{../immagini/diag_PB/diag_pack_spring.png}
    \caption{Diagramma dei package di Spring}
  \end{center}
\end{figure}

\begin{figure}[!h]
  \begin{center}
    \includegraphics[width=1\linewidth]{../immagini/diag_PB/diag_pack_vue.png}
    \caption{Diagramma dei package del modulo Acquisition}
  \end{center}
\end{figure}

\subsection{Diagrammi delle classi}
\begin{figure}[!h]
  \begin{center}
    \includegraphics[width=1\linewidth]{../immagini/diag_PB/diag_class_spring.png}
    \caption{Diagramma delle classi di Spring}
  \end{center}
\end{figure}

\subsection{Diagramma di sequenza}
\begin{figure}[!h]
  \begin{center}
    \includegraphics[width=1\linewidth]{../immagini/diag_PB/diag_seq_spring.png}
    \caption{Diagramma di sequenza di Spring}
  \end{center}
\end{figure}

\subsection{Diagramma di attività}
\begin{figure}[!h]
  \begin{center}
    \includegraphics[width=1\linewidth]{../immagini/diag_PB/diag_act_front_back.png}
    \caption{Diagramma di attività del modulo Web-app}
  \end{center}
\end{figure}

	\chapter{Test}\label{Test}
Vengono di seguito elencati gli strumenti per effettuare i test che il gruppo ha sviluppato sul prodotto.

\section{Test modulo Acquisition}\label{TestModuloAcquisition}
Per il modulo di acquisizione dati sono presenti degli Unit Test all'interno della cartella \\ \texttt{acquisition/main/test}, che verificano il corretto funzionamento di alcuni metodi contenuti dagli eseguibili \texttt{detect.py} e \texttt{weather\_forecast.py}.
Si può verificare la correttezza dei metodi eseguendo i programmi all'interno della cartella, ad esempio:
\begin{lstlisting}
  python3 test_detect.py
\end{lstlisting}

\section{Test Back-end Webapp}\label{TestBackendWebapp}
Per testare il back-end$_{\scaleto{G}{3pt}}$ dell'applicazione web sono presenti alcuni test di unità; si consiglia di utilizzare un IDE con la possibilità di eseguire test in Java direttamente dall'interfaccia grafica.
Nel caso si preferisse utilizzare il terminale, è sufficiente posizionarsi all'interno della cartella \texttt{webapp/webapp/} ed eseguire il comando:
\begin{lstlisting}
  mvn clean install
\end{lstlisting}
che eseguirà tutti i test necessari.
Tali test vengono comunque eseguiti all'avvio del back-end$_{\scaleto{G}{3pt}}$ dell'applicativo.

\section{Test Front-end Webapp}\label{TestFrontendWebapp}
Per testare il frontend dell'applicazione basta posizionarsi all'interno della cartella \\ \texttt{webapp/vue-js-client-crud/} e lanciare il comando:
\begin{lstlisting}
  npm run test:unit
\end{lstlisting}
che eseguirà tutti i test riguardanti il front-end$_{\scaleto{G}{3pt}}$ con Vue.

	\chapter{Informazione aggiuntive}\label{InformazioniAggiuntive}

\section{Aggiunta di una webcam}\label{InformazioniAggiuntiveAggiuntaDiUnaWebcam}
L'aggiunta di una nuova webcam al \textit{modulo di acquisizione} è possibile attraverso dei pochi semplici passi:
\begin{enumerate}
	\item Trovare una webcam disponibile all'interno del sito \url{https://www.whatsupcams.com/};
	\item Inserire il link all'interno del file \textit{webcams.json}, presente nella cartella \textit{acquisition} seguendo lo schema prestabilito per impostare i parametri della webcam;
	\item Salvare il file per ultimare l'aggiunta.
\end{enumerate}
Per una questione di codifica, il link della webcam dev'essere conforme a quelle già presenti, ovvero provenire da \url{https://www.whatsupcams.com/}.

\section{Tracciamento degli errori}\label{InformazioniAggiuntiveTracciamentoDegliErrori}
Per tracciare gli errori è stato creato il file \textit{test.log}, presente nel percorso \textit{/acquisition/main/test/}, che tiene traccia delle eventuali eccezioni che si verificano all'interno del file \textit{detect.py}.
In \textit{test.log} vengono rappresentati dei messaggi quali:
\begin{itemize}
  \item \textit{Debug}: rappresenta la richiesta effettuata all'\textit{API Weather Forecast} per prelevare le informazioni meteo;
  \item \textit{INFO}: rappresenta il verificarsi di un'eccezione specificando data e ora di quest'ultima;
  \item \textit{Error}: specifica il tipo di errore e in quale locazione si è verificato.
\end{itemize}

	\chapter{Glossario} \label{Glossario}
\textbf{A}\\
\\
\textbf{Application client-server} \\
Indica un'applicazione basata su un'architettura di rete nella quale, generalmente, un computer si collega ad un server$_G$ per le fruizione di und eterminato servizio.\\
\\
\textbf{Applicazioni single-page} \\
S'intende una web-app$_G$ o un sito internet che può essere usato, o consultato, tramite una singola pagina, fornendo così un'esperienza più fluida e intuitiva all'utente.\\
\\
\textbf{B} \\
\\
\textbf{Back-end} \\
Interfaccia con la quale il gestore di un sito web dinamico ne gestisce i contenuti e le funzionalità. A differenza del frontend$_G$, l'accesso al backend è riservato agli amministratori del sito che possono accedere dopo essersi autenticati.\\
\\
\textbf{Build automation} \\
In informatica è l'atto di scrivere o automatizzare un'ampia varietà di compiti che gli sviluppatori software$_G$ fanno nelle loro attività quotidiane di sviluppo.\\
\\
\textbf{C} \\
\\
\textbf{Client} \\
In ambito informatico si intendo i dispositivi collegati ad un server ed in grado di scambiarvici informazioni.\\
\\
\textbf{D} \\
\\
\textbf{Database} \\
Insieme strutturati, ovvero omogenei per contenuti e formato, rappresentanti, digitalmente, un archivio dati.\\
\\
\textbf{Document-object manager} \\
Abbreviato in "DOM", in italiano è tradotto letteralmente modello a oggetti del documento, è una forma di rappresentazione dei documenti strutturati come modello orientato agli oggetti.\\
\\
\textbf{F} \\
\\
\textbf{Framework}\\
Utilizzato per descrivere la struttura operativa nella quale viene elaborato un dato software$_{\scaleto{G}{3pt}}$.
Un framework$_{\scaleto{G}{3pt}}$, in generale, include software di supporto, librerie, un linguaggio per gli script$_G$ e altri software$_{\scaleto{G}{3pt}}$ che possono aiutare a mettere insieme le varie componenti di un progetto.\\
\\
\textbf{Front-End/Front end/Frontend} \\
Parte di un sistema software che gestisce l'interazione con l'utente o con sistemi esterni che producono dati di ingresso.
Tali dati sono poi utilizzabili dal Back-end$_G$.\\
\\
\textbf{G}\\
\\
\textbf{Git}\\
Sistema di controllo gratuito a versione distribuita progettato per tenere traccia del lavoro svolto durante l'intero periodo di sviluppo del software.
Utilizzato anche per tenere traccia di tutte le modifiche fatte nei file.
I suoi punti di forza sono l'integrità dei dati e il supporto per flussi di lavoro distribuiti e non lineari.\\
\\
\textbf{GitHub}\\
GitHub$_G$ è un servizio di hosting per progetti software. Il nome deriva dal fatto che esso è una implementazione dello strumento di controllo versione distribuito Git$_G$.\\
\\
\textbf{H}\\
\\
\textbf{Heat-map}\\
Rappresentazione grafica dei dati dove i singoli valori contenuti in una matrice sono rappresentati da colori.\\
\\
\textbf{L}\\
\\
\textbf{Layout}\\
In informatica si intende la disposizione degli elementi che costituiscono una pagina internet.\\
\\
\textbf{Linux}\\
Si tratta di una famiglia di sistemi operativi open-source$_G$ pubblicati poi in varie distribuzioni.\\
\\
\textbf{M}\\
\\
\textbf{Machine-learnig}\\
Metodo di analisi dati che automatizza la costruzione di modelli analitici. È una branca dell'Intelligenza Artificiale e si basa sull'idea che i sistemi possono imparare dai dati, identificare modelli autonomamente e prendere decisioni con un intervento umano ridotto al minimo.\\
\\
\textbf{MVC}\\
È un pattern architetturale, l'acronimo MVC$_G$ sta per \textit{Model-View-Controller}.
Il \textit{model} si occupa della rappresentazione dei dati in oggetti.
Il \textit{view} gestisce la rappresentazione grafica di essi.
Infine il \textit{controller} si occupa delle interazioni degli utenti.\\
\\
\textbf{N}\\
\\
\textbf{NoSQL}\\
 È un movimento che promuove sistemi software dove la persistenza dei dati è in generale caratterizzata dal fatto di non utilizzare il modello relazionale, di solito usato dalle basi di dati tradizionali.\\
\\
\textbf{NPM}\\
Gestore di pacchetti per il linguaggio di programmazione JavaScript$_{\scaleto{G}{3pt}}$, consiste in un client$_{\scaleto{G}{3pt}}$ da linea di comando, chiamato anch'esso npm, e un database$_{\scaleto{G}{3pt}}$ online di pacchetti pubblici e privati, chiamato npm registry.\\
\\
\textbf{O}\\
\\
\textbf{Open source}\\
Un software open-source$_{\scaleto{G}{3pt}}$ è reso tale per mezzo di una licenza attraverso cui i detentori dei diritti favoriscono la modifica, lo studio, l'utilizzo e la redistribuzione del codice sorgente.\\
\\
\textbf{P}\\
\\
\textbf{Pom}\\
Acronimo di Project Object Model è l'unità fondamentale di lavoro di Maven al cui interno sono presenti le configurazioni e le informazioni riferite al progetto.\\
\\
\textbf{R}\\
\\
\textbf{Real-time}\\
Tradotto in italiano: in tempo reale.\\
\\
\textbf{Repository-Repo}\\
Ambiente di un sistema informativo in cui vengono conservati e gestiti file, documenti e metadati relativi ad un’attività di progetto.\\
\\
\textbf{Run-time system}\\
Termine utilizzato per indicare un software che fornisce i servizi necessari all'esecuzione di un programma.\\
\\
\textbf{S}\\
\\
\textbf{Script}\\
File contenente codice eseguibile.\\
\\
\textbf{Server}\\
È un componente, o sotto sistema, adibito all'elaborazione e gestione del traffico dati fornito da servizi verso altre componenti.\\
\\
\textbf{Software}\\
È l'insieme delle procedure e delle istruzioni in un sistema di elaborazione dati.\\
\\
\textbf{Spring}\\
In informatica Spring è un framework$_{\scaleto{G}{3pt}}$ open-source$_{\scaleto{G}{3pt}}$ per lo sviluppo di applicazioni su piattaforma Java$_{\scaleto{G}{3pt}}$.\\
\\
\textbf{Streaming}\\
Identifica un flusso di dati audio/video trasmessi da una sorgente a una o più destinazioni tramite una rete telematica. Questi dati vengono riprodotti man mano che arrivano a destinazione.\\
\\
\textbf{T}\\
\\
\textbf{Tomcat}\\
(Apache) è un server web open-source$_{\scaleto{G}{3pt}}$.\\
\\
\textbf{Topic}\\
Nell'ambito di Apache Kafka, si intende una categoria per utilizzata per raggruppare i messaggi.\\
\\
\textbf{U}\\
\\
\textbf{Ubuntu}\\
Sistema operativo basato su Linux$_G$, più precisamente su Debian.\\
\\
\textbf{W}\\
\\
\textbf{Web API}\\
Un'API Web è un'interfaccia di programmazione dell'applicazione per un server Web o un browser Web.
In questo caso è un sito sul quale si appoggia il software$_G$ per acquisire delle informazioni.\\
\\
\textbf{Web-app/Applicazioni web}\\
In ambito informatico si intende un'applicazione web, ovvero applicazioni fruibili mediante via web, come un sito internet che offre determinati servizi al client$_{\scaleto{G}{3pt}}$.\\
\\
\textbf{X}\\
\\
\textbf{XML}\\
È un formato di file appertenente a script$_{\scaleto{G}{3pt}}$ scritti linguaggio col medesimo nome.
Si tratta di un linguaggio di markup ossia basato su un meccanismo sintattico che consente di definire e controllare il significato degli elementi contenuti in un documento o in un testo.\\
\\


	%  glossary would go here

\end{document}
