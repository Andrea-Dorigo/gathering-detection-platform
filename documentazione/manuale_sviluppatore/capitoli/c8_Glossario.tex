\chapter{Glossario} \label{Glossario}
\textbf{A}\\
\\
\textbf{Application client-server} \\
Indica un'applicazione basata su un'architettura di rete nella quale, generalmente, un computer si collega ad un server$_G$ per le fruizione di und eterminato servizio.\\
\\
\textbf{Applicazioni single-page} \\
S'intende una web-app$_G$ o un sito internet che può essere usato, o consultato, tramite una singola pagina, fornendo così un'esperienza più fluida e intuitiva all'utente.\\
\\
\textbf{B} \\
\\
\textbf{Back-end} \\
Interfaccia con la quale il gestore di un sito web dinamico ne gestisce i contenuti e le funzionalità. A differenza del frontend$_G$, l'accesso al backend è riservato agli amministratori del sito che possono accedere dopo essersi autenticati.\\
\\
\textbf{Build automation} \\
In informatica è l'atto di scrivere o automatizzare un'ampia varietà di compiti che gli sviluppatori software$_G$ fanno nelle loro attività quotidiane di sviluppo.\\
\\
\textbf{C} \\
\\
\textbf{Client} \\
In ambito informatico si intendo i dispositivi collegati ad un server ed in grado di scambiarvici informazioni.\\
\\
\textbf{D} \\
\\
\textbf{Database} \\
Insieme strutturati, ovvero omogenei per contenuti e formato, rappresentanti, digitalmente, un archivio dati.\\
\\
\textbf{Document-object manager} \\
Abbreviato in "DOM", in italiano è tradotto letteralmente modello a oggetti del documento, è una forma di rappresentazione dei documenti strutturati come modello orientato agli oggetti.\\
\\
\textbf{F} \\
\\
\textbf{Framework}\\
Utilizzato per descrivere la struttura operativa nella quale viene elaborato un dato software$_{\scaleto{G}{3pt}}$.
Un framework$_{\scaleto{G}{3pt}}$, in generale, include software di supporto, librerie, un linguaggio per gli script$_G$ e altri software$_{\scaleto{G}{3pt}}$ che possono aiutare a mettere insieme le varie componenti di un progetto.\\
\\
\textbf{Front-End/Front end/Frontend} \\
Parte di un sistema software che gestisce l'interazione con l'utente o con sistemi esterni che producono dati di ingresso.
Tali dati sono poi utilizzabili dal Back-end$_G$.\\
\\
\textbf{G}\\
\\
\textbf{Git}\\
Sistema di controllo gratuito a versione distribuita progettato per tenere traccia del lavoro svolto durante l'intero periodo di sviluppo del software.
Utilizzato anche per tenere traccia di tutte le modifiche fatte nei file.
I suoi punti di forza sono l'integrità dei dati e il supporto per flussi di lavoro distribuiti e non lineari.\\
\\
\textbf{GitHub}\\
GitHub$_G$ è un servizio di hosting per progetti software. Il nome deriva dal fatto che esso è una implementazione dello strumento di controllo versione distribuito Git$_G$.\\
\\
\textbf{H}\\
\\
\textbf{Heat-map}\\
Rappresentazione grafica dei dati dove i singoli valori contenuti in una matrice sono rappresentati da colori.\\
\\
\textbf{L}\\
\\
\textbf{Layout}\\
In informatica si intende la disposizione degli elementi che costituiscono una pagina internet.\\
\\
\textbf{Linux}\\
Si tratta di una famiglia di sistemi operativi open-source$_G$ pubblicati poi in varie distribuzioni.\\
\\
\textbf{M}\\
\\
\textbf{Machine-learnig}\\
Metodo di analisi dati che automatizza la costruzione di modelli analitici. È una branca dell'Intelligenza Artificiale e si basa sull'idea che i sistemi possono imparare dai dati, identificare modelli autonomamente e prendere decisioni con un intervento umano ridotto al minimo.\\
\\
\textbf{MVC}\\
È un pattern architetturale, l'acronimo MVC$_G$ sta per \textit{Model-View-Controller}.
Il \textit{model} si occupa della rappresentazione dei dati in oggetti.
Il \textit{view} gestisce la rappresentazione grafica di essi.
Infine il \textit{controller} si occupa delle interazioni degli utenti.\\
\\
\textbf{N}\\
\\
\textbf{NoSQL}\\
 È un movimento che promuove sistemi software dove la persistenza dei dati è in generale caratterizzata dal fatto di non utilizzare il modello relazionale, di solito usato dalle basi di dati tradizionali.\\
\\
\textbf{NPM}\\
Gestore di pacchetti per il linguaggio di programmazione JavaScript$_{\scaleto{G}{3pt}}$, consiste in un client$_{\scaleto{G}{3pt}}$ da linea di comando, chiamato anch'esso npm, e un database$_{\scaleto{G}{3pt}}$ online di pacchetti pubblici e privati, chiamato npm registry.\\
\\
\textbf{O}\\
\\
\textbf{Open source}\\
Un software open-source$_{\scaleto{G}{3pt}}$ è reso tale per mezzo di una licenza attraverso cui i detentori dei diritti favoriscono la modifica, lo studio, l'utilizzo e la redistribuzione del codice sorgente.\\
\\
\textbf{P}\\
\\
\textbf{Pom}\\
Acronimo di Project Object Model è l'unità fondamentale di lavoro di Maven al cui interno sono presenti le configurazioni e le informazioni riferite al progetto.\\
\\
\textbf{R}\\
\\
\textbf{Real-time}\\
Tradotto in italiano: in tempo reale.\\
\\
\textbf{Repository-Repo}\\
Ambiente di un sistema informativo in cui vengono conservati e gestiti file, documenti e metadati relativi ad un’attività di progetto.\\
\\
\textbf{Run-time system}\\
Termine utilizzato per indicare un software che fornisce i servizi necessari all'esecuzione di un programma.\\
\\
\textbf{S}\\
\\
\textbf{Script}\\
File contenente codice eseguibile.\\
\\
\textbf{Server}\\
È un componente, o sotto sistema, adibito all'elaborazione e gestione del traffico dati fornito da servizi verso altre componenti.\\
\\
\textbf{Software}\\
È l'insieme delle procedure e delle istruzioni in un sistema di elaborazione dati.\\
\\
\textbf{Spring}\\
In informatica Spring è un framework$_{\scaleto{G}{3pt}}$ open-source$_{\scaleto{G}{3pt}}$ per lo sviluppo di applicazioni su piattaforma Java$_{\scaleto{G}{3pt}}$.\\
\\
\textbf{Streaming}\\
Identifica un flusso di dati audio/video trasmessi da una sorgente a una o più destinazioni tramite una rete telematica. Questi dati vengono riprodotti man mano che arrivano a destinazione.\\
\\
\textbf{T}\\
\\
\textbf{Tomcat}\\
(Apache) è un server web open-source$_{\scaleto{G}{3pt}}$.\\
\\
\textbf{Topic}\\
Nell'ambito di Apache Kafka, si intende una categoria per utilizzata per raggruppare i messaggi.\\
\\
\textbf{U}\\
\\
\textbf{Ubuntu}\\
Sistema operativo basato su Linux$_G$, più precisamente su Debian.\\
\\
\textbf{W}\\
\\
\textbf{Web API}\\
Un'API Web è un'interfaccia di programmazione dell'applicazione per un server Web o un browser Web.
In questo caso è un sito sul quale si appoggia il software$_G$ per acquisire delle informazioni.\\
\\
\textbf{Web-app/Applicazioni web}\\
In ambito informatico si intende un'applicazione web, ovvero applicazioni fruibili mediante via web, come un sito internet che offre determinati servizi al client$_{\scaleto{G}{3pt}}$.\\
\\
\textbf{X}\\
\\
\textbf{XML}\\
È un formato di file appertenente a script$_{\scaleto{G}{3pt}}$ scritti linguaggio col medesimo nome.
Si tratta di un linguaggio di markup ossia basato su un meccanismo sintattico che consente di definire e controllare il significato degli elementi contenuti in un documento o in un testo.\\
\\
