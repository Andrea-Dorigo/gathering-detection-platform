\chapter{Glossario} \label{Glossario}
\textbf{A} \\
\textbf{Application client-server} \\
\textbf{Applicazioni single-page} \\	
\\
\textbf{B} \\
\\
\textbf{Back-end} \\
Interfaccia con la quale il gestore di un sito web dinamico ne gestisce i contenuti e le funzionalità. A differenza del frontend$_G$, l'accesso al backend è riservato agli amministratori del sito che possono accedere dopo essersi autenticati.\\
\\
\textbf{Bootstrap} \\
\textbf{Build automation} \\
In informatica è l'atto di scrivere o automatizzare un'ampia varietà di compiti che gli sviluppatori software fanno nelle loro attività quotidiane di sviluppo.\\
\\
\textbf{C} \\
\\
\textbf{Client} \\
\textbf{CSS} \\
Linguaggio utilizzato per la formattazione di documenti HTML.
Serve per separare i contenuti delle pagine HTML dalla loro formattazione o layout, inoltre permette una programmazione più chiara e una più facile manutenibilità del codice stesso.\\
\\
\textbf{D} \\
\\
\textbf{Database} \\
\textbf{Document-object manager} \\
\textbf{Duplicare} \\
\\
\textbf{F} \\
\\
\textbf{Framework}\\
Utilizzato per descrivere la struttura operativa nella quale viene elaborato un dato software.
Un framework, in generale, include software di supporto, librerie, un linguaggio per gli script e altri software che possono aiutare a mettere insieme le varie componenti di un progetto.\\
\\
\textbf{H}\\
\\
\textbf{Heat-map}\\
Rappresentazione grafica dei dati dove i singoli valori contenuti in una matrice sono rappresentati da colori.\\
\\
\textbf{J}
\\
\textbf{Java}\\
Linguaggio di programmazione ad alto livello, orientato agli oggetti e a tipizzazione statica, che si appoggia sull'omonima piattaforma software di esecuzione, specificamente progettato per essere il più possibile indipendente dalla piattaforma hardware di esecuzione. Le principali caratteristiche di Java sono la portabilità, cioè il codice sorgente e' compilato in bytecode e può essere eseguito su ogni PC che ha JVM (Java Virtual Machine), e la robustezza.\\
\\
\textbf{Javascript}\\
Linguaggio di programmazione orientato ad oggetti ed eventi.
Utilizzato specificatamente per la programmazione Web lato client.
Aggiunge al sito effetti dinamici tramite funzioni invocate da un'azione eseguita sulla pagine Web(es. click del mouse, movimento del mouse, caricamento pagina).\\
\\
\textbf{L}\\
\\
\textbf{Layout}\\
\textbf{Leaflet}\\
Libreria open-source$_G$ di JavaScript che permette la realizzazione di mappe interattive che funzionano efficientemente sia su desktop che su mobile.\\
\textbf{Linux}\\
\\
\textbf{M}\\
\\
\textbf{Machine-learnig}\\
Metodo di analisi dati che automatizza la costruzione di modelli analitici. È una branca dell'Intelligenza Artificiale e si basa sull'idea che i sistemi possono imparare dai dati, identificare modelli autonomamente e prendere decisioni con un intervento umano ridotto al minimo.\\
\textbf{Maven}\\
(Apache) Maven è uno strumento di gestione progetti software basati su Java e Build Automation.
Supporta anche altri linguaggi quali C\#, Ruby e Scala.
Mediante il file POM.xml vengono descritte le dipendenze fra il progetto e le varie versioni di librerie necessarie nonché le dipendenze fra di esse.
Maven effettua automaticamente il download delle librerie necessarie tra i vari repository$_G$ definiti scaricandoli in locale o in un repository centralizzato lato sviluppo.
Ciò permette un maggior controllo in caso si debba andare a cercare una determinata libreria.\\
\textbf{MongoDB}\\
\textbf{MVC}\\
\\
\textbf{N}\\
\\
\textbf{Node.js}\\
\textbf{NoSQL}\\
\\
\textbf{O}\\
\\
\textbf{Open source}\\
Un software open-source è reso tale per mezzo di una licenza attraverso cui i detentori dei diritti favoriscono la modifica, lo studio, l'utilizzo e la redistribuzione del codice sorgente.\\
\\
\textbf{P}\\
\\
\textbf{Pom}\\
\textbf{Python}\\
Python è un linguaggio di programmazione ad alto livello, rilasciato pubblicamente per la prima volta nel 1991 dal suo creatore Guido van Rossum, supporta diversi paradigmi di programmazione, come quello orientato agli oggetti (con supporto all'ereditarietà multipla), quello imperativo e quello funzionale, ed offre una tipizzazione dinamica forte.
Python è un linguaggio pseudocompilato: un interprete si occupa di analizzare il codice sorgente e, se sintatticamente corretto, di eseguirlo. Questa caratteristica rende Python un linguaggio portabile. Una volta scritto un sorgente, esso può essere interpretato ed eseguito sulla gran parte delle piattaforme attualmente utilizzate, semplicemente basta la presenza della versione corretta dell’interprete.\\
\\
\textbf{R}
\\
\textbf{Real-time}\\
\textbf{Repository}\\
Ambiente di un sistema informativo in cui vengono conservati e gestiti file, documenti e metadati relativi ad un’attività$_G$ di progetto.\\
\\
\textbf{S}
\\
\textbf{Script}\\
\textbf{Server}\\
\textbf{Software}\\
\textbf{Spring}\\
In informatica Spring è un framework$_G$ open-source$_G$ per lo sviluppo di applicazioni su piattaforma Java$_G$.\\
\textbf{Spring Boot}\\
\\
\textbf{U}
\\
\textbf{Ubuntu}\\
\\
\textbf{V}
\\
\textbf{Vue.js}\\
Framework$_{\scaleto{G}{3pt}}$ open-source$_{\scaleto{G}{3pt}}$ per lo sviluppo di applicazioni web, interfacce utente e applicazioni a singola pagina.\\
\\
\textbf{W}
\\
\textbf{Web-app}\\
\\
\textbf{X}
\\
\textbf{XML}\\
\\