\chapter{Test}\label{Test}
Vengono di seguito elencati gli strumenti per effettuare i test che il gruppo ha sviluppato sul prodotto.

\section{Test modulo Acquisition}\label{TestModuloAcquisition}
Per il modulo di acquisizione dati sono presenti degli Unit Test all'interno della cartella \\ \texttt{acquisition/main/test}, che verificano il corretto funzionamento di alcuni metodi contenuti dagli eseguibili \texttt{detect.py} e \texttt{weather\_forecast.py}.
Si può verificare la correttezza dei metodi eseguendo i programmi all'interno della cartella, ad esempio:
\begin{lstlisting}
  python3 test_detect.py
\end{lstlisting}

\section{Test Back-end Webapp}\label{TestBackendWebapp}
Per testare il back-end$_{\scaleto{G}{3pt}}$ dell'applicazione web sono presenti alcuni test di unità; si consiglia di utilizzare un IDE con la possibilità di eseguire test in Java direttamente dall'interfaccia grafica.
Nel caso si preferisse utilizzare il terminale, è sufficiente posizionarsi all'interno della cartella \texttt{webapp/webapp/} ed eseguire il comando:
\begin{lstlisting}
  mvn clean install
\end{lstlisting}
che eseguirà tutti i test necessari.
Tali test vengono comunque eseguiti all'avvio del back-end$_{\scaleto{G}{3pt}}$ dell'applicativo.

\section{Test Front-end Webapp}\label{TestFrontendWebapp}
Per testare il frontend dell'applicazione basta posizionarsi all'interno della cartella \\ \texttt{webapp/vue-js-client-crud/} e lanciare il comando:
\begin{lstlisting}
  npm run test:unit
\end{lstlisting}
che eseguirà tutti i test riguardanti il front-end$_{\scaleto{G}{3pt}}$ con Vue.
