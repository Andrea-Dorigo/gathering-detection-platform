\chapter{Procedure di installazione}\label{ProceduraDiInstallazione}
Questa sezione esporrà le procedure di installazione all'interno del sistema operativo Linux$_G$, più precisamente Ubuntu$_G$ 20.04 LTS, in quanto utilizzato anche per lo sviluppo del software stesso.
Rimane comunque possibile installare il software su altri sistemi operativi soddisfando le dipendenze necessarie, ma non verrà qui esplicitato.

\section{Download della repository}\label{ProceduraDiInstallazioneDownloadRepo}
Per scaricare correttamente i contenuti della repository$_G$ è necessario installare \texttt{git} e \texttt{git-lfs}(\textit{Git Large File Storage}).
Su Ubuntu$_{\scaleto{G}{3pt}}$ 20.04, questo è possibile eseguendo il comando:
\begin{lstlisting}
  sudo apt install git git-lfs
\end{lstlisting}
assumendo che le principali repository$_{\scaleto{G}{3pt}}$ per i pacchetti di Ubuntu$_{\scaleto{G}{3pt}}$ siano attive (Universe, Multiverse).

Questo passaggio è richiesto poiché GitHub$_G$ (il sito che ospita la repository del progetto) consente l'upload di file con dimensioni massime fino a 100MB.
L'utilizzo di \textit{Git Large File Storage} permette l'upload e il download di file che superano questo limite, ed in particolare permette l'upload e download dei pesi necessari all'algoritmo YOLOv3 per il rilevamento di oggetti (più precisamente per il rilevamento delle persone in un'immagine), il quale ha una dimensione maggiore di 200MB. Maggiori informazioni riguardo \textit{Git Large File Storage} sono reperibili all'indirizzo:
\begin{center}
  \item \url{https://git-lfs.github.com} .
\end{center}

È dunque possibile scaricare correttamente la repository$_{\scaleto{G}{3pt}}$ relativa al progetto \textit{Gathering-Detection-Platform} con il seguente comando:
\begin{lstlisting}
  git clone https://github.com/Andrea-Dorigo/gathering-detection-platform.git
\end{lstlisting}
% \begin{center}
%   \item \url{https://github.com/Andrea-Dorigo/gathering-detection-platform}
% \end{center}

\section{Installazione delle dipendenze}\label{ProceduraDiInstallazioneInstallazioneDipendenze}
Dopo aver eseguito il passo sopra descritto, è necessario installare le dipendenze necessarie a far eseguire il prodotto software$_{\scaleto{G}{3pt}}$ adeguatamente.
Per fare ciò è sufficiente aprire il terminale all'interno della cartella \textit{gathering-detection-platform$_{\scaleto{G}{3pt}}$}, ed eseguire il seguente comando:
\begin{lstlisting}
  sudo apt install python3-opencv python3-pip mongo maven npm && pip3 install mongoengine
\end{lstlisting}

Una volta conclusa questa operazione con esito positivo, il programma potrà essere eseguito.

\section{Inizializzazione del modulo di acquisizione}\label{ProceduraDiInstallazioneInizializzazioneModuloAcquisition}
Per eseguire il modulo di acquisizione, in modo da iniziare a raccogliere i dati dalle webcam salvate, basterà posizionarsi all'interno della cartella \texttt{acquisition/main/}, e da terminale eseguire il comando:
\begin{lstlisting}
    python3 detect.py
\end{lstlisting}
Se i passi precedenti sono stati eseguiti correttamente allora si visualizzeranno sul terminale i vari passaggi che svolge il modulo.

\subsection{Aggiunta di una webcam}
L'aggiunta di una nuova webcam al modulo di acquisizione è possibile attraverso dei pochi semplici passi:
\begin{enumerate}
	\item Trovare una webcam disponibile all'interno del sito \url{https://www.whatsupcams.com/};
	\item Inserire il link all'interno del file \textit{webcams.json} seguendo lo schema prestabilito per impostare i parametri della webcam;
	\item Salvare il file per ultimare l'aggiunta.
\end{enumerate}
Per una questione di codifica, il link della webcam dev'essere conforme a quelle già presenti, ovvero provenire da \url{https://www.whatsupcams.com/}.


\section{Inizializzazione modulo Prediction}\label{ProceduraDiInstallazioneInizializzazioneModuloPrediction}
Per eseguire il modulo di predizione, che tramite il machine-learning$_G$ si occupa di calcolare, appunto, le predizioni del periodo di tempo futuro, bisogna posizionarsi all'interno della cartella "prediction" ed eseguire il seguente comando da terminale:

\begin{lstlisting}
    python3 DataPrediction.py
\end{lstlisting}
In tal modo verrà attivato il modulo per le predizioni sui dati.


\section{Inizializzazione modulo Web-App}\label{ProceduraDiInstallazioneInizializzazioneModuloWebApp}
Per avviare la web-app$_G$, e le sue funzioni, si devono eseguire alcuni comandi, sempre da terminale, a partire dalla cartella \textit{webapp}.
\begin{enumerate}
  \item posizionarsi all'interno della cartella "webapp" ed eseguire:
  \begin{lstlisting}
    mvn spring-boot:run
  \end{lstlisting}
  \item posizionarsi all'interno della cartella "vue-js-client-crud" ed eseguire:
  \begin{lstlisting}
    npm install
  \end{lstlisting}
  \item infine, all'interno della stessa cartella bisogna eseguire:
  \begin{lstlisting}
    npm run
  \end{lstlisting}
\end{enumerate}
Il primo comando inizializza il server di Spring$_G$ che fornisce i servizi per prelevare le informazioni dal database, mentre i comandi successivi installano prima npm$_G$ e successivamente le dipendenze di ogni file necessarie alla web-app$_{\scaleto{G}{3pt}}$.
