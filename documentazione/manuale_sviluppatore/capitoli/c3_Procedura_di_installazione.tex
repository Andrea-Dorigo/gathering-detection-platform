\chapter{Procedure di installazione}\label{ProceduraDiInstallazione}
Questa sezione terrà conto delle procedure di installazione all'interno del sistema operativo Linux, più precisamente Ubuntu, in quanto utilizzato anche per lo sviluppo del software stesso.

\section{Download repo}\label{ProceduraDiInstallazioneDownloadRepo}
Come prima fase bisogna clonare la repo relativa al progetto \textit{Gathering-Detection-Platform} al seguente sito:
\begin{center}
  \item \url{https://github.com/Andrea-Dorigo/gathering-detection-platform}
\end{center}

\section{Installazione dipendenze}\label{ProceduraDiInstallazioneInstallazioneDipendenze}
Dopo eseguito il passo sopra descritto, è obbligatorio installare le dipendenze necessarie a far eseguire il prodotto software adeguatamente.
Per fare ciò sarà necessario aprire il terminale, all'interno della cartella \textit{gathering-detection-platform}, e copiare il seguente comando:
  \begin{lstlisting}
    sudo apt install python3-opencv python3-pip mongo maven npm Git Git-LFS
    && pip3 install mongoengine
  \end{lstlisting}

Una volta conclusa questa operazione sarà possibile utilizzare il software con tutte le sue funzionalità.

\begin{comment}
Dopo aver clonato la repo nella propria macchina, bisogna accedervi.
Vi sono due modi per farlo:

\begin{itemize}
  \item mediante il terminale: \textit{cd Path/To/File};
  \item mediante lo spostamento all'interno delle cartelle tramite mouse.
\end{itemize}

Una volta nella cartella del software bisogna eseguire il file da terminale

\begin{center}
  \textit{dipendenze.sh}
\end{center}

Successivamente bisogna aspettare che vengano installate tutte le dipendenze necessarie per poter permettere al software di funzionare adeguatamente.

\section{Lancio del software}\label{ProceduraDiInstallazioneLancioDelSoftware}
Finita l'installazione non rimane che eseguire effettivamente il software completo.
Per fare ciò basterà eseguire il file

\begin{center}
  \textit{start.sh}
\end{center}

Ciò farà iniziare la corretta esecuzione del prodotto.
\end{comment}
