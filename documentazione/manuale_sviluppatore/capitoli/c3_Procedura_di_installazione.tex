\chapter{Procedure di installazione}\label{ProceduraDiInstallazione}
Questa sezione terrà conto delle procedure di installazione all'interno del sistema operativo Linux$_G$, più precisamente Ubuntu$_G$, in quanto utilizzato anche per lo sviluppo del software stesso.

\section{Download repo}\label{ProceduraDiInstallazioneDownloadRepo}
Come prima fase bisogna dublicare$_G$ la repository$_G$ relativa al progetto \textit{Gathering-Detection-Platform$_{\scaleto{G}{3pt}}$} al seguente sito:
\begin{center}
  \item \url{https://github.com/Andrea-Dorigo/gathering-detection-platform}
\end{center}

\section{Installazione dipendenze}\label{ProceduraDiInstallazioneInstallazioneDipendenze}
Dopo eseguito il passo sopra descritto, è obbligatorio installare le dipendenze necessarie per eseguire il prodotto software$_{\scaleto{G}{3pt}}$ adeguatamente.
Per fare ciò sarà necessario aprire il terminale, all'interno della cartella \textit{gathering-detection-platform$_{\scaleto{G}{3pt}}$}, e copiare il seguente comando:
\begin{lstlisting}
  sudo apt install python3-opencv python3-pip mongo maven npm Git Git-LFS
  && pip3 install mongoengine
\end{lstlisting}

Una volta conclusa questa operazione il programma potrà essere eseguito senza problemi.

\section{Inizializzazione modulo Acquisition}\label{ProceduraDiInstallazioneInizializzazioneModuloAcquisition}
Per eseguire il modulo di acquisizione, in modo da iniziare a raccogliere i dati dalle webcam salvate, basterà posizionarsi all'interno della cartella "acquisition", dopodiché nella cartella "main" e da terminale eseguire il comando:
\begin{lstlisting}
    python3 detect.py
\end{lstlisting}
Se i passi precedenti sono stati eseguiti correttamente allora si visualizzeranno sul terminale i vari passaggi che svolge il modulo.

\subsection{Aggiunta di una webcam}
Per aggiungere una webcam al modulo Acquisition è molto semplice.
Per una questione di codifica, il link della webcam dev'essere conforme a quelle già presenti, ovvero provenire da \url{https://www.whatsupcams.com/}.
Dopo aver aggiunto il link tratta di accedere al file \textit{webcams.json} ed inserire la webcam prendendo spunto dallo schema delle altre già inserite.


\section{Inizializzazione modulo Prediction}\label{ProceduraDiInstallazioneInizializzazioneModuloPrediction}
Per eseguire il modulo di predizione, che tramite il machina-learning si occupa di calcolare, appunto, le predizioni per lassi di tempo futuri, bisognerà posizionarsi all'interno della cartella "prediction" ed eseguire il seguente comando da terminale:

\begin{lstlisting}
    python3 DataPrediction.py
\end{lstlisting}
In tal modo verrà attivato il modulo per le predizioni sui dati.


\section{Inizializzazione modulo Web-App}\label{ProceduraDiInstallazioneInizializzazioneModuloWebApp}
Per avviare la web-app, e le sue funzioni, si devono eseguire alcuni comandi, sempre da terminale, a partire dalla cartella webapp.
\begin{enumerate}
  \item posizionarsi all'interno della cartella "webapp" ed eseguire:
  \begin{lstlisting}
    mvn spring-boot:run
  \end{lstlisting}
  \item posizionarsi all'interno della cartella "vue-js-client-crud" ed eseguire:
  \begin{lstlisting}
    npm install
  \end{lstlisting}
  \item infine, all'interno della stessa cartella bisogna eseguire:
  \begin{lstlisting}
    npm run
  \end{lstlisting}
\end{enumerate}
