\chapter{Tecnologie coinvolte}\label{TecnologieCoinvolte}
In questa sezione vengono elencate le tecnologie, e librerie di terze parti, utilizzate per sviluppare il prodotto software \textit{Gathering-Detection-Platform}.

\subsection{Tecnologie}\label{Tecnologie}
\subsubsection{Python}\label{TecnologiePython}
\textit{Python$_{\scaleto{G}{3pt}}$} è un linguaggio di programmazione definito "ad alto livello" rispetto alla maggior parte di essi.
Si tratta di un linguaggio orientato ad oggetti, utile a sviluppare script, computazione numerica e sviluppare software.
Nel progetto \textit{Gathering-Detection-Platform}, Python$_{\scaleto{G}{3pt}}$ è il linguaggio su cui si basa tutto il backend$_{\scaleto{G}{3pt}}$, compreso il modulo del machine-learning$_{\scaleto{G}{3pt}}$.

\begin{itemize}
  \item Versione utilizzata: 3.8.x;
  \item Link download: \url{https://www.python.org/downloads/} .
\end{itemize}

\subsubsection{MongoDB}\label{TecnologieMongoDB}
\textit{MongoDB} è stato scelto come database$_{\scaleto{G}{3pt}}$ nel quale salvare i dati ottenuti dal modulo di acquisizione e dal modulo di machine-learning$_{\scaleto{G}{3pt}}$.
Si tratta di un database$_{\scaleto{G}{3pt}}$ non relazionale e orientato ai documenti.
Classificato come tipo NoSQL$_{\scaleto{G}{3pt}}$, MongoDB$_{\scaleto{G}{3pt}}$ non utilizza la classica struttura basata su tabelle ma invece si basa su tipi di documenti JSON$_{\scaleto{G}{3pt}}$, facilitando così l'integrazioni di alcuni tipi di dati.

\begin{itemize}
    \item Versione utilizzata: 4.4.4;
    \item Link download: \url{https://www.mongodb.com/it }.
\end{itemize}


\subsubsection{HTML 5}\label{TecnologieHTML}

\subsubsection{CSS}\label{TecnologieCSS}

\subsubsection{Bootstrap}\label{TecnologieBootstrap}
