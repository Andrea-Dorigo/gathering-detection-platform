\chapter{Tecnologie coinvolte}\label{TecnologieCoinvolte}
In questa sezione vengono elencate le tecnologie, e librerie di terze parti, utilizzate per sviluppare il prodotto software$_G$ \textit{Gathering-Detection-Platform}.

\section{Tecnologie}\label{Tecnologie}
\subsection{Python}\label{TecnologiePython}
Si tratta di un linguaggio di programmazione definito "ad alto livello" rispetto alla maggior parte di essi.
Si tratta di un linguaggio orientato ad oggetti, utile a sviluppare script$_G$, computazione numerica e sviluppare software$_{\scaleto{G}{3pt}}$.
Nel progetto \textit{Gathering-Detection-Platform}, Python è il linguaggio su cui si basa tutto il backend$_G$, compreso il modulo del machine-learning$_G$.

\begin{itemize}
  \item versione utilizzata: 3.8.x;
  \item link download: \url{https://www.python.org/downloads/} .
\end{itemize}

\subsection{API Weather Forecast}\label{APIWeatherForecast}
Si tratta di una Web-API$_G$ che permette la raccolta dati delle previsioni del meteo, sia presenti che future.
Viene utilizzata nel modulo Acquisition per associare i dati raccolti dalle webcams con i dati meteo della giornata, così da ottenere più informazioni relative ad una determinata città.
Queste previsioni possono essere anche scaricate come file CSV$_G$ o JSON.
\begin{itemize}
  \item link download: \url{https://openweathermap.org/history} .
\end{itemize}


\subsection{Kafka}\label{TecnologieKafka}
Kafka viene utilizzato, dal modulo acquisizione, come tramite per inviare i dati ad altre applicazioni utilizzando uno standard di comunicazione comune.
È stato scelto in quanto acquisisce flussi di dati da diverse fonti e permette a molte applicazioni di scambiarsi dati mediante esso, il suo scopo è quello di "centro smistamento" dei dati.

\begin{itemize}
  \item versione utilizzata: 2.8.0;
  \item link download: \url{https://kafka.apache.org/downloads} .
\end{itemize}

\subsection{MongoDB}\label{TecnologieMongoDB}
MongoDB è stato scelto come database$_G$ nel quale salvare i dati ottenuti dal modulo di acquisizione e dal modulo di machine-learning$_{\scaleto{G}{3pt}}$.
Si tratta di un database$_{\scaleto{G}{3pt}}$ non relazionale e orientato ai documenti.
Classificato come tipo NoSQL$_G$, MongoDB non utilizza la classica struttura basata su tabelle, invece si basa sui tipi di documenti JSON, facilitando così l'integrazioni di alcuni tipi di dati.

\begin{itemize}
    \item versione utilizzata: 4.4.4;
    \item link al sito: \url{https://www.mongodb.com/it} .
\end{itemize}

\subsection{Spring Boot} \label{TecnologieSpring}
Spring$_G$ è un framework$_G$ open source$_G$ per lo sviluppo di applicazioni su piattaforma Java. A  questo framework$_{\scaleto{G}{3pt}}$ sono associati altri progetti, in particolare Spring Boot che permette di creare una applicazione autoconfigurata che avvia un server$_G$, il quale mette a disposizione i servizi sviluppati attraverso il codice.
\begin{itemize}
	\item versione utilizzata: 2.4.4;
	\item link al sito: \url{https://spring.io/projects/spring-boot} .
	%\item link download: \url{https://start.spring.io/}??
\end{itemize}

\subsection{Apache Maven}\label{TecnologieMaven}
È una tecnologia utilizza per la gestione software$_{\scaleto{G}{3pt}}$ basati su Java e build automation$_G$.
Per la gestione si serve di un costrutto denominato POM (Project Object Model)$_G$, ovvero un file XML$_G$ in cui vengono dichiarate le dipendenze necessarie fra il progetto e le varie librerie utilizzate.
Maven si occupa di scaricare automaticamente eventuali librerie o plug-in$_G$ mancanti in una cartella predefinita.

\begin{itemize}
	\item versione utilizzata: 3.6.3;
	\item link al sito: \url{https://maven.apache.org/download.cgi} .
	%\item link download: \url{https://start.spring.io/}??
\end{itemize}

\subsection{Java}\label{TecnologieJava}
Si tratta di una piattaforma che ha come caratteristica principale il fatto di rendere possibile scrittura ed esecuzione di applicazioni indipendenti dall'hardware$_G$ di esecuzione.
Il risultato è una virtualizzazione dalla piattaforma stessa, che rende così il linguaggio Java, e i relativi programmi, portabili su piattaforme hardware$_{\scaleto{G}{3pt}}$ diverse.

\begin{itemize}
	\item versione utilizzata: 11.x;
	\item link download: \url{https://www.java.com/it/download/} .
\end{itemize}

\subsection{HTML 5}\label{TecnologieHTML}
HTML, acronimo di HyperText Markup Language, è un linguaggio di mark up per siti web.
Era stato ideato per la formattazione e impaginazione di pagine ipertestuali sul web.
Oggi giorno viene utilizzato soprattutto per gestire la separazione tra la struttura logica della pagina web e la sua rappresentazione, gestita dal CSS.
Nel progetto questo linguaggio viene utilizzato per sviluppare la parte di web-app$_{\scaleto{G}{3pt}}$, interagendo con anche Java, CSS, Bootstrap e Vue.js.


\subsection{CSS 3}\label{TecnologieCSS}
Il CSS è il principale linguaggio utilizzato per definire la formattazione dei siti e pagine web.
L'utilizzo del CSS permette di separare i contenuti della pagina HTML dal proprio layout$_G$, ma anche di rendere la programmazione più chiara e facile da utilizzare, garantendo il riutilizzo di codice e facilitando la manutenzione.
Nel progetto viene utilizzato per formattare il layout$_{\scaleto{G}{3pt}}$ estetico della web-app$_{\scaleto{G}{3pt}}$.

\subsection{Leaflet} \label{TecnologieLeaflet}
Leaflet è una libreria JavaScript$_G$ per sviluppare mappe geografiche, utilizzata nel progetto per realizzare la heat-map$_G$.
\begin{itemize}
	\item versione utilizzata: 1.7.1;
	\item link al sito: \url{https://leafletjs.com/} .
\end{itemize}


\subsection{Vue.js}\label{TecnologieVue}
È un framework$_{\scaleto{G}{3pt}}$ JavaScript, configurato come Model-Control-View$_G$ per la creazione di interfacce utente e applicazione single-page$_G$.
Supporta molte funzionalità, anche avanzate, grazie ad una serie di librerie di supporto dedicate che sono ufficialmente mantenute.

\begin{itemize}
  \item versione utilizzata: 2.6.12;
  \item link al sito: \url{https://vuejs.org/} .
\end{itemize}

\subsection{Node.js}\label{TecnologieNode}
È un runtime system open-source$_{\scaleto{G}{3pt}}$, orientato ad oggetti, per l'esecuzione di codice JavaScript.
Molti moduli di questa tecnologia sono proprio scritti in JavaScript, ed essendo appunto open-source$_{\scaleto{G}{3pt}}$, programmatori esterni possono crearne ed aggiungerne altri.
A differenza di JavaScript che in origine era lato client$_G$, Node.js$_G$ viene utilizzato lato server$_{\scaleto{G}{3pt}}$, ad esempio per produzioni di pagine dinamiche.
Implementa il paradigma "JavaScript everywhere" in modo da unificare lo sviluppo di applicazioni web intorno ad un unico linguaggio di programmazione, JavaScript.

\begin{itemize}
  \item versione utilizzata: 14.16.x;
  \item link download: \url{https://nodejs.org/it/download/} .
\end{itemize}

\subsection{Bootstrap}\label{TecnologieBootstrap}
Framework$_{\scaleto{G}{3pt}}$ open-source$_{\scaleto{G}{3pt}}$ che, mediante le proprie librerie, viene utilizzato per uniformare i vari componenti che compongono un'interfaccia web, oltre che per crearli.

\begin{itemize}
  \item versione utilizzata: 4.3.1;
  \item link download: \url{https://getbootstrap.com/docs/5.0/getting-started/download/} .
\end{itemize}


\subsection{JSON}\label{TecnologieJson}
Si tratta di un formato testuale necessario per l'esportazione ed importazione dei dati presenti nel \textit{modulo} di salvataggio dati, mediante MongoDB, ed esterni al database$_{\scaleto{G}{3pt}}$.
È un formato dati diffuso per lo scambio di essi in applicazioni client-server$_G$.
Basato su oggetti, ovvero coppie chiave/valore, e supporta una moltitudine di dati diversi. Infine è di facile lettura per l'utente e non necessita particolari procedure per modificarlo.


\section{Librerie di terze parti}\label{LibrerieDiTerzeParti}
Insieme alle tecnologie sopra citate, sono state anche integrate delle librerie di terze parti.

\subsection{OpenCV}\label{LibrerieOpenCV}
OpenCV è una libreria software$_{\scaleto{G}{3pt}}$ multipiattaforma specializzata nella visione artificiale in tempo reale.
È stata integrata nel \textit{modulo} adibito alla cattura immagini in tempo reale, in linguaggio python.

\begin{itemize}
  \item versione utilizzata: 4.2.0;
  \item link al sito: \url{https://opencv.org/} .
\end{itemize}

\subsection{Yolo V3}\label{LibrerieYoloV3}
Si tratta di uno script$_{\scaleto{G}{3pt}}$ in linguaggio python per il riconoscimento real-time$_G$ di oggetti in una foto.
Viene utilizzato nel \textit{modulo} di acquisizione dati per il riconoscimento e conteggio delle persone presenti in un singolo frame.

\begin{itemize}
  \item link al sito: \url{https://pjreddie.com/darknet/yolo/} .
\end{itemize}

\subsection{Pandas}\label{LibreriePandas}
È una libreria veloce, potente e flessibile creata appositamente per modellare dati e manipolarli mediante appositi strumenti.
Utilizzata nel \textit{modulo} di machine-learning$_{\scaleto{G}{3pt}}$, basata su python.

\begin{itemize}
  \item versione utilizzata: 1.2.1;
  \item link all'installazione: \url{https://pandas.pydata.org/getting_started.html} .
\end{itemize}

\subsection{Scikit-learn}\label{LibrerieScikitLearn}
È una libreria open source di apprendimento automatico per il linguaggio di programmazione Python.
Al suo interno sono presenti numerosi algoritmi, per la manipolazione dati, tra cui quelli di regressione, utilizzati nel \textit{modulo} di machine-learning$_{\scaleto{G}{3pt}}$.

\begin{itemize}
  \item versione utilizzata: 0.24.1;
  \item link all'installazione: \url{https://scikit-learn.org/stable/install.html} .
\end{itemize}

\subsection{Mongoengine}\label{LibrerieMongoengine}
Si tratta di un document-object mapper$_G$ basato su python ed ideato per lavorare assieme a MongoDB da Python.

\begin{itemize}
  \item versione utilizzata: 0.24.1;
  \item link alla repo$_G$: \url{https://github.com/MongoEngine/mongoengine} .
\end{itemize}

\subsection{NumPy}\label{LibrerieNumpy}
Libreria open-source$_{\scaleto{G}{3pt}}$, basata sul linguaggio python.
Fornisce un grosso supporto a grandi matrici e array multidimensionli, inoltre integra molte funzioni matematiche adatte a lavorare su tali strutture dati.

\begin{itemize}
  \item versione utilizzata: 1.20.1;
  \item link all'installazione: \url{https://numpy.org/install/} .
\end{itemize}

\subsection{Pylint}\label{LibreriePylint}
Strumento utilizzato per l’analisi statica del codice sorgente Python. Viene quindi adottato per controllare la presenza di errori nel codice, con l’obiettivo di applicare uno standard codifica e di promuovere buone prassi di scrittura del codice

\begin{itemize}
  \item versione utilizzata: 2.7.3;
  \item link al sito: \url{https://pypi.org/project/pylint/} .
\end{itemize}

\subsection{PyUnit}\label{LibreriePyUnit}
Si tratta di una libreria per Python dedicata a creare e testare unit-test per programmi scritti con tale linguaggio.

\begin{itemize}
  \item versione utilizzata: segue la stessa versione del linguaggio Python essendo una sua libreria integrata.
  \item link  alla documentazione: https://docs.python.org/3/library/unittest.html .
\end{itemize}


\subsection{Checkstyle}\label{LibrerieCheckstyle}
Strumento che permette di eseguire l’analisi statica del codice java utilizzato nello sviluppo di un progetto software$_{\scaleto{G}{3pt}}$.

\begin{itemize}
  \item versione utilizzata: 2.17;
  \item link alla repo$_{\scaleto{G}{3pt}}$: \url{https://github.com/checkstyle/checkstyle} .
\end{itemize}

\subsection{ESLint}\label{LibrerieESLint}
Strumento di analisi del codice statico per identificare i le problematiche trovate nel codice JavaScript.

\begin{itemize}
  \item versione utilizzata: 2.1.19;
  \item link alla repo$_{\scaleto{G}{3pt}}$: \url{https://www.npmjs.com/package/eslint} .
\end{itemize}

\subsection{Prettier}\label{LibreriePrettier}
Strumento per il controllo automatico della formattazione del codice scritto in linguaggio JavaScript.

\begin{itemize}
  \item versione utilizzata: 6.3.2;
  \item link al sito: \url{https://prettier.io/} .
\end{itemize}

\subsection{JUnit}\label{LibrerieJUnit}
È un semplice framework$_{\scaleto{G}{3pt}}$ per scrivere unit-test ripetibili.

\begin{itemize}
  \item versione utilizzata: 4.12;
  \item link al sito: \url{https://junit.org/junit5/} .
\end{itemize}

\subsection{Mockito}\label{LibrerieMockito}
Si tratta di un framework$_{\scaleto{G}{3pt}}$ open-source$_{\scaleto{G}{3pt}}$ di testing per Java. Consente la creazione di doppi oggetti di test in unit test automatici.

\begin{itemize}
  \item versione utilizzata: 3.9.0;
  \item link al sito: \url{https://site.mockito.org/} .
\end{itemize}
