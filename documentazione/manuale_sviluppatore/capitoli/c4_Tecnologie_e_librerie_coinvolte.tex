\chapter{Tecnologie coinvolte}\label{TecnologieCoinvolte}
In questa sezione vengono elencate le tecnologie, e librerie di terze parti, utilizzate per sviluppare il prodotto software \textit{Gathering-Detection-Platform}.

\section{Tecnologie}\label{Tecnologie}
\subsection{Python}\label{TecnologiePython}
Si tratta di un linguaggio di programmazione definito "ad alto livello" rispetto alla maggior parte di essi.
Si tratta di un linguaggio orientato ad oggetti, utile a sviluppare script, computazione numerica e sviluppare software.
Nel progetto \textit{Gathering-Detection-Platform}, Python$_G$ è il linguaggio su cui si basa tutto il backend$_G$, compreso il modulo del machine-learning$_G$.

\begin{itemize}
  \item Versione utilizzata: 3.8.x;
  \item Link download: \url{https://www.python.org/downloads/} .
\end{itemize}

\subsection{MongoDB}\label{TecnologieMongoDB}
MongoDB$_G$ è stato scelto come database$_G$ nel quale salvare i dati ottenuti dal modulo di acquisizione e dal modulo di machine-learning$_{\scaleto{G}{3pt}}$.
Si tratta di un database$_{\scaleto{G}{3pt}}$ non relazionale e orientato ai documenti.
Classificato come tipo NoSQL$_G$, MongoDB$_{\scaleto{G}{3pt}}$ non utilizza la classica struttura basata su tabelle ma invece si basa su tipi di documenti JSON$_G$, facilitando così l'integrazioni di alcuni tipi di dati.

\begin{itemize}
    \item Versione utilizzata: 4.4.4;
    \item Link download: \url{https://www.mongodb.com/it }.
\end{itemize}


\subsection{HTML 5}\label{TecnologieHTML}
HTML$_G$, acronimo di HyperText Markup Language, è un linguaggio di mark up per siti web.
Era stato ideato per la formattazione e impaginazione di pagine ipertestuali sul web.
Oggi giorno viene utilizzato soprattutto per gestire la separazione tra la struttura logica della pagina web e la sua rappresentazione, gestita dal CSS$_G$.
Nel progetto questo linguaggio viene utilizzato per sviluppare la parte di web-app$_{\scaleto{G}{3pt}}$, interagendo con anche Java$_G$, CSS$_{\scaleto{G}{3pt}}$, Bootstrap$_G$ e Vue.js$_G$.


\subsection{CSS 3}\label{TecnologieCSS}
Il CSS$_G$ è il principale linguaggio utilizzato per definire la formattazione dei siti e pagine web.
L'utilizzo del CSS$_{\scaleto{G}{3pt}}$ permette di separare i contenuti della pagina HTML$_{\scaleto{G}{3pt}}$ dal proprio layout ma anche di rendere la programmazione più chiara e facile da utilizzare, garantendo il riutilizzo di codice e facilitando la manutenzione.
Nel progetto viene utilizzato per formattare il layout estetico della web-app$_{\scaleto{G}{3pt}}$.


\subsection{Java}\label{TecnologieJava}
Si tratta di una piattaforma che ha come caratteristica principale il fatto di rendere possibile scrittura ed esecuzione di applicazioni indipendenti dall'hardware di esecuzione.
Il risultato è una virtualizzazione dalla piattaforma stessa, che rende così il linguaggio Java$_{\scaleto{G}{3pt}}$, e i relativi programmi, portabili su piattaforme hardware diverse.

\begin{itemize}
  \item Versione utilizzata: 11.x;
  \item Link download: \url{https://www.java.com/it/download/}.
\end{itemize}

\subsection{Vue.js}\label{TecnologieVue}
È un framework JavaScript$_G$, configurato come Model-Control-View per la creazione di interfacce utente e applicazione single-page.
Supporta molte funzionalità, anche avanzate, grazie ad una serie di librerie di supporto dedicate che sono ufficialmente mantenute.

\begin{itemize}
  \item Versione utilizzata: 4.5.x;
  \item Link download: \url{https://vuejs.org/}.
\end{itemize}

\subsection{Node.js}\label{TecnologieNode}
È un runtime system open-source$_G$, orientato ad oggetti, per l'esecuzione di codice JavaScript$_{\scaleto{G}{3pt}}$.
Molti moduli di questa tecnologia sono proprio scritti in JavaScript$_{\scaleto{G}{3pt}}$, ed essendo appunto open-source$_{\scaleto{G}{3pt}}$, programmatori esterni possono crearne ed aggiungerne altri.
A differenza di JavaScript$_{\scaleto{G}{3pt}}$ che in origine era lato client$_G$, Node.js$_G$ viene utilizzato lato server$_G$, ad esempio per produzioni di pagine dinamiche.
Implementa il paradigma "JavaScript everywhere" in modo da unificare lo sviluppo di applicazioni web intorno ad un unico linguaggio di programmazione, JavaScript$_{\scaleto{G}{3pt}}$.

\begin{itemize}
  \item Versione utilizzata: 14.16.x;
  \item Link download: \url{https://nodejs.org/it/download/}.
\end{itemize}

\subsection{Bootstrap}\label{TecnologieBootstrap}
Framework open-source che, mediante le proprie librerie, viene utilizzato per uniformare i vari componenti che compongono un'interfaccia web, oltre che per crearli.

\begin{itemize}
  \item Versione utilizzata: 4.x;
  \item Link download: \url{https://getbootstrap.com/docs/5.0/getting-started/download/}.
\end{itemize}



\subsection{JSON}\label{TecnologieJson}
Si tratta di un formato testuale necessario per l'esportazione ed importazione dei dati presenti nel \textit{modulo} di salvataggio dati, mediante MongoDB$_{\scaleto{G}{3pt}}$, ed esterni al database$_{\scaleto{G}{3pt}}$.
È un formato dati diffuso per lo scambio di essi in applicazioni client-server$_G$.
Basato su oggetti, ovvero coppie chiave/valore, e supporta una moltitudine di dati diversi. Infine è di facile lettura per l'utente e non necessita particolari procedure per modificarlo.
