\chapter{Introduzione}\label{Introduzione}

\section{Scopo del documento}\label{IntroduzioneScopoDelDocumento}
Il documento si propone come guida introduttiva del software \textit{GDP: Gathering Detection Platform}, indirizzata agli sviluppatori che ci lavoreranno. Nello specifico è presentata l'architettura del prodotto e l'organizzazione del codice sorgente ed inoltre sono indicate la procedura di installazione in locale e le tecnologie coinvolte.
\section{Scopo del prodotto}\label{1.2}
In seguito alla pandemia del virus COVID-19 è nata l'esigenza di limitare il più possibile i contatti fra le persone, specialmente evitando la formazione di assembramenti. Il progetto \textit{GDP: Gathering Detection Platform} di \textit{Sync Lab} ha pertanto l'obiettivo di \textbf{creare una piattaforma in grado di rappresentare graficamente le zone potenzialmente a rischio di assembramento, al fine di prevenirlo.}
Il prodotto finale è rivolto specificatamente agli organi amministrativi delle singole città, cosicché possano gestire al meglio i punti sensibili di affollamento, come piazze o siti turistici.
Lo scopo che il software intende raggiungere non è solo quello della rappresentazione grafica real-time ma anche quella di poter riuscire a prevedere assembramenti in intervalli futuri di tempo.

Al tal fine il gruppo \textit{Jawa Druids} si prefigge di sviluppare un prototipo software in grado di acquisire, monitorare ed analizzare i molteplici dati provenienti dai diversi sistemi e dispositivi, a scopo di identificare i possibili eventi che concorrono all’insorgere di variazioni di flussi di utenti. Il gruppo prevede inoltre lo sviluppo di un'applicazione web da interporre fra i dati elaborati e l'utente, per favorirne la consultazione.
\section{Glossario}\label{1.3}
Allo scopo di evitare ambiguità a lettori esterni si aggiunge in appendice un glossario dei
termini ambigui o specifici utilizzati nel presente documento che verranno segnalati con una \textit{G} a pedice.

