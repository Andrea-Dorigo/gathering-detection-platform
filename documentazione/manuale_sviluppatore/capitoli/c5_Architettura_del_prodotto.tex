\chapter{Architettura del Prodotto}\label{ArchitetturaDelProdotto}
L'architettura generale del software$_{\scaleto{G}{3pt}}$ \textit{Gathering-Detection-Platform} è un'architettura monolitica distribuita.
Si tratta di due file eseguibili, scritti in Python$_{\scaleto{G}{3pt}}$, che sono rispettivamente il modulo Acquisition e il modulo Prediction.
Il primo viene utilizzato per acquisire le informazione estrapolate dalle live webcams delle città, il secondo invece viene utilizzato per calcolare predizioni su periodi di tempo futuri, utilizzando il machine-learning$_{\scaleto{G}{3pt}}$.
Entrambi sono collegati ad un database, il quale serve per archiviare ed esportare i dati in esso.
Il terzo, ed ultimo, modulo riguarda la web-app$_{\scaleto{G}{3pt}}$ vera e propria, che permette all'utente utilizzatore di visualizzare la heat-map$_{\scaleto{G}{3pt}}$ relativa alla città.
La web-app$_{\scaleto{G}{3pt}}$ è composta da due sotto-moduli definiti rispettivamente back-end$_{\scaleto{G}{3pt}}$ e front-end$_{\scaleto{G}{3pt}}$, all'interno del primo è presente l'applicazione di Spring, la quale imposta un server per eseguire i servizi per il front end, mentre la seconda è la parte dell'applicazione che mostra i dati all'utente finale. Il back-end$_{\scaleto{G}{3pt}}$ e front-end$_{\scaleto{G}{3pt}}$ dialogano attraverso richieste di tipo HTTP, il back-end$_{\scaleto{G}{3pt}}$ mette a disposizione un server$_{\scaleto{G}{3pt}}$ con servizi di tipo REST che effettuano solo query al database di tipo GET. Il front-end$_{\scaleto{G}{3pt}}$ una volta ottenuta l'informazione modifica attraverso metodi javascript$_{\scaleto{G}{3pt}}$ la mappa che l'utente visualizza insieme alle sue componenti.
Nella sezione successiva vengono inseriti i diagrammi di ogni modulo descritto, questo permette di avere una visione globale di ogni modulo nelle sue dipendenze a librerie esterne, nelle sue attività che svolgono e nelle sue componenti. Questi diagrammi sono stati scritti seguendo i principi UML.

\section{Architettura modulo Acquisition}
\subsection{Diagramma dei Package}
Il modulo Acquisition utilizza due librerie esterne a Python$_{\scaleto{G}{3pt}}$, la prima è la libreria di Kafka$_{\scaleto{G}{3pt}}$ per creare applicativi di tipo consumer e producer e la seconda è MongoEngine$_{\scaleto{G}{3pt}}$ per creare una connessione ad un database di MongoDB$_{\scaleto{G}{3pt}}$.
\begin{figure}[H]
  \begin{center}
    \includegraphics[scale=0.6]{../immagini/diag_PB/diag_pack_acqui.png}
    \caption{Diagramma dei package del modulo Acquisition}
  \end{center}
\end{figure}

\subsection{Diagrammi di attività}
Di seguito vengono descritte le attività più importanti svolte nel modulo Acquisition.
\begin{figure}[H]
  \begin{center}
    \includegraphics[scale=0.8]{../immagini/diag_PB/detection.png}
    \caption{Diagramma di attività dell'eseguibile Detection}
  \end{center}
\end{figure}
\begin{figure}[H]
  \begin{center}
    \includegraphics[scale=0.8]{../immagini/diag_PB/detection.png}
    \caption{Diagramma di sotto-attività dell'acquisizione delle previsioni meteo}
  \end{center}
\end{figure}
\begin{figure}[H]
  \begin{center}
    \includegraphics[scale=0.65]{../immagini/diag_PB/download_e_cut_frames.png}
    \caption{Diagramma di sotto-attività di download e taglio frame}
  \end{center}
\end{figure}
\begin{figure}[H]
  \begin{center}
    \includegraphics[scale=0.7]{../immagini/diag_PB/conta_persone.png}
    \caption{Diagramma di sotto-attività del conta persone}
  \end{center}
\end{figure}
\begin{figure}[H]
  \begin{center}
    \includegraphics[scale=0.8]{../immagini/diag_PB/kafka.png}
    \caption{Diagramma di attività di Kafka}
  \end{center}
\end{figure}


\section{Architettura modulo Prediction}
\subsection{Diagramma dei package}
Nel modulo Prediction vengono utilizzate le librerie esterne Pandas$_{\scaleto{G}{3pt}}$, MongoEngine$_{\scaleto{G}{3pt}}$ e Scikit-Learn$_{\scaleto{G}{3pt}}$ (abbreviato in Sklearn nella libreria). La libreria più importante è Scikit-Learn$_{\scaleto{G}{3pt}}$ della quale utilizziamo i metodi per il Preprocessing dei dati, la creazione di modelli con Model\_selection e il tipo di modello per generare le predizioni, il Random Forest Regression.
\begin{figure}[H]
  \begin{center}
    \includegraphics[scale=0.8]{../immagini/diag_PB/diag_pack_pred.png}
    \caption{Diagramma dei package del modulo Acquisition}
  \end{center}
\end{figure}

\subsection{Diagramma di attività}
\begin{figure}[H]
  \begin{center}
    \includegraphics[scale=0.6]{../immagini/diag_PB/prediction_activity.png}
    \caption{Diagramma di attività dell'eseguibile Detection}
  \end{center}
\end{figure}



\section{Architettura modulo Web-app}
\subsection{Diagrammi dei package}
Il modulo della web-app$_{\scaleto{G}{3pt}}$, come descritto in precedenza, è diviso in due sotto-moduli rispettivamente per il back-end$_{\scaleto{G}{3pt}}$ e front-end$_{\scaleto{G}{3pt}}$. Di seguito vengono visualizzate le dipendenze dei due sotto-moduli, per il back-end$_{\scaleto{G}{3pt}}$ è solo necessaria la libreria del framework$_{\scaleto{G}{3pt}}$ Spring$_{\scaleto{G}{3pt}}$, mentre per il front-end$_{\scaleto{G}{3pt}}$ sono necessarie varie librerie per ogni componente della web-app$_{\scaleto{G}{3pt}}$.
\begin{figure}[H]
  \begin{center}
    \includegraphics[scale=0.8]{../immagini/diag_PB/diag_pack_spring.png}
    \caption{Diagramma dei package di Spring}
  \end{center}
\end{figure}

\begin{figure}[H]
  \begin{center}
    \includegraphics[scale=0.65]{../immagini/diag_PB/diag_pack_vue.png}
    \caption{Diagramma dei package del modulo Acquisition}
  \end{center}
\end{figure}

\subsection{Diagrammi delle classi}
\begin{figure}[H]
  \begin{center}
    \includegraphics[scale=0.6]{../immagini/diag_PB/diag_class_spring.png}
    \caption{Diagramma delle classi di Spring}
  \end{center}
\end{figure}

\subsection{Diagramma di sequenza}
\begin{figure}[H]
  \begin{center}
    \includegraphics[scale=0.8]{../immagini/diag_PB/diag_seq_spring.png}
    \caption{Diagramma di sequenza di Spring}
  \end{center}
\end{figure}

\subsection{Diagramma di attività}
\begin{figure}[H]
  \begin{center}
    \includegraphics[scale=0.8]{../immagini/diag_PB/diag_act_front_back.png}
    \caption{Diagramma di attività del modulo Web-app}
  \end{center}
\end{figure}
