\newpage
	\section{Informazioni Generali}
	\begin{enumerate}
		\item \textbf{Luogo:} \normalfont Server Discord \textbf{Jawa Druids};
		\item \textbf{Data:} \normalfont 04-12-2020;
		\item \textbf{Orario inizio:} \normalfont 15.15;
		\item \textbf{Orario fine:} \normalfont 17.15;
		\item \textbf{Partecipanti:}
		\begin{itemize}
			\item Mattia Cocco (nella prima parte della riunione); 
			\item Andrea Dorigo;
			\item Margherita Mitillo;
			\item Igli Mezini;
			\item Andrea Cecchin;
			\item Roveroni Emma;
			\item Graziano Alfredo.
		\end{itemize}
		\item \textbf{Assenti:}
		\begin{itemize}
			\item Mattia Cocco (nella seconda parte della riunione).
		\end{itemize}
	\end{enumerate}
	
	\section{Ordine del giorno}
	Una volta che i membri del \textbf{gruppo di lavoro} erano tutti presenti la riunione ha avuto inizio.
	L'incontro si è incentrato su tre punti:
	\begin{enumerate}
		\item Riassunto del lavoro svolto dal 27/11/2020;
		
		\item Affrontati problemi legati ai documenti di \textbf{Piano di Qualifica} e \textbf{Norme di progetto};
		
		\item Cambio giorno per i meeting del gruppo;
		
		\item Brainstorming per domande da rivolgere all'azienda \textit{Sync Lab} in una email;
		
		\item Suddivisione dei nuovi task fra i membri del gruppo.
	\end{enumerate}
	\section{Riassunto del lavoro svolto dal 27/11/2020}
	Per prima cosa, tutti i presenti al meeting hanno aggiornato gli altri membri del gruppo sul proprio lavoro svolto durante la settimana.
	
	\section{Affrontati problemi legati ai documenti di Piano di Qualifica e Norme di progetto}
	Durante il meeting sono sorti alcuni problemi legati ai documenti di \textbf{Piano di Qualifica} e \textbf{Norme di progetto}. Infatti, il primo, non può essere completato in quanto non è ancora disponibile il materiale che serve per stilarlo. \\
	Per quanto riguarda il secondo, invece, si è deciso di concentrarsi sul completamento del capitolo sulle norme per la documentazione poiché estremamente utili per l'ultimazione degli altri documenti.
	
	\section{Cambio giorno per il meeting del gruppo}
	Il gruppo aveva deciso di sentirsi regolarmente al venerdì pomeriggio, ma, per motivi organizzativi, serve trovare un altro giorno per la riunione. Per questo si è creato un sondaggio sul canale \textbf{Discord} del gruppo per decidere il nuovo giorno.
	
	\section{Brainstorming per le domande da rivolgere \\all'azienda Sync Lab in una email} 
	Il gruppo ha deciso di scrivere un'email all'azienda \textit{Sync Lab}. Tale azienda è la proponente del capitolato C3, ovvero \textit{GDP - Gathering Detection Platform}, che è il capitolato che ha attirato maggiormente l'attenzione e l'interesse di tutti i membri del gruppo \textit{Jawa Druids}. Il gruppo ha individuato alcune domande da rivolgere all'azienda per risolvere qualche dubbio e curiosità riguardo quanto l'azienda ha mostrato durante la presentazione del progetto da loro proposto. 
	
	\section{Suddivisione Task}
	Si è fatta la suddivisione dei \textbf{Task} per continuare la stesura dei documenti attualmente in corso.
	Si è così deciso:
	\begin{enumerate}
		\item Scrittura del \textbf{Verbale Interno} della riunione avvenuta in data 04/12/2020: \textbf{Emma Roveroni};
		\item Scrittura del capitolo sulle norme della documentazione, appartenente al documento \textbf{Norme di Progetto}: \textbf{Igli Mezini} e \textbf{Andrea Dorigo};
		\item Verifica delle \textbf{Norme di Progetto}: \textbf{Margherita Mitillo};
		\item Stesura email per \textit{Sync Lab}: \textbf{Andrea Dorigo};
		\item Verifica \textbf{Verbali Interni} : \textbf{Emma Roveroni};
		\item Stesura scheletro \textbf{Analisi Dei Requisiti} : \textbf{Andrea Cecchin}.
	\end{enumerate}
