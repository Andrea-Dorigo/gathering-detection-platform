\newpage
\section{Informazioni Generali}
\begin{enumerate}
  \item \textbf{Luogo:} \normalfont Server Discord \textbf{Jawa Druids};
  \item \textbf{Data:} \normalfont 29-01-2021;
  \item \textbf{Orario inizio:} \normalfont 15.00;
  \item \textbf{Orario fine:} \normalfont 16.45;
  \item \textbf{Partecipanti:}
  \begin{itemize}
    \item Mattia Cocco;
    \item Andrea Dorigo;
    \item Margherita Mitillo;
    \item Igli Mezini;
    \item Andrea Cecchin;
    \item Emma Roveroni;
    \item Alfredo Graziano.
  \end{itemize}
  \item \textbf{Assenti:}
  \begin{itemize}
    \item Nessuno
  \end{itemize}
\end{enumerate}
\section{Ordine del giorno}
Una volta che tutti membri del gruppo di lavoro erano pronti è iniziata la riunione. L'incontro si è incentrato sui segenti punti:
\begin{itemize}
  \item lettura giudizio sui documenti della RR;
  \item discussione su come modificare i documenti non approvati.
\end{itemize}

\section{Lettura giudizio sui documenti della RR}
Il gruppo ha letto attentamente la relazione stilata dai \textit{Prof. Vardanega} e \textit{Prof. Cardin} riguardo i documenti consegnati previsti per la RR.


\section{Discussione su come modificare i documenti non approvati}
Dopo aver appreso che il giudizio è stato \textbf{Giudizio Sospeso}, per via, soprattutto, del documento \textit{Analisi dei Requisiti}, il gruppo ha subito iniziato a discutere riguardo i punti da modificare in esso.
Inoltre sono stati individuati i punti da modificare anche negli altri documenti, per migliorarli come consigliato nel giudizio.


\section{Decisioni derivate dal colloquio}
  \begin{itemize}
    \item immediata rilettura e studio generale del documento \textit{Analisi dei Requisiti} con lo scopo di renderlo il più approfondito possibile;
    \item introduzione dei diagrammi dei casi d'uso;
    \item individuare e modificare il resto dei documenti;
    \item provvedere ad organizzare un nuovo meeting con l'azienda proponente riguardo l'aggiunta di nuove funzionalità.
  \end{itemize}
