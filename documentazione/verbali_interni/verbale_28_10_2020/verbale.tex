	\newpage
	\section{Informazioni Generali}
	\begin{enumerate}
		\item \textbf{Luogo:} \normalfont Server Discord \textbf{Jawa Druids};
		\item \textbf{Data:} \normalfont 28-10-2020;
		\item \textbf{Orario inizio:} \normalfont 15.30;
		\item \textbf{Orario fine:} \normalfont 17.00;
		\item \textbf{Partecipanti:}
		\begin{itemize}
			\item Mattia Cocco;
			\item Andrea Dorigo;
			\item Margherita Mitillo;
			\item Igli Mezini;
			\item Andrea Cecchin;
			\item Emma Roveroni;
			\item Alfredo Graziano.
		\end{itemize}
		\item \textbf{Assenti:}
		\begin{itemize}
			\item Nessuno.
		\end{itemize}
	\end{enumerate}
	\section{Ordine del giorno}
	Una volta che i membri del \textbf{gruppo di lavoro} erano tutti presenti, il meeting, tenutosi su \textbf{Zoom}, ha avuto inizio.
	L'incontro si è incentrato su tre punti:
	\begin{enumerate}
		\item \textbf{Presentazione Membri};		
		\item Decisione della scelta dei \textbf{Capitolati};
		\item Creazione del \textbf{Nome} e del \textbf{Logo} del \textbf{Gruppo Di Lavoro}.
	\end{enumerate}

	\section{Presentazione Dei Membri}
	All'inizio della riunione, ogni membro del gruppo si è presentato agli altri, dichiarando anche i propri esami arretrati.

	\section{Decisione Capitolati}
	Il gruppo ho svolto un'analisi di tutti i \textbf{Capitolati} proposti dai \textbf{Committenti}, cercando di valutare i punti di forza e le difficoltà di ciascuno. 
	E' stato anche predisposto un \textbf{Sondaggio}, in modo tale che ogni membro del gruppo potesse esprimere le proprie preferenza. 
	La votazione ha evidenziato una propensione verso tre \textbf{Capitolati} che, in ordine di preferenza, sono : C3, C2, C7.
	 
	\section{Creazione Nome e Logo}
	Si è scelto il \textbf{Nome} del gruppo, ossia \textbf{Jawa Druids}.
	Infine, \textbf{Margherita Mitillo} ha creato il  \textbf{Logo} del gruppo, che è il seguente:\\
	\begin{center}
			\includegraphics[width=0.5\linewidth]{../../immagini/DRUIDSLOGO.jpg}\\[4ex]
	\end{center}

