\newpage
	\chapter{Informazioni Generali}
	\begin{enumerate}
		\item \textbf{Luogo:} \normalfont Server Discord \textbf{Jawa Druids};
		\item \textbf{Data:} \normalfont 27-11-2020
		\item \textbf{Orario inizio:} \normalfont 15.35;
		\item \textbf{Orario fine:} \normalfont 17.30;
		\item \textbf{Partecipanti:}
		\begin{itemize}
			\item Mattia Cocco;
			\item Andrea Dorigo;
			\item Margherita Mitillo;
			\item Igli Mezini;
			\item Andrea Cecchin;
			\item Emma Roveroni;
			\item Alfredo Graziano;
		\end{itemize}
		\item \textbf{Assenti:}
	\end{enumerate}
	\newpage
	\chapter{Punti Di Discussione}	
	Una volta che i membri del \textbf{gruppo di lavoro} erano tutti presenti, la riunione ha avuto inizio.
	L'incontro si è incentrato su cinque punti:
	\begin{enumerate}
		\item Riassunto del lavoro svolto dal 19/11/2020
		\item Aggiunti task su Trello
		\item Scelte di utilizzo del comando "commit"
		\item Divisione del materiale su Git
		\item Organizzazione e Suddivisione dei \textbf{Task}
	\end{enumerate}
	
	\chapter{Riassunto del lavoro svolto dal 19/11/2020}
	Inizialmente è stato fatto un riepilogo del lavoro svolto singolarmente dall'ultimo meeting, così da aggiornare gli altri componenti del gruppo.
	
	\chapter{Aggiunti Task su Trello}
	Una volta valutato il punto della situazione dei \textbf{Task} già assegnati precedentemente, sono stati aggiunti nuovi \textbf{Task} da svolgere.
	
	\chapter{Scelte di utilizzo del comando "commit"}
	Si è deciso di utilizzare il titolo del commit su Git in lingua inglese per una questione di comprensione più rapida e compatta, sia da parte dei componenti del gruppo sia da esterni.
	Per quanto riguarda la descrizione del commit, invece, è stato concordato di mantenerla comunque in lingua italiana.
	
	\chapter{Divisione del materiale su Git}
	Per una maggior efficacia il gruppo ha deciso di creare nel \textbf{Branch} "develop", un feature-\textbf{Branch} in cui verranno caricati i \textbf{Documenti}
	in attesa di essere verificati. 
	Una volta revisionati, verranno spostati nel \textbf{Branch} "develop", per poi, infine, essere caricati nel \textbf{Branch}  "Main" per l'approvazione.
	
	\chapter{Organizzazione e Suddivisione dei Task}
	I \textbf{Task} decisi sono stati così suddivisi:
	\begin{itemize}
		\item Stesura \textbf{Verbale} 27/11/2020 e \textbf{Glossario}: Mattia Cocco
		\item Creazione Template LaTeX per i \textbf{Documenti}: Margherita Mitillo
		\item \textbf{Piano di Progetto}: Andrea Dorigo
		\item \textbf{Piano di Qualifica}: Emma Roveroni e Alfredo Graziano
		\item Prosecuzione stesura \textbf{Norme di Progetto}: Igli Mezini e Andrea Cecchin		
	\end{itemize}
	