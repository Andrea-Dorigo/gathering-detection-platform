\documentclass[a4paper,12pt]{report}
\usepackage[utf8]{inputenc}
\usepackage{graphicx}
\usepackage{float}
\usepackage{tabularx}
\usepackage{makecell}
\usepackage{titlesec}
\usepackage{fancyhdr}
\usepackage{lastpage}
\usepackage{xurl}
\usepackage{hyperref}
\usepackage{geometry}

\setcounter{tocdepth}{5}
\setcounter{secnumdepth}{5} %SONO I DUE COMANDI PER AVERE LE SUBSUBSECTION NUMERATE E PRESENTI NELL'INDICE
\titleformat{\chapter}[display]
{\normalfont\bfseries}{}{0pt}{\LARGE} %QUESTO SERVE PER AVERE SOLO IL NOME DEL CAPITOLO CHE VOGLIO IO

\titlespacing*{\chapter}{0cm}{0cm}{0.2cm}

\geometry{
    left=20mm,
    right=20mm,
}

\renewcommand{\contentsname}{Indice} %QUESTO SERVE PER AVERE L'INDICE CON IL NOME CHE VOGLIO IO

\pagestyle{plain}
\fancypagestyle{plain}{
\fancyhf{}
\lhead{\includegraphics[width=3cm]{../immagini/minilogo.jpg}}
\chead{}
\rhead{\fontsize{12}{10}Studio di fattibilità}
\lfoot{}
\cfoot{\thepage\ di \pageref{LastPage}}
\rfoot{}
}

\pagestyle{plain}

\begin{document}
	
	\makeatletter
	\begin{titlepage}
		\begin{center}
			\vspace*{-4,0cm}
			\author{Jawa Druids} 
			\title{Verbale Interno}
			\date{27-11-2020} %LASCIARE QUESTO CAMPO VUOTO, SE LO TOLGO STAMPA LA DATA CORRENTE
			\includegraphics[width=0.7\linewidth]{../immagini/DRUIDSLOGO.jpg}\\[4ex]
			{\huge \bfseries  \@title }\\[2ex] 
			{\LARGE  \@author}\\[50ex]
			\vspace*{-8,0cm}
			\begin{table}[H]
				\centering
				\begin{tabular}{c|c}
					\textbf{Versione} & x.x.x \\
					\textbf{Data approvazione} & xx-xx-xxxx\\
					\textbf{Responsabile} & Nome Cognome\\
					\textbf{Redattori} & Mattia Cocco \\
					\textbf{Verificatori} & \makecell{Nome Cognome \\ Nome Cognome} \\
					%MAKECELL SERVE PER POI ANDARE A CAPO ALL'INTERNO DELLA CELLA
					\textbf{Stato} & stato\\
					\textbf{Lista distribuzione} & \makecell{Jawa Druids \\ Tullio \\ NomeAziendaProponente}\\
					\textbf{Uso} & Interno            
				\end{tabular}
			\end{table}
			\fontsize{16}{10}\textbf{Sommario} \\
			Verbale Interno
		\end{center}
	\end{titlepage}
	\makeatother
	\begin{enumerate}
		\item \textbf{Luogo:} \normalfont Server Discord \textbf{Jawa Druids};
		\item \textbf{Orario inizio:} \normalfont 15.35;
		\item \textbf{Orario fine:} \normalfont 17.30;
		\item \textbf{Partecipanti:}
		\begin{itemize}
			\item Mattia Cocco;
			\item Andrea Dorigo;
			\item Margherita Mitillo;
			\item Igli Mezini;
			\item Andrea Cecchin;
			\item Emma Roveroni;
			\item Alfredo Graziano;
		\end{itemize}
		\begin{itemize}
		\item \textbf{Assenti:}
		\end{itemize}
	\end{enumerate}
	\newpage
	\tableofcontents{}
	\chapter{Punti Di Discussione}	
	Una volta che i membri del \textbf{gruppo di lavoro} erano tutti pronti, la riunione ha avuto inizio.
	L'incontro si è incentrato su cinque punti:
	\begin{enumerate}
		\item Riassunto del lavoro svolto dal 19/11/2020
		\item Aggiunti task su Trello
		\item Organizzazione e Suddivisione dei \textbf{Task}
		\item Scelte di utilizzo del comando "commit"
		\item Divisione del materiale su Git
	\end{enumerate}
	
	\chapter{Riassunto del lavoro svolto dal 19/11/2020}
	Si è fatto un riepilogo del lavoro svolto singolarmente dall'ultimo meeting svolto, così da aggiornare i componenti del gruppo.
	
	\chapter{Aggiunti Task su Trello}
	Una volta valutato il punto della situazione dei \textbf{Task} già assegnati precedentemente, si sono aggiunti nuovi \textbf{Task} da svolgere.
	
	\chapter{Scelte di utilizzo del comando "commit"}
	Si è deciso di utilizzare il titolo del commit su Git in inglese per una questione di comprensione più rapida e compatta, sia da parte dei componenti del gruppo sia da esterni.
	Invece la descrizione del commit si farà comunque in lingua italiana.
	
	\chapter{Organizzazione e Suddivisione dei Task}
	I \textbf{Task} decisi sono stati così suddivisi:
	\begin{itemize}
		\item Stesura \textbf{Verbale} 27/11/2020 e \textbf{Glossario}: Mattia Cocco
		\item Creazione Template LaTeX per i \textbf{Documenti}: Margherita Mitillo
		\item \textbf{Piano di Progetto}: Andrea Dorigo
		\item \textbf{Piano di Qualifica}: Emma Roveroni e Alfredo Graziano
		\item Prosecuzione stesura \textbf{Norme di Progetto}: Igli Mezini e Andrea Cecchin		
	\end{itemize}
	
	\chapter{Divisione del materiale su Git}
	Per una maggior efficacia il gruppo ha deciso di creare nel \textbf{Branch} "develop", un feature-\textbf{Branch} in cui verranno caricati i \textbf{Documenti}
	ancora da essere verificati. 
	Una volta che i \textbf{Documenti} saranno revisionati allora verranno spostati nel \textbf{Branch} "develop", per poi infine essere caricati nel "Main" per l'approvazione.
	
	
	
	
\end{document}
