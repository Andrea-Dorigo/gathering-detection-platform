\newpage
\section{Informazioni Generali}
\begin{enumerate}
  \item \textbf{Luogo:} \normalfont Server Discord \textbf{Jawa Druids};
  \item \textbf{Data:} \normalfont 17-02-2021;
  \item \textbf{Orario inizio:} \normalfont 16.30;
  \item \textbf{Orario fine:} \normalfont 17.30;
  \item \textbf{Partecipanti:}
  \begin{itemize}
    \item Mattia Cocco;
    \item Andrea Dorigo;
    \item Margherita Mitillo;
    \item Igli Mezini;
    \item Andrea Cecchin;
    \item Emma Roveroni;
    \item Alfredo Graziano.
  \end{itemize}
  \item \textbf{Assenti:}
  \begin{itemize}
    \item Nessuno
  \end{itemize}
\end{enumerate}
\section{Ordine del giorno}
Una volta che tutti membri del gruppo di lavoro erano pronti è iniziata la riunione. L'incontro si è incentrato sui segenti punti:
\begin{itemize}
  \item verifica delle modifiche sulla documentazione da riconsegnare;
  \item aggiornamento sulla codifica dei moduli.
\end{itemize}

\section{Verifica delle modifiche sulla documentazione da riconsegnare}
Il gruppo ha verificato le modifiche apportate sia all'\textit{Analisi dei Requisi} sia agli altri documenti necessitanti di esse.


\section{Aggiornamento sulla codifica dei moduli}
È stato fatto un primo aggiornamento sull'avanzamento della codifica dei \textit{moduli}.


\section{Decisioni derivate dal colloquio}
  \begin{itemize}
    \item \textbf{I\_17-02-2021.1}: proseguo delle modifiche dell'\textit{Analisi dei Requisi} e perfezionamento di quelle fatte al resto della documentazione;
    \item \textbf{I\_17-02-2021.2}: aumento del ritmo di codifica dei \textit{moduli} mantenendo il codice leggibile.
  \end{itemize}
