\documentclass[a4paper,12pt]{report}
\usepackage[utf8]{inputenc}
\usepackage{graphicx}
\usepackage{float}
\usepackage{tabularx}
\usepackage{makecell}
\usepackage{titlesec}
\usepackage{fancyhdr}
\usepackage{lastpage}
\usepackage{xurl}
\usepackage{hyperref}
\usepackage{geometry}
\usepackage{makeidx}

\makeindex

\titleformat{\chapter}[display]
{\normalfont\bfseries}{}{0pt}{\LARGE} %QUESTO SERVE PER AVERE SOLO IL NOME DEL CAPITOLO CHE VOGLIO IO

\titlespacing*{\chapter}{0cm}{0cm}{0.2cm}

\geometry{
	left=20mm,
	right=20mm,
}

\renewcommand{\contentsname}{Indice} %QUESTO SERVE PER AVERE L'INDICE CON IL NOME CHE VOGLIO IO

\pagestyle{plain}

\fancypagestyle{plain}{
	\fancyhf{}
	\lhead{\includegraphics[width=3cm]{../../immagini/minilogo.jpg}}
	\chead{}
	\rhead{\fontsize{12}{10}Verbale Interno}
	\lfoot{}
	\cfoot{\thepage\ di \pageref{LastPage}}
	\rfoot{}
}

\pagestyle{plain}
\begin{document}
	
	\makeatletter
	\begin{titlepage}
		\begin{center}
			\vspace*{-4,0cm}
			\author{Jawa Druids} 
			\title{Verbale Interno}
			\date{28-10-2020} %LASCIARE QUESTO CAMPO VUOTO, SE LO TOLGO STAMPA LA DATA CORRENTE
			\includegraphics[width=0.7\linewidth]{../../immagini/DRUIDSLOGO.jpg}\\[4ex]
			{\huge \bfseries  \@title }\\[2ex] 
			{\LARGE  \@author}\\[50ex]
			\vspace*{-8,0cm}
			\begin{table}[H]
				\centering
				\begin{tabular}{c|c}
					\textbf{Verbale Numero} & x.x.x \\
					\textbf{Data approvazione} & xx-xx-xxxx\\
					\textbf{Responsabile} & Nome Cognome\\
					\textbf{Redattori} & Mattia Cocco \\
					\textbf{Verificatori} & \makecell{Nome Cognome \\ Nome Cognome} \\
					%MAKECELL SERVE PER POI ANDARE A CAPO ALL'INTERNO DELLA CELLA
					\textbf{Stato} & stato\\
					\textbf{Lista distribuzione} & \makecell{Jawa Druids \\ Tullio \\ NomeAziendaProponente}\\
					\textbf{Uso} & Interno            
				\end{tabular}
			\end{table}
			\fontsize{16}{10}\textbf{Sommario} \\
			Verbale Interno			
		\end{center}
	\end{titlepage}
	\makeatother	
	\newpage
	\begin{enumerate}
		\item \textbf{Luogo:} \normalfont Server Discord \textbf{Jawa Druids};
		\item \textbf{Orario inizio:} \normalfont 15.30;
		\item \textbf{Orario fine:} \normalfont 17.00;
		\item \textbf{Partecipanti:}
		\begin{itemize}
			\item Mattia Cocco;
			\item Andrea Dorigo;
			\item Margherita Mitillo;
			\item Igli Mezini;
			\item Andrea Cecchin;
			\item Emma Roveroni;
			\item Alfredo Graziano;
		\end{itemize}
		\begin{itemize}
			\item \textbf{Assenti:}
		\end{itemize}
	\end{enumerate}
	\newpage
	\tableofcontents{}
	\chapter{Punti Di Discussione}
	Una volta che i membri del \textbf{gruppo di lavoro} erano tutti pronti, il meeting, tenutosi su \textbf{Zoom}, ha avuto inizio.
	L'incontro si è incentrato su tre punti:
	\begin{enumerate}
		\item \textbf{Presentazione Membri}.		
		\item Decisione della scelta dei \textbf{Capitolati}.
		\item Creazione del \textbf{Nome} e del \textbf{Logo} del \textbf{Gruppo Di Lavoro}.
	\end{enumerate}

	\chapter{Presenzatione Dei Membri}
	Ad inizio riunione i membri del gruppo si sono presentati per farsi conoscere dagli altri partecipanti, dichiarando gli esami arretrati.
	
	\chapter{Decisione Capitolati}
	I membri del gruppo hanno votato 3 preferenze tramite \textbf{Sondaggio} riguardanti la scelta dei \textbf{Capitolati} proposti dai \textbf{Committenti}.
	La votazione risultante ha visto, in ordine di preferenze, i seguenti \textbf{Capitolati}: C3, C2, C7.
	 
	\chapter{Creazione Nome e Logo}
	Infine si è optato per creare sia il \textbf{Nome} che il \textbf{Logo} del gruppo.
	Il gruppo si chiama \textbf{Jawa Druids} e il \textbf{Logo}, creato da \textbf{Margherita Mitillo} è il seguente:\\
	\includegraphics[width=0.5\linewidth]{../../immagini/DRUIDSLOGO.jpg}\\[4ex]

\end{document}
